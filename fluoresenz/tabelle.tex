\documentclass[12pt]{article}
\usepackage{amsmath,mathtools}
\usepackage[usenames,dvipsnames]{xcolor}
\usepackage[bitstream-charter]{mathdesign}
\usepackage{microtype}
\usepackage[utf8]{inputenc}
\usepackage[T1]{fontenc}
\usepackage{libertine}
\usepackage{graphicx}
\usepackage{siunitx}
\usepackage[german]{babel}
\usepackage{comment}
\usepackage{nicefrac}
\usepackage{booktabs}
\usepackage{float}
\usepackage{tikz}
\usetikzlibrary{arrows,chains,matrix,positioning,scopes,patterns,decorations.pathmorphing,shapes}
\usepackage{pgfplots}
\pgfplotsset{compat=newest}
\usepackage[version=4]{mhchem}
\usepackage[backend=biber,sorting=none,autocite = superscript,natbib=true]{biblatex} \addbibresource{books.bib}

\floatstyle{plaintop}
\restylefloat{table}
\usepackage[justification=justified,singlelinecheck=false]{caption}
% header
\usepackage{fancyhdr}
\fancyhf{}
\rhead{02.12.2016}
\lhead{Nevroz Arslan, Justin König Gruppe 6}
\setlength{\headheight}{15pt}
\rfoot{\thepage }
\lfoot{Versuch 1: Fluoreszenzspektroskopie}
\pagestyle{fancy}
%\renewcommand{\thesection}{\arabic{section}}% Remove section references...
%\renewcommand{\thesubsection}{\arabic{subsection}}%... from subsections
%\usepackage[backend=biber,sorting=none]{biblatex} \addbibresource{books.bib}

\usepackage{url}
\usepackage{csquotes}
\sisetup{
  round-mode          = figures,
  round-precision     = 2,
  inter-unit-product =\cdot,
   group-digits=true,          %% Zifferngruppierung an/aus
        scientific-notation = true,
}

\renewcommand{\arraystretch}{1.5}
\definecolor{skyblue1}{RGB}{135, 206, 250}
\definecolor{flame}{RGB}{226, 88, 34}
\definecolor{scarletred1}{RGB}{252, 40, 71}
\definecolor{cyanblau}{RGB}{0, 158, 224}
\definecolor{charcoal}{rgb}{0.21, 0.27, 0.31}

\begin{document}


\begin{table}[!ht]
 \begin{tabular}{lccccc}
 pH-Wert & $\lambda_{An.}$ \si{\nano\meter} & $\lambda_{max}$ \si{\nano\meter} & $I_{max}$ & $I_{350}$ & $I_{420}$ \\
\hline
2.000& 320 & 354 & \num{1.3172177e5}& \num{1.312e5} & \num{1.7145e4} \\
3.986& 320 & 351 & \num{1.12065891e5}& \num{1.11589e5} & \num{4.75454e4}\\
5.925& 320 & 355 & \num{1.2808172e5}& \num{1.12685039e5} & \num{4.9373e4}\\
8.158& 320 & 355 & \num{1.1583568e5}& \num{1.11267211e5}& \num{5.589716e4}\\
9.973& 320 & 415 & \num{1.144213915e5}& \num{2.25556e4} & \num{1.12899375e5}\\
12.198& 320 & 414 & \num{1.61930719e5}& \num{8.50769e2} & \num{1.59646828e5} \\
\end{tabular}
  % \renewcommand\thetable{3}
  \caption{Die Intensitäten aus der Emissionsspektren bei der Anregungswellenlänge $\lambda_{An.}$ = 320 }
\end{table}

\begin{table}[!ht]
 \begin{tabular}{lcccc}
 pH-Wert & $\lambda_{An.}$ \si{\nano\meter} & $\lambda_{max}$ \si{\nano\meter} & $I_{max}$ & $I_{420}$  \\
\hline
2.000& 360 &  411 & \num{2.2639e3}& \num{1.779e2}  \\
3.986& 360 &  411 & \num{2.31473e3}& \num{1.9123e2} \\
5.925& 360 &  411 & \num{2.316608e3}& \num{2.42999e2} \\
8.158& 360 &  411& \num{1.08663e4}& \num{8.87501e3}\\
9.973& 360 &  411 & \num{1.25291e5}& \num{1.2253e5}\\
12.198& 360 & 411 & \num{1.89882531e5}& \num{1.86032e5}  \\
\end{tabular}
  % \renewcommand\thetable{3}
  \caption{Die Intensitäten aus der Emissionsspektren bei der Anregungswellenlänge $\lambda_{An.}$ = 360 }
\end{table}




\end{document}
