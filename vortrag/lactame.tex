% Page Layout
\documentclass[12pt]{beamer}
\usepackage[latin1]{}
\usepackage{tikz}
\usepackage{chemfig}
\usepackage{xcolor}

%\documentclass[12pt]{article}

%\usepackage{fancyhdr}
%\usepackage{chemfig}
%\usepackage{sectsty}
%\usepackage{setspace}
%\usepackage{helvet}
%\usepackage{graphicx}
%\usepackage{overcite}
%\usepackage{parskip}
%\usepackage{siunitx}
\usepackage[version=4]{mhchem}
\usepackage[method=mhchem]{chemmacros}
%\usepackage{libertine}
%\usepackage[scaled=.83]{beramono}
\usepackage{graphicx}
%\usepackage{hyperref}
%\usepackage[a4paper]{geometry}
%\usepackage{unicode-math}
%\usepackage{microtype}
%\usepackage{xcolor}
%\usepackage{chemformula}
\usetikzlibrary{positioning,calc,arrows}
\usepackage{url}
\usepackage[german]{babel}
%\usepackage{libertine}
%\renewcommand*\familydefault{\sfdefault}  %%
\usepackage[default]{opensans}
%\usepackage{gillius2}
%\usepackage{lato}
\usepackage{adjustbox}
\usepackage[utf8]{inputenc}

\usepackage[T1]{fontenc}

\usepackage{array} % needed for \arraybackslash
\usepackage{graphicx}
\usepackage{adjustbox} % for \adjincludegraphics
\usepackage{biblatex}
\usepackage{tabularx}
\addbibresource{lactam.bib}

 % Avoid an error due to a lack of registers
\renewcommand*\printatom[1]{{\fontsize{14}{14}\selectfont\ensuremath{\mathsf{#1}}}}
\definesubmol\cycleoplus{-[,0.25,,,draw=none]\oplus}
\definesubmol{e}{-[,.1,,,draw=none]}
\definesubmol\nobond{[,0.2,,,draw=none]}

\definecolor{orange}{RGB}{255,127,0}
%\include{head}
\title[]{Lactame und ihre Verwendung}
\author[N. Arslan]{Nevroz Arslan}
\date[14.10.16]{16. Oct 2016}
%\titlegraphic{\includegraphics[width=\textwidth,height=.5\textheight]{bogen1200.png}}
\setbeamerfont{page number in head/foot}{size=\large}
\beamertemplatenavigationsymbolsempty
\setbeamertemplate{footline}[frame number]

\setdoublesep{4pt}
\setatomsep{2em}

\begin{document}
\setbeamercolor{block title}{use=structure,fg=black,bg=white}
\setbeamercolor{block body}{use=structure,fg=black,bg=white}
%\input{title.tex}
%\setbeamerfont{bibliography item}{size=\scriptsize}
%\setbeamerfont{bibliography entry author}{size=\scriptsize}
%\setbeamerfont{bibliography entry title}{size=\scriptsize}
%\setbeamerfont{bibliography entry location}{size=\scriptsize}
%\setbeamerfont{bibliography entry note}{size=\scriptsize}
%\setbeamerfont{bibliography entry url}{size=\scriptsize}


%%%%%%%%%%%% Start of content %%%%%%%%%%%%

%\input{seite11.tex}
\usebackgroundtemplate{

\includegraphics[width=1.0\paperwidth,height=0.17\paperheight]{bogen1200}
\begin{tikzpicture}[overlay, remember picture]
    \node[xshift=-10.80cm,yshift=1.15cm] at (0,0)    {\includegraphics[scale=0.6]{logo}};
\end{tikzpicture}
}

\addtobeamertemplate{frametitle}{\vskip+5ex}{}
\setbeamercolor{frametitle}{fg=black}


\setbeamertemplate{itemize/enumerate body begin}{\normalsize}

\frame{\titlepage}

\newcommand\setpolymerdelim[2]{\def\delimleft{#1}\def\delimright{#2}}
\def\makebraces[#1,#2]#3#4#5{%
\edef\delimhalfdim{\the\dimexpr(#1+#2)/2}%
\edef\delimvshift{\the\dimexpr(#1-#2)/2}%
\chemmove{%
\node[at=(#4),yshift=(\delimvshift)]
{$\left\delimleft\vrule height\delimhalfdim depth\delimhalfdim
width0pt\right.$};%
\node[at=(#5),yshift=(\delimvshift)]
{$\left.\vrule height\delimhalfdim depth\delimhalfdim
width0pt\right\delimright_{\rlap{$\scriptstyle#3$}}$};}}
\setpolymerdelim()
\setatomsep{2em}
\definesubmol{&}{-[,,,,draw=none]}
\begin{frame}
  \begin{itemize}
    \item Lactame werden von Carbonsäuren abgeleitet
    \item \scalebox{0.5}{
    \schemestart
\chemname{\chemfig{H_2N-[:-30]-[:30]-[:-30]-[:30](=[2]O)-[:-30]OH}}{\\ \\ $\gamma$-Aminobuttersäure }
\arrow{->[][][7mm]}
\chemname{\chemfig{ *5(-\chembelow{N}{H}-(=O)---)}}{\\ \\ $\gamma$-Butyrolactam}
\+{,,7pt}
\arrow(--[yshift=8mm]){0}[,0]
\chemfig{H_2O}
\schemestop
\chemnameinit{}
}
  \end{itemize}
\end{frame}


\begin{frame}
\frametitle{Wichtige Lactame}

\setdoublesep{3pt}
\setbondstyle{line width=1pt}
  \centering
\adjustbox{max height=\dimexpr\textheight-5.5cm\relax,
           max width=\textwidth}{
 \begin{tabularx}{\textwidth}{rXXX}
 ~ &  ~ &  ~  & Ausgangsstoff \\[4ex]
\scalebox{0.6}{\chemfig{ *5([:-35]--*6(--*6(--{NH}-(=[:40]O)---[,,,,draw=none])-[,,,,draw=none]--)-[,,,,draw=none]--)}} &\scriptsize  Laurinlactam & \scriptsize Beckman & \scalebox{0.6}{\chemfig{=[:30]-[:-30]=[:30]}}  \\[4ex]
\scalebox{0.6}{\chemfig{ *7(---{NH}-(=[:60,0.8]O)---)}} & \scriptsize $\epsilon$-caprolactam & \scriptsize Beckman & \scalebox{0.6}{\chemfig{*6(----(=O)--)}} \\[4ex]
\scalebox{0.6}{\chemfig{ *6(---{NH}-(=[,0.8]O)--)} }& \scriptsize $\delta$-valerolactam & \scriptsize Beckman& \scalebox{0.6}{\chemfig{*5(---(=O)--)}} \\[4ex]
\scalebox{0.6}{\chemfig{ *5(--{NH}-(=[,0.8]O)--)} }&\scriptsize $\gamma$-butyrolactam  & \scriptsize Kondensation & \scalebox{0.6}{\chemfig{H_2N-[:-30]-[:30]-[:-30]-[:30](=[2]O)-[:-30]OC_2H_5} } \\[4ex]
\scalebox{0.6}{\chemfig{ *4(-{NH}-(=[:60,0.8]O)---)} }& \scriptsize $\beta$-lactam & \scriptsize Biosynthese & \scalebox{0.6}{\chemfig{[:-90]\chemabove{N}{H}(-[:210](-[:150]R)=[:-90]O)-[:-30]*4(-(=O)-N*5(-(-(=[::-60]O)-[::+60]OH)-(-[::+0])(-[::-108])-S-)--)}}\\[4ex]
\end{tabularx}
}

\end{frame}

\begin{frame}
  \frametitle{Lactame zur Faserherstellung}
\begin{itemize}
  \item langkettige Polyamiden
  \item üblich über Anionische Ringöffnende Polymerisation
\end{itemize}
\end{frame}

\begin{frame}{Anionische Ringöffnende Polymerisation}
\scalebox{0.5}{
    \schemestart
\chemname{\chemfig{ *7(---\chemabove{N}{\hspace{3mm}\scriptstyle -}-(=[:60]O)---)}}{Cprolactam}
\+
\chemfig{ *7(---N(-H)-(=[:60]O)---)}
\arrow{<=>}
\chemname{\chemfig{ *7(---N(-C(=[2]O)-C|{(CH_2)_5}N-\chemabove{H}{\hspace{2mm}\scriptstyle -})-(=[:60]O)---)}}{Polymer}
\schemestop
}
\vspace{10mm}
\scalebox{0.5}{
 \schemestart
\chemfig{ *7(---\chemabove{N}{\hspace{3mm}\scriptstyle -}-(=[:60]O)---)}
\+
\chemfig{ *7(---N(-H)-(=[:60]O)---)}
\arrow{<=>}
\chemfig{ *7(---N(-C(=[2]O)-C|{(CH_2)_5}\chemabove{N}{\hspace{2mm}-H\scriptstyle -})-(=[:60]O)---)}
\schemestop
}
\end{frame}
\begin{frame}

\end{frame}

\end{document}
