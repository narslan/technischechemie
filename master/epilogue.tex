\section{Epilog}
Was sind das für Geschichten, die wir alle brauchen, und die in
keinem Geschichtsbuch stehen? Was sind das für Geschichten, die nicht in den
Klassenzimmern, sondern in den Häusern der Schüler\_innen erzählt werden? Was
sind das für Geschichten, die sich uns erst eröffnen, wenn wir die Bücher
schließen und uns der Dichtung und dem Spiel zuwenden?

Meine Auseinandersetzungen mit den unerzählten Geschichten, den Leerstellen, die
das hegemoniale Denken an der Universität konstituieren hat mich zu den
Erfahrungen der Subjekte geführt, die in diesem hegemonialen Denken an den Rand
gedrängt werden. Dass diese Erfahrungen in der hegemonialen Sprache
unbeschreibbar bleiben und somit auch hier nicht erzählt werden  können, zeigt,
dass die Räume

\begin{myenv}
\textit{Das Schwarze Subjekt ist eine Spiegelung und, sobald es sich selbst erfährt und
es erfährt sich immerzu selbst, eine Selbst-Spiegelung; es ist eine Spiegelung
von Spiegelungen von Spiegelungen. Es kommen Sätze vor, Erinnerungen,
vergebliche Versuche, es kommt eine Bitterkeit vor, die nur aus der Erfahrung
rassistischer Herabwürdigung resultieren kann, deren Subtilität die Sprachen
verschlägt, deren Unverfrorenheit Wut gefriert. Es kommen Sätze über Sätze,
Erinnerungen über Erinnerungen, Sprachlosigkeiten über Sprachlosigkeiten vor,
die sich zu einem Leben aufschichten, das an den Beulen, die es sich in der
Praxis der Gebundenheit zufügt, Befreiung ebenso wie Resignation erfährt, das
gewusst wird, das unerreichbar ist, das sich dem Lebenden entzieht, und ihn als
Schwarzes Subjekt in die Welt einführt, ihn in der Welt bestätigt.\footnotemark
\footnotetext{Mecheril, „Der doppelte Mangel, der das Schwarze Subjekt hervorbringt“ in Mythen, Masken und Subjekte, herausgegeben von Maureen M. Eggers ( unrast Verlag: Münster, 2005), 78.}}
\end{myenv}

Subjekte sind mehr und gleichzeitig weniger als die Summe ihrer Erfahrungen.
Weniger, weil den individuellen Erfahrungen stets ein eine Geschichte zu Grunde
liegt, durch die hindurch die Erfahrung sichtbar wird, erinnert werden kann, in
Sätze geformt und sich in den Körper einschreiben kann. Und sie sind mehr, weil
die Erfahrungen, als gesellschaftlich informierte Zugänge, die zwischen dem
Innen und Außen vermitteln, niemals beständig und klar, sich niemals lückenlos
und ohne Widerstände in Körper und Geist einfügen können. Sie widerfahren viel
eher einem Subjekt, das sich aneignend oder abweisend gegenüber ihnen verhält,
lernt die Erfahrung zu nutzen, sie zu transformieren. Ein Subjekt, das
Strategien entwickelt um sich die Erfahrungen zu eigen oder sie sie zu einem
anderen zu machen. Eines, das um die kontingente Lesbarkeit der eigenen
Erfahrungen weiß, das mehr ist als ihre Summe, es ist das, was zwischen Erlebtem
und Diskurs vermittelt.
\\

Diese Arbeit kann unter anderem zeigen, dass ein kritischer Erfahrungsbegriff
einen reflexiven Umgang mit erfahrungsorientiertem Sprechen ermöglichen kann in
dem ein Subjektverständnis gelebt und praktiziert wird, das sich nicht nur als
Spiegel von Diskursen und auch nicht als davon unabhängig begreift. Eines um
diese Pole weiß und das sich darum kritisch mit dem Absolutheitsanspruch der
eigenen Erfahrungen auseinandersetzen kann, ohne darin zu verfallen, die eigenen
Erfahrungen zu negieren.
\\

Ein Subjekt, das sich bejaht und zugleich aber auch Erfahrung als das erkennt,
was ihm eine partikulare Perspektive ermöglicht. So kann ein Dialog unter
Bedingungen von Differenz jenseits der Identität erprobt werden, durch den die
Erzählenden zugleich Teil der Geschichte und Möglichkeit ihrer Veränderung
werden. Veränderung schließt dabei immer auch das Selbst mit ein.
