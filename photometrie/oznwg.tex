\documentclass[10pt]{article}
\usepackage{amsmath}
%encoding
%--------------------------------------

%----------------------------
\usepackage{microtype}
%\usepackage{fourier}
\usepackage{helvet}
\usepackage{arevtext,arevmath}

\usepackage[utf8]{inputenc}
\usepackage[T1]{fontenc}
\usepackage[german]{babel}
\usepackage{mathastext}
%units
%--------------------------------------
\usepackage{siunitx}
%drawing
%------------------------------------
\usepackage{tikz} % To generate the plot from csv
\usepackage{pgfplots}
\usepackage{pgfplotstable}
\usepackage{graphicx}

\usepackage{caption}
\usepackage{color, colortbl}
\usepackage{tabularx}
\usepackage{setspace}
\usepackage{mhchem}
\usepackage{fancyhdr}
\pagestyle{fancy}

\rhead{ Gruppe 24}
\lhead{\textbf{Photometrie - Berechnung der NWG}}
\pagenumbering{gobble}

\makeatletter
\setlength{\@fptop}{0pt}
\makeatother

\renewcommand{\familydefault}{\sfdefault}
\captionsetup{font=footnotesize}

\definecolor{LightCyan}{rgb}{0.88,1,1}
\usetikzlibrary{datavisualization}
\pgfplotsset{compat=newest} % Allows to place the legend below plot
\usepgfplotslibrary{units} % Allows to enter the units nicely
\usepackage{overcite}
\renewcommand\citeform[1]{[#1]}

\sisetup{
  round-mode          = places,
  round-precision     = 2,
  inter-unit-product =\ensuremath{{}\cdot{}}
}
\usepackage[landscape]{geometry}

%\usetikzlibrary{arrows, positioning, calc, datavisualization}


\usepackage{url}

\begin{document}

\begin{table}[ht!]
  \centering
 \begin{tabularx}{\textwidth}{lllllllll}
$x$ & $y_1$ & $y_2$ &$y_3$ & $y_4$ & $y_1^2$ & $y_2^2$ & $y_3^2$ &$y_4^2$  \\
\hline
0.00 &-0.075126 & -0.075132 &  -0.075398 & -0.075585 &0.005644 &0.005645 &0.005685 & 0.005713 \\
0.25 &-0.040223 & -0.040661 &  -0.040852 & -0.040569 &0.001618 &0.001653 &0.001669 & 0.001646 \\
0.50 &0.015565  & 0.015389  &  0.015438  &  0.016336 &0.000242 &0.000237 &0.000238 & 0.000267 \\
1.00 &0.075279  & 0.075220  &  0.074893  &  0.074960 &0.005667 &0.005658 &0.005609 & 0.005619 \\
2.00 &0.153527  & 0.153527  &  0.153389  &  0.153601 &0.023571 &0.023571 &0.023528 & 0.023593 \\
\end{tabularx}
   \renewcommand\thetable{2}
\end{table}

\begin{table}[ht!]
  \centering
 \begin{tabularx}{\textwidth}{XXXXX}
$x^2$ & $xy_1$ & $xy_2$ &$xy_3$ &$xy_4$\\
\hline
  0.0000& -0.000000 & -0.000000& -0.000000 &-0.000000\\
  0.0625& -0.010056 & -0.010165& -0.010213 &-0.010142\\
  0.2500& 0.007783  & 0.007694  &  0.007719 &0.008168 \\
  1.0000& 0.075279  & 0.075220  &  0.074893 &0.074960 \\
  4.0000& 0.307054  & 0.307054  &  0.306778 &0.307202 \\
\end{tabularx}
   \renewcommand\thetable{2}
\end{table}
\begin{table}[ht!]
  \centering
 \begin{tabularx}{\textwidth}{XXXXXXXX}
Anzahl & $\sum(x)$ & $\sum(y)$ &$\sum(x^2)$ & $\sum(y^2)$ & $\sum(xy)$ &   $\overline x$ & $\overline y$ \\
 5 & 3.75 & 0.513578 & 5.3125 & 0.607841 & 1.519228 &  0.1875 & 0.025679 \\
\hline
  & $Q_{xx}$&  & m & b & $s_{y,x}$ & $s_{x0}$&  \\
 & 4.609375 & & 0.308704 & -0.032203 & 0.092912 & 0.3&  \\

\end{tabularx}
   \renewcommand\thetable{2}
\end{table}


\end{document}

%%\addplot[domain=0:1200]{-1.42e-5*x^(2)+2.836e-2*x+45.47};

