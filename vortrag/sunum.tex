\documentclass{article}
\usepackage{amsmath}


%encoding
%--------------------------------------

%----------------------------
\usepackage{microtype}
\usepackage{fourier}
\usepackage[utf8]{inputenc}
\usepackage[T1]{fontenc}

\begin{document}
Folie Nr. 1
Ich möchte heute über Lactame und ihre Verwendung sprechen. 
Dabei gehe ich sowohl auf die Chemie der Laktame als auch auf ihre alltägliche Bedeutung ein. 


Folie Nr. 2
Zunächst möchte ich über die Struktur von Lactamen sprechen: 
	
	Lactame sind eine Stoffklasse aus der organischen Chemie und werden von Aminosäuren abgeleitet.
	Ein Beispiele dafür ist: 
		Das Gammabytorolactam wird aus GammaAminobuttersäure abgeleitet.
	Ein anderes Beispiel ist: 
		Das Bettalactam wird aus Betta Alanin abgeleitet. 

Bei Lactamen ist es sinnvoll folgende Nomenklatur zu verwenden: 
		Die griechischen Buchstaben geben an, wie viele Kohlenstoffatome im Ring vorhanden sind: 
		Alpha steht für zwei, Betta für drei uns so weiter und Omega steht für n, ( Also wenn die Ringgröße größer als 6 ist, dann benutzt man n anstelle einer Zahl)


Folie 3
Bevor ich über die Chemie der Lactame spreche, möchte ich jene Lactame zeigen, die in der industriellen Anwendung eine wichtige Rolle einnehmen. Dabei unterscheide ich zwischen 
		1.) Lactamen ohne funktionelle Gruppen und
		2)  und Lactamen mit funktionellen Gruppen

Auf dieser Tabelle sind die Lactame, die nicht funktionalisiert sind abgebildet. 
Diese Laktame werden für die Faserherstellung benutzt. 
Das Polymerisationsprodukt dieser Lactame sind Polyamidketten. 
Die Monomere werden durch Peptidbindungen miteinander verknüpft ( wie Proteine)
Sie stellen deen Charakter eines Polyamids dar. 
Polymere werden häufig durch Nummern und Buchstaben charakterisiert. Zum Beispiel: Pa6 ( Pa steht für Polymamid und 6 steht für die Anzahl der C Atome)
Der robuste Polyamidcharakter ermöglicht vielseitige technische Verwendungsmöglichkeiten. 


Folie 4
Im Alltag begegnen wir diesen Laktamen (ohne funktionelle Gruppen) im Textilbereich.
Da die Polyamide sich durch eine gute Reißfestigkeit auszeichnen werden sie für Textilien verwendet, die extremen Belastungen standhalten müssen wie z.B. Fallschirme und Airbags.

Ein weiteres Merkmal neben der Reisfestigkeit ist die Elastizität der Polyamide. Darum werden die Polyamide z.B. in Teppiche und Nylonstrümpfe eingearbeitet. 



Folie 5
"Wie stellt man diese Polyamide eigentlich her?"
Jetzt komme ich zur Synthese für die Faserherstellung.

Die Polymerisation der Lactame gelingt über drei Wege:
1. Hydrolytische Polymerisation
2) Anionische Polymerisation
3) Kationische Polymerisation

Das wichtigste Verfahren ist die Hydrolytische Polymerisation. Der Grund dafür ist, dass bei der  Anionischen Polymerisation die Reaktionswärme schlecht reguliert werden kann. Bei der Kationischen Polymerisation hingegen gibt es nur geringe Umsätze. Darum wird am häufigsten die Hydrolytische Polymerisation angewendet. 




Folie 6
Nun möchte ich auf die Chemie der Polymerisation eingehen und dafür die Anionische Polymerisation näher betrachten: 

Die Anionische Polymerisation hat drei Phasen: 
1) In der ersten Phase wird das Lactam deprotoniert. Hierbei entsteht ein starkes Nykleophil
2) In der zweiten Phase greift dieses starke Nekleophil ein anderes Lactam an und lagert sich ein. (Dabei entsteht ein Oligomer. )
3) In der dritten Phase können zwei Dinge passieren: 
	Entweder, das Oligomer addiert sich (wie in der zweiten Phase)
	Oder das Oligomer wird ein anderes Lactam deprotonieren und dadurch die Polymerisation beenden. 




Folie 7
Die Reaktivität aller Lactame wird durch die Ringgröße beinflusst. 
Aus dieser Tabelle kann entnommen werden, dass die Reaktivität bei der Polymerisation mit zunehmender Ringgröße steigt. Der Grund dafür ist die Zunahme von Freiheitsgraden beim Übergang zur Kette. 



Folie 8
Damit schließe ich das Thema der nicht funktionalisierten Laktame ab und komme zum Abschluss noch zu  den funktionalisierten Lactamen.

Durch Subsitution des H-Atoms wird der polare Charakter des Lactams aufgehoben. 
Die wichtigen industriellen Anwendungen von diesen Lactamen, abgesehen von Penicillin, sind Lösmittel. 

Um diese komplexen Lactame zu verstehen, möchte ich auf die allgemeine MoldekülChemie eingehen. 
Die Lactame haben ein stark acides Proton.  Und die Acidität ist hoch genug um mit starken Basen das Proton zu abstrahieren. 

Auf der rechten Seite sehen wir die PKS Werte von verschiedenen Aminen. Hier ist erkennbar, dass die Resonanzstabilisierung des entsprechenden Amid eine Aciditätserhöhung bewirkt.

Dabei ensteht ein sehr starkes Nukleophil. Das wichtigste Phänomen bei der Chemie der Lactame ist Resonanzstabilisierung. 
Dieses Phänomen steht im Mittelpunkt bei einer Substanz, und zwar bei Penicilin.



Folie 9
Die Penicilline sind eine Gruppe von antibiotisch wirskamen Substanzen. Penicillin ist das stoffwechselprodukt einiger Schmimmelpilze. Das biologisch wirksame Prinzip ist der Bettalactamring. Die Verhinderung der Resonanz wird als Kriterium für die biologische Aktivität angesehen. Auf dieser Tabelle ist deutlich zu sehen, dass die Elektronenabgebenden Gruppen die Aktivität des Penicillins erhöhen. 


Mit diesem Vortrag wollte ich zeigen, dass die Lactame eine vielfältige und nützliche Chemie ermöglichen.
Vielen Dank fürs Zuhören. 

\end{document}
