% !TEX encoding   = UTF-8
% !TEX program    = LuaLaTeX
% !TEX spellcheck = de_DE
\documentclass[12pt]{article}
\usepackage{amsmath,mathtools}
\usepackage[usenames,dvipsnames]{xcolor}
%\usepackage[bitstream-charter]{mathdesign}
\usepackage{microtype}
\usepackage{fontspec}
\usepackage{graphicx}
\usepackage{siunitx}
\usepackage[german]{babel}
\usepackage{comment}
\usepackage{nicefrac}
\usepackage{booktabs}
\usepackage{float}
\usepackage{tikz}
\usetikzlibrary{arrows,chains,matrix,shapes}
\usepackage{pgfplots}
\pgfplotsset{compat=newest}
%\usepackage[backend=biber,sorting=none,autocite = superscript,natbib=true]{biblatex} \addbibresource{books.bib}
\usepackage{caption}
\usepackage{subcaption}
\usepackage{luacode}
\floatstyle{plaintop}
\restylefloat{table}
%\usepackage[justification=justified,singlelinecheck=false]{caption}
\usepgfplotslibrary{units}
\usetikzlibrary{pgfplots.groupplots}
%\usepackage{gensymb}
\usepackage{wrapfig}

\usepackage{geometry}
\usepackage{fancyhdr}
\fancyhf{}
\rhead{13.12.2016}
\lhead{Nevroz Arslan, Justin König Gruppe 6}
\setlength{\headheight}{15pt}
\rfoot{\thepage }
\lfoot{Versuch 1: Fluoreszenzspektroskopie}
\pagestyle{fancy}

\usepackage{url}
\usepackage{csquotes}
\sisetup{
  round-mode          = figures,
  round-precision     = 2,
  inter-unit-product =\cdot,
  group-digits=true,          %% Zifferngruppierung an/aus
  scientific-notation = true,
}
\renewcommand{\arraystretch}{1.5}

\definecolor{skyblue1}{RGB}{135, 206, 250}
\definecolor{flame}{RGB}{226, 88, 34}
\definecolor{scarletred1}{RGB}{252, 40, 71}
\definecolor{cyanblau}{RGB}{0, 158, 224}
\definecolor{darkblue}{RGB}{0, 0,139}
\definecolor{charcoal}{rgb}{0.21, 0.27, 0.31}
\definecolor{turquoise}{rgb}{0 0.41 0.41}
\definecolor{rouge}{rgb}{0.79 0.0 0.1}
\definecolor{vert}{rgb}{0.15 0.4 0.1}
\definecolor{mauve}{rgb}{0.6 0.4 0.8}
\definecolor{violet}{rgb}{0.58 0. 0.41}
\definecolor{orange}{rgb}{0.8 0.4 0.2}
\definecolor{bleu}{rgb}{0.39, 0.58, 0.93}
\definecolor{azulen}{RGB}{0,218,255}
\definecolor{left} {HTML}{c1f0c1}

\newcommand\addplotzr{\directlua{drawZZ()}}
\begin{document}
\begin{figure}[!htbp]
  \begin{tikzpicture}[]{
			 \draw (0,1) -- (4,1) node[left=-12mm] {$J-1$};
			 \draw (0,2) -- (4,2) node[left=-5mm] {$J$};
			 \draw (0,3) -- (4,3) node[left=-12mm] {$J+1$};
			 
			 \draw (0,-2) -- (4,-2) node[left=-12mm] {$J+1$};
			 \draw (0,-3) -- (4,-3) node[left=-5mm] {$J$};
			 \draw (0,-4) -- (4,-4) node[left=-12mm] {$J-1$};

			 \draw[color=violet] (0.5,-4) -- (0.5,2) node[pos=0.6,above=2pt,rotate=90] {$\tilde{\nu} _R (J-1)$};
			 \draw[color=violet] (1.5,-2) -- (1.5,2) node[pos=0.44,above=2pt,rotate=90] {$\tilde{\nu} _P (J+1)$};
			 \draw[color=violet] (2.5,-3) -- (2.5,3) node[pos=0.45,above=2pt,rotate=90] {$\tilde{\nu} _R (J)$};
			 \draw[color=violet] (3.8,1) -- (3.8,-3) node[pos=0.4,above=2pt,rotate=90] {$\tilde{\nu} _P (J)$};
			 \node [shading = axis,rectangle, left color=left, right color=left!30!white, anchor=north, minimum height=2cm] (box) at (-1,-2){$B_0$};
			 \node [shading = axis,rectangle, left color=left, right color=left!30!white, anchor=north, minimum height=2cm] (box) at (-1,3){$B_1$};

			}
    \end{tikzpicture}
    \caption{Das Verfahren der Kombinatonsdifferenzen}
  \end{figure}
  \end{document}
