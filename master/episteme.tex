\section{Episteme} \epigraph{\textit{ Eine von postkolonialer Theorie
inspirierte 'Pädagogik' richtet ihren Blick dabei insbesondere auf die
gelernte Vergessenheit, auf die aktiv produzierten Amnesien und deren
Komplizenschaft mit dem imperialistischen Projekt. In den Hochschulen wie auch
in den katholischen Kindergärten kann 'Wissen' sich immer auch
widerständig gegen die 'Institution' selbst wenden. }}{María do Mar
Castro Varela\footnotemark} \footnotetext{María do Mar Castro Varela, \glqq
Verlernen und die Strategie des unsichtbaren Ausbesserns. Bildung und
Postkoloniale Kritik,\grqq download unter:
\url{http://www.igbildendekunst.at/bildpunkt/2007/widerstand-macht-wissen/varela},
am 26.11.2014.} 

Im letzten Kapitel habe ich eine Idee von Universität skizziert,
die Bildung und Kritik zur Aufgabe eines universitären Selbstverständnisses
erklärt, das an Reflexivität orientiert ist. Mit Mohanty und hooks argumentierte
ich anschließend für die Notwendigkeit, als Untersuchungsgegenstand konkrete,
bildungs-pädagogische Praxen in in den Blick zu nehmen. Eine machtkritische
Analyse fordert hier, Mecheril et al. folgend ein, Differenz, also Praxen des
Unterscheidens zu untersuchen, die in jenen Formen des 'sich bildens' und
'Kritik übens' subjektivierend wirken. Dazu haben hooks und Mohanty das Konzept
der Erfahrung, als Sprechen aus und über Erfahrung diskutiert und dessen
Bedeutung für emanzipatorische Prozesse an der Universität befragt. Hier weisen
sie auf die Notwendigkeit hin, Subjekt-Objekt Verhältnisse in
Wissensproduktionen zu historisieren und aufzubrechen.
Meine Auseinandersetzung
war dabei von der Frage geleitet, woran sich eine Reflexion, die sich mit der
Bedeutung von Erfahrung in Wissensproduktion und Bildungsprozessen aber auch
Praxen der Kritik beschäftigt, orientieren kann. Dabei konnte ich aufzeigen,
dass das Sprechen aus und über Erfahrung sowohl stabilisierend als auch
irritierend auf hegemoniale Ordnungen, darüber was von wem sagbar ist, wirkt.

Wann, von wem oder wogegen Kritik geübt wird und wessen Selbst- und
Weltverhältnisse in den benannten Bildungsprozessen zur Disposition stehen wurde
dabei insbesondere von Mohanty und hooks zum Thema gemacht. Sie markieren die
herrschenden Differenzverhältnisse in universitären Seminarkontexten als
sexistisch und rassistisch und Mohanty fordert in ihrem Text explizit die
Dekolonisierung von Universität. Damit macht sie deutlich, dass Rassismus und
Sexismus im kolonialen Verhältnis situiert sind und über eine historisierende
Analyse nachvollzogen werden müssen. Ich knüpfe an diese Forderung an, und
richte in diesem Kapitel einen dekolonialen Blick auf Episteme. Dabei
interessiere ich mich dafür, welchen Ort Erfahrung in der
Legitimations\textbf{geschichte} von Wissen eingenommen hat und wie dies aus sexismus-
und rassismuskritischer Perspektive diskutiert wird. Von einer historisierenden
Analyse verspreche ich mir einen differenzierteren Zugang dazu, wie
Kolonialismus als (welt)gesellschaftliches Verhältnis in Wissensproduktionen
eingewirkt hat und von welchen Positionen Erfahrung in Wissensproduktionen
nutzbar gemacht werden konnte.

Dazu möchte ich in diesem Kapitel erkenntnistheoretische Überlegungen
anschließen, die sich einer rassismus- und sexismuskritischen
\textbf{Geschichts}schreibung widmen. Die Verbindung von  epistemologischer Praxis und
Geschichtsschreibung liegt hier nahe, da sich Erkenntnisse in die Narrative
über die Vergangenheit und Gegenwart einschreiben und so ihre diskursive Wirkung
entfalten können.
\\
María do Mar Castro Varela versteht, wie im eingangs gewählten Zitat deutlich
wird, unter einer postkolonialen Pädagogik die Aufgabe, sich dem Vergessen
entgegenzustellen. 'Vergessen' beschreibt sie nicht wie gemeinhin üblich als
unbeabsichtigten, oder gar unvermeidbaren Nebeneffekt der Wissensaneignung,
sondern als aktives und gelerntes Verhalten. Dekoloniale Geschichtsschreibung
setzt hier an und untersucht die Bedingungen, unter denen das aktive Vergessen
zur gewöhnlichen Praxis wird, um so Möglichkeiten des Erinnerns aufzuzeigen. Das
Feld, auf dem jene Auseinandersetzungen um die Leerstellen des Wissens unter
einer historisierenden Perspektive geführt wird, bildet somit den Rahmen dieses
Kapitels.

Bevor ich mich wie nun angekündigt mit sexismus- und rassismuskritischer
Epistemologie beschäftige, möchte ich zunächst ganz allgemein skizzieren, was
unter Episteme verstanden werden kann.\\

Foucault unterscheidet zwei Interessensgebiete auf denen
Wissenschaftshistoriker\_innen, also jene Theoretiker\_innen, die sich mit der
Geschichte des Wissens auseinandersetzen, gemeinhin arbeiten. So würden sie sich
zum einen mit den öffentlichen, wissenschaftlichen Auseinandersetzungen
beschäftigen, indem sie die Errungenschaften spezifischer Disziplinen und damit
einhergehende Kontroversen nachzeichneten. Zum Anderen würden sie sich den
verborgenen oder impliziten Einflüssen zuwenden, um diese als Hindernisse,
Störungen und Verzerrungen, die die wissenschaftliche Qualität mindern,
zurückzuweisen.\footnotemark \footnotetext{Michel Foucault, \textit{Die Ordnung
der Dinge}, (Frankfurt am Main:Suhrkamp Verlag, [1966 frz] 1974).} \\
Epistemologie
wird hier also als eine historisierende Analyse des Wissens beschrieben, mit dem
Zweck letzteres zu prüfen und zu ordnen.  Dies knüpft jedoch nicht an die
eingangs formulierte Forderung von Castro Varela an, nach den Leerstellen und
Auslassungen zu forschen, die jene Wissenschaftsgeschichte begleiten, und u.U
sogar konstituieren. Stattdessen scheint es viel eher darum zu gehen, Paradigmen
zu skizzieren, um sie dann auf ihre wissenschaftliche Objektivität hin zu
prüfen.

Das Verständnis von Episteme, das ich dieser Arbeit zu Grunde legen möchte
grenzt sich entsprechend von jenem, beschriebenen Konzept ab und orientiert sich
an dem Episteme Begriff den Foucault in seinem erstmals 1966 in Frankreich
erschienenen Werk \glqq Die Ordnung der Dinge\grqq \footnotemark \footnotetext{Foucault, \textit{Die Ordnung der
Dinge.} } entwickelt: \\

Foucault kritisiert hier die eben beschriebenen, konventionellen Vorstellungen
der Wissenschaftshistoriker\_innen, nach denen das explizite dem impliziten
Wissen gegenübergestellt werden kann, um ersteres von letzterem zu bereinigen.
Sein Verständnis von Episteme grenzt sich von jenen dargestellten
Interessensgebieten ab, da ähnlich wie Castro Varela auch er sich insbesondere
für das interessiert, was gemeinhin als Störfaktor ausrangiert werden soll. \\
Das
Implizite, so vermutet er in seiner Studie, ist zwar außerhalb des Bewusstseins
des\_der Theoretiker\_in, greift jedoch als Gesetzmäßigkeit in den Aufbau
wissenschaftlicher Erkenntisgenerierung ein und bestimmt darüber \textit{wie}
gedacht wird.\footnotemark \footnotetext{Ebd., 11.} 
Es könne somit weder als komplementär, noch als verzichtbar
gegenüber dem ausgesprochenen Wissen verstanden werden.\footnotemark
\footnotetext{Ebd., 13.} 
 Das Unbewusste wird
also nicht analysiert um es auszulöschen, sondern um darüber in Erfahrung zu
bringen, wie das Denkbare vom Undenkbaren unterschieden werden kann, bzw. was
das Denkbare bedingt und das Undenkbare möglich macht. Leerstellen, so können
Foucault und Varela hier zusammengebracht werden, befinden sich also nicht
außerhalb, sondern innerhalb des Wissens. Sie greifen in das Wissen ein, machen
es erst möglich oder wie Foucault schreibt 'denkbar'. 

Die Suche nach den
Leerstellen, so kann dies im Umkehrschluss interpretiert werden, dient nicht
einer Ergänzung des bereits ausgesprochenen, legitimierten Wissens. Die Annahme,
dass das nicht Denkbare das Denkbare bestimmt, lässt indes vermuten, dass die
Suche nach den Leerstellen mitsamt ihrer Markierung eine tiefgreifendere
Erschütterung hervorruft. Denn wenn Leerstellen das Fundament darstellen, auf
dem das Wissen ruht, können sie nicht einfach zum Gerüst werden, ohne das Gerüst
zum einstürzen zu bringen.\\
Episteme versteht Foucault entsprechend als machtvolles Gefüge, das darüber
bestimmt, welche von allen möglichen Aussagen als wissenschaftliche Aussagen
akzeptiert und Eingang in den wissenschaftlichen Diskursraum erhalten, um dort
diskutiert zu werden.\footnotemark \footnotetext{ Foucault, \textit{Dispositive
der Macht. Über Sexualität, Wissen und Wahrheit} (Berlin:Merve Verlag, 1978), 122. } 


Die Befragung der Episteme richtet ihren Blick nun auf diejenigen Mechanismen,
die legitime von illegitimen Aussagen unterscheiden, um erstere sicht- und
letztere unsichtbar zu machen. So leitet er seine Studie mit der Frage ein:
\glqq Was ist eigentlich für uns unmöglich zu denken? Um welche Unmöglichkeit
handelt es sich? \grqq \footnotemark \footnotetext{Foucault, \textit{Die Ordnung
der Dinge}, 17.} 
\\

Um herausfinden zu können, was für 'uns' unmöglich zu denken ist, scheint es mir
ratsam, zunächst die Entstehung und Manifestierung dessen zu untersuchen, was
für 'uns' ganz selbstverständlich denkmöglich ist. Dabei lehne ich es in
Abgrenzung zu Foucault ab, von einem kollektiven 'wir' oder 'uns' auszugehen, da
dies mit einer sexismus- und rassismuskritischen Perspektive im Widerspruch
steht. Statt mich identitär zu verorten, präferiere ich die Konkretisierung
meiner Perspektive als Interesse für die geschichtlichen Möglichkeitsbedingungen
eines euro- und androzentrischen Wissens. \\
Dazu werde ich mich auf ein Kollektiv
beziehen, das ausgehend von der Idee der \glqq Verwestlichten Universität \grqq 
\footnotemark \footnotetext{Capucine Boidin et al. \glqq Introduction: From
University to Pluriversity: A Decolonial Approach to the Present Crisis  of
Western Universitiesn \grqq in Human Architecture: Journal of the Sociology of
Self Knowledge Vol.10 Issue 1 \textit{Decolonizing the University: Practicing
Pluriversity}, (2012).} den 
symbolischen Ort markiert, auf den ich mich beziehe. In diesem Zusammenhang
möchte ich erstens klären, von welcher Universität in Zeit und Raum ich ausgehe,
zweitens markieren aus welchem Standpunkt ich versuche sie zu beschreiben und
drittens diskutieren wie die Geschichte dieser Universität (hier immer als
symbolischer Ort gedacht, vgl. voriges Kapitel) zum Gegenstand widerständiger
epistemologischer Praxen gemacht wird. Mein Interesse nach der Bedeutung von
Erfahrung in der Legitimationsgeschichte von Wissen erfordert also als ersten
Schritt, jene partikulare Wissensgeschichte,  mit der die Geschichte der
Universität verwoben ist, zu skizzieren.

Diese, eine Geschichte auf die ich mich beziehe und die notwendig eine Auswahl aus den vielfältigen Geschichten
darstellt, die über die Universität geschrieben und erzählt worden sind, lässt
sich u.A. dadurch kennzeichnen, dass sie einen Zusammenhang zwischen
Herrschafts- und Eroberungspraxen und der Erfindung von Identität aufzeigt. \\
Dies
macht sie zum Einen an den eingangs vorgestellten Kritik- und Bildungsbegriff
anschlussfähig und zum Anderen ermöglicht sie mir die Klärung eines für
sexismus- und rassismuskritische Theorien grundlegenden und zugleich umkämpften
Bezugspunkt: \\
Identität dient emanzipatorischen Bewegungen als Grundlage, von der
aus Unterdrückung und somit auch Herrschaft untersucht werden kann, da sie einen
zentralen Effekt von Differenzverhältnissen darstellt. Umkämpft hingegen ist der
strategische Umgang mit ihr, also wer sich unter welchen Voraussetzung ihrer
Kategorie bedient und damit in wessen Interessen handelt. 

Eine kritische
Auseinandersetzung mit Identität erfordert nun, so Foucault, sich mit ihren
Entstehungsbedingungen auseinanderzusetzen, und er fragt: 
\begin{myenv} \textit{
\glqq    Von welchem historischen Apriori aus ist es möglich gewesen, das große
    Schachbrett der deutlichen Identitäten zu definieren, das sich auf dem
    verwirrten, undefinierten, gesichtslosen und gewissermaßen indifferenten
    Hintergrund der Unterschiede erstellt? […] Die Geschichte der Ordnung der
    Dinge wäre die Geschichte des Gleichen, (du meme), das für eine Zivilisation
    gleichzeitig dispers und verwandt ist, also durch Markierungen zu
    unterscheiden und in Identitäten aufzufassen ist.\grqq \footnotemark
  \footnotetext{Foucault, \textit{Die Ordnung der Dinge}, 27.} }
\end{myenv}

Unterschiede, so verstehe ich Foucault, erhalten erst durch ihre Definition ein
'Gesicht'. Sie werden geordnet um sich dann in Identitäten zu manifestieren. In
diesem Kapitel möchte ich nun jenes Wissenschaftsverständnis, das sich in der
Europäischen Moderne durchsetzte, und das, in unmittelbaren Zusammenhang mit der
von Foucault konstatierten Erfindung der Identität steht, untersuchen. Dazu
werde ich es anhand der Subjekt- und Erkenntnis-Theorien bei Descartes, Kant,
Hegel und Marx aus einer sexismus- und rassismuskritischen Perspektive
explizieren um daran anschließend die Konturen einer dekolonalen Epistemologie
aufzuzeigen.
\\

Hierfür möchte ich mich zunächst auf Gayatri Chakravorty Spivak beziehen, da
sie die o.g. Philosophen\_*, in direkten Zusammenhang mit der Etablierung einer
\glqq menschlichen Norm \grqq \footnotemark \footnotetext{Barbara
Gabel-Kunningham et al. \glqq Vorwort der Übersetzer und Übersetzerinnen \grqq, In:
\textit{Spivak, Gayatri Chakrabarty: Kritik der Postkolonialen Vernunft. Hin zu einer
Geschichte der verrinnenden Gegenwart}, (Stuttgart: Kohlhammer Verlag),7.}

stellt, die unter Bezugnahme von \glqq inferiorisierenden Alteritätskonstrukten
\grqq das Mittel- und Westeuropäische als natürliche Identität und Denkstruktur
konstruierten. 

Jene Denkansätze, müssen, so fordert sie, in ihrer unmittelbaren
Komplizenschaft mit dem kolonialen Projekt untersucht werden und ihre
Grundannahmen als kolonialpolitische Normierung dekonstruiert werden.
\footnotemark \footnotetext{Gabel-Kunningham, \glqq Vorwort \grqq, ebd.}
Barbara Gabel-Kühnung die mit ihren Kolleg\_innen Spivaks Monographie \glqq Die Postkoloniale Vernunft. Hin zu einer verrinnenden
Geschichte der Gegenwart \grqq \footnotemark \footnotetext{Gayatri Chakrabarty
  Spivak: \textit{Kritik der Postkolonialen Vernunft. Hin zu einer Geschichte
der verrinnenden  Gegenwart}, (Stuttgart: Kohlhammer Verlag)} vom Englischen ins Deutsche übersetzt hat schreiben dazu: 
\begin{myenv} 
  \textit{ \glqq Reale ethische
  Handlungsmöglichkeiten in einer globalen Welt sind nach Spivak nur möglich,
wenn die Geisteswissenschaften die weitreichenden kolonialen Komplizenschaften
'Europas' durch dekonstruktive Lektüren bewusst machen, Komplizenschaften die
in unsere Denktradition eingeschrieben sind, gerade deswegen 'natürlich'
erscheinen und insofern oft der Kritik entgehen \grqq \footnotemark
\footnotetext{Gabel-Kunningham, \glqq Vorwort,\grqq 14.}} 
\end{myenv} 
Sie führen fort, dass gerade in der allgemeingültig verstandenen Kategorien 'Mensch' die

\glqq einheimische Informantin \grqq verworfen wird:
\begin{myenv} \textit{
    \glqq Die einheimische Informantin ist in Spivaks Studie eine Leerstelle,
die in westliche ethnographische, philosophische, kulturelle, literarische und
historische Diskurse eingeschrieben wurde \grqq \footnotemark \footnotetext{Gabel-Kunningham, 15.}} 
\end{myenv}

Die Leerstelle, oder das was unmöglich ist zu denken, ist also untrennbar mit
der Kategorie 'Mensch' verbunden, bzw. in sie eingeschrieben. Eine Analyse der
Leerstellen erfordert nun also die Bedingungen zu untersuchen, die der
Nutzbarmachung der Kategorie 'Mensch' zu Grunde liegen. \\
Meine Auseinandersetzung mit
der Geschichte der Universität, oder dem Denken, das die Idee der Universität
ermöglicht, werde ich entsprechend mit einer Auseinandersetzung um die
Kategorie 'Mensch', und ihren Ausschlüssen beginnen. 

Die Kategorie 'Mensch', so
meine Vermutung, hat ein Denken über 'Mensch' sein ermöglicht, das auf der
Verunmöglichung eines anderen 'Menschsein' beruht.
\\

 Für meine Arbeit bedeutet
dies, dass ich mich mit einer (Idee von) Universität auseinandersetze, deren
Geschichte bis auf die Europäische Moderne zurückgeht und im Anschluss an
Spivak somit in einer unmittelbaren Kompliz\_innenschaft mit dem kolonialen
Projekt steht. Um diese Leerstelle beschreiben zu können, bedarf es zweifellos
der Perspektiven jener, die durch die Kategorie 'Mensch' kategorisch und
faktisch vom Status des Subjektes ausgeschlossen wurden. Das dekoloniale
Projekt stellt sich diesem schwierigen Anspruch der Repräsentation.

%\subsection{Hegemoniales Wissen} 
%\subsubsection{Perspektiven auf hegemoniales
%Wissen} \subsubsection{Entstehung und Durchsetzung hegemonialen Wissens}
%\subsubsection{Methodologie hegemonialen Wissens} \subsection{Widerständiges
%Wissen}
