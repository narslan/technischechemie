% !TEX encoding   = UTF-8
% !TEX program    = LuaLaTeX
% !TEX spellcheck = de_DE
\documentclass[12pt]{article}
\usepackage{amsmath,mathtools}
\usepackage[usenames,dvipsnames]{xcolor}
%\usepackage[bitstream-charter]{mathdesign}
\usepackage{microtype}
\usepackage{fontspec}
\usepackage{graphicx}
\usepackage{siunitx}
\usepackage[german]{babel}
\usepackage{comment}
\usepackage{nicefrac}
\usepackage{booktabs}
\usepackage{float}
\usepackage{tikz}
\usetikzlibrary{arrows,chains,matrix,shapes}
\usepackage{pgfplots}
\pgfplotsset{compat=newest}
\usepackage[backend=biber,sorting=none,autocite = superscript,natbib=true]{biblatex} \addbibresource{books.bib}
\usepackage{caption}
\usepackage{subcaption}
\usepackage{luacode}
\floatstyle{plaintop}
\restylefloat{table}
%\usepackage[justification=justified,singlelinecheck=false]{caption}
\usepgfplotslibrary{units}
\usetikzlibrary{pgfplots.groupplots}
%\usepackage{gensymb}

\usepackage{geometry}
\usepackage{fancyhdr}
\fancyhf{}
\rhead{13.12.2016}
\lhead{Nevroz Arslan, Justin König Gruppe 6}
\setlength{\headheight}{15pt}
\rfoot{\thepage }
\lfoot{Versuch 1: Fluoreszenzspektroskopie}
\pagestyle{fancy}

\usepackage{url}
\usepackage{csquotes}
\sisetup{
  round-mode          = figures,
  round-precision     = 2,
  inter-unit-product =\cdot,
  group-digits=true,          %% Zifferngruppierung an/aus
  scientific-notation = true,
}
\renewcommand{\arraystretch}{1.5}
%\usepackage{unicode-math}

%\setmainfont[Numbers = OldStyle,Ligatures = TeX,SmallCapsFeatures = {Renderer=Basic}]{Minion Pro}
%\setsansfont[Numbers={OldStyle,Proportional},Scale=MatchLowercase]{Minion Pro}
%\setmonofont[Numbers=OldStyle,Scale=MatchLowercase]{Minion Pro}
%\setmathfont{MnSymbol}
%\setmathfont[range=\mathup/{num,latin,Latin,greek,Greek}]{Minion Pro}
%\setmathfont[range=\mathbfup/{num,latin,Latin,greek,Greek}]{MinionPro-Bold}
%\setmathfont[range=\mathit/{num,latin,Latin,greek,Greek}]{MinionPro-It}
%\setmathfont[range=\mathbfit/{num,latin,Latin,greek,Greek}]{MinionPro-BoldIt}
%\setmathrm{Minion Pro}
%usepackage{sansmath} % Enables turning on sans-serif math mode, and using other environments

\definecolor{skyblue1}{RGB}{135, 206, 250}
\definecolor{flame}{RGB}{226, 88, 34}
\definecolor{scarletred1}{RGB}{252, 40, 71}
\definecolor{cyanblau}{RGB}{0, 158, 224}
\definecolor{darkblue}{RGB}{0, 0,139}
\definecolor{charcoal}{rgb}{0.21, 0.27, 0.31}
\definecolor{turquoise}{rgb}{0 0.41 0.41}
\definecolor{rouge}{rgb}{0.79 0.0 0.1}
\definecolor{vert}{rgb}{0.15 0.4 0.1}
\definecolor{mauve}{rgb}{0.6 0.4 0.8}
\definecolor{violet}{rgb}{0.58 0. 0.41}
\definecolor{orange}{rgb}{0.8 0.4 0.2}
\definecolor{bleu}{rgb}{0.39, 0.58, 0.93}
\definecolor{azulen}{RGB}{0,218,255}

\newcommand\addplotzr{\directlua{drawZZ()}}
\begin{document}
\begin{figure}[!htbp]
  \begin{tikzpicture}[decoration=snake,
      spiegel/.style={rectangle,gray,fill=black!20,thin,minimum width=3cm,minimum height=0.1cm},
      teiler/.style={rectangle,black,dashed,rotate=45,minimum width=3.2cm,minimum height=0.07cm},
      probe/.style={rectangle,black,fill=black!20,dashed,rotate=90,minimum width=1.5cm,minimum height=0.7cm},
      ]{
     % \draw (0,0) rectangle (1,1) node[above] (b) {\footnotesize Strahlungsquelle};
      \path ( 5,4) node [spiegel,draw] (sspiegel) {}
        ( 5,0) node [teiler,draw] (steiler) {}
        ( 4.8,-4) node [probe,draw] (probe) {}
        ( 5,-7) node [circle,draw] (detektor) {\footnotesize Detektor}
        ( 8,0) node [spiegel,draw,rotate=90] (bild) {}
        ( 10,0) node [spiegel,draw,rotate=90,dashed] (spiegelbild) {}
        (-1,0) node [circle,draw,inner sep=0.3cm] (source) {};
      }
      \draw [->,decorate,decoration={snake,amplitude=1mm}] (steiler) -- (probe);
      \draw [->,decorate,decoration={snake,amplitude=2mm},
        transform canvas={xshift=-3mm,yshift=1mm}] (steiler) -- (probe);
      \draw [<->,skyblue1] (source) -- (steiler);
      \draw [<->,skyblue1] (sspiegel) -- (steiler);
      \draw [->,skyblue1] (steiler) -- (detektor);
      \draw [->,skyblue1,transform canvas={xshift=-3mm}] (steiler) -- (detektor);
      \draw [<->,skyblue1] (steiler) -- (bild);
      \draw [<->] (bild) -- (spiegelbild) node [midway,yshift=3mm] {\tiny Abstand $= \frac{\lambda}{4} $};
      \node [above] at (source.north) {\textsf{Strahlungsquelle}};
      \node [above] at (9,2) {\footnotesize{Beweglicher Spiegel \textbf{M}}};
      \node [above] at (sspiegel) {\footnotesize{stationärer Spiegel \textbf{S}}};
      \node [above,rotate=90] at (probe.north) {\footnotesize{Probe}};
      \node [below,yshift=-15mm] at (bild.south) {\footnotesize{ $ \delta=0$}};
      \node [below,yshift=-15mm] at (spiegelbild.south) {\footnotesize{ $ \delta= \frac{\lambda}{2} $}};
      \node [above] (texta) at (2,-4) {\tiny{die Lichtwellen von den Spiegeln M und S }};
        \node [above] at (1.8,-4.2) {\tiny{befinden sich zueinander gewöhnlich}}; 
        \node [above] at (1.6,-4.4) {\tiny{nicht in der Phase}}; 

        \node [above] (textb) at (3,1) {\footnotesize{Strahlungsteiler}};
      \draw [->] (texta) -- (4.5,-1.8);
      \draw [->] (textb) -- (steiler);

      \node [above] (textb) at (4.8,0) {\footnotesize{\textbf{O}}};
    \end{tikzpicture}
    \caption{Schematische Darstellung des Michelson-Interferometer}
  \end{figure}
  \end{document}
