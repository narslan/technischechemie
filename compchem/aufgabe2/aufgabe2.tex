\documentclass[12pt]{article}
\usepackage{amsmath,mathtools}
\usepackage[usenames,dvipsnames]{xcolor}
\usepackage[german]{babel}
\usepackage[utf8]{inputenc}
\usepackage[T1]{fontenc}
\usepackage{textcomp}
%\usepackage{libertine}
%\usepackage{helvet}

\usepackage{microtype}
% Minion and Myriad fonts
\usepackage[minionint,mathlf]{MinionPro}
\renewcommand{\sfdefault}{Myriad-LF}
\usepackage{siunitx}
\usepackage{fancyhdr}
\usepackage{sectsty}
\usepackage{setspace}
\usepackage{booktabs} % To thicken table lines
\usepackage[version=4]{mhchem}
\usepackage[draft]{graphicx}
\usepackage[labelfont=bf]{caption}
\usepackage{subcaption}
\usepackage{chemstyle}

%\usepackage[compatibility=4.7,language=german]{chemmacros}
\renewcommand{\familydefault}{\sfdefault}
\sisetup{detect-all}
\usepackage{chngcntr}
\counterwithin{table}{section}
\counterwithin{figure}{section}

\usepackage{titlesec}

\titleformat*{\section}{\large\bfseries}
\titleformat*{\subsection}{\normalsize\bfseries}

\usepackage{geometry}
 \geometry{
 a4paper,
 left=20mm,
 top=30mm,
 right=20mm
 }

\begin{document}

\begin{onehalfspace}

\section{Geometrieoptimierung und Frequenzberechnung an Dichlorethen}
Ziel dieser Aufgabe ist es eine Geometrieoptimierung von cis- und trans- 1,2-Dichlorethen durchzuführen und anschließend die Schwingungsfrequenzen zu berechnen.
\textbf{Auswertung 1 }\\
\begin{figure}[!hptb]
    \centering
    \begin{subfigure}[b]{0.4\textwidth}
        \includegraphics[width=\textwidth]{data/cis_darstellung.png}
    \end{subfigure}
    ~ %add desired spacing between images, e. g. ~, \quad, \qquad, \hfill etc.
      %(or a blank line to force the subfigure onto a new line)
    \begin{subfigure}[b]{0.4\textwidth}
        \includegraphics[width=\textwidth]{data/trans_darstellung.png}
    \end{subfigure}
    \caption{Optimierte Geometrien von cis- und trans-1,2-Dichlorethen}
\end{figure}\\
In Folgenden Tabellen sind die Ergebnisse zur Geometrieoptimierung und die Schwingungsfrequenzen zu cis-und trans-1,2-Dichlorethen mit der RHF Methode mit dem Basissatz 6-31+G(d,p) aufgeführt:\\
\begin{table}[!htpb]
\centering
\caption{cis-1,2-Dichlorethen RHF mit Basissatz 6-31+G(d,p) }
\begin{tabular}{lrrr}
\toprule
Darstellung &   Schwingungsfrequenz \si{\per\centi\meter} & \multicolumn{2}{c}{Schwingungsintensität} \\
&&IR&Raman\\
\midrule
$A _1$ & 182 & 0.44 & 2.15\\
$A _2$ & 462 & 0.00 & 3.90\\
$B _2$ & 619 & 8.08 & 6.22\\
$A _1$ & 768 & 26.36 & 15.10\\
$B _1$ & 810 & 72.43 & 0.29\\
$B _2$ & 931 & 94.09 & 0.02\\
$A _2$ & 1060 & 0.00 & 6.86\\
$A _1$ & 1333 & 0.01 & 20.48\\
$B _2$ & 1451 & 41.10 & 0.50\\
$A _1$ & 1823 & 39.79 & 70.01\\
$B _2$ & 1451 & 15.46 & 50.59\\
$A _1$ & 1823 & 1.73 & 141.62\\
\bottomrule
\end{tabular}
\end{table}

\begin{table}[!htpb]
\centering
\caption{trans-1,2-Dichlorethen RHF mit Basissatz 6-31+G(d,p) }
\begin{tabular}{lrrr}
\toprule
Darstellung & Schwingungsfrequenz \si{\per\centi\meter} & \multicolumn{2}{c}{Schwingungsintensität} \\
&&IR&Raman\\
\midrule
$A _u$ & 235 & 0.59 & 0.00\\
$A _u$ & 259 & 4.64 & 0.00\\
$B _g$ & 379 & 0.00 & 10.33\\
$A _u$ & 886 & 147.81 & 0.00\\
$B _g$ & 921 & 0.00 & 10.33\\
$B _g$ & 929 & 0.00 & 11.75\\
$A _u$ & 1053 & 79.24 & 0.00\\
$A _u$ & 1344 & 27.36 & 0.00\\
$B _g$ & 1427 & 0.00 & 31.78\\
$A _g$ & 1823 & 0.00 & 29.61\\
$B _u$ & 3415 & 17.88 & 0.00\\
$A _g$ & 3420 & 0.00 & 115.20\\
\bottomrule
\end{tabular}
\end{table}


\begin{table}[!htpb]
\centering
\caption{cis-1,2-Dichlorethen mit MP2 Methode mit Basissatz 6-31+G(d,p)}
\begin{tabular}{lrr}
\toprule
Darstellung & Schwingungsfrequenz \si{\per\centi\meter} & \multicolumn{1}{c}{Schwingungsintensität} \\
&&IR\\
\midrule
$A _1$ & 170.7 & 0.22\\
$A _2$ & 401.2 & 0.00\\
$B _2$ & 588.5 & 4.10\\
$A _1$ & 721.9 & 68.26\\
$B _1$ & 748.7 & 19.38\\
$B _2$ & 891.5 & 0.00\\
$A _2$ & 901.8 & 76.80\\
$A _1$ & 1256.8 & 0.01 \\
$B _2$ & 1367.6 & 27.83\\
$A _1$ & 1664.6 &35.93\\
$B _2$ & 3295.9 & 14.23\\
$A _1$ & 3316 & 1.97\\
\bottomrule
\end{tabular}
\end{table}

\begin{table}[!htpb]
\centering
\caption{trans-1,2-Dichlorethen mit MP2 Methode mit Basissatz 6-31+G(d,p)}
\begin{tabular}{lrr}
\toprule
Darstellung & Schwingungsfrequenz \si{\per\centi\meter} & \multicolumn{1}{c}{Schwingungsintensität} \\
&&IR\\
\midrule
$A _u$ & 210.5 & 0.13\\
$A _u$ & 244.6 & 3.18\\
$B _g$ & 360.7 & 0.00\\
$A _u$ & 767.2 & 0.00\\
$B _g$ & 868.9 & 113.44\\
$B _g$ & 896.4 & 0.00\\
$A _u$ & 944.2 & 70.43\\
$A _u$ & 1276.0 & 21.18\\
$B _g$ & 1343.6 & 0.00 \\
$A _g$ & 1662.3 & 0.00 \\
$B _u$ & 3309.3 & 14.86\\
$A _g$ & 3312 & 0.00 \\
\bottomrule
\end{tabular}
\end{table}

\begin{table}[!htpb]
\centering
\caption{ Zusammenfassung Moleküle}
\begin{tabular}{llcrc}
\toprule
Molekül & Methode &   Basissatz & Energie \si{\hartree} & Symmetrie\\
\midrule
cis-1,2-Dichlorethen   & RHF& 6-31+G(d,p)& -995.83649439 &$C_ {2v}$\\
cis-1,2-Dichlorethen   & MP2& 6-31+G(d,p)&-996.36801361  &$C_ {2v}$\\
trans-1,2-Dichlorethen & RHF& 6-31+G(d,p)& -995.83661280 &$C_ {2h}$ \\
trans-1,2-Dichlorethen & MP2& 6-31+G(d,p)& -996.36710269 &$C_ {2h}$\\
\bottomrule
\end{tabular}\\
\end{table}


\textbf{Auswertung 2}

\begin{figure}[!hptb]
    \centering
    \begin{subfigure}[b]{0.4\textwidth}
        \includegraphics[width=\textwidth]{data/cis_ir_spektrum.png}
    \end{subfigure}
    ~ %add desired spacing between images, e. g. ~, \quad, \qquad, \hfill etc.
      %(or a blank line to force the subfigure onto a new line)
    \begin{subfigure}[b]{0.4\textwidth}
        \includegraphics[width=\textwidth]{data/trans_ir_spektrum.png}
    \end{subfigure}
    \caption{CIS UND TRANS IR SPEKTREN}
\end{figure}





\end{onehalfspace}
\end{document}