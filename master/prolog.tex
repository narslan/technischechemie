\section{Prolog}
\epigraph{ We all need histories that no history book can
  tell, but they are not in the classroom, not the history classrooms, anyway.
  They are in the lessons we learn at home, in poetry and childhood games, in
  what is left of history when we close the history books with their verifiable
facts.}{Michel-Rolph Trouillot\footnotemark} \footnotetext{Michel-Rolph
Trouillot, \textit{Silencing the Past. Power and the Production of History} (Boston: Beacon Press, 1995).}

Was sind das für Geschichten, die wir alle brauchen, und die in keinem
Geschichtsbuch stehen? Was sind das für Geschichten, die nicht in den
Klassenzimmern, sondern in den Häusern der Schüler\_innen erzählt werden? Was
sind das für Geschichten, die sich uns erst eröffnen, wenn wir die Bücher
schließen und uns der Dichtung und dem Spiel zuwenden?

Die oben stehenden Zeilen bilden die letzten Worte eines Vorwortes, mit dem
Michel-Rolph Trouillot das Kapitel „The Haitian Revolution as a
Non-Event“\footnotemark\footnotetext{Trouillot, „The Haitian Revolution as a
Non-Event“ in \textit{Silencing the Past. Power and the Production of History} (Boston: Beacon Press, 1995).}
einleitet. In dem Vorwort teilt Trouillot mit den Leser\_innen zwei sehr
unterschiedliche Erfahrungen aus seiner Lehre an US-amerikanischen
Universitäten. Das eine Mal wird er von einer
Studentin\_*\footnotemark\footnotetext{Die Schreibweisen \textit{w}eiß; Schwarz, of
Color und der Gender Gap bzw. das * und ' weisen darauf hin, dass es sich hier
nicht um widerspruchsfreie Kategorien handelt denen ein ontologischer Zustand
vorausgeht. Viel eher sollen damit politische Positionen und Perspektiven
in rassistischen und sexistischen Verhältnissen zum Ausdruck kommen, die in
dieser Arbeit verhandelt werden. Die Schreibweise Schwarz und of Color, stellt
den Versuch dar, Selbstpositionierungen von 'Menschen' die Rassismus erfahren
zu berücksichtigen. Die Bezeichnung \textit{w}eiß mit dem kursiven \textit{w} soll das
Ungleichgewicht, dass in rassistischen Verhältnissen zumeist die von Rassismus
betroffenen markiert werden, die von Rassismus privilegiert und normalisierten
darum unbemerkt bleiben können entgegen wirken. Da es sich hier jedoch nicht um
eine Selbstbezeichnung im Sinne einer Wiederaneignung von Fremdbezeichnungen
geht, sondern um die Markierung von rassistischen Verhältnissen privilegiert zu
werden, wird hier eine andere Schreibweise als eben diskutierte Schreibweise
mit Großbuchstaben gewählt. Das Gender\_Gap zeigt den Zwischenraum auf, indem
sich alle 'Menschen' im Zweigeschlechterzwang verorten und das Sternchen weißt
daraufhin, dass die Leser\_in die Subjektkategorien Frau\_* wie auch andere
Subjektkategorien innerhalb einer Vielzahl von Differenzkategorien verorten
kann und sollte. Die Ein-Strich Anführungseichen betont einzelne Begriffe die
in dieser Arbeit auf ihren Normalitätsanspruch hin befragt werden. Dies
geschieht punktuell, an Stellen da mir die Betonung auf den
Konstruktionscharakter sinnvoll erscheint. Alle Schreibweisen sind im
strategischen Sinne zu verstehen, ich gehe nicht davon aus das die Frage nach
der Möglichkeit der Repräsentation von Subjekten durch Sprachordnungen damit
für alle Zeiten geklärt ist. Viel eher soll die Schreibweise mich und die
Leser\_innen dazu anregen, die sprachliche Ordnung als Werkzeug zu verstehen,
mit dem bestehende Bezeichnungspraktiken und damit auch Differenzpraktiken in
Frage gestellt werden können. } dazu aufgefordert, von den Kämpfen der
Sklav\_innen zu sprechen, anstatt nur \textit{w}eiße Autor\_innen zu
rezipieren:
\begin{myenv}
  \textit{„Mr. Trouillot, you make us read all those white scholars. What can
they know about slavery? Where were they when we were jumping off the boats?
When we chose death over misery and killed our own children to spare them from
a life of rape?“\footnotemark\footnotetext{Troillot, „The Haitian Revolution“, 70.}}
\end{myenv}
Das andere Mal und viele Jahre später fordert eine andere
Studentin\_* ihn dazu auf, endlich über die Schwarzen Millionäre zu sprechen:
„I'm tired to hear about that slavery stuff. Can we hear the stories of the
black millionaires“?\footnotemark\footnotetext{Ebd., 71.}

Troillot stellt den Leser\_innen beide Student\_innen als Schwarze Frauen\_* vor
und wünscht sich, die Zeit zurückdrehen und diese beiden Student\_innen
miteinander ins Gespräch bringen zu können. Worüber würden sie wohl sprechen?
Was verbindet und was trennt sie?

Die Worte der Student\_innen lösen vieles in mir aus. Nähe und Distanz,
Erinnerung und Sehnsucht, Unsicherheit und Mut.

Nähe, weil ich mit dieser
Geschichte die (nicht) erzählt wird verbunden bin. Es ist auch meine
Geschichte, weil Kolonialismus und Sklaverei auch eine Geschichte über
\textit{w}eiße ist.

Distanz, weil es nicht um mich, nicht um meine Gefühle
geht oder gehen sollte sondern um die Perspektiven und Forderungen derer, deren
Geschichte nicht erzählt wird.

Erinnerung, weil ich ähnliche
Auseinandersetzungen miterlebt habe und die Sehnsucht nach universitären Räumen
bleibt, in denen Subjekte mit all ihren Erfahrungen als Subjekte in diesen
Räumen vorkommen können, ohne auf diese Erfahrungen reduziert zu werden.
Unsicherheit, weil es unmöglich scheint, die Interventionen als Impuls
aufzugreifen ohne sich zugleich aneignend ihnen gegenüber zu verhalten.  Mut,
weil sie ein Zeichen dafür sind, dass es Widerstand gibt.

Die Gedanken die nun
folgen haben mehr mit mir zu tun als mit den Subjekten und dem Kontext aus dem
sie entstanden sind. Es sind meine Lesarten und sie sind damit keine absoluten
Deutungen sondern Versuche, jenen Interventionen aufzugreifen, um von ihnen zu
lernen.

Die Intervention der Schwarzen Frauen\_* in das Seminar, so unterschiedlich sie
auch sein mögen, verbinde ich mit einer geteilten Idee von Universität. Es sind
Forderungen nach einer Universität als Ort, an dem Geschichten erzählt werden
in denen sie, Schwarze Frauen\_*, sich wiederfinden. Geschichten, die ihnen
etwas darüber erzählen, woher sie kommen und was sie sein können. Und so
unterschiedlich ihre Vorstellungen davon auch sind, so ähnlich ist doch ihr
Verlangen, ihre eigene Geschichte zu hören oder vielleicht auch zu erzählen.

Die Interventionen dieser beiden Schwarzen Frauen\_* zeigen, dass Erfahrungen
Subjekte hervorbringen, die sich annehmend und abweisend gegenüber jenen
Erfahrungen verhalten können. Erfahrung geht nicht notwendig mit einem
unmittelbaren Erlebnis einher. Erfahrung schreibt sich über Generationen in die
Körper und Seelen der Menschen ein und sie hinterlässt Spuren.
