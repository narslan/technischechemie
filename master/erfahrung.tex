\subsection{Erfahrung und Universität}
\epigraph{ 
 Feminists writing about race and racism have had a lot to 
say about scholarship, but perhaps our pedagogical and institutional practices and their relation to scholarship have not been examined with quite the same care and attention. 
  }{Chandra Talpade Mohanty\footnotemark} \footnotetext{Mohanty, „On Race and Voice,“ 183.} 

Im Folgenden möchte ich die Auseinandersetzung damit, wie Bildung und Kritik
an der Universität gedacht und praktiziert werden können, fortsetzen und im
Hinblick auf die zuletzt von Foucault markierte Bedeutung von Erfahrung
konkretisieren.

Was es heißen kann, Erfahrung im universitären Kontext, zugleich als
vermittelnde und vermittelte Größe zu begreifen, werde ich im Kontext
antisexistischer und antirassistischer Universitätspolitiken diskutieren.
Anhand verschiedener Zu- und Umgänge mit Erfahrung, die von Chandra Talpade
Mohanty und bell hooks analysiert werden, versuche ich damit auch meinem
Anspruch nachzukommen, mich stärker den \textbf{Praxen} zu widmen, die in jenem Üben,
das Denkbare zu überschreiten, aufgehen. An dieser Stelle wäre es durchaus auch
möglich  erziehungswissenschaftlichen Auseinandersetzungen um die Bedeutung von
Erfahrung für wissenschaftstheoretische Positionen und deren Anwendung für
Bildungsprozesse  aufzugreifen. Da sich die deutschsprachigen
erziehungswissenschaftlichen Debatten\footnotemark \footnotetext{Vgl. Johannes
  Bilstein und Helga Peskoller, Herausgeber\_innen von \textit{Erfahrung,
  Erfahrungen}
(Wiesbaden: Chronos Verlag,[1968] 2013). Sowie Otto F. Bollnow „Der
Erfahrungsbegriff in der Pädagogik.“ In ebd. Und Helga Peskoller, Helga
„Erfahrung/en.“ im selbigen Band. Außerdem Susann Fegter und Nadine Rose.
„Herstellung von Legitimität. Zum Rekurs auf Erfahrung in der Lehre.“ In
Differenz unter Bedigungen von \textit{Differenz. Zu Spannungsverhältnissen
universitärer Lehre}, herausgegeben von Paul Mecheril,et al. (Wiesbaden:
Springer Verlag, 2013). } jedoch stärker mit innerdisziplinären Fragen und
weniger grundsätzlichen Einordnung in postkolonialen (welt)gesellschftlichen
Verhältnissen befassen, ziehe ich ihnen die Überlegungen von  Mohanty und hooks
für meine Arbeit vor.

Mohanty kritisiert, dass Feminist\_innen, die sich mit 'Race' und Rassismus
auseinandersetzen, ihre Analysen vielfach lediglich auf Publikationen beziehen
anstatt sich mit der Aneignung, Umdeutung und Herstellung dieser Erkenntnisse
in der Seminarpraxis auseinanderzusetzen. Mohantys Texte „On Race and
Voice“\footnotemark \footnotetext{Ebd.} und
„Locating the Politics of Experience“\footnotemark\footnotetext{Mohanty, „Feminist Encounters“.} greifen
die Frage auf, wie angesichts der Tatsache, dass sich die Universität nicht
außerhalb der Verhältnisse bewegt, die sie zu beschreiben beansprucht,
überhaupt kritisches Wissen über z.B. Rassismus entstehen kann und welche
Bedeutung dabei die Erfahrungen jener erhalten, die Rassismus tagtäglich
erleben – in und außerhalb der Universität. Dabei nehmen ihre Texte den
Charakter von Manifesten an, die eine Universität fordern, die sich nicht im
Diversity-Vokabular verliert, sondern Differenz als einen Angriff auf
Normalitätsvorstellungen begreift und einfordert.

Auch bell hooks beschäftigt sich in dem Kapitel „Essentialism and
Experience“\footnotemark\footcitetext{bellhooks}
mit der politischen Wirkung, die ein erfahrungsorientiertes
Sprechen\_Schreiben\_Lesen in universitären Kontexten entfalten kann. hooks
bezieht sich in dem benannten Kapitel auf einen Text von Diana Fuss mit dem
Titel „Essentially Speaking“\footnotemark\footcitetext[77]{bellhooks} und stellt ihm einen kritischen
Essentialismusbegriff entgegen.

Diana Fuss problematisiert in dem von hooks diskutierten Kapitel
erfahrungsbasierte Beiträge in universitären Seminarkontexten. Das Sprechen
über Erfahrung, so Fuss, sei automatisch mit einer Autorität besetzt, die
gerade von solchen Studierenden genutzt würde, die sie als marginalisiert
beschreibt. Letztere würden die Diskussion im Seminar durch den Rückgriff auf
eine essentialistische Auslegung ihres marginalisierten Standpunktes dominieren
und so andere Stimmen zum Schweigen bringen.\footnotemark\footnotetext{Diana Fuss in bell hooks, ebd., 81.}

Auch Mohanty bezieht zieht sich zunächst kritisch auf erfahrungsbasiertes
Sprechen. Sie merkt an, dass oftmals statt einer differenzierten
Auseinandersetzung darüber, wie sich politische Verhältnisse auf einer
strukturellen Ebene auf Differenzdynamiken z.B. im Seminar auswirken, eine
oberflächliche Gegenüberstellung von weiß und of Color bzw. Schwarz
positionierten Studierenden vorgenommen wird. Differenz werde hier nicht in
ihrer historischen Gewordenheit rekonstruiert, sondern vordergründig auf der
Ebene individueller, persönlicher Erfahrungen von z.B. Rassismus verstanden,
die von den vermeintlich nicht Betroffenen eine erhöhte Sensibilität oder
Feinfühlichkeit erfordere. Student\_innen of Color oder Schwarzen Studierenden
werde dadurch sowohl ein Expert\_innenstatus als auch eine erhöhte
Verletzbarkeit zugeschrieben, wohingegen weiß positionierte Studierende
lediglich als neutrale Beobachter\_innen fungieren. „In other words, white
students are constructed as marginal observers and students of color as the
real 'knowers' in such a liberal or left
classroom.“\footnotemark\footnotetext{Mohanty, „On Race and Voice,“ 194.}

Während sich also Fuss Kritik auf die vermeintlich unrechtmäßige Ermächtigung
Schwarzer Studierender und Studierender of Color bezieht, die ihrer Ansicht
nach durch erfahrungsbasiertes Argumentieren entsteht, problematisiert Mohanty
das Differenzverständnis, das jenem Sprechen zu Grunde liegt. So würden jene zu
Expert\_innen gemacht, die von Rassismus negativ betroffen sind, die Erfahrungen
von \textit{w}eiß Positionierten in rassistischen Verhältnissen und die damit
verbundenen Privilegien hingegen werden verschwiegen. Mohantys Darstellung
entlarvt die machtvolle Dynamik, die in diesen Zuschreibungen liegen. Der
Expert\_innenstatus wird per se allen sogenannten Betroffenen zugesprochen und
ihre Kompetenz ergo nicht als selbst erworbenes Wissen anerkannt, sondern als
natürliche Fähigkeit essentialisiert. Verletzlichkeit wird indessen nicht im
Zusammenhang mit kollektiven Traumata und Diskriminierungserfahrungen, sondern
als persönliche Schwäche im Sinne einer übertriebenen Selbstbezogenheit oder
mangelnden Souveränität gewertet. Zuschreibungen, das wird hier deutlich,
wählen mal kollektive und mal individuelle Erfahrungen als Ausgang, um Menschen
Fähigkeiten zu- und abzusprechen. 

Nicht ob, sondern in welcher Weise Erfahrung zum Thema gemacht wird, muss
stattdessen thematisiert werden. Mohanty kritisiert ein Verständnis von
Differenz, das sich nur auf die zu Anderen gemachten bezieht und das ich hier
als Betroffenheitszuschreibung markiert habe. Sie macht deutlich, dass die
politisch-historische Dimension, die hinter dem Zeichen der Differenz steht,
thematisiert werden muss.  Differenz wird von ihr nicht als Abweichung von
einer Norm, sondern als Infragestellung eben jener Norm verstanden:

\begin{myenv}
  \textit{„Difference seen as benign variation (diversity), for instance,
  rather than as conflict, struggle, or the threat of disruption, bypasses
power as well as history to suggest a harmonious, empty
pluralism.“\footnotemark\footnotetext{Ebd., 181.}}
\end{myenv}

Als problematisch an Fuss Argumentation lässt sich Mohanty folgend darum
herausstellen, dass sie nicht nach den Bedingungen fragt, die dem Sprechen über
eigene Erfahrungen, wie es u.a. auch von marginalisierten Sprecher\_innen
genutzt wird, unterliegen. Sie verkennt, dass der Sprachraum, der hier
Schwarzen Student\_innen und Studierenden of Color zugesprochen wird, oftmals
auf jene Themen begrenzt wird, in denen ihnen auf Grund ihrer zugeschriebenen
Erfahrung ein Stimme gewährt wird. Er unterliegt somit klaren Vorstellungen
davon, wie über das Thema gesprochen werden kann. 

\textit{W}eiße Dominanzverhältnisse, so lassen sich Mohanty und hooks hier lesen, werden
nicht durch verändertes Sprechen durchbrochen, wenn Zeitpunkt und Gegenstand
und damit auch Reichweite des Gesagten nach wie vor durch rassistische
Strukturen bedingt ist.

Darum wird es notwendig, die machtvollen rassistischen und sexistischen
Dynamiken, die auch in akademischen Kontexten wirken und darüber entscheiden,
wer zu welchem Thema und auf welche Art und Weise sprechen kann, in den Blick
zu nehmen und unter Umständen zu problematisieren. Dynamiken, die zwischen
Wissen und Erfahrung unterscheiden und bestimmten Gruppen nicht ermöglichen
ihre Erfahrungen als sogenannte Betroffene einzubringen, sondern dies geradezu
einfordern, müssen hier als hegemonial/paternalistische Gesten zurückgewiesen
werden.

Denn der Essentialismus liegt nicht bei denjenigen, die in diesen Momenten das
Wort ergreifen, sondern in den rassistischen und sexistischen Zuschreibungen,
die sie dazu auffordern als Expert\_innen 'ihrer' Gruppe persönliche Erfahrungen
zu teilen. Nicht selten, so argumentiert hooks weiter, werden an diesen Stellen
tatsächlich essentialistische Vorstellungen, wie sie innerhalb der
rassistischen Matrix entworfen werden,
bedient.\footnotemark\footcitetext[81]{bellhooks}

Der ausschließliche Fokus auf die Praxis der Marginalisierten, übersieht dabei,
so hooks, wie selbstverständlich und entsprechend unmarkiert, durch Rassismus
und Sexismus privilegierte Studierende ihre Erfahrung z.B. als weiße Männer\_*
zum Anlass nehmen, um selbstbewusst und unmarkiert ihre Sichtweisen zu
propagieren. Aus Erfahrung zu sprechen, oder Erfahrungen zum Gegenstand des
Beitrags zu machen, sind für hooks, so verstehe ich diese Beispiele, damit
nicht grundsätzlich verschieden. Doch während letzteres vielfach
problematisiert wird, kann ersteres unhinterfragt Hierarchien darüber, wer sich
zum Sprechen autorisiert fühlen darf, aufrecht
erhalten.\footnotemark\footnotetext{Ebd., 82.}

Der Terminus 'marginalisiert' umgeht dabei, ähnlich wie der Diskurs um die
Betroffenen, dass Rassismen und Sexismen auf alle Subjekte wirken. Ohne
Zentrierung, keine Marginalisierung –be\_troffen sind darum immer alle, nur
ge\_troffen werden alle unterschiedlich.  Es sind diese, subtilen und indirekten
Wege, über die essentialistische Selbstverständnisse von Dominanz und Intellekt
gerade bei den privilegierten Gruppen fungieren, und die von Fuss, so
argumentiert hooks, unberücksichtigt bleiben.  Essentialismus ist dabei stets
auf eine Projektionsfläche angewiesen, um das vermeintliche Wissen über 'die
Anderen' zu bestätigen. Mohanty beschreibt diese Projektionsfläche als Falle
der Repräsentation\footnotemark\footnotetext{Auf den Begriff der Repräsentation werde ich später noch ausführlicher eingehen.}. So könne sie, führt sie als Beispiel auf, nicht über
Verhältnisse z.B. der sogenannten 'Dritten Welt'\footnotemark\footnotetext{
Den Begriff  'Dritte Welt'  verstehe ich im Anschluss an Jürgen Dinkel als
„Ordnungsmuster“ das nicht auf realen Verhältnissen, sondern „ auf der
Phantasie der Sprechenden  [basiert] und [...] so koloniale Idealisierungen und
Stereotype in das postkkoloniale Zeitalter überführen [konnte].“  Dinkel
beschreibt sowohl aus historischer als auch semantischer Perspektive, wie das
das Konzept 'Dritte Welt' während des kalten Krieges, als Gegenpol zu den
einander konkurrierenden Lagern des  kapitalistischen Westen und
sozialistischen Osten konstruiert wurde. Seine Analyse verdeutlicht, wie
Prozesse der Aneignung und Umdeutung von Fremdzuschreibungen ( Der Begriff
wurde in Europa erfunden) auf globaler Ebene geführt wurden. Vgl.:
\url{https://docupedia.de/zg/Dritte_Welt}. 
} sprechen, ohne als Sprecherin \textbf{für} die 'Dritte Welt' missverstanden
zu werden.

\begin{myenv}
  \textit{
  „For I often come to embody the 'authentic' authority and experience for many
  of my students; indeed, they construct me as a native informant in the same
  way that left-liberal white students sometimes construct all people of color
  as the authentic voices of their people.“\footnotemark \footnotetext{Mohanty, „On Race and Voice,“ 194.}
    }
\end{myenv}

Die Zuschreibung, die Mohanty hier sowohl selbst erfährt als auch zwischen den
Studierenden beschreibt, basiert auf einer Vorstellung von Differenz a priori.
Differenz wird nach dieser Vorstellung nicht erst durch
Normalitätsvorstellungen hervorgebracht, sondern existiert unabhängig von
ihnen. Mohanty argumentiert nun, dass jene Naturalisierung von Differenz in
einer Kontinuität mir der kolonial-rassistischen Gewalt steht:

\begin{myenv}
  \textit{
    „In other words, I suggest that educational practices as they are shaped
  and reshaped at these sites cannot be analyzed as merely transmitting already
codified ideas of difference. These practices often produce, codify, and even
rewrite histories of race and colonialism in the name of
difference.“\footnotemark \footnotetext{Ebd., 184.}
    }
\end{myenv}

Betroffenheit, so lässt sich Mohanty hier lesen, wird in Bildungskontexten als
Mittel missbraucht, um kolonial-rassistische Strukturen fortzuschreiben und
dadurch an der Aufrechterhaltung von Herrschaftsverhältnissen mitzuwirken.
Erfahrung dient dabei als Anker, durch das sich das Betroffenheitsdenken
artikulieren und festsetzen kann.

Differenz und die in diesem Zusammenhang mächtigen Normalitätsvorstellungen
haben jedoch keine fixierten Bedeutungen. Die Bedeutungen werden immer wieder
neu verhandelt und können, so hooks, für die Stabilisierung von Dominanz- und
Herrschaftsverhältnissen, aber auch für ihre Infragestellung im Kontext einer
strategische Bündnispolitik nutzbar gemacht werden. Die Erfahrung von Differenz
kann somit immer Antrieb für Veränderungen strategisch genutzt werden und so
eine empowernde Wirkung entfalten:

\begin{myenv}
  \textit{„Identity politics emerges out of the struggles of oppressed or
    exploited groups to have a standpoint on which to critique dominant
    structures, a position that gives purpose and meaning to
    struggle.“\footnotemark \footcitetext[88]{bellhooks}
    }
\end{myenv}

Der entscheidende Unterschied ist hierbei, dass anders als bei zugeschriebener
Differenz, in strategische Identitätspolitiken die \textbf{eigene} Verortung in
Differenzverhältnissen als Ausgangspunkt dient und davon ausgegangen wird, dass
die damit einhergehenden Erfahrungen nicht individuell bedingt sind, sondern
einer Systematik unterliegen. Erfahrung spielt somit auch im emanzipatorischen
Umgang mit Differenz eine zentrale Rolle. Schließlich beabsichtigen
strategische Identitätspolitiken, die Systematik von zugeschriebener Differenz
aufzuzeigen, als ungerecht fort zuweisen und diejenigen, die davon profitieren,
in die Verantwortung zu ziehen. Die Thematisierung der kollektiven Dimension
persönlicher Erfahrungen eröffnet somit neue Interpretations- und
Handlungsräume, die in Bündnissen aufgehen und u. U. eine empowernde und
emanzipatorische Wirkung entfalten können. 

Wenn (Krisen-)Erfahrungen zum Gegenstand universitärer Bildung gemacht werden,
kann sich dies sowohl befreiend als auch beschränkend auf die Subjekte
auswirken. Die vorherrschende Ordnung, die ich mit Mohanty und hooks mit einer
rassismus- und sexismusthematisierenden Perspektive beschreibe situiert jene
(Krisen)Erfahrung hierbei in historisch gewachsenen, umkämpften
Differenzverhältnissen. Dies macht deutlich, dass Subjekte unterschiedliche
positioniert sind und ihre (Krisen-)Erfahrungen entsprechend unterschiedlich
stark Gefahr laufen, vereinnahmt zu werden. Bildungskontexte, die den Anspruch
haben, für politische Prozesse der Ermächtigung Räume zu schaffen, stehen damit
vor einer Herausforderung: Sie beabsichtigen Erfahrungen von Rassismus als
relevante Wissensquellen anzuerkennen, um die strukturelle Dimension von
Rassismus analysieren und seine Funktionsweisen verstehen zu können. Die
Verknüpfung von individuellen Erfahrungen mit Theorien über rassistische
Verhältnisse muss hierbei zugleich gegenüber einer wechselseitigen
Vereinnahmung achtsam sein. Denn es besteht immer die Möglichkeit, dass eine
konkrete Erfahrung notwendig für die Legitimierung einer Theorie, oder dass
eine bestimmte Theorie zur Voraussetzung für die Artikulation einer Erfahrung
erklärt wird. Hier ist es unabdingbar, dass der eingeführte Unterschied
zwischen Differenztheorie und Betroffenheitsdenken berücksichtigt wird.

Lernprozesse, in denen Interpretationsmöglichkeiten für eigene Erfahrungen
erarbeitet werden, oder Theorien angesichts von artikulierten Erfahrungen in
Frage gestellt werden, sind hier, so hooks,  stets auf tastende Bemühungen
angewiesen, die eine Offenheit sowohl gegenüber der Infragestellung etablierter
Theorien als auch gegenüber etablierten Interpretationsmustern von
individuellen Erfahrungen beabsichtigen.\footnotemark\footnotetext{hooks, ebd., 88ff.}

Für hooks steht darum statt einer orthodoxen Haltung für oder gegen die
Thematisierung von Erfahrung ein selbstreflexiver Umgang mit
erfahrungsbasiertem Sprechen im Vordergrund. Sie diktiert keine Regeln oder
Rezepte, sondern regt die Studierenden dazu an, sich über Anliegen und
Konsequenzen verschiedener Standpunkte zur Relevanz von erfahrungsbasierten
Beitragen im Seminar Gedanken zu machen. Eine Auseinandersetzung über Anliegen
und Effekte erfahrungsbasierter Beiträge müsse jedoch berücksichtigen, dass das
Sprechen über die Erfahrungen anderer, insbesondere unterdrückter Gruppen
systematisch praktiziert werde. Zunächst sei es darum unabdingbar, dass dieses
Ungleichgewicht erkannt und problematisiert werde.\footnotemark\footnotetext{Ebd., 89.} hooks verschiebt hier den
Fokus weg von der Frage, ob eine Ermächtigung aus einer marginalisierten
Position legitim ist, hin zu dem Umstand, dass sich universitäres Sprechen
vielfach auf die vermeintlich Anderen und wenig auf die eigene Position oder
Lebenswelt bezieht. 

\subsubsection{Dekoloniale Bildung}

Doch wie kann diese machtvolle Dynamik, die vermeintlich Anderen im Sprechen
über die vermeintlich Anderen zu erzeugen, durchbrochen werden? Eine
Voraussetzung, um neue Wege der Auseinandersetzung mit der Geschichte und
Gegenwart von Sexismus, Rassismus, Heterosexismus etc. bestreiten zu können,
muss Mohanty zufolge darin liegen, den Beitrag, den konventionelle
Wissensproduktionen und Pädagogiken an der Universität zur systematischen
Marginalisierung der Geschichte und der Erfahrungen von Menschen aus der sog.
'Dritten Welt' leisteten und leisten, zu erkennen und zu poblematisieren. Erst
diese Einsicht ermögliche eine ernsthafte und nachhaltige „dekolonisierung
unserer disziplinären und pädagogischen Praxen“\footnotemark\footnotetext{Mohanty, „On Race and Voice,“ 191.}. Dabei ergibt sich für Mohanty
folgende grundsätzliche Frage:

\begin{myenv}
  \textit{„The crucial question is how we teach about the West and its Others
    so that education becomes the practice of liberation. This question becomes
    all the more important in the context of the significance of education as a
    means of liberation and advancement for Third World and post colonial
    peoples and their/our historical belief in education as a crucial form of
    resistance to the colonization of hearts and minds.“\footnotemark
    \footnotetext{Ebd.}
    }
\end{myenv}

Bildung wird hier von Mohanty als Ort von Befreiung theoretisiert und damit zu
einem zentralen, auch geschichtlichen Schauplatz antikolonialen Widerstands
erklärt. Entsprechend stellt sie die Frage, wie jene Universität der Befreiung
geschaffen werden kann, also wie Lehr-Lernverhältnisse gestaltet werden können,
die der „Kolonisierung von Herz und Geist“\footnotemark\footnotetext{Ebd.} widerstehen.  Mit Mohanty wird
damit die eingangs geführte Auseinandersetzung um das emanzipatorische Moment
in Bildung in kolonialen Verhältnissen verortet. Ihre Verortung weist dabei
darauf hin, dass ein emanzipatorischer Anspruch an Bildung immer vor dem
Hintergrund der Geschichte von Bildung und damit der Geschichte von Kolonialismus geltend gemacht werden muss. Bildung  war und ist damit Teil hegemonialer Praxen von Ausschluss und Entmachtung ebenso wie von Widerstand und Aneignung.

Dekolonisierung ist ihrer Ansicht nach unmittelbar an die Möglichkeit der
Selbstdefinitionen geknüpft und findet in universitären Räumen statt, die das
Ziel haben historisch ausgeschlossenen Perspektiven sicht- und hörbar zu
machen.\footnotemark\footnotetext{Ebd., 184.} Dabei sind jene Räume stets damit beauftragt, sich gegenüber der Aneignung und Vereinnahmung durch apolitische Vielfaltbestrebungen zu wehren:
\begin{myenv} 
  \textit{„By their very location in the academy, fields such
    as women's studies are grounded in definitions of difference, difference
    that attempts to resist incorporation and appropriation by providing a
    space for historically silenced peoples to construct
    knowledges.“\footnotemark\footnotetext{Ebd.} } \end{myenv}

Mohanty betont immer wieder das, was an anderer Stelle in dieser Arbeit mit
der Überwindung von Subjekt-Objekt-Verhältnissen in der Wissensproduktion
diskutiert wird. Damit kristallisiert sich bereits hier die grundlegende
Frage, wem in Erkenntnisprozessen ein Subjektstatus zugesprochen wird und
auf welchen Ausschlüssen das Subjektverständnis basiert. Hiermit erweitert
sie die Diskussion der Fragmentierung des Selbst in Bildungsprozessen um
die Frage, wem in Bildungsprozessen überhaupt das Privileg zukommt, als
Subjekt angerufen zu werden und weist auf Praxen des Silencing hin.

Die Möglichkeit, individuelle Erfahrungen nicht nur zu artikulieren, sondern
sie als legitimen Wissensbestand, der sich durchaus auch im Dissens zu anderen
Wissensbeständen behaupten kann, gebrauchen zu können, kann hier als notwendige
Bedingung gesetzt werden, um Bildungsräume zu schaffen, in denen Lernende als
Subjekte agieren können: „The authorization of experience is thus a crucial
form of empowerment for students-a way for them to enter the classroom as
speaking subjects.“\footnotemark\footnotetext{Ebd., 193.} Bildung ist somit auf Subjekte angewiesen, die sich selbst
und damit ihre Stimme als notwendigen Teil von Bildungsprozessen begreifen.
Kritische Bildung verfolgt hierbei den Anspruch, jenen subjektiven Status nicht
als selbstverständlich anzunehmen, sondern die Mechanismen seiner Herstellung
in den Blick zu nehmen: 
\begin{myenv} 
  \textit{
„Without this analysis of culture and of experience in the classroom, there is no way to develop and nurture oppositional practices. After all, critical education concerns the production of subjectivities in relation to discourses of knowledge and power.“\footnotemark\footnotetext{Ebd., 196.} } \end{myenv}

Subjekte müssen also immer in ihrer Beziehung zu Diskursen, Wissen und Macht
untersucht werden. Subjektivierung, also der Prozess der Subjektwerdung, lässt
sich, so verstehe ich hier Mohanty, über eine Analyse von Erfahrung
untersuchen, die Erfahrung immer im Kontext historisch gewachsener Verhältnisse
verortet. Erst wenn Erfahrung zum Gegenstand der Analyse gemacht wird, kann
widerständiges Wissen entstehen. Dabei ist es unabdingbar die
(welt-)gesellschaftlichen Verhältnisse als Ausdruck und Teil ihrer Geschichte
zu begreifen. Sie müssen historisiert werden, um herrschende Differenzordnung
und ihre Wirkung auf Subjekte einbeziehen zu können. Mohantys Position knüpft
hier an die bereits explizierten Verständnisse von Bildung und Kritik an und
erweitert sie um die vielschichtigen und stets kontextgebundenen Bedeutungen,
die Erfahrungen in den Prozessen der Selbst- und Fremdzuschreibung von
Subjektpositionen einnehmen. Prozesse der Ermächtigung können hierbei nicht
losgelöst von Prozessen der Unterwerfung gedacht werden. 

Die kritische Auseinandersetzung mit Erfahrung kann dabei insbesondere im
Seminar, so führt sie fort, einen zentralen Stellenwert einnehmen. Erfahrung
müsse hier als Ausdruck von Erlebtem und von textueller und historischer
Repräsentation diskutiert werden. Nicht die Autorität der Erfahrung, sondern
eine Leidenschaft für Erfahrung und Erinnerung erlaube es, erfahrungs- und
analytisch orientierte Wissensbestände miteinander in Bezug zu
setzen.\footnotemark\footcitetext{bellhooks}

An dieser Stelle wird das eingangs erwähnte Verhältnis, das zwischen Erfahrung
als vermittelnder und vermittelter Größe besteht, konkreter. Mit Erfahrung als
vermittelnde Größe verstehe ich eben jenen Prozess der Krisenerfahrung, den
Mohanty hier als das Erleben beschreibt. Erfahrung als vermittelte Größe
fokussiert indessen den Prozess, in dem eigene oder Erfahrungen anderer als
Werkzeug für Bildungsprozesse zu sogenannten Vermittlern über gesellschaftliche
Verhältnisse werden, die es ermöglichen einen intersubjektiven Austausch
einzugehen. Während sich erstere also stärker auf das Erlebte und dessen intime
Verarbeitung beziehen, steht bei letzterem der Einzug dieses Erlebten in den
Diskurs und den damit möglicherweise einhergehenden Impulsen, die dies bei
Anderen in ihren Bildungsprozessen anstößt im Vordergrund.

Die Diskussion von Mohanty und hooks in Hinblick auf diese Differenzierung
zeigen jedoch eines: Eine Trennung von Erfahrung als Erlebtem und Erfahrung als
Erzähltem darf nicht darüber hinwegtäuschen, dass sowohl das Erleben als auch
das Erzählen von hegemonialen Strukturen nicht losgelöst sind, die Einfluss
darauf haben, was sag-,  denk- und damit auch erfahrbar ist. Und so wird es
dringend notwendig, dass „[...] within the classroom, [...] teachers and
students develop a critical analysis of how experience itself is named,
constructed, and legitimated in the academy.“\footnotemark\footnotetext{Mohanty, „On Race and Voice,“ 196.}

Dieser Schlussappell ist charakteristisch für Mohantys Anliegen, Lehrende und
Lernende in die Verantwortung zu ziehen und den Leser\_innen weniger Rezepte
als viel mehr Aufgaben mitzugeben. Sie fordert, eine kritische Analyse der
Verortung, Konstruktion und Legitimierung von Erfahrung in der Wissenschaft,
kurz, einen reflexiven Umgang mit Erfahrungen in universitärer
Wissensproduktion. Es geht weder Mohanty noch hooks darum die Leerstellen, die
durch hegemoniale Wissensstrukturen entstanden sind, nun mit Erfahrungen der
Marginalisierten zu füllen. Stattdessen weisen hooks und Mohanty darauf hin,
dass das Sprechen aus und über Erfahrung bereits ein zentraler Bestandteil in
der Auseinandersetzung mit Wissen einnimmt. Was fehlt ist jedoch eine reflexive
Auseinandersetzung damit, wie Erfahrung eigentlich beschaffen ist und welchen
Ort sie in der Produktion und Legitimation von Wissen erhält. 

Für mich stellt sich hier die Frage, ob der Erfahrungsbegriff den
Identitätsbegriff unter Umständen ablösen, und so einen kontingenten,
historisch verorteten und zugleich zutiefst subjektive Dimension dessen, was
Menschen verbindet und trennt, erahnen lässt.  Was, wenn Wissen nicht mehr an
Identitäten, sondern an Erfahrungen gekoppelt wird und wenn diese Erfahrungen
nicht als statisch, sondern wandelbare Interpretationen des Erlebten, ja
Widerfahrenen verstanden werden? 
