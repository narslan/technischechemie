\documentclass[12pt]{article}
\usepackage{amsmath,mathtools}
\usepackage[usenames,dvipsnames]{xcolor}
\usepackage[german]{babel}
\usepackage[utf8]{inputenc}
\usepackage[T1]{fontenc}
\usepackage{textcomp}
%\usepackage{libertine}
%\usepackage{helvet}
\usepackage{lscape}

\usepackage{microtype}
% Minion and Myriad fonts
\usepackage[minionint,mathlf]{MinionPro}
\renewcommand{\sfdefault}{Myriad-LF}
\usepackage{siunitx}
\usepackage{fancyhdr}
\usepackage{sectsty}
\usepackage{setspace}
\usepackage{booktabs} % To thicken table lines
\usepackage[version=4]{mhchem}
\usepackage[draft]{graphicx}
\usepackage[labelfont=bf]{caption}
\usepackage{subcaption}
\usepackage{chemstyle}
\usepackage{tabularx}
%\usepackage[compatibility=4.7,language=german]{chemmacros}
\renewcommand{\familydefault}{\sfdefault}
\sisetup{detect-all}
\usepackage{chngcntr}
\counterwithin{table}{section}
\counterwithin{figure}{section}

\usepackage{titlesec}

\titleformat*{\section}{\large\bfseries}
\titleformat*{\subsection}{\normalsize\bfseries}

\usepackage{geometry}
 \geometry{
 a4paper,
 left=20mm,
 top=30mm,
 right=20mm
 }

\begin{document}

\begin{onehalfspace}

\section{Geometrieoptimierung und Frequenzberechnung an Dichlorethen}
Ziel dieser Aufgabe ist es eine Geometrieoptimierung von cis- und trans- 1,2-Dichlorethen
durchzuführen und anschließend die Schwingungsfrequenzen zu berechnen.
\textbf{Auswertung 1 }\\
\begin{figure}[!hptb]
    \centering
    \begin{subfigure}[b]{0.4\textwidth}
        \includegraphics[width=\textwidth]{data/cis_darstellung.png}
    \end{subfigure}
    ~ %add desired spacing between images, e. g. ~, \quad, \qquad, \hfill etc.
      %(or a blank line to force the subfigure onto a new line)
    \begin{subfigure}[b]{0.4\textwidth}
        \includegraphics[width=\textwidth]{data/trans_darstellung.png}
    \end{subfigure}
    \caption{Optimierte Geometrien von cis- und trans-1,2-Dichlorethen}
\end{figure}
\begin{table}[!htpb]
\begin{tabularx}{\textwidth}{lllll}
\toprule
Bindung in \si{\angstrom} &cis(RHF) & cis(MP2) & trans(RHF) & trans(MP2)\\
\midrule
C-Cl  & 1.723659 &1.715588   & 1.732743 & 1.723728\\
C-C  & 1.311110 & 1.338136   & 1.308159 & 1.335534\\
C-H  & 1.071179 & 1.082586   & 1.070440 & 1.082339\\
\bottomrule
\end{tabularx}
\label{tab:cistrans}
\end{table}

In Folgenden Tabellen sind die Ergebnisse zur Geometrieoptimierung und die Schwingungsfrequenzen zu cis-und trans-1,2-Dichlorethen mit der RHF Methode mit dem Basissatz 6-31+G(d,p) aufgeführt:\\
\begin{landscape}

\begin{table}[!htpb]

\caption{cis-1,2-Dichlorethen mit Basissatz 6-311G(d,p)}
\begin{tabularx}{\textwidth}{llll|lll|llll}
\toprule
\multicolumn{4}{l}{RHF}&\multicolumn{3}{l}{MP2}&\multicolumn{4}{l}{Experimentell} \\
Irrep &   SF \si{\per\centi\meter} & \multicolumn{2}{c}{Schwingungsintensität} &
Irrep &   SF \si{\per\centi\meter} & Schwingungsintensität  &
Irrep &   SF \si{\per\centi\meter} & \multicolumn{2}{c}{Schwingungsintensität}\\
& & IR & Raman& & & IR && & IR & Raman\\
\midrule
$A _1$ & 183 & 0.41 & 2.160   &$A _1$  & 173 & 0.21     & & & &\\
$A _2$ & 458 & 0 & 4.86       &$A _2$  & 411 & 0        & & & &\\
$B _2$ & 614 & 7.93 & 6.74    &$B _2$  & 584 & 3.03     & & & &\\
$A _1$ & 759 & 27.36 & 12.33  &$A _1$  & 714 & 69.15    & & & &\\
$B _1$ & 803 & 73.62 & 1.13   &$B _1$  & 747 & 19.58    & & & &\\
$B _2$ & 920 & 106.94 & 0.01  &$B _2$  & 888 & 0        & & & &\\
$A _2$ & 1051 & 0 & 4.54      &$A _2$  & 897 & 84.06    & & & &\\
$A _1$ & 1324 & 0.14 & 20.88  &$A _1$  & 1238 & 0.02    & & & &\\
$B _2$ & 1439 & 36.02 & 1.05  &$B _2$  & 1341 & 22.88   & & & &\\
$A _1$ & 1818 & 34.83 & 59.81 &$A _1$  & 1648 &31.96    & & & &\\
$B _2$ & 3373 & 15.14 & 53.92 &$B _2$  & 3249 & 14.28   & & & &\\
$A _1$ & 3397 & 2.93 & 141.85 &$A _1$  & 3270 & 2.77    & & & &\\
\bottomrule
\end{tabularx}
\label{tab:cisvergleich}

\end{table}
\end{landscape}

\begin{table}[!htpb]
\centering
\caption{cis-1,2-Dichlorethen RHF mit Basissatz 6-311G(d,p) }
\begin{tabular}{llll}
\toprule
Darstellung &   Schwingungsfrequenz \si{\per\centi\meter} & \multicolumn{2}{c}{Schwingungsintensität} \\
&&IR&Raman\\
\midrule
$A _1$ & 183 & 0.41 & 2.160\\
$A _2$ & 458 & 0 & 4.86\\
$B _2$ & 614 & 7.93 & 6.74\\
$A _1$ & 759 & 27.36 & 12.33\\
$B _1$ & 803 & 73.62 & 1.13 \\
$B _2$ & 920 & 106.94 & 0.01\\
$A _2$ & 1051 & 0 & 4.54\\
$A _1$ & 1324 & 0.14 & 20.88\\
$B _2$ & 1439 & 36.02 & 1.05\\
$A _1$ & 1818 & 34.83 & 59.81\\
$B _2$ & 3373 & 15.14 & 53.92\\
$A _1$ & 3397 & 2.93 & 141.85\\
\bottomrule
\end{tabular}
\end{table}



\begin{landscape}

\begin{table}[!htpb]

\caption{trans-1,2-Dichlorethen  mit Basissatz 6-311G(d,p) }
\begin{tabularx}{\textwidth}{llll|lll|llll}
\toprule
\multicolumn{4}{l}{RHF}&\multicolumn{3}{l}{MP2}&\multicolumn{4}{l}{Experimentell} \\
Irrep &   SF \si{\per\centi\meter} & \multicolumn{2}{c}{Schwingungsintensität} &
Irrep &   SF \si{\per\centi\meter} & Schwingungsintensität  &
Irrep &   SF \si{\per\centi\meter} & \multicolumn{2}{c}{Schwingungsintensität}\\
& & IR & Raman& & & IR && & IR & Raman\\
\midrule
$A _u$ & 233  & 0.73   & 0     &   $A _u$ & 215 & 0.15     & & & &\\
$B _u$ & 257  & 4.58   & 0     &   $B _u$ & 243 & 3.09     & & & &\\
$A _g$ & 376  & 0.0    & 9.82  &   $A _g$ & 360 & 0        & & & &\\
$B _u$ & 875  & 156.36 & 0     &   $B _g$ & 753 & 0        & & & &\\
$B _g$ & 912  & 0      & 6.61  &   $B _u$ & 869 & 115.7    & & & &\\
$A _g$ & 920  & 0      & 9.56  &   $A _g$ & 892 & 0        & & & &\\
$A _u$ & 1049 & 82.03  & 0     &   $A _u$ & 939 & 73.37    & & & &\\
$B _u$ & 1333 & 27.08  & 0     &   $B _u$ & 1258 & 20.34  & & & &\\
$A _g$ & 1417 & 0      & 24.55 &   $A _g$ & 1325 & 0      & & & &\\
$A _g$ & 1815 & 0      & 49.17 &   $A _u$ & 1645 & 0      & & & &\\
$B _u$ & 3393 & 17.72  & 0     &   $B _u$ & 3263 & 15.29  & & & &\\
$A _g$ & 3400 & 0      & 115.68&   $A _g$ & 3267 & 0      & & & &\\
\bottomrule
\end{tabularx}
\label{tab:transvergleich}

\end{table}

\end{landscape}


\begin{table}[!htpb]
\centering
\caption{trans-1,2-Dichlorethen RHF mit Basissatz 6-311G(d,p) }
\begin{tabular}{llll}
\toprule
Darstellung & Schwingungsfrequenz \si{\per\centi\meter} & \multicolumn{2}{c}{Schwingungsintensität} \\
&&IR&Raman\\
\midrule
$A _u$ & 233 & 0.73& 0\\
$B _u$ & 257 & 4.58 & 0\\
$A _g$ & 376 & 0.0 & 9.82\\
$B _u$ & 875 & 156.36 & 0\\
$B _g$ & 912 & 0 & 6.61\\
$A _g$ & 920 & 0 & 9.56\\
$A _u$ & 1049 & 82.03 & 0\\
$B _u$ & 1333 & 27.08 & 0\\
$A _g$ & 1417 & 0 & 24.55\\
$A _g$ & 1815 & 0 & 49.17\\
$B _u$ & 3393 & 17.72 & 0\\
$A _g$ & 3400 & 0 & 115.68\\
\bottomrule
\end{tabular}
\end{table}

\begin{table}[!htpb]
\centering
\caption{ Zusammenfassung Moleküle}
\begin{tabular}{llllc}
\toprule
Molekül & Methode &   Basissatz & Energie \si{\hartree} & Symmetrie\\
\midrule
cis-1,2-Dichlorethen   & RHF& 6-311G(d,p)& -995.89694703 &$C_ {2v}$\\
cis-1,2-Dichlorethen   & MP2& 6-311G(d,p)& -996.45258595  &$C_ {2v}$\\
trans-1,2-Dichlorethen & RHF& 6-311G(d,p)& -995.89737964 &$C_ {2h}$ \\
trans-1,2-Dichlorethen & MP2& 6-311G(d,p)& -996.45188 &$C_ {2h}$\\
\bottomrule
\end{tabular}\\
\end{table}


\textbf{Auswertung 2}



\begin{figure}[!hptb]
    \centering
    \begin{subfigure}[b]{0.4\textwidth}
        \includegraphics[width=\textwidth,height=5cm]{data/cis_ir_spektrum.png}
        \subcaption{IR-Spektrum von cis-1,2-Dichlorethen auf RHF/6-311G(d,p)}
    \end{subfigure}
    ~ %add desired spacing between images, e. g. ~, \quad, \qquad, \hfill etc.
      %(or a blank line to force the subfigure onto a new line)
    \begin{subfigure}[b]{0.4\textwidth}
        \includegraphics[width=\textwidth,height=5cm]{data/cis_ir_mp2.png}
        \subcaption{IR-Spektrum von cis-1,2-Dichlorethen auf MP2/6-311G(d,p)}
    \end{subfigure}
\end{figure}


\begin{figure}[!hptb]
    \centering
    \begin{subfigure}[b]{0.4\textwidth}
        \includegraphics[width=\textwidth,height=5cm]{data/trans_ir_spektrum.png}
        \subcaption{IR-Spektrum von trans-1,2-Dichlorethen auf RHF/6-311G(d,p)}
    \end{subfigure}
    ~ %add desired spacing between images, e. g. ~, \quad, \qquad, \hfill etc.
      %(or a blank line to force the subfigure onto a new line)
    \begin{subfigure}[b]{0.4\textwidth}
        \includegraphics[width=\textwidth,height=5cm]{data/trans_ir_mp2.png}
        \subcaption{IR-Spektrum von trans-1,2-Dichlorethen auf MP2/6-311G(d,p)}
    \end{subfigure}
\end{figure}


\begin{figure}[!hptb]
    \centering
    \begin{subfigure}[b]{0.4\textwidth}
        \includegraphics[width=\textwidth,height=5cm]{data/cis_raman_spektrum.png}
        \subcaption{Raman-Spektrum von cis-1,2-Dichlorethen auf RHF/6-311G(d,p)}
    \end{subfigure}
    ~ %add desired spacing between images, e. g. ~, \quad, \qquad, \hfill etc.
      %(or a blank line to force the subfigure onto a new line)
    \begin{subfigure}[b]{0.4\textwidth}
        \includegraphics[width=\textwidth,height=5cm]{data/trans_raman_spektrum.png}
        \subcaption{Raman-Spektrum von trans-1,2-Dichlorethen auf MP2/6-311G(d,p)}
    \end{subfigure}
\end{figure}

\textbf{Auswertung 6}
\end{onehalfspace}
\end{document}