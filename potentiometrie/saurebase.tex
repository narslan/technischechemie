\documentclass{article}
\usepackage{amsmath}


%----------------------------
\usepackage{microtype}
%\usepackage{fourier}
%\usepackage{mathpazo} % add possibly `sc` and `osf` options

%\usepackage[default,osfigures,scale=0.85]{opensans} %% Alternatively
%% use the option 'defaultsans' instead of 'default' to replace the
%% sans serif font only.

%\usepackage[default]{gillius}
\usepackage{listings}
\usepackage[utf8]{inputenc}
\usepackage{tgheros}
\renewcommand*\familydefault{\sfdefault} %% Only if the base font of the document is to be sans serif

\usepackage[T1]{fontenc}
\usepackage[german]{babel}

\usepackage{mathastext}




%units
%--------------------------------------
\usepackage{siunitx}
%drawing
%------------------------------------
\usepackage{tikz} % To generate the plot from csv
\usepackage{pgfplots}
\usepackage{pgfplotstable}
\usepackage{graphicx}

\usepackage{caption}
\usepackage{color, colortbl}
\usepackage{tabularx}
\usepackage{setspace}
\usepackage{mhchem}
\usepackage{fancyhdr}
%\usepackage[landscape]{geometry}

\pagestyle{fancy}

\cfoot{\thepage}

\rhead{\footnotesize Gruppe 24}
\lhead{\footnotesize Potentiometrische Titration - Auswertung}
\setlength{\headheight}{15pt}


\renewcommand{\familydefault}{\sfdefault}
\captionsetup{font=footnotesize}

\definecolor{LightCyan}{rgb}{0.88,1,1}
\usetikzlibrary{datavisualization}
\pgfplotsset{compat=newest} % Allows to place the legend below plot
\usepgfplotslibrary{units} % Allows to enter the units nicely
\usetikzlibrary{pgfplots.groupplots}
\pagenumbering{gobble}

\usepackage{overcite}
\renewcommand\citeform[1]{[#1]}

\sisetup{
  round-mode          = places,
  round-precision     = 2,
  inter-unit-product =\ensuremath{{}\cdot{}}
}
\makeatletter
\setlength{\@fptop}{0pt}
\makeatother
%\usetikzlibrary{arrows, positioning, calc, datavisualization}

\begin{document}
\tikzset{every mark/.append style={scale=0.5}}
\begin{figure}[!t]
\caption{  $\Delta E$ vs \ce{V_{AgNO3}} Titrationskurve der Joghurtprobe mit 0.1 M \ce{AgNO3} bei T = 20 \si{\celsius}}

\begin{tikzpicture}
\begin{groupplot}[
    group style={
        group name=my plots,
        group size=1 by 3,
        xlabels at=edge bottom,
        xticklabels at=edge bottom,
        vertical sep=0pt
    },
    width=12cm,
    height=7cm,
    xlabel= \ce{V_{\ce{AgNO3}}} / \si{\milli\liter},
    xmin=0, xmax=10,
    tickpos=left,
    ytick align=outside,
    xtick align=outside,
    legend style={draw=none,legend pos=outer north east},
]
\nextgroupplot[ylabel=$\Delta E$,y unit=\si{\milli\volt},xtick={1,2,3,4,5,6,7,8,9},xticklabels={1,2,3,4,5,6,7,8,9}]
\addplot[black,thick,smooth] file[skip first]{ag.txt};
\node[coordinate,pin=right:{\scriptsize ÄP bei $V_{\ce{AgNO3}}$ = \SI{1.85}{\milli\liter}}]
at (axis cs:1.87,-142) {};
\addlegendentry{\scriptsize $\Delta E = E_{IE}-E_{RE}$ }
\nextgroupplot[ylabel=$\ce{\frac{\partial pH}{\partial V}}$,xtick={1,2,3,4,5,6,7,8,9},xticklabels={1,2,3,4,5,6,7,8,9}]
\addplot[mark=diamond] file[skip first] {agd.txt};
\nextgroupplot[ylabel=$\ce{\partial \frac{(\partial pH)}{(\partial V)^2}}$]
\addplot[mark=square] file[skip first] {agdd.txt };
\end{groupplot}
\end{tikzpicture}
\end{figure}


\begin{figure}[!t]
\caption{  pH vs $\ce{V_{NaOH}}$ Titrationskurve einer 50 \si{\milli\liter} 0.005 M \ce{H2SO4}-Lösung   mit 0.1 M \ce{NaOH} in 0.1 \si{\milli\liter} Schritten}

\begin{tikzpicture}
\begin{groupplot}[
    group style={
        group name=my plots,
        group size=1 by 3,
        xlabels at=edge bottom,
        xticklabels at=edge bottom,
        vertical sep=0pt
    },
    width=12cm,
    height=7cm,
    xlabel= \ce{V_{\ce{NaOH}}} / \si{\milli\liter},
    xmin=0, xmax=6,
    tickpos=left,
    ytick align=outside,
    xtick align=outside,
    legend style={draw=none,legend pos=outer north east},
]
\nextgroupplot[ymax=14,ylabel=pH,xtick={1,2,3,4,5},xticklabels={1,2,3,4,5}]
\addplot[mark=o.thick] file[skip first] {naohp.txt};
\node[coordinate,pin=right:{\scriptsize ÄP bei pH = 7 und $\ce{V_{NaOH}}$  = 2.75 \si{\milli\liter}}] at (axis cs:2.75,7) {};
\addlegendentry{\scriptsize $\,$Experimentelle Werte}
\nextgroupplot[ylabel=$\ce{\frac{\partial pH}{\partial V}}$,xtick={1,2,3,4,5},xticklabels={1,2,3,4,5}]
\addplot[mark=diamond] file[skip first] {naohpd.txt};
\nextgroupplot[ylabel=$\ce{\partial \frac{(\partial pH)}{(\partial V)^2}}$]
\addplot[mark=square] file[skip first] {naohpdd.txt };
\end{groupplot}
\end{tikzpicture}
\end{figure}

\begin{figure}[!t]
\caption{ Der Vergleich einer theoretischen Titrationskurve (50 \si{\milli\liter} einer 0.025 M \ce{H2SO4}-Lösung mit 0.1 M \ce{NaOH} in 0.1 \si{\milli\liter} Schritten)
mit der experimentellen Titrationskurve (50 \si{\milli\liter} einer 0.005 M \ce{H2SO4}-Lösung mit 0.1 M \ce{NaOH} in 0.1 \si{\milli\liter} Schritten)}

\begin{tikzpicture}
\begin{groupplot}[
    group style={
        group name=my plots,
        group size=1 by 3,
        xlabels at=edge bottom,
        xticklabels at=edge bottom,
        vertical sep=0pt
    },
    width=12cm,
    height=7cm,
    xlabel= $\ce{V_{NaOH}}$ ,
    x unit= \si{\milli\liter},
    tickpos=left,
    ytick align=outside,
    xtick align=inside,
    legend style={draw=none,legend pos=outer north east},
]
\nextgroupplot[ymax=14,ylabel=pH,xtick={1,5,10,15,20,25,30},xticklabels={1,5,10,15,20,25,30},xlabel=$\ce{V_{NaOH}}$, x unit =\si{\milli\liter}]
\addplot[mark=triangle] file[skip first]{naoht.txt};
\node[coordinate,pin=left:{\scriptsize ÄP bei pH = 7 und $\ce{V_{NaOH}}$ = 25 \si{\milli\liter}}] at (axis cs:25,7) {};
\addlegendentry{\scriptsize Theoretische Werte}
\nextgroupplot[ymax=14,ylabel=pH,xtick={1,2,3,4,5},xticklabels={1,2,3,4,5}]
\addplot[mark=o.thick] file[skip first] {naohp.txt};
\node[coordinate,pin=right:{\scriptsize ÄP bei pH = 7 und $\ce{V_{NaOH}}$ = 2.75 \si{\milli\liter}}] at (axis cs:2.75,7) {};
\addlegendentry{\scriptsize Experimentelle Werte}
\end{groupplot}
\end{tikzpicture}
\end{figure}


\begin{figure}[!t]
\caption{ pH vs $\ce{V_{NaOH}}$ Titrationskurve einer 50 \si{\milli\liter} 0.00333 M \ce{H3PO4}-Lösung mit 0.1 M \ce{NaOH} in 0.1 \si{\milli\liter} Schritten}

\begin{tikzpicture}
\begin{groupplot}[
    group style={
        group name=my plots,
        group size=1 by 3,
        xlabels at=edge bottom,
        xticklabels at=edge bottom,
        vertical sep=0pt
    },
    width=12cm,
    height=7cm,
    xlabel= \ce{V_{\ce{NaOH}}} / \si{\milli\liter},
    xmin=0, xmax=8,
    tickpos=left,
    ytick align=outside,
    xtick align=outside
]
\nextgroupplot[ymax=12,ylabel=pH,xtick={1,2,3,4,5,6,7},xticklabels={1,2,3,4,5,6,7}]
\addplot file[skip first] {phosphor.txt};
\node[coordinate,pin=right:{\tiny ÄP bei pH=4.45}]
at (axis cs:1.45, 4.45) {};
\node[coordinate,pin=right:{\tiny ÄP bei pH=9.04}]
at (axis cs:3.85, 9.04) {};
\nextgroupplot[ylabel=$\ce{\frac{\partial pH}{\partial V}}$,xtick={1,2,3,4,5,6,7},xticklabels={1,2,3,4,5,6,7}]
\addplot  file[skip first] {phosphord.txt};
\nextgroupplot[ylabel=$\ce{\partial \frac{(\partial pH)}{(\partial V)^2}}$]
\addplot  file[skip first] {phosphordd.txt};
\end{groupplot}
\end{tikzpicture}
\end{figure}



\end{document}


%\addplot[domain=0:1200]{-1.42e-5*x^(2)+2.836e-2*x+45.47};


