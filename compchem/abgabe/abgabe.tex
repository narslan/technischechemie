\documentclass[12pt]{article}
\usepackage{amsmath,mathtools}
\usepackage[usenames,dvipsnames]{xcolor}
\usepackage[german]{babel}
\usepackage[utf8]{inputenc}
\usepackage[T1]{fontenc}
\usepackage{textcomp}
%\usepackage{libertine}
%\usepackage{helvet}
\usepackage{lscape}

\usepackage{microtype}
% Minion and Myriad fonts
\usepackage[minionint,mathlf]{MinionPro}
\renewcommand{\sfdefault}{Myriad-LF}
\usepackage{siunitx}
\usepackage{fancyhdr}
\usepackage{sectsty}
\usepackage{setspace}
\usepackage{booktabs} % To thicken table lines
\usepackage[version=4]{mhchem}
\usepackage[draft]{graphicx}
\usepackage[labelfont=bf]{caption}
\usepackage{subcaption}
\usepackage{chemstyle}
\usepackage{tabularx}
\usepackage[framemethod=tikz]{mdframed}
\usepackage{pgfplots}
\usepackage{pgfplotstable}
\usepackage[backend=biber,sorting=none,autocite = superscript,natbib=true]{biblatex} \addbibresource{books.bib}
\usepackage{tikzorbital}
\usepackage{chemfig}

\pgfplotsset{compat=newest}
\usepgfplotslibrary{units}
\mdfdefinestyle{mystyle}{%
  innerleftmargin=10,
  innerrightmargin=10,
  outerlinewidth=3pt,
  topline=false,
  rightline=false,
  leftline=false,
  bottomline=false,
  skipabove=\topsep,
  skipbelow=\topsep
}
%\usepackage[compatibility=4.7,language=german]{chemmacros}
\renewcommand{\familydefault}{\sfdefault}
\sisetup{detect-all}
\usepackage{chngcntr}
\counterwithin{table}{section}
\counterwithin{figure}{section}

\usepackage{titlesec}

\titleformat*{\section}{\large\bfseries}
\titleformat*{\subsection}{\normalsize\bfseries}

\usepackage{geometry}
 \geometry{
 a4paper,
 left=20mm,
 top=30mm,
 right=20mm
 }

\begin{document}
\tableofcontents

\begin{onehalfspace}


\section{Berechnung der Hartree-Fock-Energie von Wasser}
Ziel dieser Aufgabe ist es, die Hartree-Fock-Energie von Wasser für verschiedene Geometrien und Methoden zu berechnen.
\subsection{Berechnung der Gesamtenergie einer Geometrie von Wasser}
Im ersten Teil soll die Gesamtenergie einer Geometrie von Wasser berechnet werden. Hierfür wird wie im Skript erläutert ein Wassermolekül erstellt und die Geometrie wie im Datenblatt angegeben angepasst. Als Basissatz wird cc-pVTZ verwendet. \\
\textbf{Auswertung 1} Fünf grundlegende Parameter einer quantenmechanischen Rechnung sind die Geometrie, die Ladung, die Multiplizität, die Basisparameter und die QC-Methode. \\
\textbf{Auswertung 2} Die sich ergebene Gesamtenergie beträgt -76.06099515 \si{\hartree}.
Die Punktgruppe der verwendeten Geometrie ist $C _s$ .\\
\subsection{Gesamtenergie in Abhängigkeit von der Geometrie}
In dem zweiten Teil soll die Abhängigkeit der Gesamtenergie von der Geometrie gezeigt werden, wofür dies Geometrie optimiert wird.\\
\textbf{Auswertung 1}:\\
\begin{figure}[!htbp]
\centering
\includegraphics[width=.9\textwidth,height=.9\textheight,keepaspectratio]{data/h2Omolekul_teil1.png}
  \caption{Optimierte Geometrie von Wasser (HF/cc-pVQZ)}
\end{figure}
\noindent
Die sich für die optimierte Geometrie ergebene Gesamtenergie beträgt -76.06551853 \si{\hartree}. \\
\begin{table}[htbp]
\centering
\caption{Die Partialladungen von Sauerstoffatom und Wasserstoffatom im Wassermolekül}
\begin{tabular}{lr}
\toprule
Atom &  Partialladungen (in e)\\
O & -0.512\\
H & 0.256\\
\midrule
\bottomrule

\end{tabular}
\end{table}
\noindent
Summiert man alle Partialladungen auf ergibt sich 0, was für ein neutral geladenes Molekül spricht.
Für das Dipolmoment ergibt sich ein Wert von + 1.948 Debye und für die Richtung des Dipols in negativer Z-Achsenrichtung (-z).
Der Wechsel der Punktgruppe von Aufgaben Teil 1 zu Teil 2 ist auf die Geometrieoptimierung in Teil 2 zurückzuführen.
 Es wurde mit einer unsymmetrischen Geometrie gestartet, die als Punktgruppe $C _s$ ausgab.
  Bei der Geometrieoptimierung in Teil 2 wurde die Symmetrie erhöht und es wurde die Punktgruppe $C _{2v}$ ausgegeben,
  da eine Spiegelebene entlang der $C _2$ Achse existiert.\\
\noindent
\textbf{Auswertung 2}\\
Als nächstes soll die Optimierung der Geometrie nachvollzogen werden, was durch die Auftragung der Hartree-Energie sichtbar wird.
\begin{figure}[!htp]
\centering
  \includegraphics[width=.9\textwidth,height=.9\textheight,keepaspectratio]{data/water.png}%
  \caption{Auftragung der Gesamtenergie gegen die Optimierungsschritte (RHF/cc-pVTZ)}
\end{figure}
Es ist deutlich zu erkennen, dass die Energie ab dem 2. Optimierungsschritt deutlich langsamer sinkt als zuvor. Diese langsamere Verbesserung hat zur Folge, dass sich der Kosten/Nutzen-Faktor bei den späteren Schritten verschlechtert.\\
\newpage
\noindent
\textbf{Auswertung 3}\\
Die Terme des Hamilton-Operators, welche von der Änderung der Geometrie betroffen sind, sind die der potentiellen Energien
$\hat{V} _{K,K}(\vec{R})$, $\hat{V} _{K,el}(\vec{r},\vec{R})$ und $\hat{V} _{el,el}(\vec{r})$.
Diese hängen vom Abstand des Elektrons bzw. des Kerns ab, den sie zueinander haben. Das kann am Beispiel von Helium gezeigt werden \supercite{atkins219}:\\
\begin{align*}
\hat{V} &= - \hat{V} _{K,el1}(\vec{r},\vec{R}) - \hat{V} _{K,el2}(\vec{r},\vec{R})  + \hat{V} _{el,el}(\vec{r}) \\
  &= - \frac{2e^2}{4\pi \varepsilon _0 r _1} - \frac{2e^2}{4\pi \varepsilon _0 r _2} +  \frac{e^2}{4\pi \varepsilon _0 r _{12}}
\end{align*}
$\hat{V}$ wird mit abnehmendem Kern-Elektronenabstand $\vec{r}$ negativer (stabilisierend) und mit abnehmendem Elektron-Elektronenabstand $\vec{r} _{12}$ positiver (destabilisierend).

\subsection{Gesamtenergie in Abhängigkeit von der quantenchemischen Methode}
In diesem Teil soll die Abhängigkeit der Gesamtenergie von der quantenmechanischen Methode gezeigt werden. Hierfür wird die Gesamtenergie des Wassermoleküls mit dem Basissatz cc-pVTZ berechnet, allerdings die genutzte Methode variiert. Die zu genutzten Methoden sind MP2, CCSD und CCSD(T).
\newpage
\noindent
\textbf{Auswertung 1}\\
\begin{table}[!htpb]
\centering
\caption{Vergleich der Basissätze}
\begin{tabular}{lrr}
\toprule
 &
STO-3G &
cc-pVDZ \\
\midrule
Punktgruppe & $C _{2v}$ & $C _{2v}$\\
Energie \si{\hartree}    & -74.95896952 & -76.02690370  \\
$S _O / \textit{e}$ & -0.386 & -0.287  \\
$S _H / \textit{e}$ & +0.193 & +0.144  \\
$\nu / \textit{Debye}$ & 1.7368 & 2.0130  \\
\bottomrule
\end{tabular}
\end{table}\\
\textbf{Auswertung 2}\\
Anhand der in der Tabelle aufgelisteten Werte ist zu erkennen,
 dass im Vergleich mit den zuvor berechneten Werten, der cc-pVDZ Basissatz eine niedrigere Gesamtenergie ergibt als der STO-3G.
 Ebenso verhält es sich mit dem Dipolmoment ($\mu$ [Debye]). Bei den Partialladungen weisen beide Basissätze große Unterschiede
 nicht nur zueinander, sondern auch zu der optimierten Geometrie aus Teil 2 der ersten Aufgabe auf.
\subsection{Gesamtenergie in Abhängigkeit von der quantenchemischen Methode}
In diesem Teil soll die Abhängigkeit der Gesamtenergie von der quantenmechanischen Methode gezeigt werden.
Hierfür wird die Gesamtenergie des Wassermoleküls mit dem Basissatz cc-pVTZ berechnet, allerdings die genutzte Methode variiert.
Die zu genutzten Methoden sind MP2, CCSD und CCSD(T).\\
\textbf{Auswertung 1}\\

\begin{table}[!htpb]
\centering
\caption{ Vergleich der QM-Methoden}
\begin{tabular}{ll}
\toprule
Methode &   Energie \si{\hartree} \\
\midrule
HF/STO-3G & -74.95896952 \\
HF/CC-pVDZ& -76.0269037\\
HF/CC-pVQZ& -76.06099515\\
MP2 & -76.34694522\\
CCSD & -76.35028785 \\
CCSD(T) & -76.35909660\\
\bottomrule
\end{tabular}
\end{table}
\noindent
Es ist zu erkennen, dass die sich für die Gesamtenergie errechneten Werte
relativ nah beieinander liegen,
jedoch in den Nachkommastellen voneinander abweichen.
Die CCSD(T) Methode ist die Komplexeste, welche zugleich die niedrigste der drei Energien aufweist.

\section{Geometrieoptimierung und Frequenzberechnung an Dichlorethen}
In dieser Aufgabe wird die Bedeutung der Geometrieoptimierung näher untermauert.
Dafür werden die Schwingungsfrequenzen von \textit{cis-} und \textit{trans}-1,2-Dichlorethen auf HF-Niveau und MP2-Niveau berechnet.
Die angewandten quantumchemischen Methoden werden im Zusammenhang von Elektronenkorrelation verglichen.\\
\textbf{Auswertung 1 }
\begin{figure}[!hptb]
    \caption{Die optimierten Geometrien}
    \begin{subfigure}[b]{0.4\textwidth}
       \fbox{  \includegraphics[width=\textwidth]{data/cis_darstellung.png}}
                \subcaption{\textit{cis}-1,2-Dichlorethen }
    \end{subfigure}
    ~ %add desired spacing between images, e. g. ~, \quad, \qquad, \hfill etc.
      %(or a blank line to force the subfigure onto a new line)
    \begin{subfigure}[b]{0.4\textwidth}
       \fbox{  \includegraphics[width=\textwidth]{data/trans_darstellung.png}}
        \subcaption{ \textit{trans}-1,2-Dichlorethen }
    \end{subfigure}
    \label{figure:opt}
\end{figure}


\begin{table}[!htpb]

\caption{Gesamtenergien und Symmetrien von \textit{cis-} und \textit{trans}-1,2-Dichlorethen}
\begin{tabular}{llllc}
\toprule
Molekül & Methode &   Basissatz & Ges. En. \si{\hartree} & Symm. \\
\midrule
cis-1,2-Dichlorethen   & RHF& 6-311G(d,p)& -995.89694703 &$C_ {2v}$  \\
trans-1,2-Dichlorethen & RHF& 6-311G(d,p)& -995.89737964 &$C_ {2h}$ \\
cis-1,2-Dichlorethen   & MP2& 6-311G(d,p)& -996.45258595  &$C_ {2v}$  \\
trans-1,2-Dichlorethen & MP2& 6-311G(d,p)& -996.45188 &$C_ {2h}$    \\
\bottomrule
\label{table:energie}
\end{tabular}
\end{table}
 %\num{4.32e-4}
 %\num{7.05e-4}





\begin{table}[!htpb]

\caption{theoretische und experimentelle Schwingungsfrequenzen mit irreduziblen Darstellungen für \textit{cis}-1,2-Dichlorethen}
\begin{tabular}{llllll}
\multicolumn{2}{c}{RHF/6-311G(d,p)}&\multicolumn{2}{c}{MP2/6-311G(d,p)}&\multicolumn{2}{c}{Experimentell\supercite{cisvib}} \\
\midrule
Irrep &  $\tilde{\nu}$ \si{\per\centi\meter} & Irrep &   $\tilde{\nu}$ \si{\per\centi\meter} & Irrep &  $\tilde{\nu}$\si{\per\centi\meter} \\
\midrule
$A _1$ & 183  &$A _1$  & 173    & $A_1$&   173 \\
$A _2$ & 458  &$A _2$  & 411    & $A_2$&    406 \\
$B _2$ & 614  &$B _2$  & 584    & $B_1$&     571  \\
$A _1$ & 759  &$A _1$  & 714    & $B_2$& 697  \\
$B _1$ & 803  &$B _1$  & 747    & $A_1$& 711  \\
$B _2$ & 920  &$B _2$  & 888    & $B_1$& 857 \\
$A _2$ & 1051 &$A _2$  & 897    & $A_2$& 876 \\
$A _1$ & 1324 &$A _1$  & 1238   & $A _1$& 1179\\
$B _2$ & 1439 &$B _2$  & 1341   & $B _1$ &1303 \\
$A _1$ & 1818 &$A _1$  & 1648   & $A _1$ &1587 \\
$B _2$ & 3373 &$B _2$  & 3249   & $B _1$ & 3072\\
$A _1$ & 3397 &$A _1$  & 3270   & $A _1$ & 3077\\
\bottomrule
\end{tabular}
\label{tab:cisvergleich}

\end{table}





\begin{table}[!htpb]

\caption{Schwingungsintensitäten mit irreduziblen Darstellungen und Schwingungsmodi für
\textit{cis}-1,2-Dichlorethen}
\begin{tabular}{llll}
\midrule
Irrep & IR-Int(RHF)/rel & Raman-Int(RHF)/rel & Schwingungsmodus~\supercite{cisvib} \\
\midrule
$A _1$ & 0.41 & 2.160   & CCCl Deformationss.\\
$A _2$ & 0 & 4.86       & Torsion\\
$B _2$ & 7.93 & 6.74    &  CCCl Deformationss.\\
$A _1$ & 27.36 & 12.33  & C-H Deformationss.\\
$B _1$ & 73.62 S & 1.13   & C-Cl Strecks.\\
$B _2$ & 106.94 & 0.01  & C-Cl Strecks.\\
$A _2$ & 0 & 4.54       & C-H Deformationss.\\
$A _1$ & 0.14 & 20.88   & C-H Deformationss.\\
$B _2$ & 36.02 & 1.05   & C-H Deformationss.\\
$A _1$ & 34.83 & 59.81  & C-C Strecks.\\
$B _2$ & 15.14 & 53.92  & C-H Strecks.\\
$A _1$ & 2.93 & 141.85  & C-H Strecks.\\
\bottomrule
\end{tabular}
\label{tab:cisschwings}

\end{table}



\begin{table}[!htpb]

\caption{theoretische und experimentelle Schwingungsfrequenzen mit irreduziblen Darstellungen für \textit{trans}-1,2-Dichlorethen}
\begin{tabular}{llllll}
\multicolumn{2}{c}{RHF/6-311G(d,p)}&\multicolumn{2}{c}{MP2/6-311G(d,p)}&\multicolumn{2}{c}{Experimentell}~\supercite{transvib} \\
\midrule
Irrep &  $\tilde{\nu}$ \si{\per\centi\meter} & Irrep &   $\tilde{\nu}$ \si{\per\centi\meter} & Irrep &  $\tilde{\nu}$\si{\per\centi\meter} \\
\midrule
$A _u$ & 233  & $A _u$ & 215  & $A_u$ & 227\\
$B _u$ & 257  & $B _u$ & 243  & $B_u$ & 250\\
$A _g$ & 376  & $A _g$ & 360  & $A_g$ & 350\\
$B _u$ & 875  & $B _g$ & 753  & $B_g$ & 763\\
$B _g$ & 912  & $B _u$ & 869  & $B_u$ & 828\\
$A _g$ & 920  & $A _g$ & 892  & $A_g$ & 846\\
$A _u$ & 1049 & $A _u$ & 939  & $A_u$ & 900\\
$B _u$ & 1333 & $B _u$ & 1258 & $B_u$ & 1200\\
$A _g$ & 1417 & $A _g$ & 1325 & $A_g$ & 1274\\
$A _g$ & 1815 & $A _u$ & 1645 & $A_g$ & 1578\\
$B _u$ & 3393 & $B _u$ & 3263 & $A_g$ & 3073\\
$A _g$ & 3400 & $A _g$ & 3267 & $B_u$ & 3090\\
\bottomrule
\end{tabular}
\label{tab:transvergleich}

\end{table}


\begin{table}[!htpb]
\caption{Schwingungsintensitäten mit irreduziblen Darstellungen und Schwingungsmodi für \textit{trans}-1,2-Dichlorethen }
\begin{tabular}{llll}
\midrule
Irrep & IR-Int(RHF)/rel & Raman-Int(RHF)/rel & Schwingungsmodus~\supercite{transvib}   \\
\midrule
$A _u$ & 0.73 w& 0 & Torsion\\
$B _u$ & 4.58 w& 0 & CCCl Deformationss.\\
$A _g$ & 0.0 & 9.82 w&CCCl Deformationss.\\
$B _u$ & 156.36 s& 0 & C-H Deformationss.  \\
$B _g$ & 0 & 6.61 w & C-Cl Strecks.\\
$A _g$ & 0 & 9.56 w & C-Cl Strecks.\\
$A _u$  & 82.03 m & 0 & C-H Deformation \\
$B _u$ & 27.08 w& 0 & C-H Deformation\\
$A _g$ & 0 & 24.55 w& C-H Deformation\\
$A _g$ & 0 & 49.17 m& C-C Strecks.\\
$B _u$ & 17.72 w& 0 & C-H Streck. \\
$A _g$ & 0 & 115.68 s& C-H Strecks.\\
\bottomrule
\end{tabular}
\end{table}

\textbf{Auswertung 2}

\begin{figure}[!hptb]
    \centering
    \begin{subfigure}[b]{0.4\textwidth}
         \fbox{   \includegraphics[width=\textwidth,height=5cm]{data/cis_ir_spektrum.png}}
        \subcaption{auf RHF/6-311G(d,p) Niveau}
    \end{subfigure}
    ~ %add desired spacing between images, e. g. ~, \quad, \qquad, \hfill etc.
      %(or a blank line to force the subfigure onto a new line)
    \begin{subfigure}[b]{0.4\textwidth}
         \fbox{   \includegraphics[width=\textwidth,height=5cm]{data/cis_ir_mp2.png}}
        \subcaption{auf MP2/6-311G(d,p) Niveau}
    \end{subfigure}
    \caption{Die berechneten IR-Spektren von \textit{cis}-1,2-Dichlorethen}
    \label{figure:vergleichmethodec}
\end{figure}

\begin{figure}[!hptb]

\begin{tikzpicture}[font=\sffamily]
\begin{axis}[xlabel=Wellenzahl, ylabel=Intensität,  enlargelimits=true,axis x line=middle,
    axis y line=middle,
   y label style={at={(axis description cs:-0.15,.5)},anchor=south,rotate=90},
    x tick label style={rotate=90,anchor=east},
    x label style={at={(axis description cs:0.5,-.3)},anchor=south},
     xmin=0
    ]
\addplot[blue] table[x=frequency ,y=intensity] {data/cisex.txt};
\end{axis}
\end{tikzpicture}
    \caption{Das experimentelle IR-Spektrum von \textit{cis}-1,2-Dichlorethen \supercite{cisir} }
\label{figure:vergleichcis}
\end{figure}

\begin{figure}[!hptb]
    \centering
    \begin{subfigure}[b]{0.4\textwidth}
         \fbox{   \includegraphics[width=\textwidth,height=5cm]{data/trans_ir_spektrum.png}}
        \subcaption{auf RHF/6-311G(d,p) Niveau}
    \end{subfigure}
    ~ %add desired spacing between images, e. g. ~, \quad, \qquad, \hfill etc.
      %(or a blank line to force the subfigure onto a new line)
    \begin{subfigure}[b]{0.4\textwidth}
         \fbox{   \includegraphics[width=\textwidth,height=5cm]{data/trans_ir_mp2.png}}
                \subcaption{auf MP2/6-311G(d,p) Niveau}
    \end{subfigure}
        \caption{Die berechneten IR-Spektren von \textit{trans}-1,2-Dichlorethen}
\label{figure:vergleichmethodet}

\end{figure}

\begin{figure}[!hptb]

\begin{tikzpicture}[font=\sffamily]
\begin{axis}[xlabel=Wellenzahl, ylabel=Intensität,  enlargelimits=true,axis x line=middle,
    axis y line=middle,
   y label style={at={(axis description cs:-0.15,.5)},anchor=south,rotate=90},
    x tick label style={rotate=90,anchor=east},
    x label style={at={(axis description cs:0.5,-.3)},anchor=south},
     xmin=0
    ]
\addplot[blue] table[x=frequency ,y=intensity] {data/trans.txt};
\end{axis}
\end{tikzpicture}
    \caption{Das experimentelle IR-Spektrum von \textit{trans}-1,2-Dichlorethen~\supercite{transir}}
\label{figure:vergleichtrans}
\end{figure}


\begin{figure}[!hptb]
    \centering
    \begin{subfigure}[b]{0.4\textwidth}
          \fbox{  \includegraphics[width=\textwidth,height=5cm]{data/cis_raman_spektrum.png}}
        \subcaption{ \textit{cis}-1,2-Dichlorethen }
    \end{subfigure}
    ~ %add desired spacing between images, e. g. ~, \quad, \qquad, \hfill etc.
      %(or a blank line to force the subfigure onto a new line)
    \begin{subfigure}[b]{0.4\textwidth}
         \fbox{   \includegraphics[width=\textwidth,height=5cm]{data/trans_raman_spektrum.png}}
        \subcaption{\textit{trans}-1,2-Dichlorethen}
    \end{subfigure}
            \caption{Raman-Spektren auf RHF/6-311G(d,p) Niveau}
\end{figure}

.\\
\textbf{Auswertung 3}\\
Die MP2-Methode ist für die Berechnung von Schwingungsfrequenzen besser geeignet als die RHF-Methode,
da die Spektren der MP2-Methode näher an den experimentellen Spektren sind.
Dies liegt daran, dass die RHF-Methode die Elektronenkorrelation (die Wechselwirkung der Elektronen) vernachlässigt,
die MP2 Methode diese jedoch berücksichtigt.\\
\textbf{Auswertung 4}\\
Mit Elektronenkorrelation ist gemeint, dass die Coulombabstoßung $\dfrac{e^2}{\epsilon r_{12}}$ zwischen den beiden Elektronen einer Bindung dafür sorgt, dass sich
die Elektronen nicht zu nahe kommen. Ist das eine Elektron momentan in der Umgebung des einen Kerns,
so ist die Wahrscheinlichkeit gross, dass sich das andere Elektron in der Umgebung des anderen Kerns aufhält. Die Elektronen bewegen sich derart \glqq
korreliert\grqq, dass sie sich nicht nahe kommen. Die Elektronenkorrelation wirkt sich auf die Geometrie bzw. den
Energiezustand einer Geometrie aus.\supercite{coulson}  \\
\textbf{Auswertung 5}\\
Beim \textit{trans}-1,2-Dichlorethen fällt auf, dass es entweder nur ramanaktiv oder infrarotaktiv ist.
Das \textit{cis}-1,2-Dichlorethen hingegen
 ist bei beiden aktiv. Dies liegt daran, dass das \textit{trans}-1,2-Dichlorethen ein
  Inversionszentrum hat und das \textit{cis}-1,2-Dichlorethen nicht.
Das Inversionszentrum verbietet bei einer symmetrischen Schwingung die
Infrarotaktivität und bei einer asymmetrischen Schwingung die Ramanaktivität.\\
\textbf{Auswertung 6}\\

Anhand der Werte für die Hartree-Energien aus ~\ref{table:energie} lässt sich sagen, dass
bei der Verwendung der RHF Methode für das \textit{trans}-1,2-Dichlorethen eine
kleinere Hartree-Energie ermittelt wurde als für das cis-1,2-Dichlorethen.
Dementsprechend ist unter Verwendung der RHF Methode das cis-Isomer der
Verbindung die stabilere. Betrachtet man jedoch die Werte, die anhand der MP2
Methode berechnet wurden, so lässt sich feststellen, dass hierbei das \textit{cis}-Isomer
eine kleinere Hartree-Energie aufweist als das trans-Isomer. Der Grund dafür ist, dass die MP2-Methode die destabilisierende Elektronenabstoßung
 zwischen Elektronen der vicinalen Cl-Atomen von \textit{cis}-1,2-Dichlorethen in Betracht zieht.

\section{Potentialkurve von \ce{N_2}}
Ziel dieser Aufgabe ist es, die Potentialkurve des Stickstoffmoleküls zu berechnen und anschließend die Orbitale
 visualisiert und ein MO-Diagramm erstellt werden.
\subsection{Berechnung der Potenzialkurve auf Hartree-Fock Niveau}
\textbf{Auswertung 1 und 2}
\begin{table}[!htpb]
\centering
\caption{}
\begin{tabularx}{\textwidth}{lllclll}
\toprule
Molekül &
Methode &
Basissatz &
\parbox[t]{2cm}{Gitter\\ \scriptsize{(Min, Max, Incr)}} &
Symmetrie &
GGW-Abstand &
GGW-Energie $E _h$ \\
\midrule
\ce{N _2} & RHF & 6-311G(d,p) & 0.6, 4.6, 0.2 \si{\angstrom}& $D _{\infty h}$ & 1 \si{\angstrom} & -108.95140449 \\
\bottomrule
\end{tabularx}
\end{table}
\begin{figure}[!htpb]
\centering
  \fbox{\includegraphics[width=0.5\textwidth]{data/morse.png}}
  \caption{Potenzialkurve für Stickstoff auf RHF/6-311G(d,p)-Niveau}
\end{figure}
Das Morse Potential beschreibt den Verlauf des elektronischen Potentials eines zweiatomigen Moleküls
in Abhängigkeit vom Kern­bindungsabstand $R$ durch eine exponentielle Näherung:
\begin{equation}
 E(r) = D_e (1-e^\alpha{r-r_e})^2
\end{equation}
 Darin ist $r$ der Kernabstand der beiden betrachteten Atome, $r_e$ der Kernabstand bei der
geringsten potentiellen Energie $D_e$. $D_e$ ist die minimale Energie, auch spektroskopische
Dissoziationsenergie genannt, $\alpha$ ist eine spezifische Stoffkonstante, abhängig von den
betrachteten Atomen.
Es ist an der ~\ref{figure:morse} erkennen, dass der Verlauf des berechneten Potentials
 den exponentiellen Morse-Verlauf in guter Näherung wiedergibt.


\noindent
\textbf{Auswertung 3}\\
Die  Bindungsspaltung war eine heterolytische Bindungsspaltung. wird
der Input der Rechnung betrachtet, so weist die Vorgabe „charge: 0, Spin:
Singlet“ darauf hin.

\textbf{Auswertung 4 }\\

\begin{table}[!htpb]
\begin{tabular}{c|ccc}
 \large Homolytisch & &\multicolumn{2}{c}{\large Heterolytisch}\\
 & &\\
 \ce{\Lewis{0.2.4:6.,N}} & & \ce{N+} & \ce{N-}\\
  & &\\
\begin{tikzpicture}
%\draw [->,ultra thick] (-1,-2) --(-1,4) node[above] { Energie};
\drawLevel[elec = updown,pos = {(0,0)},    width = 1]{d1};
\drawLevel[elec = updown,pos = {(0,1.3)},  width = 1]{};
\drawLevel[elec = up,pos = {(0,2.6)},  width = 1]{};
\drawLevel[elec = up,pos = {(1.3,2.6)},  width = 1]{};
\drawLevel[elec = up,pos = {(2.6,2.6)},  width = 1]{};
\node[right] at (right d1) { Quartett} ;
\end{tikzpicture}
& &
\begin{tikzpicture}

%\draw [->,ultra thick] (-1,-2) --(-1,4) node[above] { Energie};
\drawLevel[elec = updown,pos = {(0,0)},    width = 1]{d1};
\drawLevel[elec = updown,pos = {(0,1.3)},  width = 1]{};
\drawLevel[elec = up,pos = {(0,2.6)},  width = 1]{};
\drawLevel[elec = up,pos = {(1.3,2.6)},  width = 1]{};
\drawLevel[pos = {(2.6,2.6)},  width = 1]{};
\node[right] at (right d1) { Triplett} ;
\end{tikzpicture}
&
\begin{tikzpicture}
%\draw [->,ultra thick] (-1,-2) --(-1,4) node[above] { Energie};
\drawLevel[elec = updown,pos = {(0,0)},    width = 1]{d1};
\drawLevel[elec = updown,pos = {(0,1.3)},  width = 1]{};
\drawLevel[elec = updown,pos = {(0,2.6)},  width = 1]{};
\drawLevel[elec = up,pos = {(1.3,2.6)},  width = 1]{};
\drawLevel[elec = up,pos = {(2.6,2.6)},  width = 1]{};
\node[right] at (right d1) { Triplett} ;
\end{tikzpicture}\\
&&&\\
&&&\\
\begin{tikzpicture}
%\draw [->,ultra thick] (-1,-2) --(-1,4) node[above] { Energie};
\drawLevel[elec = updown,pos = {(0,0)},    width = 1]{d1};
\drawLevel[elec = updown,pos = {(0,1.3)},  width = 1]{};
\drawLevel[elec = updown,pos = {(0,2.6)},  width = 1]{};
\drawLevel[elec = up,pos = {(1.3,2.6)},  width = 1]{};
\drawLevel[pos = {(2.6,2.6)},  width = 1]{};
\node[right] at (right d1) { Dublett} ;
\end{tikzpicture}
&&
\begin{tikzpicture}
%\draw [->,ultra thick] (-1,-2) --(-1,4) node[above] { Energie};
\drawLevel[elec = updown,pos = {(0,0)},    width = 1]{d1};
\drawLevel[elec = updown,pos = {(0,1.3)},  width = 1]{};
\drawLevel[elec = updown,pos = {(0,2.6)},  width = 1]{};
\drawLevel[pos = {(1.3,2.6)},  width = 1]{};
\drawLevel[pos = {(2.6,2.6)},  width = 1]{};
\node[right] at (right d1) { Singlet} ;
\end{tikzpicture}
&
\begin{tikzpicture}
%\draw [->,ultra thick] (-1,-2) --(-1,4) node[above] { Energie};
\drawLevel[elec = updown,pos = {(0,0)},    width = 1]{d1};
\drawLevel[elec = updown,pos = {(0,1.3)},  width = 1]{};
\drawLevel[elec = updown,pos = {(0,2.6)},  width = 1]{};
\drawLevel[elec = updown,pos = {(1.3,2.6)},  width = 1]{};
\drawLevel[pos = {(2.6,2.6)},  width = 1]{};
\node[right] at (right d1) { Singlet} ;
\end{tikzpicture}\\

\end{tabular}

\caption{Die Elektronenbesetzungschemen der Stickstoff-Fragmenten von möglichen Dissoziationsarten \supercite{holleman}}
\label{table:besetzung}
\end{table}

Die anderen Möglichkeiten zur Dissoziation und der Multiplizitäten sind oben in
der ~\ref{table:besetzung} dargestellt.
Bei einer heterolytischen Dissoziation erhält man ein
Stickstoffkation und ein Stickstoffanion, die Singlet- und Dublett-Zustand aufweisen.
Möglich ist eine homolytische Bindungsspaltung,
bei der man zwei Stickstoffradikale erhält. Die dabei möglichen Multiplizitäten sind
in der ~\ref{table:besetzung} zu sehen; es ist der Quartett- und der Dublett-Zustand
möglich.\\
\textbf{Auswertung 5}
Nach der Hundschen Regel die Besetzung entarteter Orbitale so erfolgt, dass die größtmögliche Zahl
 ungepaarter Elektronen erreicht wird (maximale Spinmultiplizität).
 Solche Zustände sind energetisch stabiler als Zustände, bei denen die entarteten Orbitale nicht so besetzt werden,
dass die Anzahl der ungepaarten Elektronen maximal wird. Demnach lautet die energetische Reihenfolge der
Dissoziationskanäle wie folgt (nach aufsteigender Energie):\\
 Quartett < Dublett < Triplett < Singlet.\\
Daraus ergibt es sich dass es eine homolytische Dissoziation
  energetisch günstiger als eine heterolytische Dissoziation.



\noindent
\textbf{Auswertung 3}\\
Die  Bindungsspaltung war eine heterolytische Bindungsspaltung. wird
der Input der Rechnung betrachtet, so weist die Vorgabe „charge: 0, Spin:
Singlet“ darauf hin.

\textbf{Auswertung 4 }\\

\begin{table}[!htpb]
\begin{tabular}{c|ccc}
 \large Homolytisch & &\multicolumn{2}{c}{\large Heterolytisch}\\
 & &\\
 \ce{\Lewis{0.2.4:6.,N}} & & \ce{N+} & \ce{N-}\\
  & &\\
\begin{tikzpicture}
%\draw [->,ultra thick] (-1,-2) --(-1,4) node[above] { Energie};
\drawLevel[elec = updown,pos = {(0,0)},    width = 1]{d1};
\drawLevel[elec = updown,pos = {(0,1.3)},  width = 1]{};
\drawLevel[elec = up,pos = {(0,2.6)},  width = 1]{};
\drawLevel[elec = up,pos = {(1.3,2.6)},  width = 1]{};
\drawLevel[elec = up,pos = {(2.6,2.6)},  width = 1]{};
\node[right] at (right d1) { Quartett} ;
\end{tikzpicture}
& &
\begin{tikzpicture}

%\draw [->,ultra thick] (-1,-2) --(-1,4) node[above] { Energie};
\drawLevel[elec = updown,pos = {(0,0)},    width = 1]{d1};
\drawLevel[elec = updown,pos = {(0,1.3)},  width = 1]{};
\drawLevel[elec = up,pos = {(0,2.6)},  width = 1]{};
\drawLevel[elec = up,pos = {(1.3,2.6)},  width = 1]{};
\drawLevel[pos = {(2.6,2.6)},  width = 1]{};
\node[right] at (right d1) { Triplett} ;
\end{tikzpicture}
&
\begin{tikzpicture}
%\draw [->,ultra thick] (-1,-2) --(-1,4) node[above] { Energie};
\drawLevel[elec = updown,pos = {(0,0)},    width = 1]{d1};
\drawLevel[elec = updown,pos = {(0,1.3)},  width = 1]{};
\drawLevel[elec = updown,pos = {(0,2.6)},  width = 1]{};
\drawLevel[elec = up,pos = {(1.3,2.6)},  width = 1]{};
\drawLevel[elec = up,pos = {(2.6,2.6)},  width = 1]{};
\node[right] at (right d1) { Triplett} ;
\end{tikzpicture}\\
&&&\\
&&&\\
\begin{tikzpicture}
%\draw [->,ultra thick] (-1,-2) --(-1,4) node[above] { Energie};
\drawLevel[elec = updown,pos = {(0,0)},    width = 1]{d1};
\drawLevel[elec = updown,pos = {(0,1.3)},  width = 1]{};
\drawLevel[elec = updown,pos = {(0,2.6)},  width = 1]{};
\drawLevel[elec = up,pos = {(1.3,2.6)},  width = 1]{};
\drawLevel[pos = {(2.6,2.6)},  width = 1]{};
\node[right] at (right d1) { Dublett} ;
\end{tikzpicture}
&&
\begin{tikzpicture}
%\draw [->,ultra thick] (-1,-2) --(-1,4) node[above] { Energie};
\drawLevel[elec = updown,pos = {(0,0)},    width = 1]{d1};
\drawLevel[elec = updown,pos = {(0,1.3)},  width = 1]{};
\drawLevel[elec = updown,pos = {(0,2.6)},  width = 1]{};
\drawLevel[pos = {(1.3,2.6)},  width = 1]{};
\drawLevel[pos = {(2.6,2.6)},  width = 1]{};
\node[right] at (right d1) { Singlet} ;
\end{tikzpicture}
&
\begin{tikzpicture}
%\draw [->,ultra thick] (-1,-2) --(-1,4) node[above] { Energie};
\drawLevel[elec = updown,pos = {(0,0)},    width = 1]{d1};
\drawLevel[elec = updown,pos = {(0,1.3)},  width = 1]{};
\drawLevel[elec = updown,pos = {(0,2.6)},  width = 1]{};
\drawLevel[elec = updown,pos = {(1.3,2.6)},  width = 1]{};
\drawLevel[pos = {(2.6,2.6)},  width = 1]{};
\node[right] at (right d1) { Singlet} ;
\end{tikzpicture}\\

\end{tabular}

\caption{Die Elektronenbesetzungschemen der Stickstoff-Fragmenten von möglichen Dissoziationsarten \supercite{wiberg98}}
\label{table:besetzung}
\end{table}

Die anderen Möglichkeiten zur Dissoziation und der Multiplizitäten sind oben in
der ~\ref{table:besetzung} dargestellt.
Bei einer heterolytischen Dissoziation erhält man ein
Stickstoffkation und ein Stickstoffanion, die Singlet- und Dublett-Zustand aufweisen.
Möglich ist eine homolytische Bindungsspaltung,
bei der man zwei Stickstoffradikale erhält. Die dabei möglichen Multiplizitäten sind
in der ~\ref{table:besetzung} zu sehen; es ist der Quartett- und der Dublett-Zustand
möglich.\\
\textbf{Auswertung 5}
Nach der Hundschen Regel die Besetzung entarteter Orbitale so erfolgt, dass die größtmögliche Zahl
 ungepaarter Elektronen erreicht wird (maximale Spinmultiplizität).
 Solche Zustände sind energetisch stabiler als Zustände, bei denen die entarteten Orbitale nicht so besetzt werden,
dass die Anzahl der ungepaarten Elektronen maximal wird. Demnach lautet die energetische Reihenfolge der
Dissoziationskanäle wie folgt (nach aufsteigender Energie):\\
 Quartett < Dublett < Triplett < Singlet.\\
Daraus ergibt es sich dass es eine homolytische Dissoziation
  energetisch günstiger als eine heterolytische Dissoziation.

\subsection{Geometrieoptimierung des Stickstoffmoleküls}
Mit der gleichen Methode (RHF) und dem gleichen Basissatz 6-311G(d,p)
wurde die Geometrie des Stickstoffmoleküls optimiert.
\textbf{Auswertung 1 und 2}
\begin{table}[!htpb]
\caption{}
\begin{tabular}{lllll}
\toprule
Molekül &
Methode &
Basissatz &
Bindungslänge \si{\angstrom} &
Gesamtenergie \si{\hartree}\\
\midrule
\ce{N _2} & RHF & 6-311G(d,p) & 1.07027 \si{\angstrom} & -108.98655 \\
\bottomrule
\end{tabular}
\end{table}

Im Vergleich zur Gesamtenergie aus dem ersten Aufgabenteil ist die Energie der optimierten Geometrie geringfügig niedriger.\\

\textbf{Auswertung 3}\\
\begin{table}[!htpb]
\centering
\caption{Der experimentelle Gasphasenabstand im \ce{N_2} Molekül }
\begin{tabular}{ccc}
\toprule
Abstand Teil 1 & Abstand Teil 2 (opt. Geometrie)  & experimenteller Abstand ~\supercite{holleman} \\
1.0 \si{\angstrom} & 1.07027 \si{\angstrom} & 1.0976 \si{\angstrom} \\
\midrule
\bottomrule
\end{tabular}
\end{table}\\
\noindent
\textbf{Auswertung 4}\\


Beim Vergleich der berechneten Orbitale mit denen aus der Literatur ~\supercite{ritaatom} fällt auf,
 dass die Orbitale von Stickstoffmoleküls in richtiger Reihenfolge aufgeführt sind. Es liegt daran

\begin{figure}[!hptb]

        \fbox{\includegraphics[width=\textwidth,height=8cm,keepaspectratio]{data/mohf.png}}

\caption{MO-Diagramm für ein Stickstoff-Molekül auf RHF/6-311G(d,p) und aus der Literatur ~\supercite{ritaatom}}

\end{figure}





\textbf{Auswertung 5 und 6} \\
Im Folgenden werden die Orbitale als
Abbildungen dargestellt und mit Hauptquantenzahl, Drehimpuls und Symmetrie
gekennzeichnet.
\begin{figure}[!hptb]
    \centering
    \begin{subfigure}[b]{0.4\textwidth}
        \fbox{\includegraphics[width=0.4\textwidth]{data/orbitale/1.png}}
      \subcaption*{$1 \sigma _g$  }
    \end{subfigure}
    \begin{subfigure}[b]{0.4\textwidth}
        \fbox{\includegraphics[width=0.4\textwidth]{data/orbitale/2.png}}
         \subcaption*{$1 \sigma _u$ }
    \end{subfigure}

\end{figure}

\begin{figure}[!hptb]
    \centering
    \begin{subfigure}[b]{0.4\textwidth}
        \fbox{\includegraphics[width=0.4\textwidth]{data/orbitale/3.png}}
        \subcaption*{$2 \sigma _g$ }
    \end{subfigure}
    ~ %add desired spacing between images, e. g. ~, \quad, \qquad, \hfill etc.
      %(or a blank line to force the subfigure onto a new line)
    \begin{subfigure}[b]{0.4\textwidth}
       \fbox{ \includegraphics[width=0.4\textwidth]{data/orbitale/4.png}}
     \subcaption*{$2 \sigma _u$}
    \end{subfigure}

\end{figure}

\begin{figure}[!hptb]
    \centering
    \begin{subfigure}[b]{0.4\textwidth}
        \fbox{\includegraphics[width=0.4\textwidth]{data/orbitale/5.png}}
        \subcaption*{$1 \pi _u^x$  }
    \end{subfigure}
    ~ %add desired spacing between images, e. g. ~, \quad, \qquad, \hfill etc.
      %(or a blank line to force the subfigure onto a new line)
    \begin{subfigure}[b]{0.4\textwidth}
        \fbox{\includegraphics[width=0.4\textwidth]{data/orbitale/6.png}}
        \subcaption*{$1 \pi _u^y$}
    \end{subfigure}
\label{figure:orbitalen2}
\end{figure}
\begin{figure}[!hptb]
    \centering
    \begin{subfigure}[b]{0.4\textwidth}
        \fbox{\includegraphics[width=0.4\textwidth]{data/orbitale/7.png}}
                \subcaption*{$3 \sigma _g$}

    \end{subfigure}
    ~ %add desired spacing between images, e. g. ~, \quad, \qquad, \hfill etc.
      %(or a blank line to force the subfigure onto a new line)
    \begin{subfigure}[b]{0.4\textwidth}
        \fbox{\includegraphics[width=0.4\textwidth]{data/orbitale/8.png}}
               \subcaption*{$1 \pi _g$}

    \end{subfigure}
\end{figure}

\begin{figure}[!hptb]
    \centering
    \begin{subfigure}[b]{0.4\textwidth}
        \fbox{\includegraphics[width=0.4\textwidth]{data/orbitale/9.png}}
               \subcaption*{$1 \pi _g$}
    \end{subfigure}
    ~ %add desired spacing between images, e. g. ~, \quad, \qquad, \hfill etc.
      %(or a blank line to force the subfigure onto a new line)
    \begin{subfigure}[b]{0.4\textwidth}
        \fbox{\includegraphics[width=0.4\textwidth]{data/orbitale/10.png}}
          \subcaption*{ $3 \sigma _u$}
    \end{subfigure}

\end{figure}


\section{Aufgabe 5 S\textsubscript{N}2-Substitution des Br- durch Cl- an CH\textsubscript{3}Br}

Ziel dieser Aufgabe ist die Bestimmung des Übergangszustandes bei der
Substitution des Bromidions durch ein Chloridion an CH\textsubscript{3}Br,
sowie die Auswahl der Startgeometrie (die Geometrie am höchsten höchsten Punkt
des Energieprofiles, bei der schrittweisen annäherung des Chloridions an an
das Methylbromid) und die Berechnung der Aktivierungsbarriere und der Ausbeute
dieser Reaktion.


\subsection{Bestimmung des Reaktionspfades und Auswahl einer geeigneten Startgeometrie}



\textbf{Auswertung 1}:

\begin{figure}[!htbp]
\centering
  \includegraphics[width=0.8\textwidth]{data/a5_teil1_scan.png}%
  \caption{Optimierte Geometrie des Übergangszustandes}
\end{figure}
\noindent

 Die Gesamtenergie für die Geometrie beträgt -3071.56972781 \si{\hartree}.
 Der Basissatz ist aug-cc-pVDZ. Die Punktgruppe beträgt C\textsubscript{1}.

\textbf{Auswertung 2}\\

Nach der Optimierung kommt es zur änderung der Punktgruppe von
C\textsubscript{1} zu C\textsubscript{3V}. Die Gesamtenergie beträgt nun
-3071.56971543 \si{\hartree}. Die Bildungsenthalpie beträgt -3071.560004
\si{\hartree} Die optimierte Bindungslänge der C-Cl bindung beträgt 2.45
\si{\angstrom}

Der Zustand besitzt die imaginäre Schwingungsfrequenz -390.12
cm\textsuperscript{-1} Die Bewegung bei dieser Frequenz entspricht der
Walden'sche Umkehr.

\begin{figure}[!htbp]
\centering
  \includegraphics[width=0.5\textwidth,keepaspectratio]{data/A5_opt_darstellung.png}%
  \caption{Optimierte Geometrie des Übergangszustandes}
\end{figure}
\noindent

\subsection{Optimierung und Frequenzrechnung des Übergangszustandes}

\textbf{Auswertung 1}\\
\begin{table}[!htpb]
\centering
\begin{tabular}{ccc}
\toprule
Molekül & E/\textit{Hartree} & Bildungsenthalpie $\Delta G$  \textit{Hartree}\\
\ce{CH_3Br}  & -2612.00556924 & -2611.99026 \\
\ce{CH_3Cl}  & -499.12227340 & -499.105102\\
\ce{Br^-}  & -2572.46248971 & -2572.478665 \\
\ce{Cl^-}  & -459.56364460 & -459.578667 \\
\midrule
\bottomrule
\end{tabular}
\caption{Molekülbindungen und Energien}
\end{table}
Mit Hilfe der ermittelten Werte kann nun die freie Enthalpie wie folgt berechnet werden:
\begin{equation}
\Delta _R G = \sum\limits_{Produkte} \Delta _B G - \sum\limits_{Edukte} \Delta _B G
\end{equation}
Nach einsetzen der Werte erhält man $ \Delta _R G = -0.014834$ \si{\hartree}. Nach Umrechnung des Ergebnisses in  \si{\kilo\joule\per\mol},erhält man $ \Delta _R G = -38.95$ \si{\kilo\joule\per\mol}.
Die freie Aktivierungsenthalpie wird definiert als Energiedifferenz der Edukte und der raktiven Zwischenstufe. Und wird wie folgt berechnet:
\begin{equation}
\Delta _R G^{\neq} = \sum\limits_{Edukte} - \Delta _B G _U
\end{equation}
$\Delta _B G _U$ stellt hier die Energie der reaktiven Zwischenstufe dar und  beträgt $\Delta _B G _U = -3071.560004$ \si{\hartree} (Wert aus Aufgabe 5 Teil 1). $\Delta _R G^{\neq}$ steht für die freie Aktivierungsenthalpie. Nach der Berechnung und Umrechnung in SI-Einheiten ergibt sich für die freie Aktivierungsenthalpie: $\Delta _R G^{\neq} = 24.178$ \si{\kilo\joule\per\mol}.

\noindent
 \textbf{Auswertung 2:} Bei Betrachtung des Diagrammes ~\ref{figure:pfad} im
Skript auf S.51 mit der auftragung der Reaktionsenthalpie gegen die Ausbeute
für Reaktionen bei 25 °C bzw. 100 °C, lässst sich für den Berechneten Wert
eine Ausbeute von annähernd 100\% vorraus sagen. Aus dem niedrigen Wert für
die freie Aktivierungsenthalpie lässt sich schließen, dass die Reaktion sehr
schnell ablaufen wird, da der Graph der halbwertszeiten für die Reaktion 2.
Ordnung mit  Aktivierungsenthalpien unter $\Delta _R G^{\neq} = 90$
\si{\kilo\joule\per\mol} stark gegen Null geht. Zur Betrachtung eines Fehlers
von $\pm 10$ \si{\kilo\joule\per\mol} lässt sich sagen das sich dieser nur
gerfingfügig auf eine Vorhersage auswirkt, da es auch bei einem so hohen
Fehler nur zu unsignifikanten änderung der Reaktionszeit und der Ausbeute
kommen würde.
\subsection{Bestimmung des Reaktionspfades}
\textbf{Auswertung 1}
 \begin{figure}[!htpb]
 \centering
 \includegraphics[width=0.5\textwidth]{data/potentialkurve.png}%
\captionsetup{justification=raggedright}
 \caption{Reaktionspfad: Auftragung der Energie der optimierten Geometrien gegen den C-Cl-Bindungsabstand}
 \label{figure:pfad}
\end{figure}
\subsection{Bestimmung des Reaktionspfades}


Die Bewegung, welche bei der S\textsubscript{N}2 Reaktion abläuft, ist die
eines Rückseitenangriffs des Chloridions an das Methylbromid. Bei dieser
Reaktion kommt es zunächst zu einem Rückseitenangriff des Chloridions,
daraufhin zur Bildung eines Intermediates und letztlich zum Bindungsbruch zum
Bromidion. Diese Bewegung ist aperiodisch.

Die Vorstellung des \glqq umklappenden Regenschirms \grqq ist unpassend, da
die H-Atome bei der Übergangsbewegung starr auf ihrem Platz bleiben, somit
nicht umklappen, sondern das C-Atom die Fläche der H-Atome durchwandert
(Inversion am C-Zentrum).
\printbibliography

\end{onehalfspace}
\end{document}