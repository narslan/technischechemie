\documentclass[12pt]{article}
\usepackage{amsmath,mathtools}
\usepackage[usenames,dvipsnames]{xcolor}
\usepackage[german]{babel}
\usepackage[utf8]{inputenc}
\usepackage[T1]{fontenc}
\usepackage{textcomp}
%\usepackage{libertine}
%\usepackage{helvet}

\usepackage{microtype}
% Minion and Myriad fonts
\usepackage[minionint,mathlf]{MinionPro}
\renewcommand{\sfdefault}{Myriad-LF}
\usepackage{siunitx}
\usepackage{fancyhdr}
\usepackage{sectsty}
\usepackage{setspace}
\usepackage{booktabs} % To thicken table lines
\usepackage[version=4]{mhchem}
\usepackage[draft]{graphicx}
\usepackage[labelfont=bf]{caption}
\usepackage{subcaption}
\usepackage{chemstyle}
\usepackage{tabularx}
%\usepackage[compatibility=4.7,language=german]{chemmacros}
\renewcommand{\familydefault}{\sfdefault}
\sisetup{detect-all}
\usepackage{chngcntr}
\counterwithin{table}{section}
\counterwithin{figure}{section}

\usepackage{titlesec}

\titleformat*{\section}{\large\bfseries}
\titleformat*{\subsection}{\normalsize\bfseries}

\usepackage{geometry}
 \geometry{
 a4paper,
 left=20mm,
 top=30mm,
 right=20mm
 }

\begin{document}

\begin{onehalfspace}

\section{Geometrieoptimierung und Berechnung der chemischen Verschiebung von Toluol}
Ziel dieser Aufgabe ist es, nach einer durchgeführten Geometrieoptimierung, die \ce{^1_{}H-NMR}-Verschiebung
von Toluol gegen Tetramethylsilan (TMS) als Standard zu berechnen.\\
\textbf{Auswertung 1:}

\begin{table}[!htpb]


\caption{ Parameter für die optimierte Geometrie von Toluol}
\begin{tabularx}{\textwidth}{llllc}
\toprule
Molekül  & Methode & Basissatz & Energie \si{\hartree} & Punktgruppe \\
\midrule
 Toluol & MP2 & aug-cc-pVDZ & -270.75294 &$C _S$\\
\midrule
 Bindung & Bindungslänge \si{\angstrom} & &  &\\
 C-C (Ring) $C_3 - C_4$ & 1.409800 &&&\\
 C-C (Ring) $C_2 - C_3$ & 1.405891 &&&\\
 C-C (Ring) $C_1 - C_2$ & 1.406102 &&&\\
 C-C (Kette) & 1.511691 &&&\\
 C-H (Orto)  & 1.094643 &&&\\
 C-H (Meta)  & 1.093370 &&&\\
 C-H (Para)  & 1.093007 &&&\\
 C-H (Alkyl) & 1.100292 &&&\\
\bottomrule
\end{tabularx}
\label{tab:toluol}
\end{table}

\textbf{Auswertung 2:}

\begin{figure}[!htpb]
  \includegraphics[width=\textwidth]{data/toluol_bezzifert.png}%
  \caption{Optimierte Geometrie von Toluol auf MP2/aug-cc-pVDZ Niveau für die Bindungslänge siehe ~\ref{tab:toluol}  }
\end{figure}
\begin{figure}[!htbp]
  \includegraphics[width=\textwidth]{data/mp2fullaugccpvddz.png}%
  \caption{Chemische Verschiebung und Intensitäten auf MP2/aug-cc-pVDZ Niveau}
\end{figure}


\textbf{Auswertung 3 und 4:}
\begin{table}[!htpb]
\caption{Chemische Verschiebung der Wasserstoffatome für die optimierten Geometrie von Toluol}
\begin{tabular}{llll}
\toprule
H-Atom  & \parbox[t]{4cm}{$\delta$ Toluol/TMS \\ (MP2/aug-cc-pVDZ\\ ppm}  &  \parbox[t]{4cm}{$\delta$ Toluol\cite{zeeh} ppm} & Intensitäten\\
\midrule
H7  & 7.46 & 7.21 & 1 \\
H8  & 7.51 & 7.17 & 2  \\
H9  &  7.52 & 7.17 & 2  \\
H10 & 7.52 & 7.21 & 2 \\
H11 & 7.51 & 7.17 & 2 \\
H13 & 2.63& 2.32 & 1 \\
H14 & 2.30 & 2.32 & 1 \\
H15 & 2.30 & 2.32 & 1 \\
\bottomrule
\end{tabular}
\end{table}

\end{onehalfspace}
\end{document}