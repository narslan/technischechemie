\section{Prolog}
\epigraph{\textit{ We all need histories that no history book can
  tell, but they are not in the classroom, not the history classrooms, anyway.
  They are in the lessons we learn at home, in poetry and childhood games, in
  what is left of history when we close the history books with their verifiable
facts.}}{Michel-Rolph Trouillot\footnotemark} \footnotetext{Michel-Rolph
Trouillot, \textit{Silencing the Past. Power and the Production of History} (Boston: Beacon Press, 1995).}

Was sind das für Geschichten, die wir alle brauchen, und die in keinem
Geschichtsbuch stehen? Was sind das für Geschichten, die nicht in den
Klassenzimmern, sondern in den Häusern der Schüler\_innen erzählt werden? Was
sind das für Geschichten, die sich uns erst eröffnen, wenn wir die Bücher
schließen und uns der Dichtung und dem Spiel zuwenden?\\

\noindent Die oben zitierten Zeilen sind Teil des Vorwortes, mit dem
Michel-Rolph Trouillot das Kapitel „The Haitian Revolution as a
Non-Event“\footnotemark\footnotetext{Trouillot, „The Haitian Revolution as a
Non-Event“ in \textit{Silencing the Past. Power and the Production of History} (Boston: Beacon Press, 1995),71.}
einleitet. In dem Vorwort teilt Trouillot mit den Leser\_innen zwei sehr
unterschiedliche Erfahrungen aus seiner Lehre an US-amerikanischen
Universitäten. Das eine Mal wird er von einer
Studentin\_*\footnotemark\footnotetext{Die Schreibweisen \textit{w}eiß; Schwarz, of
Color und der Gender Gap bzw. das * und ' weisen darauf hin, dass es sich hier
nicht um widerspruchsfreie Kategorien handelt denen ein ontologischer Zustand
vorausgeht. Viel eher sollen damit politische Positionen und Perspektiven
in rassistischen und sexistischen Verhältnissen, die in
dieser Arbeit verhandelt werden, zum Ausdruck gebracht werden.\\
Die Schreibweise Schwarz und of Color, stellt hier
den Versuch dar, Selbstpositionierungen von Menschen, die Rassismus erfahren,
zu berücksichtigen. Die Bezeichnung \textit{w}eiß mit dem kursiven \textit{w} soll dem
Ungleichgewicht entgegenwirken, dass in rassistischen Verhältnissen zumeist die von Rassismus
betroffenen markiert werden und die von Rassismus privilegiert und Normalisierten
darum unbemerkt bleiben können. Da es sich hier jedoch nicht um
eine Selbstbezeichnung handelt, sondern um die Markierung einer durch rassistische Verhältnisse privilegierte Position, wird hier die kursive Schreibweise gewählt.\\
Das Gender\_Gap zeigt den Zwischenraum auf, in dem
sich Menschen im Zweigeschlechterzwang verorten und das Sternchen weist
daraufhin, dass der\_die Leser\_in die Subjektkategorien Frau\_* wie auch andere
Subjektkategorien innerhalb einer Vielzahl von Differenzkategorien verorten
kann und sollte. Die Ein-Strich Anführungseichen ' betont einzelne Begriffe, die
in dieser Arbeit auf ihren Normalitätsanspruch hin befragt werden. Dies
geschieht punktuell, an Stellen da mir die Betonung auf den
Konstruktionscharakter des Begriffs sinnvoll erscheint.\\
Alle Schreibweisen sind im
strategischen Sinne zu verstehen, ich gehe nicht davon aus, dass die Frage nach
der Möglichkeit der Repräsentation von Subjekten durch Sprachordnungen damit
für alle Zeiten geklärt ist. Viel eher soll die Schreibweise den\_die
Leser\_innen und mich dazu anregen, Schreibweisen als Werkzeuge zu verstehen,
mit dem bestehende Bezeichnungspraktiken und damit auch Differenzpraktiken in
Frage gestellt werden können. } dazu aufgefordert, von den Kämpfen der
Sklav\_innen zu sprechen, anstatt nur \textit{w}eiße Autor\_innen zu
rezipieren:
\begin{myenv}
  \textit{„Mr. Trouillot, you make us read all those white scholars. What can
they know about slavery? Where were they when we were jumping off the boats?
When we chose death over misery and killed our own children to spare them from
a life of rape?“\footnotemark\footnotetext{Troillot, „The Haitian Revolution“, 70.}}
\end{myenv}
Das andere Mal und viele Jahre später fordert eine andere
Studentin\_* ihn dazu auf, endlich über die Schwarzen Millionäre zu sprechen:
„I'm tired to hear about that slavery stuff. Can we hear the stories of the
black millionaires?"\footnotemark\footnotetext{Ebd., 71.}\\
Troillot stellt den Leser\_innen beide Student\_innen als Schwarze Frauen\_* vor
und wünscht sich, die Zeit zurückdrehen und diese beiden Student\_innen
miteinander ins Gespräch bringen zu können. Worüber würden sie wohl sprechen?
Was verbindet und was trennt sie?\\

\noindent Die Worte der Student\_innen lösen vieles in mir aus. Nähe und Distanz,
Erinnerung und Sehnsucht, Unsicherheit und Mut.\\
Nähe, weil ich mit dieser
Geschichte die (nicht) erzählt wird verbunden bin. Es ist auch meine
Geschichte, weil Kolonialismus und Sklaverei auch eine Geschichte über
\textit{w}eiße ist.\\
Distanz, weil es nicht um mich, nicht um meine Gefühle
geht oder gehen sollte sondern um die Perspektiven und Forderungen derer, deren
Geschichte nicht erzählt wird.\\
Erinnerung, weil ich ähnliche
Auseinandersetzungen miterlebt habe und die Sehnsucht nach universitären Räumen
bleibt, in denen Subjekte mit all ihren Erfahrungen als Subjekte in diesen
Räumen vorkommen können, ohne auf diese Erfahrungen reduziert zu werden.
Unsicherheit, weil es unmöglich scheint, die Interventionen als Impuls
aufzugreifen ohne sich zugleich aneignend ihnen gegenüber zu verhalten.  Mut,
weil sie ein Zeichen dafür sind, dass es Widerstand gibt.\\

\noindent Die Gedanken die nun
folgen haben mehr mit mir zu tun als mit den Subjekten und dem Kontext aus dem
sie entstanden sind. Es sind meine Lesarten und sie sind damit keine absoluten
Deutungen sondern Versuche, jenen Interventionen aufzugreifen, um von ihnen zu
lernen.\\

\noindent Die Interventionen der Schwarzen Frauen\_* in das Seminar, so unterschiedlich sie
auch sein mögen, verbinde ich mit einer geteilten Idee von Universität. Es sind
Forderungen nach einer Universität als Ort, an dem Geschichten erzählt werden
in denen sie, Schwarze Frauen\_*, sich wiederfinden. Geschichten, die ihnen
etwas darüber erzählen, woher sie kommen und was sie sein können. Und so
unterschiedlich ihre Vorstellungen davon auch sind, so ähnlich ist doch ihr
Verlangen, ihre eigene Geschichte zu hören oder vielleicht auch zu erzählen.\\
Ihre Interventionen zeigen, dass Erfahrungen
Subjekte hervorbringen, die sich annehmend und abweisend gegenüber jenen
Erfahrungen verhalten können.\\
Erfahrung geht nicht notwendig mit einem
unmittelbaren Erlebnis einher. Erfahrung schreibt sich über Generationen in die
Körper und Seelen der Menschen ein und sie hinterlässt Spuren.\\
Spuren, die in den Geschichten, die an der Universität erzählt werden,
aufgegriffen oder verschwiegen werden. Spuren, die Sehnsüchte hervorbringen,
die in den Geschichten, die an der Universität erzählt werden, gestillt oder
verstärkt werden.\\
Was die Forderungen der Schwarzen Frauen\_*, so gegensätzlich
sie auch sein mögen, verbindet, ist ihr Anspruch nach Repräsentation. Sie
möchten die Geschichte aus der Perspektive Schwarzer Menschen über Schwarze
Menschen hören, denn was sollen die \textit{w}eißen schon wissen über Schwarze Sklaven
oder Schwarze Millionäre?\\
Ein Verständnis von Erfahrungswissen, also Wissen,
das aus einer bestimmten Erfahrung heraus entsteht, teilt auch Nicola Lauré al
Samaray. Sie zeigt die Notwendigkeit eines historisierenden Erfahrungsbegriffs
für den deutschen Kontext auf, der Kolonialität mitdenkt:

\begin{myenv}
  \textit{\glqq[…] die koloniale Erfahrung [ist] die, die komplexe kulturelle
    Landkarten mit ineinander widerhallenenden dominanten und unterworfenen
    Geschichtlichkeiten hervorbrachte und noch immer hervorbringt, die sich in
    die Körper Schwarzer und weißer Männer und Frauen einschrieb und noch immer
    einschreibt, derer man sich zu entledigen oder zu erinnern versucht und die
    – weder be- noch überwältigt – Schwarze und weiße deutsche Vergangenheiten
    und Gegenwarten untrennbar miteinander verknüpft. Es ist die koloniale
    Erfahrung, die den Ausgangspunkt einer gewaltvollen hierarchischen
    Begegnungs- und Beziehungsgeschichte markiert, über die im Zuge der
    Sichtbarmachung und verknüpfenden Gegenüberstellung weißer hegemonialer und
    Schwarzer unterworfener Perspektiven Zeugnis und Zeuginnenschaft abgelegt
  wird.\grqq\footnotemark\footnotetext{Nicola Lauré Al-Samaray, „Inspirited
Topography: Über/Lebensräume, Heim-Suchungen und die Verortung der Erfahrung in
Schwarzen deutschen Kultur- und Wissenstraditionen,“ in \textit{Mythen, Masken und
Subjekte}, herausgegeben von Maureen M. Eggers et al. (unrast Verlag: Münster,
2005), 120.}} \end{myenv}
Die koloniale Erfahrung konstituiert \textit{w}eiße und Schwarze Subjekte, sie
konstituiert ihre Beziehung und sie konstituiert die Geschichten, die von
\textit{w}eißen und Schwarzen über sie erzählt werden. Es sind unterschiedliche
Erfahrungen derselben Geschichte, die sich auch dadurch unterscheiden, dass die
Erzählungen Schwarzer über diese geteilte Geschichte unterworfen werden und die
Erzählungen \textit{w}eißer Perspektiven hegemonial (gesetzt) werden.\\

\noindent Die Geschichten, die wir brauchen, finden in den Büchern und Seminaren keinen
Platz.\\
Wenn nun Trouillot seinen Prolog mit diesen Worten resümiert,
klingt das zunächst wie die Resignation eines Professors, der vermutlich einen
großen Teil seines Lebens mit diesen Büchern und in diesen Seminarräumen
verbracht hat. Doch auch in der Resignation können Fragen entstehen:\\
 Was sind
das für Geschichten? Können sie überhaupt erzählt werden? Oder liegt ihr
Geheimnis vielleicht darin, dass sie aus den Gedichten und Kinderzimmern
niemals in die wissenschaftlichen Diskurse, wie sie an der Universität geführt
werden, Einzug erhalten können?\\

\noindent Diese Arbeit besteht nicht in dem Versuch, diese Geschichten zu erzählen. Sie versucht die Leere zu beschreiben, die entsteht, weil diese Geschichten
nicht erzählt werden.\\

\noindent Meinen Ausgangspunkt bildet darum jene Geschichte, die erzählt und
die vermutlich täglich von einer\_m wie Trouillots Student\_innen unterbrochen
wird. Es ist eine Geschichte, die in Europa geschrieben wurde und sich
vornehmlich mit der Frage beschäftigt: „Was ist der Mensch?“ \\
Ich interessiere
mich dafür, welche Erfahrungen hier erinnert und welche entinnert werden. Ich
interessiere mich dafür, von wem und für wen diese Geschichte, die an der
Universität erzählt wird, geschrieben wurde. Wer sich in ihr wiederfindet, und
wer nicht.\\

Trouillot beschreibt in dem Kapitel, das auf den Prolog folgt, die Umstände,
unter denen die Haitianische Sklav\_innenrevolution stattfand. Die europäische
Antwort auf die Frage „Was ist der Mensch?“ wurde von den versklavten Menschen
in Haiti in Frage gestellt. \\
Die Revolution war nicht nur eine Befreiung aus der
Versklavung, sondern auch aus dem kolonialrassistischen Blick, mit dem Europa
die Anderen konstruiert. Doch Trouillot zeigt auch, dass diese Revolution nicht
bis zu der europäischen Erzählung über den Menschen durchdringen, sie nicht
erschüttern konnte. Er zeigt damit wie machtvoll diese Erzählung ist und wie
notwendig Gegenerzählungen sind, die aus Erfahrungen von Widerstand
beschaffen sind:
\begin{myenv}
  \textit{„The silencing of the Haitian Revolution is only a chapter within a
  narrative of global domination. It is part of the history of the West, and it
is likely to persist, even in attenuated form, as long as the history of the
West is not retold in ways that bring forward the perspectives of the
world.“\footnotemark\footnotetext{Trouillot, „The Haitian Revolution“, 107.}}
\end{myenv}
Das Verschweigen der Haitianischen Revolution, so verstehe ich Trouillot, wird als Leerstelle beschrieben, die in der vorherrschenden Erzählung neben vielen anderen Leerstellen besteht.\\
Diese Leerstellen haben sich so tief in die Erzählung des Westens eingeschrieben, dass nur eine Umerzählung, in der die 'Perspektiven der Welt' vorkommen, eine andere Erzählung ermöglichen kann. 

\noindent Vorliegende Arbeit kann nun als Suche nach den 'Perspektiven der Welt' gelesen
werden, die Trouillot als Gegenerzählung zu der Geschichte des Westens
einfordert.\\
Dazu möchte mich mich mit Praktiken des Gegenerzählens als Strategien
dekolonialer Bildung an der Universität auseinandersetzen und fragen, welche
Bedeutung Erfahrungswissen hier einnehmen kann: Was hat es mit jener großen
Erzählung auf sich, welche Erzählformen können sich ihr widersetzen und welches
(Verständnis von) Erfahrungswissen ist hierfür notwendig?


