\documentclass[12pt]{article}
\usepackage{amsmath,mathtools}
\usepackage[usenames,dvipsnames]{xcolor}
\usepackage[german]{babel}
\usepackage[utf8]{inputenc}
\usepackage[T1]{fontenc}
\usepackage{textcomp}
%\usepackage{libertine}
%\usepackage{helvet}

\usepackage{microtype}
% Minion and Myriad fonts
\usepackage[minionint,mathlf]{MinionPro}
\renewcommand{\sfdefault}{Myriad-LF}
\usepackage{siunitx}
\usepackage{fancyhdr}
\usepackage{sectsty}
\usepackage{setspace}
\usepackage{booktabs} % To thicken table lines
\usepackage[version=4]{mhchem}
\usepackage[draft]{graphicx}
\usepackage[labelfont=bf]{caption}
\usepackage{subcaption}
\usepackage{chemstyle}

%\usepackage[compatibility=4.7,language=german]{chemmacros}
\renewcommand{\familydefault}{\sfdefault}
\sisetup{detect-all}
\usepackage{chngcntr}
\counterwithin{table}{section}
\counterwithin{figure}{section}

\usepackage{titlesec}

\titleformat*{\section}{\large\bfseries}
\titleformat*{\subsection}{\normalsize\bfseries}

\usepackage{geometry}
 \geometry{
 a4paper,
 left=20mm,
 top=30mm,
 right=20mm
 }

\begin{document}

\begin{onehalfspace}

\section{\ce{S_N}2-Substitution bei \ce{CH_3Br}}
Ziele dieser Aufgabe ist Bestimmung des Reaktionspfades und Auswahl einer geeigneten Übergangszustandsgeometrie.
\newpage
\subsection{Bestimmung des Reaktionspfades und Auswahl einer geeigneten Übergangszustandsgeometrie}
\textbf{Auswertung 1}:

\begin{figure}[!htbp]
\centering
  \includegraphics[width=0.8\textwidth]{data/a5_teil1_scan.png}%
  \caption{Optimierte Geometrie des Übergangszustandes}
\end{figure}
\noindent
\textbf{Auswertung 2:}\\
 Die Gesamtenergie für die Geometrie beträgt -3071.56972 \si{\hartree}.
Der C-Cl Bindungsabstand beträgt 2.4500\si{\angstrom}.
 Der Basissatz ist aug-cc-pVDZ. Die Punktgruppe beträgt $C _{3v}$.
\begin{table}[!htpb]
\centering
\begin{tabular}{cc}
\toprule
Bindung & Abstand \si{\angstrom}\\
C-Br  & 2.5175 \si{\angstrom}\\
C-Cl  & 2.4500 \si{\angstrom} \\
\midrule
\bottomrule
\end{tabular}
\caption{Die Bindungsabstände im Übergangszustand}
\end{table}\\
\noindent
\textbf{Auswertung 3}\\
Es gibt genau eine imaginäre Schwingungsfrequenz, welche $i^*$= 390.1 \si{\per\centi\meter} beträgt. Die Bewegung, welche bei der SN2 Reaktion abläuft, ist die eines Rückseitenangriffs des Chlorid-Anions an das CH3-Br Molekül. Diese Reaktion verläuft in drei Schritten. Als Erstes kommt der Rückseitenangriff des Cl-, anschließend die Bildung des Zwischenproduktes (Intermediats) und als Letztes der endgültige Bruch der Bindung zum Br-. Diese Bewegung ist als aperiodisch zu charakterisieren.
 Dies sieht man auch daran, dass bei den Schwingungsfrequenzen eine Imaginäre Frequenz auftaucht. Vergleichen lässt sich dies mit der eulerschen Form von Sinus und Cosinus, der e-Funktion. Sinus und Cosinus besitzen keinen Imaginärteil, sind also periodische Bewegung. Nur die eulersche Form hat einen Imaginärteil. Die e-Funktion ist aperiodisch.\\

\subsection{Optimierung und Frequenzrechnung der Übergangszustandes}

\textbf{Auswertung 1}\\
\begin{table}[!htpb]
\centering
\begin{tabular}{ccc}
\toprule
Molekül & E/\textit{Hartree} & Bildungsenthalpie $\Delta G$  \textit{Hartree}\\
\ce{CH_3Br}  & -2612.00556924 & -2611.990266 \\
\ce{CH_3Cl}  & -499.12227340 & -499.105102\\
\ce{Br^-}  & -2572.46248971 & -2572.478665 \\
\ce{Cl^-}  & -459.56364460 & -459.578667 \\
\midrule
\bottomrule
\end{tabular}
\caption{Molekülbindungen und Energien}
\end{table}
Anhand der ermittelten Werte kann nun die freie Enthalpie wie folgt berechnet werden:
\begin{equation}
\Delta _R G = \sum\limits_{Produkte} \Delta _B G - \sum\limits_{Edukte} \Delta _B G
\end{equation}
Setzt man die Werte ein, so erhält man $ \Delta _R G = -0.014834$ \si{\hartree}. Rechnet man das Ergebnis nun in \si{\kilo\joule\per\mol} um, so erhält man $ \Delta _R G = -38.95$ \si{\kilo\joule\per\mol}.
Anschließend wird die freie Aktivierungsenthalpie berechnet:
\begin{equation}
\Delta _R G^{\neq} = \sum\limits_{Produkte} \Delta _B G _U - \sum\limits_{Edukte} \Delta _B G
\end{equation}
$\Delta _B G _U$ stellt hier die Energie des Übergangszustandes dar. Sie beträgt $\Delta _B G _U = -3071.56972$ \si{\hartree} (aus der Aufgabe 5 Teil 1). $\Delta _R G^{\neq}$ steht für die freie Aktivierungsenthalpie. Nach der Berechnung ergibt sich für die freie Aktivierungsenthalpie: $\Delta _R G^{\neq} = 26.167$ \si{\kilo\joule\per\mol}.

\noindent
\textbf{Auswertung 2}\\
Betrachtet man den Graphen zur Ausbeute aus dem Skript, so lässt sich bei den dargestellten Temperaturen eine sehr gute Ausbeute von ca. 98 \% voraussagen. Aus dem niedrigen Wert für die freie Aktivierungsenergie lässt sich schließen, dass die Reaktion wahrscheinlich sehr schnell ablaufen wird, da der 2. Graph für die Reaktion 2. Ordnung mit niedrigen Aktivierungsenthalpien stark gegen Null geht. Ein Fehler von $\pm 10$ \si{\kilo\joule\per\mol} wirkt sich nur sehr geringfügig auf die Vorhersage auf, da auch bei so einem vergleichsweise hohen Fehler die Reaktion fast gleichschnell ablaufen und sich die Ausbeute nur geringfügig verringern würde.
\newpage
\subsection{Bestimmung des Reaktionspfades}
\textbf{Auswertung 1}
\begin{figure}[!htpb]
\centering
  \includegraphics[width=\textwidth,height=\textheight,keepaspectratio]{data/potentialkurve.png}%
      \captionsetup{justification=raggedright}
  \caption{Reaktionspfad: Auftragung der Energie der optimierten Geometrien gegen den C-Cl-Bindungsabstand}
\end{figure}\\
\noindent
\textbf{Auswertung 2}\\
Es handelt sich bei der Übergangsbewegung um eine aperiodische. \\
\noindent
\textbf{Auswertung 3}\\
Maßgeblich an der Übergangsbewegung beteiligt sind das Brom- und Chloratom.
 Das sogenannte Bild des \glqq umklappenden Regenschirms \grqq ist nicht zutreffend, da die H-Atome bei der Übergangsbewegung starr auf ihrem Platz bleiben (nicht umklappen), wohingegen das C-Atom umklappt (Inversion am C-Zentrum).
\end{onehalfspace}
\end{document}