\documentclass{article}
\usepackage{amsmath}


%encoding
%--------------------------------------

%----------------------------
\usepackage{microtype}
%\usepackage{fourier}
%\usepackage{mathpazo} % add possibly `sc` and `osf` options

%\usepackage[default,osfigures,scale=0.85]{opensans} %% Alternatively
%% use the option 'defaultsans' instead of 'default' to replace the
%% sans serif font only.

%\usepackage[default]{gillius}
\usepackage{mhchem}
\usepackage{listings}
\usepackage[utf8]{inputenc}
\usepackage{tgheros}
\renewcommand*\familydefault{\sfdefault} %% Only if the base font of the document is to be sans serif

\usepackage[T1]{fontenc}
\usepackage[german]{babel}

\usepackage{mathastext}

%\usepackage{sansmath}
%\sansmath

%\renewcommand{\sfdefault}{Myriad-LF}
%units
%--------------------------------------
\usepackage{siunitx}
\usepackage{verbatim}
%drawing
%------------------------------------
\usepackage{tikz} % To generate the plot from csv
\usepackage{pgfplots}
\usepackage{pgfplotstable}
\usepackage{caption}
\usepackage{tabularx}
\usepackage[landscape]{geometry}
\usepackage{fancyhdr}


\pagestyle{fancy}

\cfoot{\thepage}

\rhead{Gruppe 24}
\lhead{\textbf{Coffein-Bestimmung} Ermittlung des Standardfehlers und des Vertrauensbereichs}
\setlength{\headheight}{15pt}


\renewcommand{\familydefault}{\sfdefault}

\pagenumbering{gobble}

\definecolor{LightCyan}{rgb}{0.88,1,1}
\usetikzlibrary{datavisualization}
\pgfplotsset{compat=newest} % Allows to place the legend below plot
\usepgfplotslibrary{units} % Allows to enter the units nicely
\usepackage{overcite}
\renewcommand\citeform[1]{[#1]}

\sisetup{
  round-mode          = places,
  round-precision     = 2,
  inter-unit-product =\ensuremath{{}\cdot{}}
}




\usetikzlibrary{arrows, positioning, calc, datavisualization}

\pgfplotsset{
  standard/.style={
    axis x line=bottom,
    axis y line=middle,
  }
}
\usepackage{url}

\begin{document}

\section{Ermittlung des Standardfehlers $s_{y,x}$}

 \begin{table}[ht!]
  \centering
 \begin{tabularx}{\textwidth}{XXXXX}
 $x$  & $y$ & $y^2$ & $x \cdot y$ & $(x-\overline{x})^2$\\
\hline
0   & 458.54  & 210256.18    &0.00       & 15625.00 \\
50  &1859.12  & 3456308.58   &92955.75   & 5625.00 \\
100 &4196.75  & 17612727.35  &419675.20  & 625.00 \\
150 &6710.20  & 45026851.14  &1006530.75 & 625.00 \\
200 &8742.89  & 76438143.04  &1748578.20 & 5625.00 \\
250 &11323.00 & 128210396.94 &2830750.75 & 15625.00 \\
\hline
$\sum$ &  & & & \\
750.00 & 33290.50 & 270954683.23 & 6098490.65 & 43750.00 \\
\hline
\end{tabularx}
 \caption{Die Messwerte und die Summen zur Ermittlung von $s_{y,x}$}
\end{table}
\begin{flushleft}
  Der Standardfehler der Schätzung $s_{y,x}$ wird errechnet nach:

\end{flushleft}
\begin{align}
 s_{y,x} &= \sqrt{\frac{\sum y^2 - b \cdot \sum y - m \cdot \sum xy}{(n-2)}}
\end{align}
mit m = 44.278, b = 13.624 und n = 6
\begin{align*}
  s_{y,x} &=\sqrt{ \frac{270954683.23 - 13.624 \cdot 33290.50 -44.278 \cdot 6098490.65} {6-2}} \\
  &= 342.801
\end{align*}
\newpage
\section{Ermittlung des Vertrauensbereich $ VB(\hat{x}) $}
\begin{flushleft}
  Der Vertrauensbereich $VB(\hat{x})$ wird errechnet nach:
\end{flushleft}
\begin{equation}
  VB(\hat{x}) = \hat{x} \pm \frac{t_{P,n-2} \cdot s _{y,x}}{m} \cdot \sqrt{1+\frac{1}{n}+\frac{(y-\overline{y})^2}{m^2 \cdot \sum (x -\overline{x})^2}}
\end{equation}
mit:
\begin{align*}
   y &= 7781.223 \\
   \hat{x} & = 175.428 \\
   n &= 6 \\
   s_{y,x} &= 342.81 \\
   \overline{y} & = 5548.417 \\
   t_{p,n-2} &= 2.45 \\
   m &= 44.278\\
   \sum (x -\overline{x})^2&= 43750.00
 \end{align*}
 daraus folgt:
\begin{align*}
   VB(\hat{x}) &= 175.428 \pm \frac{2.45 \cdot 342.801}{44.278} \cdot  \sqrt{1+\frac{1}{6}+\frac{(7781.223-5548.417)^2}{(44.278)^2 \cdot 43750.00 }}\\
    &=20.992 \, \si{\milli\gram\per\liter}
 \end{align*}


% \begin{thebibliography}{}
% \bibitem{crc1}
% D. R. Lide, \textit{CRC Handbook of Chemistry and Physics}, 88. Aufl., CRC Press, New York \textbf{2007}, S. 888.
% \bibitem{crc2}
% D. R. Lide, \textit{CRC Handbook of Chemistry and Physics}, 88. Aufl., CRC Press, New York \textbf{2007}, S. 1069.
% \bibitem{nist}
% \url{http://webbook.nist.gov/chemistry/fluid/}
% \bibitem{skript1}
% A. Brehm, \textit{Praktikumskript Wärmeübertragung}, Oldenburg, \textbf{2014},  S. 14.
% \end{thebibliography}
\end{document}
