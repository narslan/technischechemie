% Page Layout
\documentclass[12pt]{beamer}
\usepackage[latin1]{}
\usepackage{tikz}
\usepackage{chemfig}
\usepackage{xcolor}
%\documentclass[12pt]{article}

%\usepackage{fancyhdr}
%\usepackage{chemfig}
%\usepackage{sectsty}
%\usepackage{setspace}
%\usepackage{helvet}
%\usepackage{graphicx}
%\usepackage{overcite}
%\usepackage{parskip}
%\usepackage{siunitx}
\usepackage[version=4]{mhchem}
\usepackage[method=mhchem]{chemmacros}
%\usepackage{libertine}
%\usepackage[scaled=.83]{beramono}
\usepackage{graphicx}
%\usepackage{hyperref}
%\usepackage[a4paper]{geometry}
%\usepackage{unicode-math}
%\usepackage{microtype}
%\usepackage{xcolor}
%\usepackage{chemformula}
\usetikzlibrary{positioning,calc,arrows}
\usepackage{url}
\usepackage[german]{babel}
%\usepackage{libertine}
%\renewcommand*\familydefault{\sfdefault}  %%
\usepackage[default]{opensans}
%\usepackage{gillius2}
%\usepackage{lato}

\usepackage[utf8]{inputenc}

\usepackage[T1]{fontenc}
\usepackage{chemstyle}
\usepackage{array} % needed for \arraybackslash
\usepackage{graphicx}
\usepackage{adjustbox} % for \adjincludegraphics
 % Avoid an error due to a lack of registers
\renewcommand*\printatom[1]{{\fontsize{14}{14}\selectfont\ensuremath{\mathsf{#1}}}}
\definesubmol\cycleoplus{-[,0.25,,,draw=none]\oplus}
\definesubmol{e}{-[,.1,,,draw=none]}
\definesubmol\nobond{[,0.2,,,draw=none]}

\definecolor{orange}{RGB}{255,127,0}
%\include{head}
\title[]{Lactame und ihre Verwendung}
\author[N. Arslan]{Nevroz Arslan}
\date[14.10.16]{16. Oct 2014}
%\titlegraphic{\includegraphics[width=\textwidth,height=.5\textheight]{bogen1200.png}}
\setbeamerfont{page number in head/foot}{size=\large}
\beamertemplatenavigationsymbolsempty
\setbeamertemplate{footline}[frame number]

\setdoublesep{4pt}
\setatomsep{2em}
\definesubmol{e}{-[,.1,,,draw=none]}
\begin{document}
\setbeamercolor{block title}{use=structure,fg=black,bg=white}
\setbeamercolor{block body}{use=structure,fg=black,bg=white}
%\input{title.tex}




%%%%%%%%%%%% Start of content %%%%%%%%%%%%

%\input{seite11.tex}
\usebackgroundtemplate{

\includegraphics[width=1.0\paperwidth,height=0.17\paperheight]{bogen1200}
\begin{tikzpicture}[overlay, remember picture]
    \node[xshift=-10.80cm,yshift=1.15cm] at (0,0)    {\includegraphics[scale=0.6]{logo}};
\end{tikzpicture}
}
\addtobeamertemplate{frametitle}{\vskip+5ex}{}
\setbeamercolor{frametitle}{fg=black}


\setbeamertemplate{itemize/enumerate body begin}{\normalsize}

\frame{\titlepage}
\begin{frame}
  \frametitle{Inhaltsübersicht}
  \tableofcontents
  \end{frame}

\section{Lactamen}

\begin{frame}
\subsection{Definition}

\begin{tabular}{p{.8\textwidth} p{.2\textwidth}}
\adjincludegraphics[width=.8\linewidth,valign=t]{gaba}
&
""
\end{tabular}


\end{frame}

%\input{a1.tex}
\subsection{Herstellung}
\begin{frame}
  Zur Herstellung von Butyrolactam kann die Reaktion von $\gamma$-Butyrolacton mit
  Ammoniak bei erhöhter Temperatur genutzt werden
  Luvitec®: Adhesive Raw Materials
water solubility, non toxic
\end{frame}



%\input{a2.tex}
\subsection{Eigenschaften}
%\input{a3.tex}
\section{Reaktionen}
%\input{seite2.tex}

\begin{frame}
\begin{thebibliography}{}

\bibitem{bio}
Technisches Datenblatt,\url{http://www.dispersions-pigments.basf.com/portal/basf/ide/dt.jsp?setCursor=1_736581}
\bibitem{march}
M. Smith,J. March, \textit{March's Advanced Organic Chemistry}, 6. Aufl., Wiley, New York \textbf{2007}, S. 924-926.
\end{thebibliography}
\end{frame}
%%\input{seite11.tex}

%%%%%%%%%%%% End of content %%%%%%%%%%%%

%\input{end.tex}

%%%%%%%%%%%% Start of appendix %%%%%%%%%%%%

%%%%%%%%%%%% End of appendix %%%%%%%%%%%%

\end{document}
