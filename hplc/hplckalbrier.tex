% !TEX encoding   = UTF-8
% !TEX program    = LuaLaTeX
% !TEX spellcheck = de_DE
\documentclass[a4paper]{report}
\usepackage{fontspec}
\usepackage[usenames,dvipsnames]{color}
\usepackage{siunitx}
\usepackage{pgfplots}
\usepackage{pgfplotstable}
\pgfplotsset{compat=1.13}
\usepgfplotslibrary{units}
\usepackage{fancyhdr}
\pagestyle{fancy}
\usepackage{luacode}
\definecolor{skyblue1}{rgb}{0.447,0.624,0.812}
\definecolor{scarletred1}{rgb}{0.937,0.161,0.161}
\definecolor{chameleon1}{rgb}{0.541,0.886,0.204}
%\usepackage{helvet}
\usepackage[landscape]{geometry}

%\usepackage{libertine}
%\usepackage{beramono}
\setmainfont{Ubuntu}
\begin{luacode*}

function drawD()
local string = string
local lines = {}

for line in io.lines("daten/1.txt") do
  table.insert(lines, line)
end
local summex=0
local summey=0
local xwerte={}
local ywerte={}



for i, line in ipairs(lines) do
  if i>1 then
     local s, e = line:find("%s+", 1)
     local y=tonumber(line:sub(e))
     local x=tonumber(line:sub(1,e))
     table.insert(xwerte,x)
     table.insert(ywerte,y)
   end
end

----m
local n = #lines-1

    tex.sprint("\\addplot[".."no marks".."] coordinates{")
    -- tex.sprint("\\addplot coordinates{")
    for i=1,n do
      tex.sprint("("..xwerte[i]..","..ywerte[i]..")")
    end
    tex.sprint("};")
     tex.sprint("\\node[small dot,pin=-90:{\\texttt{3.108}}] at  (axis description cs:3.108,0){};")


 --   tex.sprint("\\addplot[".."domain="..minx..":"..maxx..", no markers".."]{"..quad.a.."*x^2+"..quad.b.."*x+"..quad.c.."};")
 --   tex.sprint("\\addplot[".."domain="..minx..":"..maxx..", no markers".."]{"..m.."*x+"..b.."};")
--   tex.sprint("\\addplot[only marks,color=red] coordinates{(".."0"..","..(-b/m)..")};")

--   tex.sprint("\\node[small dot,pin=90:{"..round(math.abs(-b/m),2).."}] at (axis description cs:0.182,0) {};")

   -- tex.sprint("\\addplot[only marks] coordinates {("..nwg..","..mess(nwg)..")};")

end
function drawD2()
local string = string
local lines = {}

for line in io.lines("daten/2.txt") do
  table.insert(lines, line)
end
local summex=0
local summey=0
local xwerte={}
local ywerte={}



for i, line in ipairs(lines) do
  if i>1 then
     local s, e = line:find("%s+", 1)
     local y=tonumber(line:sub(e))
     local x=tonumber(line:sub(1,e))
     table.insert(xwerte,x)
     table.insert(ywerte,y)
   end
end

----m
local n = #lines-1

    tex.sprint("\\addplot[".."no marks".."] coordinates{")
    -- tex.sprint("\\addplot coordinates{")
    for i=1,n do
      tex.sprint("("..xwerte[i]..","..ywerte[i]..")")
    end
    tex.sprint("};")
 --   tex.sprint("\\addplot[".."domain="..minx..":"..maxx..", no markers".."]{"..quad.a.."*x^2+"..quad.b.."*x+"..quad.c.."};")
 --   tex.sprint("\\addplot[".."domain="..minx..":"..maxx..", no markers".."]{"..m.."*x+"..b.."};")
--   tex.sprint("\\addplot[only marks,color=red] coordinates{(".."0"..","..(-b/m)..")};")

--   tex.sprint("\\node[small dot,pin=90:{"..round(math.abs(-b/m),2).."}] at (axis description cs:0.182,0) {};")

   -- tex.sprint("\\addplot[only marks] coordinates {("..nwg..","..mess(nwg)..")};")

end

function drawD3()
local string = string
local lines = {}

for line in io.lines("daten/3.txt") do
  table.insert(lines, line)
end
local summex=0
local summey=0
local xwerte={}
local ywerte={}



for i, line in ipairs(lines) do
  if i>1 then
     local s, e = line:find("%s+", 1)
     local y=tonumber(line:sub(e))
     local x=tonumber(line:sub(1,e))
     table.insert(xwerte,x)
     table.insert(ywerte,y)
   end
end

----m
local n = #lines-1

    tex.sprint("\\addplot[".."no marks".."] coordinates{")
    -- tex.sprint("\\addplot coordinates{")
    for i=1,n do
      tex.sprint("("..xwerte[i]..","..ywerte[i]..")")
    end
    tex.sprint("};")
 --   tex.sprint("\\addplot[".."domain="..minx..":"..maxx..", no markers".."]{"..quad.a.."*x^2+"..quad.b.."*x+"..quad.c.."};")
 --   tex.sprint("\\addplot[".."domain="..minx..":"..maxx..", no markers".."]{"..m.."*x+"..b.."};")
--   tex.sprint("\\addplot[only marks,color=red] coordinates{(".."0"..","..(-b/m)..")};")

--   tex.sprint("\\node[small dot,pin=90:{"..round(math.abs(-b/m),2).."}] at (axis description cs:0.182,0) {};")

   -- tex.sprint("\\addplot[only marks] coordinates {("..nwg..","..mess(nwg)..")};")

end

function drawD4()
local string = string
local lines = {}

for line in io.lines("daten/4.txt") do
  table.insert(lines, line)
end
local summex=0
local summey=0
local xwerte={}
local ywerte={}



for i, line in ipairs(lines) do
  if i>1 then
     local s, e = line:find("%s+", 1)
     local y=tonumber(line:sub(e))
     local x=tonumber(line:sub(1,e))
     table.insert(xwerte,x)
     table.insert(ywerte,y)
   end
end

----m
local n = #lines-1

    tex.sprint("\\addplot[".."no marks".."] coordinates{")
    -- tex.sprint("\\addplot coordinates{")
    for i=1,n do
      tex.sprint("("..xwerte[i]..","..ywerte[i]..")")
    end
    tex.sprint("};")
 --   tex.sprint("\\addplot[".."domain="..minx..":"..maxx..", no markers".."]{"..quad.a.."*x^2+"..quad.b.."*x+"..quad.c.."};")
 --   tex.sprint("\\addplot[".."domain="..minx..":"..maxx..", no markers".."]{"..m.."*x+"..b.."};")
--   tex.sprint("\\addplot[only marks,color=red] coordinates{(".."0"..","..(-b/m)..")};")

--   tex.sprint("\\node[small dot,pin=90:{"..round(math.abs(-b/m),2).."}] at (axis description cs:0.182,0) {};")

   -- tex.sprint("\\addplot[only marks] coordinates {("..nwg..","..mess(nwg)..")};")

end

\end{luacode*}
\newcommand\addplotCHIpf{\directlua{drawD()}}
\newcommand\addploty{\directlua{drawD2()}}
\newcommand\addplotxx{\directlua{drawD3()}}
\newcommand\addplotzz{\directlua{drawD4()}}

\pgfplotsset{
  standard/.style={
    axis x line=bottom,
    axis y line=middle,
     xlabel= Zeit, % Set the labels
    ylabel= Potential ,
    y unit= \si{\milli\volt} , % Set the respective units
    x unit= \si{\minute},
    legend style={at={(0.5,-0.2)},anchor=north},
    y label style={at={(axis description cs:-0.1,.5)},rotate=90,anchor=south},
    x tick label style={rotate=90,anchor=east}
  }
}
\begin{document}
\lhead{HPLC - Aufgabe 2 Gruppe 24  \footnotesize{Nevroz Arslan}}

\begin{figure}
\begin{tikzpicture}[scale=2,transform shape]
  \tikzset{
    every pin/.style={fill=scarletred1,rectangle,rounded corners=3pt,font=\tiny},
    small dot/.style={fill=chameleon1,circle,scale=0.3}
  }
  \begin{axis}[standard,
  xtick={0,  2.991 ,   5.607 ,   6.637 ,   8.316 ,   10.917},
    xticklabels={$0$,$t_m = 2.991$, $t_r = 5.607$, $t_r =  6.637 $, $t_r =  8.316 $, $t_r =  10.917$},
  ticklabel style={scale=0.5}
  ]
    \addplotCHIpf
    %\node[coordinate,pin=right:{\scriptsize 5.748}] at (axis cs:5.748, 326.559) {};
    %\node[coordinate,pin=right:{\scriptsize 6.777}] at (axis cs:6.777,260.202) {};
    %\node[coordinate,pin=right:{\scriptsize 8.448}] at (axis cs:8.448,  211.349) {};
    %\node[coordinate,pin=right:{\scriptsize 11.045}] at (axis cs:11.045,164.937) {};
  \end{axis}
\end{tikzpicture}
\caption{Chromatogramm der 4 Benzoesäureester bei 2 \si{\milli\liter\per\minute}}
\end{figure}

\begin{figure}

\begin{tikzpicture}[scale=2,transform shape]
  \tikzset{
    every pin/.style={fill=scarletred1,rectangle,rounded corners=3pt,font=\tiny},
    small dot/.style={fill=chameleon1,circle,scale=0.3}
  }
\begin{axis}[standard, xtick={0, 3.108,5.748,6.777,8.448,11.045 },
  xticklabels={$0$,$t_m = 3.108$, $t_r = 5.748$, $t_r = 6.777$, $t_r = 8.448$, $t_r = 11.045$},
  ticklabel style={scale=0.5}
]
    \addploty
  \end{axis}

\end{tikzpicture}
  \caption{Chromatogramm der 4 Benzoesäureester bei 2 \si{\milli\liter\per\minute}}

\end{figure}


\begin{figure}

\begin{tikzpicture}[scale=2,transform shape]
  \tikzset{
    every pin/.style={fill=scarletred1,rectangle,rounded corners=3pt,font=\tiny},
    small dot/.style={fill=chameleon1,circle,scale=0.3}
  }
\begin{axis}[standard,
 xtick={0, 3.981 , 7.516 , 8.893 , 11.128,14.591},
  xticklabels={$0$,$t_m = 3.981 $, $t_r = 7.516$, $t_r =  8.893$, $t_r = 11.128$, $t_r = 14.591$},
  ticklabel style={scale=0.5}]
    \addplotxx
  \end{axis}
\end{tikzpicture}
\caption{Chromatogramm der 4 Benzoesäureester bei 1.5 \si{\milli\liter\per\minute}}

\end{figure}

\begin{figure}

\begin{tikzpicture}[scale=2,transform shape]
  \tikzset{
    every pin/.style={fill=scarletred1,rectangle,rounded corners=3pt,font=\tiny},
    small dot/.style={fill=chameleon1,circle,scale=0.3}
  }

\begin{axis}[standard, xtick={0,  4.004 ,7.516 , 8.892 , 11.128, 14.601},
  xticklabels={$0$,$t_m =  4.004$, $t_r =  7.516 $, $t_r = 8.892$, $t_r = 11.128$, $t_r = 14.601$},
  ticklabel style={scale=0.5}
  ]
    \addplotzz
  \end{axis}
\end{tikzpicture}
\caption{Chromatogramm der 4 Benzoesäureester bei 1.5 \si{\milli\liter\per\minute}}

\end{figure}


\par
%\tabellenwerte
\par
\end{document}
