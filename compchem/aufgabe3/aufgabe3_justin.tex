\documentclass[12pt]{article}
\usepackage{amsmath,mathtools}
\usepackage[usenames,dvipsnames]{xcolor}
\usepackage[german]{babel}
\usepackage[utf8]{inputenc}
\usepackage[T1]{fontenc}
\usepackage{textcomp}
%\usepackage{libertine}
%\usepackage{helvet}

\usepackage{microtype}
% Minion and Myriad fonts
%\usepackage[minionint,mathlf]{MinionPro}
%\renewcommand{\sfdefault}{Myriad-LF}
\usepackage{siunitx}
\usepackage{fancyhdr}
\usepackage{sectsty}
\usepackage{setspace}
\usepackage{booktabs} % To thicken table lines
\usepackage[version=4]{mhchem}
\usepackage[draft]{graphicx}
\usepackage[labelfont=bf]{caption}
\usepackage{subcaption}
\usepackage{chemstyle}
\usepackage{tabularx}
%\usepackage[compatibility=4.7,language=german]{chemmacros}
\renewcommand{\familydefault}{\sfdefault}
\sisetup{detect-all}
\usepackage{chngcntr}
\counterwithin{table}{section}
\counterwithin{figure}{section}

\usepackage{titlesec}

\titleformat*{\section}{\large\bfseries}
\titleformat*{\subsection}{\normalsize\bfseries}

\usepackage{geometry}
 \geometry{
 a4paper,
 left=20mm,
 top=30mm,
 right=20mm
 }

\begin{document}

\begin{onehalfspace}

\section{Geometrieoptimierung und Berechnung der chemischen Verschiebung von Toluol}
In dieser Aufgabe, soll die \textsuperscript{1}H-NMR-Verschiebung gegen Tetramethylsilan als Standard zu berechnen. \\
\textbf{Auswertung 1:}

\begin{table}[!htpb]


\caption{ Parameter für die optimierte Geometrie von Toluol}
\begin{tabularx}{\textwidth}{llllc}
\toprule
Molekül  & Methode & Basissatz & Energie \si{\hartree} & Punktgruppe \\
\midrule
 Toluol & MP2 & aug-cc-pVDZ & -270.75294 &$C _S$\\
\midrule
 Bindung & Bindungslänge \si{\angstrom} & &  &\\
 C-C (Ring) $C_3 - C_4$ & 1.409800 &&&\\
 C-C (Ring) $C_2 - C_3$ & 1.405891 &&&\\
 C-C (Ring) $C_1 - C_2$ & 1.406102 &&&\\
 C-C (Kette) & 1.511691 &&&\\
 C-H (Orto)  & 1.094643 &&&\\
 C-H (Meta)  & 1.093370 &&&\\
 C-H (Para)  & 1.093007 &&&\\
 C-H (Alkyl) & 1.100292 &&&\\
\bottomrule
\end{tabularx}
\label{tab:toluol}
\end{table}

\textbf{Auswertung 2:}

\begin{figure}[!htpb]
  \includegraphics[width=\textwidth]{data/toluol_bezzifert.png}%
  \caption{Optimierte Geometrie von Toluol auf MP2/aug-cc-pVDZ Niveau für die Bindungslänge siehe ~\ref{tab:toluol}  }
\end{figure}
\begin{figure}[!htbp]
  \includegraphics[width=\textwidth]{data/mp2fullaugccpvddz.png}%
  \caption{Chemische Verschiebung und Intensitäten auf MP2/aug-cc-pVDZ Niveau}
\end{figure}
\pagebreak

\begin{table}[!htpb]
\caption{Chemische Verschiebung der Wasserstoffatome für die optimierten Geometrie von Toluol}
\begin{tabular}{lcc}
\toprule
H-Atom  & \parbox[t]{4cm}{$\delta$ Toluol/TMS \\ (MP2/aug-cc-pVDZ\\ ppm}  &   Intensitäten\\
\midrule
H-7  & 7.46 &  5 \\
H-8  & 7.51 &  3  \\
H-9  & 7.52 &  1  \\
H-10 & 7.52 &  2 \\
H-11 & 7.51 &  4 \\
H-13 & 2.63 &  1 \\
H-14 & 2.30 &  1 \\
H-15 & 2.30 &  2 \\
\bottomrule
\end{tabular}
\end{table}



\textbf{Auswertung 3:}
Die theoretische Berechnung und die reale Messung unterscheiden sich in mehrere Punkten.  Bei der theoretischen Betrachtung wird ausser acht gelassen dass das Molekül in der Realität nicht starr ist sondern Bewegungen durchläuft. Ebenso spielt die Temperatur der gemessenen Probe eine Rolle, sowie natürliche Messfehler und Verunreinigungen der Probe. Zudem kann es in der Probe zu Wechselwirkungen kommen die das Ergebniss ebenfalls abweichen lassen.

\begin{table}[!htpb]
\caption{Chemische Verschiebung der Wasserstoffatome für die optimierten Geometrie von Toluol und experimentelle Daten (aus M. Hesse, et al, Spektroskopische Methoden in der organischen Chemie, 8. Auflage, Thieme Verlag, \textbf{2012}, S. 233, Toluen}
\begin{tabular}{lcc}
\toprule
H-Atom  & \parbox[t]{4cm}{$\delta$ Toluol/TMS \\ (MP2/aug-cc-pVDZ\\ ppm}  &  \parbox[t]{4cm}{$\delta$ Toluol[Experimentelle Daten] ppm}\\
\midrule
H-7  & 7.46 & 7.17  \\
H-8  & 7.51 & 7.21   \\
H-9  &  7.52 & 7.21   \\
H-10 & 7.52 & 7.17 \\
H-11 & 7.51 & 7.17  \\
H-13 & 2.63& 2.32  \\
H-14 & 2.30 & 2.32 \\
H-15 & 2.30 & 2.32  \\
\bottomrule
\end{tabular}
\end{table}
\pagebreak
\textbf{Auswertung 4:}Der Vergleich zwischen berechneten und experimentellen Daten zeigt, das  es bis auf 14-H und 15-H der Methylgruppe deutliche Unterschiede gibt. Die größte Abweichung stellt hier das H-13 der Methylgruppe dar, gefolgt von den H-Signalen des Benzolrings. Die Unterschiede der Werte gehen wie in Auswertung 3 aufgezeigt unter anderem auf die Wechselwirkungen in der gemessenen Probe zurück.


\end{onehalfspace}
\end{document}