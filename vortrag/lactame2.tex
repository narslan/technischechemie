% Page Layout
\documentclass[10pt]{beamer}
\usepackage[latin1]{}
\usepackage{tikz}
\usepackage{chemfig}
\usepackage{xcolor}

\usepackage{siunitx}
\usepackage[version=4]{mhchem}
\usepackage[method=mhchem]{chemmacros}
\usepackage{graphicx}
\usetikzlibrary{positioning,calc,arrows}
\usepackage{url}
\usepackage[default]{opensans}
%\usepackage{gillius2}
%\usepackage{lato}
\usepackage{adjustbox}
\usepackage[utf8]{inputenc}
\usepackage{caption}

\usepackage[T1]{fontenc}

\usepackage{array} % needed for \arraybackslash
\usepackage{graphicx}
\usepackage{adjustbox} % for \adjincludegraphics
\usepackage{biblatex}
\usepackage{tabularx}

\addbibresource{lactam.bib}

\sisetup{
  round-mode          = places,
  round-precision     = 2,
  inter-unit-product =\ensuremath{{}\cdot{}}
}
 % Avoid an error due to a lack of registers
\renewcommand*\printatom[1]{{\fontsize{14}{14}\selectfont\ensuremath{\mathsf{#1}}}}
\definesubmol\cycleoplus{-[,0.25,,,draw=none]\oplus}
\definesubmol{e}{-[,.1,,,draw=none]}
\definesubmol\nobond{[,0.2,,,draw=none]}

\definecolor{orange}{RGB}{255,127,0}
%\include{head}
\title[]{Lactame und ihre Verwendung}
\author[N. Arslan]{Nevroz Arslan}
\date[14.10.16]{16. Oct 2016}
%\titlegraphic{\includegraphics[width=\textwidth,height=.5\textheight]{bogen1200.png}}
\setbeamerfont{page number in head/foot}{size=\large}
\beamertemplatenavigationsymbolsempty
\setbeamertemplate{footline}[frame number]

\setdoublesep{2pt}
\setatomsep{2em}

\begin{document}
\setbeamercolor{block title}{use=structure,fg=black,bg=white}
\setbeamercolor{block body}{use=structure,fg=black,bg=white}
%\input{title.tex}
%\setbeamerfont{bibliography item}{size=\scriptsize}
%\setbeamerfont{bibliography entry author}{size=\scriptsize}
%\setbeamerfont{bibliography entry title}{size=\scriptsize}
%\setbeamerfont{bibliography entry location}{size=\scriptsize}
%\setbeamerfont{bibliography entry note}{size=\scriptsize}
%\setbeamerfont{bibliography entry url}{size=\scriptsize}


%%%%%%%%%%%% Start of content %%%%%%%%%%%%

%\input{seite11.tex}
\usebackgroundtemplate{

\includegraphics[width=1.0\paperwidth,height=0.17\paperheight]{bogen1200}
\begin{tikzpicture}[overlay, remember picture]
    \node[xshift=-10.80cm,yshift=1.15cm] at (0,0)    {\includegraphics[scale=0.6]{logo}};
\end{tikzpicture}
}

\addtobeamertemplate{frametitle}{\vskip+7ex}{}
\setbeamercolor{frametitle}{fg=black}
\setbeamertemplate{caption}{\insertcaption}


\setbeamertemplate{itemize/enumerate body begin}{\normalsize}
\setbeamerfont{frametitle}{size=\large}
\frame{\titlepage}

\newcommand\setpolymerdelim[2]{\def\delimleft{#1}\def\delimright{#2}}
\def\makebraces[#1,#2]#3#4#5{%
\edef\delimhalfdim{\the\dimexpr(#1+#2)/2}%
\edef\delimvshift{\the\dimexpr(#1-#2)/2}%
\chemmove{%
\node[at=(#4),yshift=(\delimvshift)]
{$\left\delimleft\vrule height\delimhalfdim depth\delimhalfdim
width0pt\right.$};%
\node[at=(#5),yshift=(\delimvshift)]
{$\left.\vrule height\delimhalfdim depth\delimhalfdim
width0pt\right\delimright_{\rlap{$\scriptstyle#3$}}$};}}
\setpolymerdelim()
\setatomsep{2em}
\definesubmol{&}{-[,,,,draw=none]}



\begin{frame}[t]
\frametitle{Struktur}
  \begin{itemize}
    \item Lactame werden von Aminosäuren abgeleitet
  \end{itemize}
  \begin{center}
    \scalebox{0.5}{
    \schemestart
\chemname{\chemfig{H_2N-[:-30]-[:30]-[:-30]-[:30](=[2]O)-[:-30]OH}}{\\ \\ $\gamma$-Aminobuttersäure }
\arrow{->[][][2mm]}
\chemname{\chemfig{ *5(--{NH}-(=O)--)}}{\\ \\ $\gamma$-Butyrolactam}
\+{,,7pt}
\arrow(--[yshift=2mm]){0}[,0]
\chemfig{H_2O}
\schemestop
\chemnameinit{}
}
\end{center}
  \begin{center}
    \scalebox{0.5}{
    \schemestart
\chemname{\chemfig{H_2N-[:30]-[:-30]-[:30](=[2]O)-[:-30]OH}}{\\ \\ $\beta$-Alanin }
\arrow{->[][][2mm]}
\chemname{\chemfig{ *4(-\chembelow{N}{H}-(=O)---)}}{\\ \\ $\beta$-Lactam}
\+{,,4pt}
\arrow(--[yshift=2mm]){0}[,0]
\chemfig{H_2O}
\schemestop
\chemnameinit{}
}
\end{center}
 %Lactame entstehen aus Amino-carbonsäuren durch intramolekulare Kondensation
  \begin{itemize}
    \item \small Die griechischen Buchstaben geben an wie viele Kohlenstoffatome im Ring vorhanden sind
    \item $\alpha =2$ , $\beta = 3$, $\gamma =4$ , $\delta = 5$, $\epsilon = 6$, $\omega=n$
  \end{itemize}

\end{frame}

\begin{frame}[t]
\frametitle{\small Wichtige Lactame - \scriptsize{\textbf{nicht funktionalisiert}}}

 \begin{tabularx}{\textwidth}{llllr}
 ~ &  ~  & \scriptsize \textbf{Anwendung} & \scriptsize \textbf{Polymer} & \scriptsize \textbf{Bezeichnung}\\
\scalebox{0.3}{\chemfig{ *5([:-35]--*6(--*6(--{NH}-(=[:40]O)---[,,,,draw=none])-[,,,,draw=none]--)-[,,,,draw=none]--)}} &\scriptsize Laurinlactam  & \scriptsize Faserherstellung & \scalebox{0.3}{\chemfig{[:-30]-[@{leftd,.75}]N(-[6]H)-[:30](=[2]O)--[:30]--[:30]--[:30]--[:30]--[@{rightd,0.25}:30]}
 \makebraces[5pt,25pt]{\!\!\!n}{leftd}{rightd}} & \scriptsize PA12\\[2ex]
\scalebox{0.3}{\chemfig{ *7(---{NH}-(=[:60,0.8]O)---)}} & \scriptsize $\epsilon$-caprolactam  & \scriptsize Faserherstellung  & \scalebox{0.3}{\chemfig{[:-30]-[@{lefte,.75}]N(-[6]H)-[:30](=[2]O)--[:30]--[:30]--[@{righte,0.25}:30]} \makebraces[5pt,25pt]{\!\!\!n}{lefte}{righte}} & \scriptsize PA6 \\[2ex]
\scalebox{0.3}{\chemfig{ *5(--{NH}-(=[,0.8]O)--)} }&\scriptsize $\gamma$-butyrolactam  & \scriptsize Faserherstellung  & \scalebox{0.3}{\chemfig{[:-30]-[@{leftf,.75}]N(-[6]H)-[:30](=[2]O)--[:30]--[@{rightf,0.25}:30]} \makebraces[5pt,25pt]{\!\!\!n}{leftf}{rightf}} & \scriptsize PA4\\
\end{tabularx}
\begin{itemize}
\item \scriptsize Die Monomere werden durch Peptid-Bindungen miteinander verknüpft.(Polyamid)
  \item \scriptsize Die Polyamide werden häufig durch Nummern und/oder Buchstaben charakterisiert.
      \begin{enumerate}
      \item \scriptsize PA6 aus $\epsilon$-caprolactam
      \scalebox{0.5}{\chemfig{[:-30]-[@{leftx,.75}]N(-[6]H)-[:30](=[2]O)--[:30]--[:30]--[@{rightx,0.25}:30]}
   \makebraces[5pt,25pt]{\!\!\!n}{leftx}{rightx}}
      \item \scriptsize PA4 aus $\gamma$-lactam \scalebox{0.5}{\chemfig{[:-30]-[@{lefty,.75}]N(-[6]H)-[:30](=[2]O)--[:30]--[@{righty,0.25}:30]}
   \makebraces[5pt,25pt]{\!\!\!n}{lefty}{righty}}
    \end{enumerate}
\end{itemize}

\end{frame}

\begin{frame}[t]\frametitle{\small Wichtige Lactame - \scriptsize{\textbf{funktionalisiert}}}

 \begin{tabularx}{\textwidth}{lll}
 ~ &  ~  & \scriptsize \textbf{Anwendung} \\
%\scalebox{0.3}{\chemfig{ *5([:-35]--*6(--*6(--{NH}-(=[:40]O)---[,,,,draw=none])-[,,,,draw=none]--)-[,,,,draw=none]--)}} &\scriptsize Laurinlactam  & \scriptsize Faserherstellung PA12\\[2ex]
%\scalebox{0.3}{\chemfig{ *7(---{NH}-(=[:60,0.8]O)---)}} & \scriptsize $\epsilon$-caprolactam  & \scriptsize Faserherstellung PA6 \\[2ex]
\scalebox{0.3}{\chemfig{ *7(---N(-CH_3)-(=[:60,0.8]O)---)}} & \scriptsize $N$-Methyl-$\epsilon$-caprolactam  & \scriptsize unpolares Lösemittel \\[2ex]
\scalebox{0.3}{\chemfig{ *5(--N(-=[:60])-(=[,0.8]O)--)} }&\scriptsize $N$-Vinyl-$\gamma$-lactam  & \scriptsize Klebstoffe, Hilfsmittel PVP\\[2ex]
\scalebox{0.3}{\chemfig{ *5(--N(-CH_3)-(=[,0.8]O)--)} }&\scriptsize $N$-Methyl-$\gamma$-lactam  & \scriptsize unpolares Lösemittel NMP\\[2ex]
\scalebox{0.3}{\chemfig{[:-90]\chemabove{N}{H}(-[:210](-[:150]R)=[:-90]O)-[:-30]*4(-[,,,,red,line width=2pt](=[,,,,red,line width=2pt]\textcolor{red}{O})-[,,,,red,line width=2pt]\textcolor{red}{N}*5(-(-(=[::-60]O)-[::+60]OH)-(-[::+0])(-[::-108])-S-)-[,,,,red,line width=2pt]-[,,,,red,line width=2pt])}} & \scriptsize $\beta$-Lactam, Penicillin & \scriptsize Synthese der Antibiotika\\
\end{tabularx}
\end{frame}

\begin{frame}[t]\frametitle{Chemie der Lactame}

 \begin{itemize}
   \item \scriptsize Die Reaktivität aller Lactame wird durch dieRinggröße beeinflusst
     \end{itemize}
\begin{center}
\scalebox{0.5}{

 \begin{tabularx}{\textwidth}{XXXX}
 C-Zahl & $\Delta H$ \si{\kilo\joule\per\mol}&  $\Delta G$ \si{\kilo\joule\per\mol} & $T \Delta S$ \si{\kilo\joule\per\mol}\\[1ex]
 \hline \\[1ex]
4&  -4.6  & 4.6   & -9.2 \\[1ex]
5&  -7.1  & 0.4   & -7.5\\[1ex]
6&  -13.8 & -15.1 & 1.3\\[1ex]
7&  -22.6 & -27.6 & 5.0 \\[1ex]
8&  -35.1 & -47.8 & 12.7\\[1ex]
9&  -23.4 & -42.4 & 19\\[1ex]
10&  -11.7 & -36.8 & 25.1 \\[1ex]
11& -2.9  & -40.6 & 37.7\\
\end{tabularx}

}
\end{center}
\begin{itemize}
   \item \scriptsize steigender Entropie-Term bei größeren Ringen
   \item \scriptsize Zunahme von Freiheitsgraden beim Übergang zur Kette
\end{itemize}
\end{frame}

\begin{frame}[t]
\frametitle{Chemie der Lactame}
   \begin{columns}[onlytextwidth]
    \begin{column}{0.8\textwidth}
   \begin{itemize}
   \item \scriptsize Acides Proton \scalebox{0.4}{\chemfig{N-H}}
   \item \scriptsize Resonanzstabilizierung
        \end{itemize}
        \begin{center}
   \scalebox{0.7}{
   \centering
    \schemestart
    \chemname{\chemfig{ *5(-[0]-{NH}-(=\lewis{04,O})--)}}{}
\arrow{<->[][][1mm]}
\setbondoffset{3pt}
\chemname{\chemfig{ *5(--{\chemabove{N}{\hspace{3mm}\scriptstyle\oplus}H}=(-\chemabove{\lewis{024,O}}{\hspace{7mm}\scriptstyle\ominus})--)}}{}
\schemestop
\chemnameinit{}
}


\end{center}

    \end{column}
    \begin{column}{0.2\textwidth}
  \begin{tabularx}{\textwidth}{rX}
 ~ &  $pKs$  \\
\scalebox{0.3}{\chemfig{ *5(--{NH}-(=[:75,0.8]O)--)}} & \scriptsize 24.2 \\[2ex]
\scalebox{0.3}{\chemfig{ *6(---{NH}-(=[,0.8]O)--)} }& \scriptsize 26.6  \\[2ex]
\scalebox{0.3}{\chemfig{ *5(--{NH}---)} }& \scriptsize 41  \\[2ex]
\scalebox{0.3}{\chemfig{NH_3}} & \scriptsize 44  \\[2ex]
%\scalebox{0.6}{\chemfig{ *5(--{NH}-(=[,0.8]O)--)} }&\scriptsize $\gamma$-butyrolactam  & \scriptsize Kondensation & \scalebox{0.6}{\chemfig{H_2N-[:-30]-[:30]-[:-30]-[:30](=[2]O)-[:-30]OC_2H_5} } \\[4ex]
\end{tabularx}
    \end{column}
    \end{columns}
\end{frame}


\begin{frame}[t]
  \frametitle{\small Lactame zur Faserherstellung}
  \begin{itemize}
    \item \scriptsize Lactame werden größtenteils zur Faserherstellung eingesetzt
    \item \scriptsize Faser sind zweidimensionale Polymeren
    \item \scriptsize Seide
    \begin{figure}[htb]
\includegraphics[width=0.6\textwidth]{fibroin.png}
\end{figure}

    %%%% Polymer resimleri
  \end{itemize}
\end{frame}

\begin{frame}[t]\frametitle{Eigenschaften von Lactamefasern}

\begin{itemize}
  \item gute Reissfestigkeit (Fallschirme)
  \item hohe Schmelztemperatur
  \item \glqq{}seidenahnliches\grqq{} Verhalten (Damenstrümpfe)
\end{itemize}
\end{frame}
\begin{frame}
\begin{itemize}
  \item \scriptsize Die Polymerisation der Lactame gelingt über drei Wege
  \begin{enumerate}
    \item \scriptsize Hydrolytische Polymerisation
    \item \scriptsize Anionische Ringöffnende Polymerisation
    \item \scriptsize Kationische Ringöffnende Polymerisation
  \end{enumerate}

  \item \scriptsize Das wichtigste Verfahren für Polyamid 6-Fasern ist die hydrolytische Polymerisation.
\end{itemize}
\end{frame}




\end{document}
