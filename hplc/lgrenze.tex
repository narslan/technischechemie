\documentclass{article}
\usepackage{amsmath}


%encoding
%--------------------------------------

%----------------------------
\usepackage{microtype}
%\usepackage{fourier}
%\usepackage{mathpazo} % add possibly `sc` and `osf` options

%\usepackage[default,osfigures,scale=0.85]{opensans} %% Alternatively
%% use the option 'defaultsans' instead of 'default' to replace the
%% sans serif font only.

%\usepackage[default]{gillius}
\usepackage{mhchem}
\usepackage{listings}
\usepackage[utf8]{inputenc}
\usepackage{tgheros}
\renewcommand*\familydefault{\sfdefault} %% Only if the base font of the document is to be sans serif

\usepackage[T1]{fontenc}
\usepackage[german]{babel}

\usepackage{mathastext}

%\usepackage{sansmath}
%\sansmath

%\renewcommand{\sfdefault}{Myriad-LF}
%units
%--------------------------------------
\usepackage{siunitx}
\usepackage{verbatim}
%drawing
%------------------------------------
\usepackage{tikz} % To generate the plot from csv
\usepackage{pgfplots}
\usepackage{pgfplotstable}
\usepackage{caption}
\usepackage{tabularx}
\usepackage[landscape]{geometry}
\usepackage{fancyhdr}


\pagestyle{fancy}

\cfoot{\thepage}

\rhead{Gruppe 24}
\lhead{\textbf{Coffein-Bestimmung} Ermittlung des Standardfehlers und des Vertrauensbereichs}
\setlength{\headheight}{15pt}


\renewcommand{\familydefault}{\sfdefault}

\pagenumbering{gobble}

\definecolor{LightCyan}{rgb}{0.88,1,1}
\usetikzlibrary{datavisualization}
\pgfplotsset{compat=newest} % Allows to place the legend below plot
\usepgfplotslibrary{units} % Allows to enter the units nicely
\usepackage{overcite}
\renewcommand\citeform[1]{[#1]}

\sisetup{
  round-mode          = figures,
  round-precision     = 3,
  inter-unit-product =\cdot,
   group-digits=true,          %% Zifferngruppierung an/aus
        scientific-notation = true,
}



\usetikzlibrary{arrows, positioning, calc, datavisualization}

\pgfplotsset{
  standard/.style={
    axis x line=bottom,
    axis y line=middle,
  }
}
\usepackage{url}

\begin{document}

% section section_name (end)

\section{Ermittlung des Standardfehlers $s_{y,x}$}

 \begin{table}[ht!]
  \centering
 \begin{tabular}{llllll}
 $x$ [\si{\milli\gram\per\liter}] & $y$ [\si{\milli\volt.\second}]  & $y^2$ [\si{\milli\volt\square.\second\square}] & $x \cdot y$ [\si{\milli\gram\per\liter.\milli\volt.\second} ] &
 $(x-\overline{x})^2$ [\si{\square\milli\gram\per\square\liter}] & $x^2$ [\si{\square\milli\gram\per\square\liter}]\\
\hline
0  & \num{458.54}  & \num{210256.18 }   &\num{0}      & \num{15625}& 0\\
50  & \num{1859.12 } & \num{3456308.58}   &\num{92955.75}   &\num{ 5625} &\num{2500}\\
100 & \num{4196.75}  & \num{17612727.35}  &\num{419675.20}  &\num{ 625} &\num{10000}\\
150 &\num{6710.20}  & \num{45026851.14}  &\num{1006530.75} & \num{625} & \num{22500}\\
200 & \num{8742.89}  & \num{76438143.04}  &\num{1748578.20} &\num{ 5625}& \num{40000}\\
250 &\num{11323.00} & \num{128210396.94} &\num{2830750.75} & \num{15625}& \num{62500}\\
\hline
$\sum$ &  & & & &\\
\num{750.00} & \num{33290.50} & \num{270954683.23} & \num{6098490.65} & \num{43750.00} & \num{137500}\\
\hline
n = 6&   & & & &\\
\hline
\end{tabular}
 \caption{Die Messwerte und die Summen zur Ermittlung der Regressionskoeffizienten $m$, $b$ und des Standartfehlers der Schätzung $s_{y,x}$}
  \label{tab:mess}
\end{table}

  m wird wie folgt berechnet:
  \begin{align*}
    m &= \frac{n \cdot \sum xy - \sum x \cdot \sum y}{n \cdot \sum x^2 - (\sum x )^2} \\
      &= \frac{6 \cdot  \num{6098490.65} \, \si{\milli\gram\per\liter} \si{\milli\volt.\second} - 750 \, \si{\milli\gram\per\liter} \times \num{33290.50} \,  \si{\milli\volt.\second}}
      {6  \cdot \num{137500} \, \si{\square\milli\gram\per\square\liter}  - \num{562500} \, \si{\square\milli\gram\per\square\liter}}\\
      &= 44.3 \, \si{\liter.\milli\volt.\second\per\milli\gram}
  \end{align*}

  b wird wie folgt berechnet:

\begin{align}
  b &= \frac{\sum x^2 \cdot \sum y - \sum x \cdot \sum x y}{n \cdot \sum x^2 - (\sum x )^2}\\
    & = \frac{\num{137500} \, \si{\square\milli\gram\per\square\liter} \cdot  \num{33290.50} \,  \si{\milli\volt.\second} -   750 \, \si{\milli\gram\per\liter} \cdot  \num{6098490.65} \, \si{\milli\gram\per\liter} \si{\milli\volt.\second}}{6  \cdot \num{137500} \, \si{\square\milli\gram\per\square\liter}  - \num{562500} \, \si{\square\milli\gram\per\square\liter}}\\
    &  = 13.6 \, \si{\milli\volt.\second}
\end{align}



  Der Standardfehler der Schätzung $s_{y,x}$ wird errechnet nach:

\begin{align}
 s_{y,x} &= \sqrt{\frac{\sum y^2 - b \cdot \sum y - m \cdot \sum xy}{(n-2)}}
\end{align}
mit m = 44.3 \, \si{\liter.\milli\volt\second\per\milli\gram} , b = 13.6 \si{\milli\volt.\second} und n = 6
\begin{align*}
  s_{y,x} &=\sqrt{ \frac{ \num{270954683.23} \, \si{\square\milli\volt} \si{\square\second} - 13.6 \, \si{\milli\volt.\second} \times \num{33290.50} \, \si{\milli\volt.\second}
   -  44.3 \, \si{\liter.\milli\volt.\second\per\milli\gram} \times \num{6098490.65} \si{\milli\gram\per\liter.\milli\volt.\second} } {6-2}} \\
  &= 343 \, \si{\milli\volt.\second}
\end{align*}
\section{Ermittlung des Vertrauensbereich $ VB(\hat{x}) $}
\begin{flushleft}
  Der Vertrauensbereich $VB(\hat{x})$ wird errechnet nach:
\end{flushleft}
\begin{equation}
  VB(\hat{x}) = \hat{x} \pm \frac{t_{P,n-2} \cdot s _{y,x}}{m} \cdot \sqrt{1+\frac{1}{n}+\frac{(y-\overline{y})^2}{m^2 \cdot \sum (x -\overline{x})^2}}
\end{equation}
mit:\par\noindent
\begin{flalign*}
   y &=  \num{7781.223} \, \si{\milli\volt.\second},&\\
   \hat{x} & = 175 \, \si{\milli\gram\per\liter} ,\\
   s_{y,x} & = 343 \, \si{\milli\volt.\second} \\
    \overline{y} &= \num{5548.417} \, \si{\milli\volt.\second} ,
   \, m = 44.3 \, \si{\liter.\milli\volt.\second\per\milli\gram},\\
    \, n &= 6 ,    \, t_{p,n-2}= 2.78 \, \text{für f=4 Freiheitsgrade} \\
   \sum (x -\overline{x})^2 &= \num{43750.00} \, \si{\square\milli\gram\square\per\liter}
 \end{flalign*}
 daraus folgt:
\begin{flalign*}
   VB(\hat{x}) &= 175 \, \si{\milli\gram\per\liter} \pm \frac{2.78 \times 343 \, \si{\milli\volt.\second}}{44.3 \, \si{\liter.\milli\volt.\second\per\milli\gram}}
      \sqrt{1+\frac{1}{6}+\frac{(\num{7781.223}-\num{5548.417})^2 \si{\milli\volt^2\square\second}  }{44.3^2 \si{\square\liter.\milli\volt^2\square\second\per\milli\gram^2}
   \times \num{43750.00}  \si{\square\milli\gram\square\per\liter}}}&\\
    &=175 \, \si{\milli\gram\per\liter} \pm 23.8  \, \si{\milli\gram\per\liter}&
 \end{flalign*}


% \begin{thebibliography}{}
% \bibitem{crc1}
% D. R. Lide, \textit{CRC Handbook of Chemistry and Physics}, 88. Aufl., CRC Press, New York \textbf{2007}, S. 888.
% \bibitem{crc2}
% D. R. Lide, \textit{CRC Handbook of Chemistry and Physics}, 88. Aufl., CRC Press, New York \textbf{2007}, S. 1069.
% \bibitem{nist}
% \url{http://webbook.nist.gov/chemistry/fluid/}
% \bibitem{skript1}
% A. Brehm, \textit{Praktikumskript Wärmeübertragung}, Oldenburg, \textbf{2014},  S. 14.
% \end{thebibliography}
\end{document}
