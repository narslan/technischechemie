\documentclass[12pt]{article}
\usepackage{amsmath,mathtools}
\usepackage[usenames,dvipsnames]{xcolor}
\usepackage[german]{babel}
\usepackage[utf8]{inputenc}
\usepackage[T1]{fontenc}
\usepackage{textcomp}
%\usepackage{libertine}
%\usepackage{helvet}

\usepackage{microtype}
% Minion and Myriad fonts
\usepackage[minionint,mathlf]{MinionPro}
\renewcommand{\sfdefault}{Myriad-LF}
\usepackage{siunitx}
\usepackage{fancyhdr}
\usepackage{sectsty}
\usepackage{setspace}
\usepackage{booktabs} % To thicken table lines
\usepackage[version=4]{mhchem}
\usepackage[draft]{graphicx}
\usepackage[labelfont=bf]{caption}
\usepackage{subcaption}
\usepackage{chemstyle}

%\usepackage[compatibility=4.7,language=german]{chemmacros}
\renewcommand{\familydefault}{\sfdefault}
\sisetup{detect-all}
\usepackage{chngcntr}
\counterwithin{table}{section}
\counterwithin{figure}{section}
\usepackage{tikz}
\usetikzlibrary{arrows,chains,matrix,shapes}
\usepackage{pgfplots}
\pgfplotsset{compat=newest}
\usepgfplotslibrary{units} % Allows to enter the units nicely

\usepackage[version=4]{mhchem}
%\usepackage{titlesec}
\sisetup{
  round-mode          = figures,
  round-precision     = 1,
  inter-unit-product =\cdot,
   group-digits=true,          %% Zifferngruppierung an/aus
  scientific-notation = true,
}

% \usepackage{geometry}
%  \geometry{
%  a4paper,
%  left=20mm,
%  top=30mm,
%  right=20mm
%  }
\pgfplotsset{
  standard/.style={
    axis x line=bottom,
    axis y line=middle,
     ylabel= $\frac{V}{D}$, % Set the labels
    xlabel= $r-r_0$ ,
    x unit= \si{\angstrom},
    legend style={at={(0.1,-0.2)},anchor=north},
    y label style={at={(axis description cs:-.17,.5)},rotate=90,anchor=south},
    x tick label style={rotate=90,anchor=east},
   yticklabel style={/pgf/number format/sci},
scaled y ticks=false
  }
}
% colors
\definecolor{turquoise}{rgb}{0 0.41 0.41}
\definecolor{rouge}{rgb}{0.79 0.0 0.1}
\definecolor{vert}{rgb}{0.15 0.4 0.1}
\definecolor{mauve}{rgb}{0.6 0.4 0.8}
\definecolor{violet}{rgb}{0.58 0. 0.41}
\definecolor{orange}{rgb}{0.8 0.4 0.2}
\definecolor{bleu}{rgb}{0.39, 0.58, 0.93}



\begin{document}

\begin{onehalfspace}


% \begin{tikzpicture}
% \begin{axis}[standard]
% \addplot[domain=-0.3:6,smooth] {(1-e^(-1*(x-0.5)))^2};
% \addplot[domain=-0.3:6,smooth] {2*(1-e^(-1*(x-0.5)))^2};
% \addplot[domain=-0.3:6,smooth] {3*(1-e^(-1*(x-0.5)))^2};

% \end{axis}
% \end{tikzpicture}
\begin{figure}
\begin{tikzpicture}

    % styles
    \tikzstyle{elec} = [line width=2pt,draw=black!80]
    \tikzstyle{vib} = [thick,draw=black!30]
    \tikzstyle{trans} = [line width=2pt,->]
    \tikzstyle{transCI} = [trans,dashed,draw=vert]
    \tikzstyle{transCS} = [trans,dashed,draw=violet]
    \tikzstyle{relax} = [draw=orange,ultra thick,decorate,decoration=snake]
    \tikzstyle{rv} = [rotate=90,text=orange,pos=0.5,yshift=3mm]

    % fondamental
    \path[elec] (0,0)  -- ++ (14,0)
        node[below,pos=0.5,yshift=-1mm] {\large Grundzustand  $S_0$};
    \path[vib] (0,0.2) -- ++ (14,0);
    \path[vib] (0,0.4) -- ++ (13,0);
    \foreach \i in {1,2,...,30} {
        \path[vib] (0,0.4 + \i*0.2) -- ++ ({2 + 10*exp(-0.2*\i)},0);
    }

    % T1
    \path[elec] (11,4) -- ++ (3,0) node[anchor=south west] {\large $T_1$};
    \foreach \i in {1,2,...,6} {
        \path[vib] (11,4 + \i*0.2) -- ++ (3,0);
    }

    % S1
    \path[elec] (4,5) node[anchor=south east] {\large $S_1$} -- ++ (5,0);
    \foreach \i in {1,2,...,6} {
        \path[vib] (4,5 + \i*0.2) -- ++ (5,0);
    }
    \foreach \i in {1,2,...,12} {
        \path[vib] ({7.5 - 1*exp(-0.3*\i)},6.2+\i*0.2) -- (9,6.2+\i*0.2);
    }

    % S2
    \path[elec] (4,8) node[anchor=south east] {\large $S_2$} -- ++ (2,0);
    \foreach \i in {1,2,...,6} {
        \path[vib] (4,8 + \i*0.2) -- ++ (2,0);
    }

    % absorption
    \path[trans,draw=turquoise] (4.5,0) -- ++(0,9)
        node[rotate=90,pos=0.35,text=turquoise,yshift=-3mm] {\large Absorption};

    % fluo
    \path[trans,draw=rouge](7,5) -- ++(0,-4.4)
        node[rotate=90,pos=0.5,text=rouge,yshift=-3mm] {\large Fluoreszenz};

    % phosphorescence
    \path[trans,draw=mauve] (13,4) -- ++(0,-3.4)
        node[rotate=90,pos=0.5,text=mauve,yshift=-3mm] {\large Phosphoreszenz};

    % Conversion interne
    \path[transCI] (4,5) -- ++(-1.9,0) node[below,pos=0.5,text=vert] {\large IC};
    \path[transCI] (6,8) -- ++(1.3,0)  node[above,pos=0.5,text=vert]  {\large IC};

    % Croisement intersysteme
    \path[transCS] (9,5)  -- ++(2,0)    node[below,pos=0.5,text=violet] {\large ISC};
    \path[transCS] (11,4) -- ++(-2.5,0) node[below,pos=0.5,text=violet] {\large ISC};

    % relaxation vib
    \path[relax] (5.5,8.8) -- ++(0,-0.8) node[rv] {\textbf{D}};
    \path[relax] (8,8)     -- ++(0,-3)   node[rv] {\textbf{D}};
    \path[relax] (1,5)     -- ++(0,-5)   node[rv] {\textbf{D}};
    \path[relax] (11.5,5)  -- ++(0,-1)   node[rv] {\textbf{D}};

\end{tikzpicture}

\caption{Das Jablonski-Diagramm und die Erklärung zur thermische Variaton der Fluoreszenz und der Phosphoreszenz} \label{fig:jablonski}
\end{figure}


\end{onehalfspace}
\end{document}