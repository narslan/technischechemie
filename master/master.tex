\documentclass[11pt]{article}

\usepackage{amsmath,mathtools} \usepackage[usenames,dvipsnames]{xcolor}
\usepackage{microtype} \usepackage[ngerman]{babel}
%\usepackage[utf8]{inputenc}
\usepackage[T1]{fontenc}

\usepackage{libertine}
% header
%\setlength{\headheight}{15pt} \setlength{\topmargin}{-15mm}
\usepackage{csquotes} \usepackage{geometry} \geometry{ a4paper,
total={170mm,257mm}, left=20mm, top=20mm, }

\renewcommand{\thesection}{}% Remove section references...
\renewcommand{\thesubsection}{\arabic{subsection}}%... from subsections

\usepackage{endnotes} \let\footnote\endnote

\usepackage[ style=verbose, backend=biber, isbn=false, doi=false, url=false,
date=year ]{biblatex} \addbibresource{books.bib}

\usepackage[onehalfspacing]{setspace}

\usepackage{epigraph} \renewcommand{\epigraphrule}{0pt}

\usepackage{changepage}   % for the adjustwidth environment
\newenvironment{myenv}{\begin{adjustwidth}{2cm}{}}{\end{adjustwidth}}

\usepackage[hang]{footmisc}
\setlength\footnotemargin{10pt}



\title{Erfahrung als Kritik. Kritik der Erfahrung?} 
\author{Birgit Manz} 


\begin{document}

\begin{titlepage}
 
 \begin{center}
 % \rotatebox{0}{\includegraphics[width=0.5\textwidth]{unilogo.pdf}} 

  \begin{sffamily}
    Fakultät I - Bildungs- und Sozialwissenschaften \\[0.1em]
    Department für Pädagogik \\
  \end{sffamily}
  \vspace*{0.5cm}
  Masterstudiengang Pädagogik
  \vspace*{0.5cm}
  Masterarbeit

  \begin{sffamily}
    \huge \bfseries Erfahrung als Kritik. Kritik der Erfahrung?\\[0.75em]
  \end{sffamily}
 
  \begin{sffamily}
    \large \bfseries Praktiken des Gegenerzählens und ihre Bedeutung für eine
    dekoloniale Bildung an der Universität. \\[1em]
  \end{sffamily}

  vorgelegt von \\[0.75em]
  \large \textbf{Birgit Manz} \\[1em]
\vspace*{1cm}
Gutachter: \\[0.75em]
  \begin{bfseries}
Prof. Dr. phil. Paul Mecheril
    \\[0.2em]
Dr. Ulrike Lingen-Ali
 \end{bfseries}

  \vspace*{1.2cm}



  Oldenburg,
  \date{09.07.2017}

  \vfill


\end{center} 
\end{titlepage}


\section{Vorwort}

\setlength{\epigraphwidth}{0.7\textwidth}

\epigraph{\textit{We all need histories that no history book can tell, but they
    are not in the class room, not the history classrooms, anyway.  They are in
    the lessons we learn at home, in poetry and childhood games, in what is left
    of history when we close the history books with their verifiable
facts.}}{Michel-Rolph Trouillot \footnotemark} \footcitetext{troillot} Was sind
das für Geschichten, die wir alle brauchen, und die in keinem Geschichtsbuch
stehen? Was sind das für Geschichten, die nicht in den Klassenzimmern, sondern
in den Häusern der Schüler\_innen erzählt werden? Was sind das für Geschichten,
die sich uns erst eröffnen, wenn wir die Bücher schließen und uns der Dichtung
und dem Spiel zuwenden?

Die oben stehenden Zeilen bilden die letzten Worte eines Prologs, mit dem Michel
Rolph Trouillot das Kapitel \glqq The Haitian Revolution as a Non-Event\grqq
\footnotemark \footcitetext{haiti} einleitet.  In dem Prolog teilt Trouillot mit
den Leser\_innen zwei sehr unterschiedliche Erfahrungen aus seiner Lehre an
US-amerikanischen Universitäten.  Das eine Mal wird er von einer Studentin dazu
aufgefordert, von den Kämpfen der Sklaven zu sprechen, anstatt nur
\textit{weiße} \footnotemark \footnotetext{Die Schreibweisen weiß; Schwarz, of Color und der Gender
  Gap bzw. das * und ' weisen darauf hin, dass es sich hier nicht um
  widerspruchsfreie Kategorien handelt denen ein ontologischer Zustand vorausgeht.
  Viel eher kommen damit politische Positionen und Perspektiven in rassistischen
  und sexistischen Verhältnissen zum Ausdruck, die in dieser Arbeit verhandelt
  werden. Die Schreibweise Schwarz und of Color, stellt den Versuch dar,
  Selbstpositionierungen von 'Menschen' die Rassismus erfahren zu berücksichtigen.
  Die Bezeichnung weiß mit dem kursiven w soll das Ungleichgewicht, dass in
  rassistischen Verhältnissen zumeist die von Rassismus betroffenen markiert
  werden, die von Rassismus privilegiert und normalisierten darum unbemerkt
  bleiben können entgegen wirken. Da es sich hier jedoch nicht um eine
  Selbstbezeichnung im Sinne einer Wiederaneignung von Fremdbezeichnungen geht,
  sondern um die Markierung von rassistischen Verhältnissen privilegiert zu
  werden, wird hier eine andere Schreibweise als eben diskutierte Schreibweise mit
  Großbuchstaben gewählt. Das Gender\_Gap zeigt den Zwischenraum auf, indem sich
  alle 'Menschen' im Zweigeschlechterzwang verorten und das Sternchen weißt
  daraufhin, dass die Leser\_in die Subjektkategorien Frau\_* wie auch andere
  Subjektkategorien innerhalb einer Vielzahl von Differenzkategorien verorten kann
  und sollte. Die Ein-Strich Anführungseichen betont einzelne Begriffe die in
  dieser Arbeit auf ihren Normalitätsanspruch hin befragt werden. Dies geschieht
  punktuell, an Stellen da mit die Betonung auf den Konstruktionscharakter
  sinnvoll erscheint. Alle Schreibweisen sind im strategischen Sinne zu verstehen,
  ich gehe nicht davon aus das die Frage nach der Möglichkeit der Repräsentation
  von Subjekten durch Sprachordnungen damit für alle Zeiten geklärt ist. Viel eher
  soll die Schreibweise mich und die Leser\_innen dazu anregen, die sprachliche
  Ordnung als Werkzeug zu verstehen, mit dem bestehende Bezeichnungspraktiken und
damit auch Differenzpraktiken in Frage gestellt werden können.} Autor\_innen zu
rezipieren:

\begin{myenv}
  \textit{ \glqq Mr. Trouillot, you make us read all those white scholars. What can
    they know about slavery? Where were they when we were jumping off the boats?
    When we chose death over misery and killed our own children to spare them from a
  life of rape? \grqq} \footnotemark \footnotetext{Ebd., 70.}  
\end{myenv}

Das andere Mal und viele Jahre später fordert eine Studentin ihn dazu auf, 
endlich über die Schwarzen Millionäre zu sprechen: 
\glqq I'm tired to hear about that slavery stuff. Can we hear the stories of the
black millionaires? \grqq \footnotemark \footnotetext{Ebd., 71.}   

Troillot stellt den Leser\_innen beide Student\_innen als Schwarze Frauen\_* vor
und wünscht sich, die Zeit zurückdrehen und diese beiden Student\_innen
miteinander ins Gespräch bringen zu können. Worüber würden sie wohl sprechen?
Was verbindet und was trennt sie?

Die Worte der Student\_innen lösen vieles in mir aus. Nähe und Distanz,
Erinnerung und Sehnsucht, Unsicherheit und Mut. Nähe und Distanz, weil ich mit
der Geschichte die erzählt wird verbunden bin, es ist auch meine Geschichte,
weil Rassismus auch eine Geschichte über weiße ist und doch geht es nicht um
meine Perspektiven sondern um das Recht derer über die erzählt wurde selbst zu
erzählen. Erinnerung und Sehnsucht, weil ich ähnliche Auseinandersetzungen
miterlebt habe und die Sehnsucht nach universitären Räumen bleibt, in denen
Subjekte mit all ihren Erfahrungen als Subjekte in diesen Räumen vorkommen
können, ohne auf diese Erfahrungen reduziert zu werden. Unsicherheiten, weil es
unmöglich scheint, die Interventionen als Impuls aufzugreifen ohne sich zugleich
aneignend ihnen gegenüber zu verhalten. Mut, weil sie ein Zeichen dafür sind,
dass es Widerstand gibt.  Die Gedanken die nun folgen haben mehr mit mir als mit
den Subjekten und dem Kontext aus dem sie entstanden sind. Es sind meine
Lesarten und sie sind damit keine absoluten Deutungen sondern Versuche, jenen
Interventionen aufzugreifen um von ihnen zu lernen. 

Die Intervention der Schwarzen Frauen\_* in das Seminar, so unterschiedlich sie
auch sein mögen, verbinde ich mit einer geteilten Idee von Universität. Es sind
Forderungen nach einer Universität als Ort, an dem Geschichten erzählt werden in
denen sie, Schwarze Frauen\_*, sich wiederfinden. Geschichten, die ihnen etwas
darüber erzählen, woher sie kommen und was sie sein können. Und so
unterschiedlich ihre Vorstellungen davon auch sind, so ähnlich ist doch ihr
Verlangen, ihre eigene Geschichte zu hören oder vielleicht auch zu erzählen.
Die Interventionen dieser beiden Schwarzen Frauen\_* zeigen, dass Erfahrungen
Subjekte hervorbringen, die sich annehmend und abweisend gegenüber jenen
Erfahrungen verhalten können. Erfahrung geht nicht notwendig mit einem
unmittelbaren Erlebnis einher. Erfahrung schreibt sich über Generationen in die
Körper und Seelen der Menschen ein und sie hinterlässt Spuren.  Spuren, die in
den Geschichten, die an der Universität erzählt werden, aufgegriffen oder
verschwiegen werden. Spuren, die Sehnsüchte hervorbringen, die in den
Geschichten, die an der Universität erzählt werden, gestillt oder verstärkt
werden. Was die Forderungen der Schwarzen Frauen\_*, so gegensätzlich sie auch
sein mögen, verbindet, ist ihr Anspruch nach Repräsentation. Sie möchten die
Geschichte aus der Perspektive Schwarzer Menschen über Schwarze Menschen hören,
denn was sollen die Weißen schon wissen über Schwarze Sklaven oder Schwarze
Millionäre?  Ein Verständnis von Erfahrungswissen, also Wissen, das aus einer
bestimmten Erfahrung heraus entsteht, teilt auch Nicola Lauré al Samaray. Sie
zeigt die Notwendigkeit eines historisierenden Erfahrungsbegriffs für den
deutschen Kontext auf, der Kolonialität mitdenkt:

\begin{myenv} \textit{[...] die koloniale Erfahrung [ist] die, die komplexe
    kulturelle Landkarten mit ineinander widerhallenenden dominanten und
    unterworfenen Geschichtlichkeiten hervorbrachte und noch immer hervorbringt,
    die sich in die Körper Schwarzer und weißer Männer und Frauen einschrieb und
    noch immer einschreibt, derer man sich zu entledigen oder zu erinnern versucht
    und die – weder be- noch überwältigt – Schwarze und weiße deutsche
    Vergangenheiten und Gegenwarten untrennbar miteinander verknüpft. Es ist die
    koloniale Erfahrung, die den Ausgangspunkt einer gewaltvollen hierarchischen
    Begegnungs- und Beziehungsgeschichte markiert, über die im Zuge der
    Sichtbarmachung und verknüpfenden Gegenüberstellung weißer hegemonialer und
    Schwarzer unterworfener Perspektiven Zeugnis und Zeuginnenschaft abgelegt
wird.} \footnotemark \footcitetext{masken} \end{myenv}

Die koloniale Erfahrung konstituiert weiße und Schwarze Subjekte, sie
konstituiert ihre Beziehung und sie konstituiert die Geschichten, die von weißen
und Schwarzen über sie erzählt werden. Es sind unterschiedliche Erfahrungen
derselben Geschichte, die sich auch dadurch unterscheiden, dass die Erzählungen
Schwarzer über diese geteilte Geschichte unterworfen werden und die Erzählungen
weißer Perspektiven hegemonial (gesetzt) werden.  Die Geschichten, die wir
brauchen, finden in den Büchern und Seminaren keinen Platz.  Wenn nun Trouillot
seinen Prolog mit den oben zitierten Worten resümiert, klingt das zunächst wie
die Resignation eines Professors, der vermutlich einen großen Teil seines Lebens
mit diesen Büchern und in diesen Seminarräumen verbracht hat. Doch auch in der
Resignation können Fragen entstehen: Was sind das für Geschichten? Können sie
überhaupt erzählt werden? Oder liegt ihr Geheimnis vielleicht darin, dass sie
aus den Gedichten und Kinderzimmern niemals in die Oder liegt ihr Geheimnis
vielleicht darin, dass sie aus den Gedichten und Kinderzimmern niemals in die
wissenschaftlichen Diskurse, wie sie an der Universität geführt werden, Einzug
erhalten können?

Diese Arbeit besteht nicht in dem Versuch, diese Geschichten zu erzählen. Diese
Arbeit versucht, die Leere zu beschreiben, die entsteht, weil diese Geschichten
nicht erzählt werden.  Den Ausgangspunkt dieser Arbeit bildet darum jene
Geschichte, die erzählt und die vermutlich täglich von einer wie Troillots
Student\_innen unterbrochen wird. Es ist eine Geschichte, die sich vornehmlich
mit der Frage beschäftigt: Was ist der Mensch?  Ich interessiere mich dafür,
welche Erfahrungen hier erinnert, welche entinnert werden. Ich interessiere mich
dafür, von wem und für wen diese Geschichte, die an der Universität erzählt
wird, geschrieben wurde. Wer sich in ihr wiederfindet, und wer nicht.  Trouillot
beschreibt in dem Kapitel, das auf den Prolog folgt, die Umstände, unter denen
die Haitianische Sklavenrevolution stattfand. Die europäische Antwort auf die
Frage \glqq Was ist der Mensch?\grqq wurde von den versklavten Menschen in Haiti in Frage
gestellt. Die Revolution war nicht nur eine Befreiung aus der Versklavung,
sondern auch aus dem kolonialrassistischen Blick, mit dem Europa die Anderen
konstruiert. Dass diese Revolution nicht bis zu der europäischen Erzählung
durchdringen, sie nicht erschüttern konnte, zeigt, wie machtvoll diese Erzählung
ist und wie notwendig Gegenerzählungen sind, die aus den Erfahrungen von
Widerstand beschaffen sind:

\begin{myenv}

  \textit{ \glqq The silencing of the Haitian Revolution is only a chapter within
    a narrative of global domination. It is part of the history of the West, and
    it is likely to persist, even in attenuated form, as long as the history of
    the West is not retold in ways that bring forward the perspectives of the
world \grqq \footnotemark \footnotetext{Trouillot, 107.} } \end{myenv}

In dieser Arbeit möchte ich nach der Bedeutung fragen, die Erfahrung in den
Praktiken des Gegenerzählens einnehmen kann, um hieraus mögliche Strategien
für eine dekoloniale Bildung an und für die Universität abzuleiten.
%\printbibliography
\section{Vorhaben}
\epigraph{\textit{What, after all, is opposed by the movements of slaves and their descendents?
slavery? capitalism? coerced industrialisation? racial terror?
or the ethnocentrism and European solipsism that these processes help to reproduce?}}{Paul Gilroy \footnotemark} \footcitetext{gilroy}

\epigraph{\textit{Wenn die Erkenntnis ein imperiales Instrument der
Kolonialisierung ist, dann ist Dekolonialisierung der Erkenntnis eine der
dringlichsten Aufgaben.}}{Aníbal Quijano \footnotemark} \footcitetext{quijano}
Auf den ersten Blick scheint Aníbal Quijano auf Paul Gilroys Frage eine
unmittelbare Antwort zu liefern. Gilroy fragt, wogegen sich antikoloniale
Widerstände richten und beginnt mit einer Aufzählung: Ist es Sklaverei,
Kapitalismus, erzwungene Industrialisierung, rassistischer Terror? Oder der
Ethnozentrismus und europäische Solipsismus, mit Hilfe dessen sich jene Zustände
immer wieder von Neuem zementieren? Quijano schließt hieran an, so lässt sich
meine Collage weiterdenken, indem er die von Gilroy benannte ethnozentrische
Perspektive und solipsistische Methode der Erkenntnisproduktion als imperiales
Instrument der Kolonialisierung beschreibt. Dekolonialisierung müsse sich
entsprechend gegen dieses Instrument und damit zuallererst gegen Ethnozentrismus
und europäischen Solipsismus wenden.  Auf den zweiten Blick wird deutlich, dass
Gilroys Frage eine Beschreibung der Verhältnisse bestrebt; Quijanos Antwort
hingegen liest sich eher wie eine Forderung. Gilroy fragt, was passiert, Quijano
antwortet, was passieren sollte.  Der Unterschied mag trivial erscheinen, für
mich repräsentiert er ein grundlegendes Dilemma der vorliegenden Arbeit. Die
postkolonialen Theoretiker\_innen, mit denen ich mich in dieser Arbeit
auseinandersetze, sprechen zu unterschiedlichen Zeiten, von unterschiedlichen
Orten und aus unterschiedlichen Beweggründen. Mein Versuch sie zusammenzuführen,
kann nur gelingen, wenn der Anspruch nach einer kohärenten Erzählung der
Einsicht weicht, dass es gerade die Kohärenz ist, die sich so machtvoll, so
beschneidend auf die vielfältigen Stimmen und Dissonanzen legt, die sich
gegenüber den unterschiedlichen Kolonialismen zur Wehr setzten und setzen.
\subsection{Erzählung} Chimamanda Ngozi Adichie beschäftigt sich mit dieser
Macht, die von kohärenten Erzählungen ausgeht. Sie bezeichnet sie als Single
Story und zeigt auf, welchen Einfluss sie auf die Wirklichkeit, also auf die
Handlungs- und Denkräume von Menschen haben und wie sich ihnen widersetzt werden
kann. In ihrem Vortrag mit dem Titel \glqq The danger of a single story \grqq
zeigt sie anhand von autobiographischen Beispielen, wie sich Geschichten in die
Selbst- und Weltverständnisse ihrer Zuhörer\_innen bzw.  Leser\_innen
einschreiben. Dabei ist ihrer Ansicht der Kontext, in dem Geschichten
wirkmächtig werden, entscheidend: \begin{myenv}

  \textit{ \glqq How they are told, who tells them, when they're told, how many
stories are told, are really dependent on power. Power is the ability not just
to tell the story of another person, but to make it the definitive story of that
person \grqq \footnotemark \footnotetext{Chimamanda Ngozi Adichie, The danger of
a single story, Ted Talk, 2009.} } \end{myenv}

In dieser Arbeit suche ich nach jenen Erzählungen aus (Widerstands)erfahrungen
und frage nach den Strategien, die sie anwenden, um sich gegenüber der
hegemonialen Erzählung zu behaupten. Statt einer kategorialen Gegenüberstellung
beispielsweise feministischer und postkolonialer Ansätze, interessiere ich mich
für die verschiedenen divergierenden Möglichkeitsräume, die mit
unterschiedlichen Perspektiven aus unterschiedlichen Standorten einhergehen. Ich
möchte im Sinne D. Mignolos begreifen, \glqq [...] was diese Projekte wem
anzubieten haben. [und fragen:] Wer braucht sie? Wer profitiert von ihnen? Wer
sind die möglichen Agent\_innen von Emanzipations- oder Befreiungsprojekten?
Welche Subjektivität wird durch diese Projekte aktiviert? \grqq

\subsection{Universität}

Die Universität bildet den Rahmen für mein Nachdenken über die Möglichkeiten
widerständigen Wissens.  Denn die Universität stellt einen Ort dar, der
untrennbar mit der Entstehung, Behauptung, Kritik und Unterwanderung des
kolonialen Wissenssystems verbunden ist Das macht Universität zu einem
ambivalenten Ort, da hier die Kerben des Kolonialismus neben den Versuchen
(weiter)bestehen, ein irgendwie würdevolleres Bild vom Menschen zu zeichnen.
Mich interessiert hierbei, wie der Zustand des \glqq knowing about something
\grqq im Sinne eines Wissens um die gewaltvolle Entstehungs- und
Nutzungsgeschichte weißen Wissens in eine Praxis des \glqq learning to do
something \grqq, also in einen Umgang mit bzw. lernen aus der Kritik, an ihm
übersetzt werden kann. Ein Schritt, der Gayatri Chakravorty Spivak folgend
unabdingbar ist, \glqq as we move from the space of opposition to the menaced
space of the emergent dominant \grqq, und der für mich damit jene
Herausforderung benennt, dekoloniale Kritik auf Praxen der Bildung an der
Universität anzuwenden, umzusetzen, ja zu übersetzen.  Doch von welchem 'wir'
geht Spivak aus? Wer dekolonisiert, und wer erzählt wie über Strategien der
Dekolonialisierung?

\subsection{Dekolonisierung}

Dekolonial, wird im Glossar von Quix  als

\begin{myenv}

  \textit{ \glqq eine Haltung oder eine Vorstellung von der Welt [beschrieben],
    die versucht, Geschichte nicht von Europa aus zu denken und zu schreiben,
    und die jene Menschen und Weltgegenden, die seit der europäischen kolonialen
    Expansion im 15. Jahrhundert auf verschiedene Weise unterdrückt worden
    sind/werden, als Subjekte zu begreifen. Dekolonial bezieht sich nicht nur
    auf eine praktische politische Ent-kolonisierung von Nationalstaaten,
    sondern auf ein Dekonstruieren, Verlernen und Erneuern von Denkmustern und
    Strukturen \grqq \footnotemark \footnotetext{Quix-kollektiv, \glqq
Gender\_Sexualitäten\_Begehren \grqq, 91.} } \end{myenv}

Eine dekoloniale Haltung stellt Subjekte von Bildungsprozessen, so verstehe ich
das Quix Kollektiv, vor mehrere Aufgaben: Sie beansprucht von ihnen einen
Perspektivwechsel (Geschichte nicht von Europa aus denken), erwartet
Vorstellungskraft (Menschen als Subjekte begreifen) und erfordert Überwindung
des Bekannten und Vertrauten (bisher Gewusstes verlernen und neues Wissen
erlernen).  Hier werden, so scheint es, jene angesprochen, die die
Weltgeschichte bisher aus europäisch-kolonialer Sicht erfahren haben. Für sie
bedarf es einer Anstrengung, in Menschen das Menschliche zu erkennen, die im
Kolonialismus wie Dinge betrachtet und behandelt wurden. Dekoloniale Bildung
bedeutet für sie darum auch, Quix folgend, den sicheren Boden aus
Selbstverständlichkeiten und Gewissheiten über das Selbst und die zu Anderen
gemachten in der Welt zu verlassen. Dekolonialisierung beschränkt sich jedoch
nicht auf (Ver)Lernprozesse weiß positionierter Europäer\_innen. Die Möglichkeit
aus einer weißen Position heraus hegemoniale Wissensordnungen zu überschreiten,
wird von einigen postkolonialen Theoretiker\_innen hingegen infrage gestellt:


  \begin{myenv}
  \textit{ \glqq Indigenous knowledges are the starting point for resurgence and
  decolonization, are the medium through which we engage in the present, and are
  the possibility of an Indigenous future. Without this power base,
  decolonization becomes a domesticated industry of ideas. Decolonization is not
  always about the co-existence of knowledges, nor knowledge synthesis, which
  inevitably centers colonial logic. Whiteness does not ‘play well with others’
  but, rather, fragments and marginalizes - so it must be asked: Co-existence at
  what cost and for whose benefit? Decolonization necessarily unsettles. In the
  face of the beast of colonialism, thirsty for the blood of Indigeneity and
  drunk on conquest, assimilation is submission and decolonization calls on
  those who will 'beat the beast into submission and teach it to behave' \grqq
  \footnotemark \footnotetext{Sium, Aman, Chandni Desai und Eric Ritskes,
  \textit{Towards the 'tangible unknown':Decolonization and the Indegenous
  Future } }} 
  \end{myenv}

Aman Sium und Kolleg\_innen beschreiben das Verhältnis, weißen und Indigenen
Wissens als Kampf, der nicht ohne Verluste geführt werden kann. Hegemoniale
Diskurse spalten und verdrängen Indigenes Wissen und führen damit koloniale
Logiken fort, so ihre Argumentation. Dekolonialisierung kann also nicht in der
Zusammenführung unterschiedlicher Wissensbestände bestehen, sondern muss laut
der Autor\_innen die Unterwerfung hegemonialen Wissens zum Anliegen erklären.
Indigenes Wissen ist demnach Ausgangspunkt, Mittel und Ziel von
Dekolonialisierung. Nicht nur der Ton wird schärfer in dieser Ausführung.  Im
Unterschied zur Definition des Quix Kollektivs werden bei Aman et al.
nicht-weiße (Indigene) Subjekte adressiert, sie werden als Akteur\_innen der
Dekolonisierung angesprochen und dazu aufgerufen, sich nicht domestizieren zu
lassen, sondern selbst zu domestizieren. Gibt es also doch ein Außen, und damit
Perspektiven auf weißes Wissen, aus denen Erkenntnisse wachsen können?

Die Ansätze von Quix und Sium et al. unterscheiden sich in der Frage, wer den
Subjektstatus im dekolonialen Projekt beanspruchen und damit wer wen mit welchen
Mitteln befreien kann. Diese Perspektiven als einander konkurrierend und damit
als unvereinbar zu lesen, funktioniert jedoch nur, solange die Frage, was mit
weißem vs. Indigenem Wissen gemeint ist, nicht gestellt wird. Geht das Ergebnis
jahrhundertelanger Gewaltherrschaft in diesem Gegensatzpaar auf?

Es mag wie eine rhetorische Frage klingen, doch die Antwort ist komplex. Sie
muss aufzeigen, \glqq wie Differenz zu denken ist, ohne sie in Identität zu
vereinnahmen \grqq \footnotemark \footnotetext{Silvia Stoller und Helmuth
Vetter, \glqq Einleitung \grqq in Phänomenologie und Geschlechterdifferenz,
herausgegeben von ebd. (Wien: WUV Universitätsverlag, 1997)} 
und damit Wege finden, wie historisch verankerte, stetig
wachsende und widerstandene Machtverhältnisse und ihre Effekte anerkannt werden
können, ohne ihre Logik fortzuschreiben. Diese Arbeit befindet sich mitten in
diesem Spannungsfeld: radikale Infragestellungen europäischer Erkenntnispolitik,
auf die ich mich beziehe, bestreben weder deren Komplementierung, noch eine
vollständig unabhängige Neuschaffung eines alternativen Denk- und
Wissenssystems. Epistemische Widerstandsstrategien \footnotemark \footnotetext{
Walter D. Mignolo, Epistemischer Ungehorsam. Rhetorik der Moderne, Logik der Kolonialität und Grammatik der Dekolonialität (Wien: Turia + Kant, 2006), sowie Enrique Dussel,
\glqq Transmodernity and Interculturality. An Interpretation from the Philosophy
of Liberation,\grqq Transmodernity: Journal of Peripheral Cultural Production of the Luso-Hispanic World,
Vol. 1 Nr. 3, (2012), sowie Gayatri Chakravorty Spivak, \glqq Righting Wrongs,
\grqq zitiert in : M. Castro Varela & N. Dhawan, Hg., Postkoloniale Theorie. Eine kritische Einführung, (Bielefeld: transkript Verlag, 2015).
}
zeigen im Gegensatz dazu auf, wie die eigene Involviertheit in und gleichsame Distanzierung von
europäischen Wissensregimen genutzt werden kann, um radikale Neuordnungen und
damit einhergehende Perspektivwechsel zu ermöglichen.

\end{document}
