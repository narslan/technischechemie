\documentclass[10pt]{article}
\usepackage{amsmath}


%encoding
%--------------------------------------

%----------------------------
\usepackage{microtype}
%\usepackage{fourier}
\usepackage{helvet}
\usepackage{arevtext,arevmath}

\usepackage[utf8]{inputenc}
\usepackage[T1]{fontenc}
\usepackage[german]{babel}
\usepackage{mathastext}
%units
%--------------------------------------
\usepackage{siunitx}
%drawing
%------------------------------------
\usepackage{tikz} % To generate the plot from csv
\usepackage{pgfplots}
\usepackage{pgfplotstable}
\usepackage{graphicx}

\usepackage{caption}
\usepackage{color, colortbl}
\usepackage{tabularx}
\usepackage{setspace}
\usepackage{mhchem}
\usepackage{fancyhdr}
\pagestyle{fancy}

\cfoot{\thepage}

\rhead{\footnotesize Gruppe 24}
\lhead{\footnotesize Titration der schwachen Säure - Starken Base}
\setlength{\headheight}{15pt}


\renewcommand{\familydefault}{\sfdefault}
\captionsetup{font=footnotesize}

\definecolor{LightCyan}{rgb}{0.88,1,1}
\usetikzlibrary{datavisualization}
\pgfplotsset{compat=newest} % Allows to place the legend below plot
\usepgfplotslibrary{units} % Allows to enter the units nicely
\usepackage{overcite}
\renewcommand\citeform[1]{[#1]}

\sisetup{
  round-mode          = places,
  round-precision     = 2,
  inter-unit-product =\ensuremath{{}\cdot{}}
}
\makeatletter
\setlength{\@fptop}{0pt}
\makeatother
%\usetikzlibrary{arrows, positioning, calc, datavisualization}


\usepackage{url}

\begin{document}
%\begin{luacode*}
function string:split(sep)
        local sep, fields = sep or "%s", {}
        local pattern = string.format("([^%s]+)", sep)
        self:gsub(pattern, function(c) fields[#fields+1] = c end)
        return fields
end

function printHyperbola()
    local lines={}

    for line in io.lines("dampf.csv") do
            table.insert(lines, line)
    end
    tex.sprint("\\addplot[color=black] coordinates{")

    for i=2,#lines do
        local a=lines[i]:split()
        tex.sprint("("..a[1]..","..a[2]..")")
    end
     tex.sprint("};")
    tex.sprint("\\addplot[color=blue] coordinates{")

    for i=2,#lines do
        local a=lines[i]:split()
        tex.sprint("("..a[1]..","..a[3]..")")
    end
     tex.sprint("};")

end
\end{luacode*}
%\begin{figure}
\centering
     \begin{tikzpicture}
            \begin{axis}[standard,xlabel=Temperatur,ylabel=Druck]
                \directlua{printHyperbola()}
            \end{axis}
        \end{tikzpicture}
     \end{figure}

\noindent

\begin{table}[!htbp]
\scriptsize
\begin{tabular}{lllll}
\hline
 \ce{V_{\ce{NaOH}}} / \si{\milli\liter} &  Potential / \si{\milli\volt}  & \ce{pH}  & $\frac{\partial pH}{\partial V}$ & $\partial \frac{(\partial pH)}{(\partial V)^2}$\\
\hline
0.1 & - & - \\
0.2 & - & - \\
0.3 & - & - \\
0.4 & - & - \\
0.5 & - & - \\
0.6 & - & - \\
0.7 & - & - \\
0.8 & - & - \\
0.9 & - & - \\
1.0 & - & - \\
1.1 & - & - \\
1.2 & - & - \\
1.3 & - & - \\
1.4 & - & - \\
1.5 & - & - \\
1.6 & - & - \\
1.7 & - & - \\
1.8 & - & - \\
1.9 & - & - \\
2.0 & - & - \\
2.1 & - & - \\
2.2 & - & - \\
2.3 & - & - \\
2.4 & - & - \\
2.5 & - & - \\
2.6 & - & - \\
2.7 & - & - \\
2.8 & - & - \\
2.9 & - & - \\
3.0 & - & - \\
3.1 & - & - \\
3.2 & - & - \\
3.3 & - & - \\
3.4 & - & - \\
3.5 & - & - \\
3.6 & - & - \\
3.7 & - & - \\
3.8 & - & - \\
3.9 & - & - \\
4.0 & - & - \\
4.1 & - & - \\
4.2 & - & - \\
4.3 & - & - \\
4.4 & - & - \\
4.5 & - & - \\
4.6 & - & - \\
4.7 & - & - \\
4.8 & - & - \\
4.9 & - & - \\
5.0 & - & - \\
\hline
\end{tabular}
\caption{Messwerte der schwachen Säure - Starken Base - Titration}

\end{table}
\begin{figure}[!t]
\centering
\begin{tikzpicture}
\begin{axis}[
axis equal=false,
axis x line=bottom,
axis y line=middle,
legend style={draw=none,legend pos=outer north east},
xlabel= \ce{V_{\ce{AgNO3}}} / \si{\milli\liter},
ylabel= pH,
y label style={at={(axis description cs:-0.1,.5)},rotate=90,anchor=south}]
\addplot[only marks] table {
  X Y
  0.1 1
  0.2 2
  0.3 3
  0.4 4
};
\end{axis}
\end{tikzpicture}
\caption{Diagramm der schwachen Säure - Starken Base - Titration }
\end{figure}
%\addplot[domain=0:1200]{-1.42e-5*x^(2)+2.836e-2*x+45.47};

\end{document}

