\documentclass{article}
\usepackage{amsmath}


%encoding
%--------------------------------------

%----------------------------
\usepackage{microtype}
\usepackage{fourier}
\usepackage[utf8]{inputenc}
\usepackage[T1]{fontenc}
%units
%--------------------------------------
\usepackage{siunitx}
\usepackage{verbatim}
%drawing
%------------------------------------
\usepackage{tikz} % To generate the plot from csv
\usepackage{pgfplots}
\usepackage{pgfplotstable}
\usepackage{caption}
\usepackage{color, colortbl}
\usepackage{tabularx}

\definecolor{LightCyan}{rgb}{0.88,1,1}
\usetikzlibrary{datavisualization}
\pgfplotsset{compat=newest} % Allows to place the legend below plot
\usepgfplotslibrary{units} % Allows to enter the units nicely
\usepackage{overcite}
\renewcommand\citeform[1]{[#1]}

\sisetup{
  round-mode          = places,
  round-precision     = 2,
  inter-unit-product =\ensuremath{{}\cdot{}}
}




\usetikzlibrary{arrows, positioning, calc, datavisualization}

\pgfplotsset{
  standard/.style={
    axis x line=bottom,
    axis y line=middle,
  }
}
\usepackage{url}

\begin{document}
%\begin{luacode*}
function string:split(sep)
        local sep, fields = sep or "%s", {}
        local pattern = string.format("([^%s]+)", sep)
        self:gsub(pattern, function(c) fields[#fields+1] = c end)
        return fields
end

function printHyperbola()
    local lines={}

    for line in io.lines("dampf.csv") do
            table.insert(lines, line)
    end
    tex.sprint("\\addplot[color=black] coordinates{")

    for i=2,#lines do
        local a=lines[i]:split()
        tex.sprint("("..a[1]..","..a[2]..")")
    end
     tex.sprint("};")
    tex.sprint("\\addplot[color=blue] coordinates{")

    for i=2,#lines do
        local a=lines[i]:split()
        tex.sprint("("..a[1]..","..a[3]..")")
    end
     tex.sprint("};")

end
\end{luacode*}
%\begin{figure}
\centering
     \begin{tikzpicture}
            \begin{axis}[standard,xlabel=Temperatur,ylabel=Druck]
                \directlua{printHyperbola()}
            \end{axis}
        \end{tikzpicture}
     \end{figure}

\noindent
\section{Ziel des Versuches}
Der Versuch soll den Studenten an das Prinzip der Wärmeübertragung heranführen, indem
Wasser durch eine Rohrspirale geleitet und dabei über ein Wasserbad temperiert wird.
Davon ausgehend sollen der Wärmeübergangskoeffizient $\alpha _2$ sowie die Konstante $K$ und die
Exponenten m und n einer definierten Kriteriengleichung ermittelt werden.
\section{Versuchsdurchführung}
Das Wasserbad des Thermostaten wurde auf 40 \si{\celsius} geregelt. Die Durchflussrate des
wärmeaufnehmenden Fluidstromes wurde auf 120 \si{\liter\per\hour} eingestellt und die Ein- und die
Ausgangstemperatur wurden gemessen. Danach wurde der Vorgang wiederholt, wobei der
Fluidstrom von 120 – 420 \si{\liter\per\hour} in 50 \si{\liter\per\hour} Schritten erhöht wurde.
\section{Messwerte}
% subsection subsection_name (end)
\begin{table}[ht!]
  \centering
 \begin{tabularx}{\textwidth}{XXXX}
$\dot V$ [\si{\liter\per\hour}] & $T_{Ein}$ [\si{\celsius}]  & $T_{Aus}$ [\si{\celsius}] & $T_{Bad}$ [\si{\celsius}]  \\
\hline
\rowcolor{LightCyan}
120  & 20.2 & 32.1 & 40.0\\
170  & 19.8 & 30.6 & 40.1\\
\rowcolor{LightCyan}
220 & 19.6 & 29.2 & 40.1\\
270 & 19.5 & 27.9 & 40.1\\
\rowcolor{LightCyan}
320 & 19.4 & 26.9 & 39.8\\
370 & 19.4 & 25.8 & 38.9\\
\rowcolor{LightCyan}
420 & 19.4 & 25.1 & 38.3\\
\end{tabularx}
  \caption{Messergebnisse}
\end{table}
\section{Auswertung}
Damit der Wärmeübergangskoeffizient $\alpha _2$  berechnet werden kann, müssen zunächst der
Wärmeübergangskoeffizient $\alpha _1$ 1 sowie die Wärmedurchgangszahl $k _w$ bestimmt werden.
Berechnung von $\alpha _1$ :
Um $\alpha _1$  berechnen zu können müssen zunächst die Prandtl-, die Reynolds- und die
Nusseltzahl bestimmt werden.
Die Prandtl-Zahl berechnet sich nach folgender Formel:
\begin{equation}
  Pr_1 = \frac{\eta \cdot c _p}{\lambda _{H_2O}}
\end{equation}
Die Wärmekapazitäten sind aus Tabellenwerken zu entnehmen.
\subsection{Spezifische Wärmekapazität $c _p$}
Da die spezifische Wärmekapazität $c _p$ \cite{crc1} in 10 \si{\celsius} Schritten
tabelliert sind, wurden sie mittels R über ein Polynom vierten Grades interpoliert.
\begin{verbatim}
assign("t1", c(0,10,20,30,40,50,60,70,80,90,100))
assign("cp", c(4.2176,4.1921,4.1818,4.1784,4.1785,4.1806,
4.1843,4.1895,4.1963,4.2050,4.2159))
fitcp<-lm(cp ~ t1 +I(t1^2)+I(t1^3)+I(t1^4))
summary(fitcp)
--R2 = 0.9971
\end{verbatim}
Daraus wurden folgende Funktionen erhalten:
\begin{equation}
 cp (T) = -3.4 \cdot 10^{-9} T ^4 -8.323 \cdot 10^{-7} T ^ 3 + 7.966 \cdot 10^{-5} T ^ 2 - 3.049 \cdot 10^{-3} T + 4.217
\end{equation}
\subsection{Viskosität $\eta$}
Da die Viskosität $\eta$ \cite{crc2} in 5 \si{\celsius} Schritten
tabelliert sind, wurden sie mittels R über ein Polynom dritten Grades interpoliert.
\begin{verbatim}
assign("t2", c(0,25,50,75,100))
assign("vis", c(1.793,0.890,0.547,0.378,0.282))
fitvis<-lm(vis ~ t2 +I(t2^2)+I(t2^3))
summary(fitvis)
--R2 = 0.9992
\end{verbatim}
Daraus wurden folgende Funktionen erhalten:
\begin{equation}
 \eta (T) = -2.597 \cdot 10^{-6} T ^ 3 + 5.939 \cdot 10^{-4} T ^ 2 - 4.853 \cdot 10^{-2} T + 1.789
\end{equation}
\subsection{Wärmeleitzahl $\lambda _{H_2O}$}
Die Wärmeleitzahl wird über die Formel aus dem Skript berechnet.\cite{skript1}
\begin{equation}
  \lambda _{H_2O} = 0.100 + 1.66 \cdot 10 ^{-3} \cdot T
\end{equation}
Aus der Prandtlzahl $Pr$ lässt sich nun mithilfe der Reynoldszahl $R _{e_1} $, die für einen Rührkessel 850
beträgt, die Nusseltzahl $Nu$ berechnen:
\begin{equation}
  Nu = 3.6 \cdot Re ^{\frac{2}{3}} \cdot Pr ^{\frac{1}{3}}
\end{equation}
Über die Nusseltzahl $Nu$ lässt sich nun der Wärmeübergangskoeffizient $\alpha _1$ bestimmen:
\begin{equation}
  \alpha _1 = \frac{Nu _1 \cdot \lambda _{H_2O}}{d}
\end{equation}
Es ergibt sich folgende Tabelle:
\begin{table}[ht!]
  \centering
 \begin{tabularx}{\textwidth}{XXXXXXX}
$T _{Bad}$ [\si{\kelvin}] & $c _p$ [\si{\joule\per\gram\per\kelvin}]  & $\eta$ [\si{\meter\pascal\second}] &  $\lambda _{H_2O}$ [\si{\watt\per\meter\kelvin}]
& $Pr_1$ & $Nu _1$ & $\alpha _1$ [\si{\watt\per\square\meter\per\kelvin}] \\
\hline
\rowcolor{LightCyan}
313.15 & 4.160525 & 0.63183 & 0.6198 & 4.2411 & 522.8866 & 2700.84\\
313.25 & 4.160370 & 0.63049 & 0.6200 & 4.2308 & 522.4620 & 2699.37\\
313.25 & 4.160370 & 0.63049 & 0.6200 & 4.2308 & 522.4620 & 2699.37\\
313.25 & 4.160370 & 0.63049 & 0.6200 & 4.2308 & 522.4620 & 2699.37\\
312.95 & 4.160831 & 0.63454 & 0.6195 & 4.2619 & 523.7389 & 2703.79\\
312.05 & 4.162158 & 0.64701 & 0.6180 & 4.3575 & 527.6278 & 2717.30\\
311.45 & 4.163000 & 0.65558 & 0.6170 & 4.4233 & 530.2689 & 2726.50\\
\end{tabularx}
  \caption{Bestimmung von $Nu_1$ und $\alpha _1$}
\end{table}

\subsection{Bestimmung von $K _w$}
Um $K _w$ bestimmen zu können muss zunächst die aufgenommene Wärmemenge $\dot Q$ berechnet werden.
\begin{equation}
  \dot Q = \dot V \cdot \rho \cdot c _p \cdot (T^{Aus}-T^{Ein})
\end{equation}
Hierbei ist die Dichte $\rho$ des Wassers ebenfalls temperaturabhängig und sie lässt sich aus der Tabellenwerk \cite{crc3} entnehmen.
Damit kann die $\dot Q$ berrechnet werden:
\begin{table}[ht!]
  \centering
 \begin{tabularx}{\textwidth}{XXXXXX}
 $\dot V$ [\si{\milli\liter\per\second}] & $T _{Aus}$ [\si{\celsius}] &  $T _{Ein}$ [\si{\celsius}] & $c _p$ [\si{\joule\per\gram\per\kelvin}]
 & $\rho$ [\si{\gram\per\mol}] & $\dot Q [\si{\joule\second}] $\\
\hline
\rowcolor{LightCyan}
33.33  & 20.2 & 32.1 & 4.1605 & 0.99222 &1637.5025 \\
47.22  & 19.8 & 30.6 & 4.1604 & 0.99218 &2105.1964 \\
61.11  & 19.6 & 29.2 & 4.1604 & 0.99218 &2421.6638 \\
75.00  & 19.5 & 27.9 & 4.1604 & 0.99218 &2600.5367 \\
88.89  & 19.4 & 26.9 & 4.1608 & 0.99229 &2752.5007 \\
102.78 & 19.4 & 25.8 & 4.1622 & 0.99263 &2717.5979 \\
116.67 & 19.4 & 25.1 & 4.1630 & 0.99286 &2748.6284 \\
\end{tabularx}
  \caption{Bestimmung von $\dot Q$}
\end{table}
Aus der aufgenommen Wärmemenge $\dot Q$ lässt sich $K _w$ bestimmen über:
\begin{equation}
  \dot Q = k _w \cdot A \cdot \overline{\Delta T}
\end{equation}
wobei A = Außenfläche der Rohrspirale
 \begin{equation}
   A = \pi \cdot d \cdot l = \pi \cdot 8 \cdot 10^{-3} \si{\meter} \cdot 2.75 \si{\meter} = 6.9 \cdot 10^{-2} \si{\square\meter}
 \end{equation}
 $ \overline{\Delta T}$ = logarithmisches Mittel der Temperaturdifferenz
 \begin{equation}
   \overline{\Delta T} = \frac{\Delta T^{ein}-\Delta T^{aus}}{ln(\Delta T^{ein}/\Delta T^{aus})}
 \end{equation}
 Es erstellt sich folgende Tabelle:
 \begin{table}[ht!]
  \centering
 \begin{tabularx}{\textwidth}{XXXXX}
 $\Delta T _{Aus}$ [\si{\celsius}] &  $ \Delta T _{Ein}$ [\si{\celsius}] & $\overline{\Delta T}$ [\si{\kelvin}]
 & $\dot Q [\si{\joule\second}] $ & $ K _w$ [\si{\watt\per\square\meter\per\kelvin}] \\
\hline
\rowcolor{LightCyan}
 7.9  & 19.8 & 286.96 & 1637.5025 & 82.7015  \\
 9.5  & 20.3 & 288.02 & 2105.1964 & 105.9318 \\
 \rowcolor{LightCyan}
 10.9 & 20.5 & 288.82 & 2421.6638 & 121.5157 \\
 12.2 & 20.6 & 289.53 & 2600.5367 & 130.1730 \\
 \rowcolor{LightCyan}
 12.9 & 20.4 & 289.78 & 2752.5007 & 137.6589 \\
 13.1 & 19.5 & 289.44 & 2717.5979 & 136.0756 \\
 \rowcolor{LightCyan}
 13.2 & 18.9 & 289.19 & 2748.6284 & 137.7472 \\
\end{tabularx}
  \caption{Bestimmung von $K _w$}
\end{table}

\subsection{Bestimmung von $\alpha _2$}
Mittels $K _w$ und $\alpha _1$ lässt sich nun $\alpha _2$ und $Nu_2$ bestimmen.
\begin{equation}
  \alpha _2 = \frac{1}{-\frac{1}{\alpha _1} - \frac{\Delta z}{\lambda} + \frac{1}{K _w}}
\end{equation}

 \begin{table}[ht!]
  \centering
 \begin{tabularx}{\textwidth}{XXXX}
 $\alpha$ &  $K _w$ [\si{\watt\per\square\meter\per\kelvin}] & $ \alpha _2$ & $Nu _2 $  \\
\hline
\rowcolor{LightCyan}
2700.84 & 82.7015  & 98.93  & 522.89 \\
2699.37 & 105.9318 & 134.11 & 522.46 \\
2699.37 & 121.5157 & 160.10 & 522.46 \\
2699.37 & 130.1730 & 175.48 & 522.46 \\
2703.79 & 137.6589 & 189.38 & 523.74 \\
2717.30 & 136.0756 & 186.47 & 527.63 \\
2726.50 & 137.7472 & 189.67 & 530.27 \\
\end{tabularx}
  \caption{Bestimmung von $\alpha _2$ und $Nu _2$}
\end{table}
Mithilfe der Nusseltzahl $Nu_2$ lassen sich nun die Prandtlzahl $Pr _2$
und die Reynoldszahl $Re _2$ bestimmen.
\subsection{Reynolds Zahl $Re _2$}
Dazu wird zunächst $V _R$berechnet:
\begin{equation}
  V _R = \pi \cdot (\frac{d}{2}) \cdot l = \pi \cdot (\frac{6 \cdot 10^{-3} m}{2})^2 \cdot 2.75 m = 7.77 \cdot 10 ^{-2} l
\end{equation}
Dadurch ist es möglich die mittlere Geschwindigkeit zu berechnen:
\begin{equation}
  \overline{u} = \frac{l}{t} = \frac{l}{\frac{V_r}{\dot V}}
\end{equation}
Für die $Re_2$ gilt:
\begin{equation}
  Re _2 = \frac{v \cdot \rho \cdot d}{\eta}
\end{equation}
 \begin{table}[ht!]
  \centering
 \begin{tabularx}{\textwidth}{XXXX}
 $\overline u$ [\si{\meter\per\second}]  &  $\rho$ [\si{\kilo\gram\per\cubic\meter}] &
 $ \eta [\si{\kilo\gram\per\meter\per\second}] $ & $Re _2 $  \\
\hline
\rowcolor{LightCyan}
1.179751 & 0.9922 & 0.000632 &5094.8294 \\
1.671314 & 0.9922 & 0.000630 &7232.7840 \\
\rowcolor{LightCyan}
2.162877 & 0.9922 & 0.000630 &9360.0734 \\
2.654440 & 0.9922 & 0.000630 &11487.3628\\
\rowcolor{LightCyan}
3.146003 & 0.9923 & 0.000635 &13529.1788\\
3.637566 & 0.9926 & 0.000647 &15346.9013\\
\rowcolor{LightCyan}
4.129129 & 0.9929 & 0.000656 &17196.9626\\
\end{tabularx}
  \caption{Bestimmung von $Re _2$}
\end{table}
\section{Kriteriengleichung}
Für die Kriteriengleichung gilt:
\begin{equation}
  Nu_2 = K \cdot Re^{m}_2  \cdot \Pr^{n}
\end{equation}
Die Gleichung wird durch Logaritmieren linearisiert. Somit :
\begin{equation}
  ln Nu _2 = ln K + m \cdot ln Re _2 + n \cdot ln Pr
\end{equation}
 \begin{table}[ht!]
  \centering
 \begin{tabularx}{\textwidth}{XXX}
 $ln Nu_2 $& $ln Re_2$ & $ln Pr _1 $  \\
\hline
\rowcolor{LightCyan}
-0.043 & 8.54 & 1.44\\
 0.261 & 8.89 & 1.44\\
 0.438 & 9.14 & 1.44\\
 0.530 & 9.35 & 1.44\\
 0.607 & 9.51 & 1.45\\
 0.594 & 9.64 & 1.47\\
 0.612 & 9.75 & 1.49\\
\end{tabularx}
  \caption{Bestimmung von Kriteriengleichung $K$}
\end{table}
\begin{align}
  -4.57   & = ln K\\
   0.53   &= m
\end{align}
Die Steigung entspricht dem Exponenten
. K lässt sich aus dem Achsenabschnitt
berechnen.
(19)
Für die Kriteriengleichung folgt daher:
(20)
\begin{verbatim}
assign("nu", c(-0.043, 0.261, 0.438, 0.530, 0.607, 0.594, 0.612))
assign("re", c(8.54,8.89,9.14,9.35,9.51,9.64,9.75))
fitcp<-lm(nu ~ re +I(re^2)+I(re^3))
summary(fitcp)
--R2
\end{verbatim}
\section{Fehlerbetrachtung}
\section{Aufgabe 1.}
\section{Aufgabe 2.}
\begin{thebibliography}{}
\bibitem{crc1}
D. R. Lide, \textit{CRC Handbook of Chemistry and Physics}, 88. Aufl., CRC Press, New York \textbf{2007}, S. 888.
\bibitem{crc2}
D. R. Lide, \textit{CRC Handbook of Chemistry and Physics}, 88. Aufl., CRC Press, New York \textbf{2007}, S. 1069.
\bibitem{crc3}
\url{http://webbook.nist.gov/chemistry/fluid/}
\bibitem{skript1}
A. Brehm, \textit{Praktikumskript Wärmeübertragung}, Oldenburg, \textbf{2014},  S. 14.
\end{thebibliography}
\end{document}
