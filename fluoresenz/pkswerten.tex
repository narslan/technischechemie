\documentclass{article}
\usepackage{amsmath}


%encoding
%--------------------------------------

%----------------------------
\usepackage{microtype}
%\usepackage{fourier}
\usepackage{helvet}
\usepackage{arevtext,arevmath}

\usepackage[utf8]{inputenc}
\usepackage[T1]{fontenc}
\usepackage[german]{babel}
\usepackage{mathastext}
%units
%--------------------------------------
\usepackage{siunitx}
%drawing
%------------------------------------
\usepackage{tikz} % To generate the plot from csv
\usepackage{pgfplots}
\usepackage{pgfplotstable}
\usepackage{graphicx}

\usepackage{caption}
\usepackage{color, colortbl}
\usepackage{tabularx}
\usepackage{setspace}
\usepackage{mhchem}
\usepackage{fancyhdr}
\usepackage[landscape]{geometry}
\usepackage{listings}

\pagestyle{fancy}

\cfoot{\thepage}

\rhead{Gruppe 24}
\lhead{\textbf{Fluoreszenz} pKs-Wert Bestimmung}
\setlength{\headheight}{15pt}


\renewcommand{\familydefault}{\sfdefault}

\sisetup{
  round-mode          = places,
  round-precision     = 2,
  inter-unit-product =\ensuremath{{}\cdot{}}
}

\definecolor{LightCyan}{rgb}{0.88,1,1}
\usetikzlibrary{datavisualization}
\pgfplotsset{compat=newest} % Allows to place the legend below plot
\usepgfplotslibrary{units} % Allows to enter the units nicely
\usepackage{overcite}
\renewcommand\citeform[1]{[#1]}

\makeatletter
\setlength{\@fptop}{0pt}
\makeatother
%\usetikzlibrary{arrows, positioning, calc, datavisualization}

\begin{document}
%\begin{luacode*}
function string:split(sep)
        local sep, fields = sep or "%s", {}
        local pattern = string.format("([^%s]+)", sep)
        self:gsub(pattern, function(c) fields[#fields+1] = c end)
        return fields
end

function printHyperbola()
    local lines={}

    for line in io.lines("dampf.csv") do
            table.insert(lines, line)
    end
    tex.sprint("\\addplot[color=black] coordinates{")

    for i=2,#lines do
        local a=lines[i]:split()
        tex.sprint("("..a[1]..","..a[2]..")")
    end
     tex.sprint("};")
    tex.sprint("\\addplot[color=blue] coordinates{")

    for i=2,#lines do
        local a=lines[i]:split()
        tex.sprint("("..a[1]..","..a[3]..")")
    end
     tex.sprint("};")

end
\end{luacode*}
%\begin{figure}
\centering
     \begin{tikzpicture}
            \begin{axis}[standard,xlabel=Temperatur,ylabel=Druck]
                \directlua{printHyperbola()}
            \end{axis}
        \end{tikzpicture}
     \end{figure}

\noindent
\pagenumbering{gobble}

\begin{figure}[!t]



\begin{tikzpicture}[scale=1.5,transform shape]
\begin{axis}[
axis x line=bottom,
axis y line=middle,
xlabel= pH,
ylabel= Intensität,
legend style={draw=none,legend pos=outer north east,font=\scriptsize},
y label style={at={(axis description cs:-0.3,.5)},rotate=90,anchor=south},
  legend cell align=left,
 scaled ticks=false,
  y tick label style={/pgf/number format/sci},
 minor x tick num=10
]
\addplot[no marks,smooth,blue] table {
   X Y
2.000 1.77969086e2
3.986  1.91234894e2
 5.925 2.42990112e2
8.158  8.8750166e3
 9.973 1.22534625e5
 12.198 1.86032375e5
  };
  \addlegendentry{$I_f$=(360 nm/420 nm)}
\addplot[no marks,smooth,red] table {
   X Y
2.000 1.31233328e5
3.986  1.11589625e5
 5.925 1.12685039e5
8.158  1.11267211e5
 9.973 2.25562051e4
 12.198 8.50769043e2
  };
    \addlegendentry{$I_f$=(320 nm/350 nm)}

  \addplot[no marks,smooth,green] table {
   X Y
2.000 1.71453184e4
3.986  4.75454219e4
 5.925 4.93732383e4
8.158  5.58971602e4
 9.973 1.12899375e5
 12.198 1.59646828e5
  };
    \addlegendentry{$I_f$=(320 nm/420 nm)}
\node[coordinate,pin=right:{\tiny pH = pKs = 9.2}] at (axis cs:9.12,6.7e4) {};
\node[coordinate,pin=right:{\tiny pH = pKs = 2.9}] at (axis cs:2.9,3.2e4) {};

\end{axis}
\end{tikzpicture}
\caption{Kurvenverlauf der pH-Werte gegen Fluoreszenzintänsitäten}
\end{figure}

\end{document}


%\addplot[domain=0:1200]{-1.42e-5*x^(2)+2.836e-2*x+45.47};



