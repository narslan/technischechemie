% !TEX encoding = UTF-8
% !TEX program = lualatex
% Page Layout
\expandafter\gdef\csname ver@amssymb.sty\endcsname{9999/12/31}
\expandafter\gdef\csname ver@amsfonts.sty\endcsname{9999/12/31}
\documentclass{beamer}
%\documentclass[beamer]

\let\Tiny=\tiny
\usepackage{tikz}
\global\expandafter\let\csname ver@amssymb.sty\endcsname\relax
\global\expandafter\let\csname ver@amsfonts.sty\endcsname\relax

%\usepackage[bitstream-charter]{mathdesign}
%\usepackage{luatexja-fontspec}
\usepackage{chemfig}




\usepackage{siunitx}
\usepackage{graphicx}
\usepackage{url}
\usepackage[german]{babel}
\usepackage{adjustbox}
\usepackage{caption}
%\usepackage{microtype}

%\usepackage[T1]{fontenc}
\usepackage{fontspec,microtype}
\defaultfontfeatures{Ligatures=TeX, Scale=MatchLowercase}


\setmainfont[SmallCapsFeatures={LetterSpace=6}, Numbers={Proportional,OldStyle}]{Minion Pro}
\setsansfont[LetterSpace=3, Numbers={Proportional,OldStyle}]{Myriad Pro}

\usepackage{array} % needed for \arraybackslash
\usepackage{graphicx}
\usepackage{adjustbox} % for \adjincludegraphics
\usepackage{tabularx}

\sisetup{
  round-mode          = places,
  round-precision     = 2,
  inter-unit-product =\ensuremath{{}\cdot{}}
}
 % Avoid an error due to a lack of registers

\definecolor{orange}{RGB}{255,127,0}
%\include{head}
\title[]{Ephedrin und Ephedra-Alkaloiden}
\author[N. Arslan]{Nevroz Arslan}
\date[18.04.17]{18. Apr 2017}
%\titlegraphic{\includegraphics[width=\textwidth,height=.5\textheight]{bogen1200.png}}
\setbeamerfont{page number in head/foot}{size=\large}
\beamertemplatenavigationsymbolsempty


\setdoublesep{4pt}
\setatomsep{3em}
\let\otp\titlepage
\renewcommand{\titlepage}{\otp\addtocounter{framenumber}{-1}}

\renewcommand*\printatom[1]{\ensuremath{\mathsf{#1}}}
\begin{document}
\setbeamercolor{block title}{use=structure,fg=black,bg=white}
\setbeamercolor{block body}{use=structure,fg=black,bg=white}

\setbeamerfont{frametitle}{size=\large}
%%%%%%%%%%%% Start of content %%%%%%%%%%%%

%\input{seite11.tex}
\usebackgroundtemplate{

\includegraphics[width=1.0\paperwidth,height=0.17\paperheight]{bogen1200}
\begin{tikzpicture}[overlay, remember picture]
    \node[xshift=-10.80cm,yshift=1.15cm] at (0,0)    {\includegraphics[scale=0.6]{logo}};
\end{tikzpicture}
}


\addtobeamertemplate{frametitle}{\vskip+7ex}{}
\setbeamercolor{frametitle}{fg=black}
\setbeamertemplate{caption}{\insertcaption}

\setbeamertemplate{itemize/enumerate body begin}{\normalsize}
\setbeamerfont{frametitle}{size=\large}
\frame[plain]{\titlepage}
\defbeamertemplate{footline}{centered page number}
{%
 \hfill%
  \usebeamercolor[fg]{page number in head/foot}%
  \usebeamerfont{page number in head/foot}%
  \raisebox{.5cm}[0pt][0pt]{% <--- change here
    \insertframenumber\,/\,\inserttotalframenumber\kern1em}%
}

\setbeamertemplate{footline}[centered page number]
%\setatomsep{3em}
%\definesubmol{&}{-[,,,,draw=none]}


\begin{frame}[t,label=amphetamine]
  \frametitle{Vergleich der Alkaloide mit Amphetaminen}
\begin{table}[!htpb]
  \label{tab:label}
  \begin{tabular}{ccc}
    \chemname{\chemfig[][scale=0.35]{*6(-=-(-(<:[2]OH)-[:-30](<:[6]CH_3)-[:30]\chemabove{N}{H}-[:-30])=-=)}}{\tiny\textit{1S,2R}-ephedrin}&
    \chemname{\chemfig[][scale=0.35]{*6(-=-(-(<[2]OH)-[:-30](<[6]CH_3)-[:30]\chemabove{N}{H}-[:-30])=-=)}}{\tiny
    \textit{1R,2S}-ephedrin}&
    \chemname{\chemfig[][scale=0.35]{*6(-=-(-(<[2]OH)-[:-30](<:[6]CH_3)-[:30]\chemabove{N}{H}-[:-30]CH_3)=-=)}}{\tiny
    \textit{1R,2R}-pseudoephedrin}\\
    &&\\
    \chemname{\chemfig[][scale=0.35]{*6(-=-(-(<:[2]OH)-[:-30](<:[6]CH_3)-[:30]N-[:-30])=-=)}}{\tiny
    \textit{1S,2S}-pseudoephedrin}&
    \chemname{\chemfig[][scale=0.35]{*6(-=-(--[:-30](-[6]CH_3)-[:30])=-=)}}{\tiny
    Amphetamin }&
    \chemname{\chemfig[][scale=0.35]{*6(-=-(-(-[2]OH)-[:-30]-[:30]N-[:-30])=-=)}}{\tiny
    Epinephrine}
  \end{tabular}
\end{table}

\footnotetext[2]{Eitaro Matsumara, Motoki Matsumada in \textit{Phytochemistry, Botany and
Metabolism of Alkaloids,
Phenolics and Terpenes}, Springer Verlag \textbf{2014}, S.913}
\end{frame}


\begin{frame}[t,label=ephedraarten]
  \frametitle{Ephedra Arten}
  \begin{table}[htpb]
    \centering
  %  \caption{caption}
    \begin{tabular}{cc}
      Ephedra monosperma & adsda  \\ 
    \end{tabular}
  \end{table}
\end{frame}

\begin{frame}[t,label=ephedrin]
  \frametitle{Ephedrin}
  \begin{table}[htpb]
    \centering
  %  \caption{caption}
    \begin{tabular}{cc}
      Ephedra monosperma & adsda  \\ 
    \end{tabular}
  \end{table}
\end{frame}

\begin{frame}[t,label=pseudoephedrin]
  \frametitle{Pseudoephedrin}
    \begin{itemize}
      \item<1-> \translation{First|Erstens}.
      \item<2-> \translation{Second|Zweitens}.
      \item<3-> \translation{Third|Drittens}.
    \end{itemize}
\end{frame}



\end{document}
