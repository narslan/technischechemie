\documentclass[12pt]{article}
\usepackage{amsmath,mathtools}
\usepackage[usenames,dvipsnames]{xcolor}
\usepackage[german]{babel}
\usepackage[utf8]{inputenc}
\usepackage[T1]{fontenc}
\usepackage{textcomp}
\usepackage{libertine}
%\usepackage{helvet}
\usepackage{microtype}

\usepackage{siunitx}
\usepackage{fancyhdr}
\usepackage{sectsty}
\usepackage{setspace}
\usepackage{booktabs} % To thicken table lines
\usepackage[version=4]{mhchem}
\usepackage[draft]{graphicx}
\usepackage[labelfont=bf]{caption}
\usepackage{subcaption}
\usepackage{chemstyle}

%\usepackage[compatibility=4.7,language=german]{chemmacros}
\renewcommand{\familydefault}{\sfdefault}
\sisetup{detect-all}
\usepackage{chngcntr}
\counterwithin{table}{section}
\counterwithin{figure}{section}

\usepackage{titlesec}

\titleformat*{\section}{\large\bfseries}
\titleformat*{\subsection}{\normalsize\bfseries}

\usepackage{geometry}
 \geometry{
 a4paper,
 left=20mm,
 top=30mm,
 right=20mm
 }

\begin{document}

  \begin{titlepage}
  {\hfil \large \textbf{ Protokoll zum Praktikum Quantenchemie}\hfil}
\par
  \vspace{1cm}
\hfil \textbf{Modul AM12 - Quantenmechanik und Gruppentheorie}\hfil
\vspace{1cm}
  \end{titlepage}

%\begin{onehalfspace}
\tableofcontents
\newpage
\section{Berechnung der Hartree-Fock-Energie von Wasser}
\subsection{Berechnung der Gesamtenergie einer Geometrie von Wasser}
\textbf{Ziel dieser Aufgabe}\\
\textbf{Auswertung 1}\\
\textbf{Auswertung 2}\\
\subsection{Gesamtenergie in Abhängigkeit von der Geometrie}
\textbf{Ziel dieser Aufgabe}\\
\textbf{Auswertung 1}:\\
\begin{figure}[!htbp]
\centering
  \includegraphics[width=0.7\textwidth]{aufgabe1/molekul.png}%
  \caption{optimierte Geometrie von Wasser (RHF/BASISSATZ)}
\end{figure}
\noindent
\\
\textbf{Auswertung 2}:sdfsfdsfs\\
\textbf{Auswertung 3}\\
\newpage
\begin{figure}[!htp]
\centering
  \includegraphics{aufgabe1_teil2_geometrieoptimirung.png}%
  \caption{Auftragung der Gesamtenergie gegen die Optimierungsschritte (RHF/BASISSATZ)}
\end{figure}
\subsection{Gesamtenergie in Abhängigkeit vom Basissatz}
\textbf{Ziel dieser Aufgabe ...}\\
\textbf{Auswertung 1 }\\
\textbf{Auswertung 2 ...}
\subsection{Gesamtenergie in Abhängigkeit von der quantenchemischen Methode}
\textbf{Ziel dieser Aufgabe ...}\\
\textbf{Auswertung 1 }\\
\newpage
\textbf{Auswertung 2 }\\
\newpage
\section{Geometrieoptimierung und Frequenzberechnung an Dichlorethen}
\textbf{Ziel dieser Aufgabe ...}\\\\
\textbf{Auswertung 1 }\\
\begin{figure}[!hptb]
    \centering
    \begin{subfigure}[b]{0.4\textwidth}
        \includegraphics[width=\textwidth]{aufgabe1_teil2_geometrieoptimirung.png}
    \end{subfigure}
    ~ %add desired spacing between images, e. g. ~, \quad, \qquad, \hfill etc.
      %(or a blank line to force the subfigure onto a new line)
    \begin{subfigure}[b]{0.4\textwidth}
        \includegraphics[width=\textwidth]{aufgabe1_teil2_geometrieoptimirung.png}
    \end{subfigure}
    \caption{Optimierte Geometrien von cis- (li.) und trans-1,2-Dichlorethen (re.) (für BL und BW siehe Tab. 2.1)}
\end{figure}


\begin{table}[htbp]
\caption{Parameter für die optimierten Geometrien von cis- und trans-1,2-Dichlorethen in Abbildung 2.1}
\begin{tabular}{llrrr}
\toprule
BL\/ \si{\angstrom} bzw. BW\/\si{\degree} &  cis(RHF) & cis(MP2) &trans(RHF) &trans(MP2)\\
\midrule
\bottomrule
\end{tabular}
\end{table}

\begin{table}[htbp]
\caption{theoretische und experimentelle Schwingungsfrequenzen mit irreduziblen Darstellungen für cis-1,2-Dichlorethen}
\begin{tabular}{cc|cc|cc}
\toprule
Irrep & Freq (RHF)/\si{\per\centi\meter} &  Irrep & Freq (RHF)/\si{\per\centi\meter} & Irrep & Freq (RHF)/\si{\per\centi\meter} \\
\midrule
\bottomrule
\end{tabular}
\end{table}

\newpage


\begin{table}[htbp]
\caption{Schwingungsintensitäten mit irreduziblen Darstellungen und Schwingungsmodi für cis-1,2-Dichlorethen}
\begin{tabular}{cc|cc|cc}
\toprule
Irrep & Freq (RHF)/\si{\per\centi\meter} &  Irrep & Freq (RHF)/\si{\per\centi\meter} & Irrep & Freq (RHF)/\si{\per\centi\meter} \\
\midrule
\bottomrule
\end{tabular}
\end{table}

\begin{table}[htbp]
\caption{theoretische und experimentelle Schwingungsfrequenzen mit irreduziblen Darstellungen für trans-1,2-Dichlorethen}
\begin{tabular}{cc|cc|cc}
\toprule
Irrep & Freq (RHF)/\si{\per\centi\meter} &  Irrep & Freq (RHF)/\si{\per\centi\meter} & Irrep & Freq (RHF)/\si{\per\centi\meter} \\
\midrule
\bottomrule
\end{tabular}
\end{table}


\begin{table}[htbp]
\caption{Schwingungsintensitäten mit irreduziblen Darstellungen und Schwingungsmodi für trans-1,2-Dichlorethen}
\begin{tabular}{cc|cc|cc}
\toprule
Irrep & Freq (RHF)/\si{\per\centi\meter} &  Irrep & Freq (RHF)/\si{\per\centi\meter} & Irrep & Freq (RHF)/\si{\per\centi\meter} \\
\midrule
\bottomrule
\end{tabular}
\end{table}

\newpage
\noindent
\textbf{Auswertung 2 ...}\\
\begin{figure}[!hptb]
    \centering
    \begin{subfigure}[b]{0.4\textwidth}
        \includegraphics[width=\textwidth]{aufgabe1_teil2_geometrieoptimirung.png}
    \end{subfigure}
    ~ %add desired spacing between images, e. g. ~, \quad, \qquad, \hfill etc.
      %(or a blank line to force the subfigure onto a new line)
    \begin{subfigure}[b]{0.4\textwidth}
        \includegraphics[width=\textwidth]{aufgabe1_teil2_geometrieoptimirung.png}
    \end{subfigure}
    \caption{IR-Spektrum von cis-1,2-Dichlorethen auf RHF/- (li.) und MP2/BASISSATZ Niveau(re.)}
\end{figure}

\begin{figure}[!hptb]
    \centering
    \begin{subfigure}[b]{0.4\textwidth}
        \includegraphics[width=\textwidth]{aufgabe1_teil2_geometrieoptimirung.png}
    \end{subfigure}
    ~ %add desired spacing between images, e. g. ~, \quad, \qquad, \hfill etc.
      %(or a blank line to force the subfigure onto a new line)
    \begin{subfigure}[b]{0.4\textwidth}
        \includegraphics[width=\textwidth]{aufgabe1_teil2_geometrieoptimirung.png}
    \end{subfigure}
    \caption{IR-Spektrum von trans-1,2-Dichlorethen auf RHF/- (li.) und MP2/BASISSATZ Niveau (re.)}
\end{figure}

\begin{figure}[!hptb]
    \centering
    \begin{subfigure}[b]{0.4\textwidth}
        \includegraphics[width=\textwidth]{aufgabe1_teil2_geometrieoptimirung.png}
    \end{subfigure}
    ~ %add desired spacing between images, e. g. ~, \quad, \qquad, \hfill etc.
      %(or a blank line to force the subfigure onto a new line)
    \begin{subfigure}[b]{0.4\textwidth}
        \includegraphics[width=\textwidth]{aufgabe1_teil2_geometrieoptimirung.png}
    \end{subfigure}
    \caption{Raman-Spektrum von cis- (li.) und trans-1,2-Dichlorethen (re.) auf RHF/BASISSATZ-Niveau}
\end{figure}
\newpage
\noindent
\textbf{Auswertung 3}\\
\textbf{Auswertung 4}\\
\textbf{Auswertung 5}\\
\textbf{Auswertung 6}\\
\newpage
\section{Geometrieoptimierung und Berechnung der chemischen Verschiebung von Toluol}
\noindent
\textbf{Ziel dieser Aufgabe}\\
\textbf{Auswertung 1}\\

\begin{table}[htbp]
\caption{Parameter für die optimierte Geometrie von Toluol in Abbildung 3.1}
\begin{tabular}{cc}
\toprule
BL\/ \si{\angstrom} bzw. BW\/\si{\degree} & Toluol (MP2/BASISSATZ) \\
\midrule
\bottomrule
\end{tabular}
\end{table}
\begin{figure}[!htbp]
\centering
  \includegraphics{aufgabe1_teil2_geometrieoptimirung.png}%
  \caption{Optimierte Geometrie von Toluol (MP2/BASISSATZ, für BL und BW siehe Tab.3.1)}
\end{figure}

\begin{table}[htbp]
\caption{Chemische Verschiebung der Wasserstoffatome für die optimierten Geometrie von Toluol (Bezeichnung entsprechend Abbildung 3.1)}
\begin{tabular}{ccc}
\toprule
H-Atom & $\delta$ [Toluol (MP2/BASISSATZ)]/ppm & $\delta$ [Toluol/TMS(MP2/BASISSATZ)]/ppm\\
\midrule
\bottomrule
\end{tabular}
\end{table}
\noindent
\textbf{Auswertung 2}\\
\textbf{Auswertung 3}\\
\newpage
\textbf{Auswertung 4}\\
\section{Potentialkurve von \ce{N_2}}
\textbf{Ziele dieser Aufgabe ...}
\subsection{Berechnung der Potenzialkurve auf Hartree-Fock Niveau}
\textbf{Auswertung 1 und 2}\\
\begin{figure}[!htpb]
\centering
  \includegraphics{aufgabe4/stickstoff_potentialkurve.png}%
  \caption{Potenzialkurve für Stickstoff auf RHF/BASISSATZ-Niveau}
\end{figure}
\\
\noindent
\textbf{Auswertung 3}\\
\textbf{Auswertung 4}\\
\textbf{Auswertung 5}\\
\newpage

\subsection{Geometrieoptimierung des Stickstoffmoleküls}
\textbf{Auswertung 1}\\
\textbf{Auswertung 2}\\
\textbf{Auswertung 3}\\
\textbf{Auswertung 4} Die Orbitalenergien haben experimentell folgende Werte:\footnote[1]{G. L. Miessler, \textit{Inorganic Chemistry}, 5. Aufl., Pearson, Boston \textbf{2014}, S. 132.}\\
\begin{scheme}[!htpb]
\centering
\begin{tabular}{lrr}
\toprule
Orbital & Energie (eV) & HF-Energie (eV)\\
\ce{\sigma _u^*(2s)}  & - 18.7 & -15.66636 \\
\ce{\pi _u(2p) }& - 16.9  & -0.62551   \\
\ce{\sigma _g^*(2p) } & - 15.6 & -0.63513 \\
\midrule
\bottomrule
\end{tabular}
\end{scheme}
\begin{scheme}[!htpb]
   \centering
\includegraphics{aufgabe4/compchem}
\caption{MO-Diagramm für ein Stickstoff-Molekül (RHF/BASISSATZ-Niveau)}
\end{scheme}
\\
\textbf{Auswertung 5 und 6}\\
\newpage

\begin{figure}[!hptb]
    \centering
    \begin{subfigure}[b]{0.4\textwidth}
        \includegraphics[width=\textwidth]{aufgabe4/n2_Orbital1.png}
    \end{subfigure}
    ~ %add desired spacing between images, e. g. ~, \quad, \qquad, \hfill etc.
      %(or a blank line to force the subfigure onto a new line)
    \begin{subfigure}[b]{0.4\textwidth}
        \includegraphics[width=\textwidth]{aufgabe4/n2_Orbital2.png}
    \end{subfigure}
    \caption{1$\sigma$ gerade (li.) und ungerade (re.) (beide doppelt besetzt)}
\end{figure}

\begin{figure}[!hptb]
    \centering
    \begin{subfigure}[b]{0.4\textwidth}
        \includegraphics[width=\textwidth]{aufgabe4/n2_Orbital3.png}
    \end{subfigure}
    ~ %add desired spacing between images, e. g. ~, \quad, \qquad, \hfill etc.
      %(or a blank line to force the subfigure onto a new line)
    \begin{subfigure}[b]{0.4\textwidth}
        \includegraphics[width=\textwidth]{aufgabe4/n2_Orbital4.png}
    \end{subfigure}
    \caption{2$\sigma$ gerade (li.) und ungerade (re.) (beide doppelt besetzt)}
\end{figure}

\begin{figure}[!hptb]
    \centering
    \begin{subfigure}[b]{0.4\textwidth}
        \includegraphics[width=\textwidth]{aufgabe4/n2_Orbital5.png}
    \end{subfigure}
    ~ %add desired spacing between images, e. g. ~, \quad, \qquad, \hfill etc.
      %(or a blank line to force the subfigure onto a new line)
    \begin{subfigure}[b]{0.4\textwidth}
        \includegraphics[width=\textwidth]{aufgabe4/n2_Orbital6.png}
    \end{subfigure}
    \caption{3$\sigma$gerade (li.) und ungerade (re.) (gerade doppelt besetzt, ungerade nicht besetzt)}
\end{figure}
\newpage
\begin{figure}[!hptb]
    \centering
    \begin{subfigure}[b]{0.4\textwidth}
        \includegraphics[width=\textwidth]{aufgabe4/n2_Orbital7.png}
    \end{subfigure}
    ~ %add desired spacing between images, e. g. ~, \quad, \qquad, \hfill etc.
      %(or a blank line to force the subfigure onto a new line)
    \begin{subfigure}[b]{0.4\textwidth}
        \includegraphics[width=\textwidth]{aufgabe4/n2_Orbital8.png}
    \end{subfigure}
    \caption{1$\pi x$  (li.) und y (re.) ungerade (beide doppelt besetzt)}
\end{figure}

\begin{figure}[!hptb]
    \centering
    \begin{subfigure}[b]{0.4\textwidth}
        \includegraphics[width=\textwidth]{aufgabe4/n2_Orbital9.png}
    \end{subfigure}
    ~ %add desired spacing between images, e. g. ~, \quad, \qquad, \hfill etc.
      %(or a blank line to force the subfigure onto a new line)
    \begin{subfigure}[b]{0.4\textwidth}
        \includegraphics[width=\textwidth]{aufgabe4/n2_Orbital10.png}
    \end{subfigure}
    \caption{1$\pi x$ x (li.) und y (re.) gerade (beide nicht besetzt)}
\end{figure}
\newpage
\section{\ce{S_N}2-Substitution bei \ce{CH_3Br}}
\textbf{Ziele dieser Aufgabe ...}
\subsection{Bestimmung des Reaktionspfades und Auswahl einer geeigneten Übergangszustandsgeometrie}
\textbf{Auswertung 1}\\
\textbf{Auswertung 2}\\
\textbf{Auswertung 3}

\subsection{Optimierung und Frequenzrechnung des Übergangszustandes}
\textbf{Auswertung 1}\\
\textbf{Auswertung 2}\\
\textbf{Auswertung 3}
\begin{figure}[!htpb]
\centering
  \includegraphics{aufgabe1_teil2_geometrieoptimirung.png}%
  \caption{Optimierte Geometrie des Übergangszustands (RHF/aug-cc-pVDZ-Niveau)}
\end{figure}
\newpage
\subsection{Optimierung und Frequenzrechnung der Edukte und Produkte}
\textbf{Auswertung 1}\\
\textbf{Auswertung 2}\\
\subsection{Optimierung und Frequenzrechnung der Edukte und Produkte}
\textbf{Auswertung 1}\\
\begin{figure}[!htpb]
\centering
  \includegraphics{aufgabe1_teil2_geometrieoptimirung.png}%
  \caption{: Reaktionspfad: Auftragung der Energie der optimierten Geometrien gegen den C-Cl-Bindungsabstand}
\end{figure}
\noindent
\textbf{Auswertung 2}\\
\textbf{Auswertung 3}\\
%\end{onehalfspace}
\end{document}