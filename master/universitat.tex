\section{Vorwort}

\setlength{\epigraphwidth}{0.7\textwidth}

\epigraph{\textit{Hier kommt Vorwort}}{Nevroz \footnotemark}
\footnotetext{narslan} 

\section{Universität}

Ausgang vorliegender Arbeit bilden spezifische Überlegungen über die Idee der
Universität. Bevor ich diese Überlegungen darstelle, die zugleich als
Perspektive auf die Universität verstanden werden können, möchte ich die
Kriterien offen legen, die mich in der Auswahl jener Perspektiven leiteten.

In meinen Recherchen bin ich auf einen vordergründig deutschsprachigen Diskurs
gestoßen, der aus unterschiedlichen Positionen und entsprechend verschiedenen
Beweggründen heraus die Idee der Universität befragt und auf die Probe stellt.
Hier kommen Studierende\footnotemark \footcitetext{lohman} und Lehrende\footnotemark \footnotetext{ Die Markierung der Autor\_innen als Studierende
  und Lehrende soll darauf aufmerksam machen, dass die Autor\_innen im
  universitären Kontext durchaus unterschiedlich positioniert sind, aber
  allesamt im Bereich der Wissenschaften tätig sind. Stimmen von Angehörigen
  der Universität im Bereich der Verwaltung, Reinigung, etc. kommen hier nicht
  zu Wort.} zu Wort, die sich alle im Bereich der Geistes- und
  Sozialwissenschaften verorten und die das Anliegen verbindet, die Universität
  einer Kritik zu unterziehen. Dieses geteilte Selbstverständnis, nämlich
  Kritik zu üben, zeichnet sich durch eine Aufmerksamkeit für die Diskrepanz
  zwischen dem Gewünschten und Tatsächlichen aus, also zwischen dem, wie etwas
  sein könnte, und dem, wie es erlebt und gelebt wird.\footnotemark
  \footnotetext{Mit dem Begriff der Kritik setze ich mich in dieser Arbeit zu
  einem späteren Zeitpunkt ausführlicher auseinander.} Der Text \glqq Die
  unbedingte Universität\grqq\footnotemark \footcitetext{derrida} von Jacques Derrida dient hierbei als fester
  Ausgangs- oder Bezugspunkt, anhand dessen die Kritik an der Universität aus
  eben jenen Blickwinkeln geübt und eine Krise der Universität konstatiert
  wird.  Bei näherer Betrachtung fällt jedoch auf, dass angesichts der
  unterschiedlichen Ausgangspositionen bzw. Blickwinkel von Krisen im Plural
  gesprochen werden muss, da hier durchaus verschiedene und nicht selten
  einander widerstreitende Vorstellungen zum Ausdruck kommen - sowohl im
  Hinblick auf das, was die Universität gegenwärtig ist, als auch auf das was sie sein
  sollte. \\

  Für Julian Nida-Rümelin\footnotemark \footcitetext{nida} besteht die Krise
  der Universität in ihrer Öffnung zur Massenuniversität. Die Idee der
  Universität ist für ihn untrennbar mit dem \glqq Humboldtschen Bildungsideal\grqq\footnotemark \footnotetext{Rümelins Rückgriff auf das Humboldsche
  Bildungsideal ist meiner Ansicht nach insofern widersprüchlich, als er zum
einen Bildung als Zweck an sich und nicht als Mittel begreift und damit Bildung
von Ausbildung abgrenzt. Zum Anderen schließt er in seiner Argumentation an
Ökonomietheorien an, die Berufsausbildungen als notwendig für die Ökonomie
begreifen und eine nur an Selbstzweck ausgerichtete Bildung für die breite
Bevölkerung problematisieren. Vgl.: ebd. } verknüpft und geht angesichts der
Forderung, universitäre Bildung für alle zu gewährleisten, verloren. Denn weder
brauche die Gesellschaft (und damit ist in erster Linie die Ökonomie gemeint)
eine vollakademisierte Bevölkerung, noch sei universitäre Bildung die richtige
Form der Bildung für alle. Um das Besondere an universitärer Bildung bewahren
zu können, müsse diese streng von der Idee der Berufsausbildung getrennt werden
und dürfe nur denen vorbehalten werden, die sich in jener intellektuellen
Tätigkeit, die für ihn Bildung kennzeichnet, zurecht finden. 

Lucyna Darowska und Claudia Macholds \footnotemark \footcitetext{lucyna} Perspektive unterscheidet sich sowohl ontologisch als
auch normativ von den Ansichten Rümelins. Sie konstatieren nicht eine Öffnung
zur Massenuniversität, sondern fokussieren in ihrem Beitrag Praxen der
Grenzziehung, die zu Benachteiligung und Ausschluss führen und nach wie vor
konstitutiv für den universitären Raum gelten. Ganz im Gegensatz zu Rümelin
fordern sie dazu auf, eben jene Grenzziehungen zu problematisieren; und sie
entlarven die im Zuge der Internationalisierung vorangetriebene Pluralisierung
der Zugänge zur Universität als einseitige, auf ökonomische Verwertbarkeit
orientierte Fokussierung auf Differenz.  

Während sowohl für Rümelin als auch
für Darowska und Machold die Frage, für wen die Universität offen sein sollte,
im Vordergrund steht, problematisiert Ingrid Lohman \footnotemark
\footcitetext{lohman} die neoliberale
Vereinnahmung der Universität durch Konzerne. Die Universität als Raum für
Kritik und den Gebrauch der öffentlichen Vernunft weicht ihrer Ansicht nach der
\glqq unternehmerischen Hochschule\grqq\footnotemark \footcitetext{lohmein}, die ihren Auftrag zunehmend darin sieht, dem
Modernisierungsbedarf europäischer Konzerne entgegen zu kommen, und ihre
Wissensproduktion entsprechend nach dem ökonomischen Nutzen ausrichtet. 

\subsection{Praxen der Überschreitung}

Die Perspektive, für die ich mich im Folgenden interessiere, unterscheidet sich
von den bisher dargestellten Perspektiven nicht nur darin, was hier
problematisiert wird, sondern auch, wie es problematisiert wird. Bisher
dargestellte Positionen gehen bei aller Unterschiedlichkeit (darin, was die
Universität ist und was sie sein sollte) davon aus, dass die Krise der
Universität temporär und damit überwindbar sei. Sie beziehen sich in ihren
Analysen auf gegenwärtige gesellschaftliche Veränderungsprozesse und erläutern
die Gefahren, die von diesen für ihre Idee der Universität ausgehen. Ihre
Forderungen sind entsprechend lösungsorientiert und erinnern in ihrer
Dringlichkeit an Rettungsversuche.

Die Perspektive auf Universität, die ich in
dieser Arbeit einnehmen möchte und im Folgenden darstellen werde, begreift Krise als konstitutives Moment von
Universität. Universität ist Krise. Eine Überwindung der Krise würde das Ende
der Universität bedeuten und darum orientiert sich diese Perspektive weniger an
einer Auflösung als an einem Umgang mit Krise. Bildung und Kritik bilden hier
gewissermaßen das Spannungsfeld, auf dem die Krise zur Praxis wird und damit
nicht nur den Kern universitären Tuns, sondern auch den Kern ihres an
Reflexivität orientierten Selbstverständnisses darstellt. Die Krise wird hier
nicht als Fehlentwicklung, sondern als Notwendigkeit verstanden, um
Bildungsprozesse und damit auch Praxen der Kritik zu ermöglichen.
\\

Paul Mecheril und Birte Klingler verdeutlichen in dem Text \glqq Universität als
transgressive Lebensform. Anmerkungen, die gesellschaftliche Differenz- und
Ungleichheitsverhältnisse berücksichtigen\grqq\footnotemark
\footcitetext{metcheril} gleich zu Beginn, dass \glqq die Idee
der Universität weder präzise formuliert werden, noch sich der Wandlung
verschließen [kann].\grqq\footnotemark \footnotetext{Mecheril und Klingler,
\glqq Universität als transgressive Lebensform,\grqq 2.} Sie müsse viel eher mittels der Frage, was die
Universität ist, und was sie sein soll, ständig neu gestellt werden.

Damit wird deutlich, dass es ihnen nicht um die Suche oder gar die Verteidigung der
Idee der Universität geht. Ihr Anliegen besteht viel eher darin, Bezugspunkte
zu skizzieren, die für eine empirische und normative Auseinandersetzung mit
Universität dienlich sein können.

Dafür entwerfen sie die Universität als
einen Ort der Aufklärung, an dem \glqq irgendwie sinnvollere Welt- und
Selbstverhältnisse\grqq\footnotemark \footnotetext{Ebd. 85.} erprobt und damit Bildungsprozesse ermöglicht werden, in denen bisherige Deutungsmuster in Frage gestellt werden können. Bildung
verstehen sie dabei im emanzipatorischen Sinne als Ermächtigung und die
Universität als Zugang zu einer Lebensform, in der eine transgressive Praxis
der Überschreitung von (epistemischen) Grenzen ge- und erlebt werden kann. Vor
dem Hintergrund dieser normativen Skizze werfen die Autor\_innen anschließend
die Frage auf, wie unter gegebenen Verhältnissen, Bildung als Bürger\_innenrecht
eingefordert werden kann, durch das Praxen der Kritik möglich werden. Dabei
problematisieren sie die faktische Mitwirkung der Universität an der
Aufrechterhaltung von (welt)gesellschaftlichen Differenz- und
Ungleichheitsverhältnissen.

Michael Wimmers Beitrag \glqq Die Agonalität des
Demokratischen und die Aporetik der Bildung. Zwölf Thesen zum Verhältnis
zwischen Politik und Pädagogik\grqq\footnotemark \footcitetext{wimmer} knüpft
an dieser Stelle an und setzt sich mit dem Prozess der Entdemokratisierung und
der Verdrängung des Pädagogischen\footnotemark \footnotetext{Wimmer, \glqq Die
Agonalität des Demokratischen,\grqq 33.} auseinander. Eine an Bildung und Kritik
orientierte Universität weicht seiner Ansicht nach zunehmend \glqq einer an
ökonomischen Effizienz und Steigerung orientierten Steuerungslogik\grqq\footnotemark \footnotetext{Ebd. 48.} mit dem
Effekt der Transformation von Bildung in Humankapital. In seiner durchgehend
Kapitalismus thematisierenden Analyse setzt er sich mit dem Verhältnis von
Politik zu Pädagogik und Bildung zu Demokratie auseinander und weist auf die
Gefahren hin, die in dem Diktat der Selbstoptimierung liegen.

Beide Beiträge skizzieren sowohl einen Bildungs- als auch einen Kritikbegriff, der sich durch
Reflexivität auszeichnet und damit das Spannungsfeld universitärer Krise
andeutet, das den Ausgangspunkt vorliegender Arbeit bildet. 

\subsubsection{Bildung an und für die Universität}

Die Auseinandersetzung darum, was Universität sein soll und was sie faktisch
ist, führt Wimmer zu der Erkenntnis, dass es niemals eine Universität geben
kann, die ihrem Ideal entspricht, da dies seiner Ansicht nach in einem
Widerspruch zu dem Anspruch steht, sich selbst in Frage zu
stellen.\footnotemark \footnotetext{Ebd. 48} Auch für
Mecheril und Klingler besteht das charakteristische Moment der Universität in
dem Versuch einen Ort zu bewahren, \glqq der sich entzieht, mitunter nicht
konservierbar ist.\grqq\footnotemark \footnotetext{Mecheril und Klingler,
\glqq Universität als transgressive Lebensform,\grqq84.} Bildung bildet in beiden Beitragen einen zentralen
Bezugspunkt, anhand dem sie sich mit der Krise der Universität
auseinandersetzen.
\\

Doch was ist mit Bildung, und insbesondere universitärer
Bildung gemeint? \\
Für Wimmer erweist sich die viel praktizierte Gegenüberstellung des
neuhumanistischen Verständnis von Bildung mit einem ökonomistischen Verständnis
des unternehmerischen Selbst als nicht produktiv. Stattdessen ginge es darum,
\glqq den Bildungsbegriff auf das Differenzverhältnis von Singularität und
Gleichheit, Verantwortung und Gerechtigkeit zu gründen.\grqq\footnotemark
\footnotetext{Wimmer, \glqq Die Agonalität des Demokratischen,\grqq 35.} Bildung müsse dabei
in ein Verständnis von Demokratie eingebettet sein, das Demokratie als
\glqq Bewegungs- und Artikulationsform unaufhebbarer Konflikte, Antagonismen und
Differenzen\grqq versteht. Sie wird damit zu einer Praxis, die sich dem
Partikularen widmen muss, ohne dabei das Ganze aus den Augen zu verlieren. Nur
so sei es möglich, sich auf den Widerspruch als eine grundlegende Kategorie von
Bildung einzulassen.

Bildung und Demokratie, so führt Wimmer aus, sollten
hierbei nicht im instrumentellen Sinne der gegenseitigen Nutzbarmachung
verstanden werden, sondern vielmehr als theoretische Konzepte in ihrer
wechselseitigen Verschränkung theoretisiert werden. Eine an Demokratie
orientierte Gesellschaft müsse demnach dafür Sorgen, dass das Recht auf Bildung
allen offen stehe, da Bildung notwendige Voraussetzung für Beteiligung an
demokratischen Prozessen darstelle. Zugleich stelle die oben skizzierte Idee
der Demokratie als Form, die nicht die Auflösung, sondern die Bearbeitung von
Konflikten als Auftrag versteht, einen wichtigen, wenn nicht sogar notwendigen
Ausgangspunkt für Bildungsprozesse dar.\footnotemark \footnotetext{Ebd., 37.} 

Die Idee einer demokratischen Bildung,
als auch einer Demokratie, die auf Bildung aufbaut, muss sich nach Wimmer
demnach fragen, wie Bildung verstanden bzw. Bildungsräume gestaltet werden
können, damit Bildung ihrer Verschränkung mit einem demokratischen Anspruch,
gerecht werden kann. Hierin liegt jedoch gerade die Schwierigkeit, da, so
Wimmer, weder Bildung noch Demokratie einfach bestimmbare Begriffe sind. In
Anlehnung an Chantal Mouffes und Ernesto Lacleaus Idee des Politischen,
beschreibt er Demokratie und Bildung als leere Signifikanten, deren einzige
Bestimmbarkeit auf ihre Unbestimmbarkeit beruht, da sie \glqq auf irreduzible
Differenzen, konstitutive Spaltungen und unlösbare Konflikte im Sozialen wie
auch im Subjekt\grqq\footnotemark \footnotetext{Ebd., 40.} verweisen. 

Was heißt dies nun für eine Idee von Universität,
die die Ermöglichung von Bildung als eine ihrer zentralen Aufgaben versteht?
Für Wimmer muss die Universität in erster Linie ein Ort der Selbstbildung
werden, der die Krise des Selbst als Möglichkeit versteht, sich vom Anspruch
der Kohärenz zu lösen und stattdessen dem nicht greif- und abschließbaren
Selbst- und Weltverständnis zu widmen.  Denn Bildung als Selbstbildung muss
sich, so führt er fort, immer dem Selbst entbehren, das es zu bilden
beansprucht. Das Subjekt ist damit nicht nur von sich selbst, sondern von der
Welt getrennt, und damit fehlt der Bildung \glqq gerade der eigentliche Sinn eines
Gebildes als einer integrierten Ganzheit und Identität.\grqq\footnotemark \footnotetext{Ebd., 40.}  Die Fragmentierung
des Selbst stellt das Subjekt damit vor die Herausforderung sich in einer Welt
zu orientieren, über die es nicht verfügen kann, sondern die es in einen
unbestimmten Zwischenraum entlässt: 
\begin{myenv}
  \textit{ \glqq Je mehr Selbstbildung […] desto mehr
Weltbindung, was bedeuten würde, dass Bildung nur im Grenzbereich zwischen Ich
und Welt, Selbst und Anderem, Erfahrung und Widerfahrnis, agency und pathos
lokalisierbar und daher kein rein pädagogischer Begriff wäre.\grqq}
\footnotemark \footnotetext{Ebd.}  
\end{myenv}

Bildung
zeichnet sich damit erst durch eine 
\begin{myenv}
  \textit{\glqq [...]Erfahrung des Unmöglichen aus, der
Paradoxie der Fremderfahrung, die zugleich mit einer Transformation des Selbst-
und des Weltverhältnisses die Unmöglichkeit der Selbstpräsenz und völliger
Selbstverfügung erfahrbar macht und so weniger eine Steigerung der
Selbstkompetenz als viel mehr als eine zunehmende Vergegenwärtigung der
Selbstfremdheit zu verstehen wäre.\grqq}
\footnotemark \footnotetext{Ebd., 46}  
\end{myenv}

Universität als Ort der Bildung grenzt sich, so kann Wimmer gelesen werden, von
der klassischen Ausbildung bzw. Verwertbarkeit von Bildung ab und stellt in
Wimmers Anspruch eher einen Raum dar, an dem die Grenzen des Selbst- und
Welterkennens erfahrbar gemacht werden können. Die Auseinandersetzung mit dem
Selbst führt dabei jedoch nicht zu einer Bestätigung des Selbst, sondern zu
dessen Verunsicherung. Bildung zeigt damit die Grenzen der Selbsterfahrung auf,
da es die Erfassung der Nicht-Erfahrbarkeit des Selbst ist, die durch
Bildungsprozesse möglich wird. Die Erkenntnis, nicht mehr über die Grenzen des
Selbst verfügen zu können, erfordert damit die Bereitschaft, sich mit dem
Unmöglichen auseinanderzusetzen, nämlich sich des vermeintlich eigenen Selbst
zu entbehren.  

Auch für Mecheril und Klingler stellt Bildung eine
Grenzerfahrung dar. Bildung, so heißt es zunächst, ist eine Praxis des
Infragestellens.  \footnotemark \footnotetext{Mecheril und Klingler,
\glqq Universität als transgressive Lebensform,\grqq 85.} Dabei bilden Fragen nicht nur das Gerüst, durch das
Bildungsprozesse ermöglicht werden sollen, sondern auch die Praxis mittels
derer \glqq die Normalität des ordentlichen Lernens\grqq\footnotemark
\footnotetext{Ebd., 86.} überschritten werden könne.
Überschreitung meint hier sowohl die Veränderung bisheriger Deutungsgrundlagen,
also auch die Hinwendung zum Streit.\footnotemark \footnotetext{Hier wird der
Bezug zu Kollers Verständnis tranformatorischer Bildung deutlich: Koller
bezieht sich in seiner Theorie der transformatorischen Bildung auf die
Neubestimmung des Humboldschen Bildungsbegriff durch Rainer Kokemohr. Kokemohr
versteht unter Bildung eine \glqq Veränderung grundlegender Figuren des Welt- und
Selbstverhältnisses von Menschen\grqq, der in der Regel die Einsicht vorausgeht,
dass bisherige Vorstellungen des eigenen Verhältnisses zum Selbst und der Welt
nicht mehr ausreichen. Diese Einsicht, so Kokemohr tritt zumeist dann ein, wenn
Menschen mit Problemen konfrontiert werden. Diese Problemlagen, oder
Krisenerfahrungen, wie sie von Koller benannt werden, lösen demnach
Bildungsprozesse aus und führen zu einer Neubestimmung des Selbst in und zur
Welt. Vgl.: Hans-Christoph Koller, \textit{Bildung anders denken. Einführung in
die Theorie transformatorischer Bildungsprozesse}, Kohlhammer Verlag, Stuttgart, 2012, S. 16 } Nicht die Akkumulation von Wissen,
sondern die Auseinandersetzung mit Wissen und Nicht-Wissen kennzeichne damit
Bildungsprozesse, die es den Subjekten ermögliche, zwischen jenen Erklärungen,
die sich für ihre spezifischen Lebens- und Erfahrungskontexte bewähren bzw.
nicht bewähren, zu unterscheiden. \footnotemark \footnotetext{Mecheril und Klingler,
\glqq Universität als transgressive Lebensform,\grqq88.} 

In diesem Sinne wird Bildung immer auch als
Praxis der Ermächtigung verstanden, da in der Auseinandersetzung mit
verschiedenen Formen des Wissens und ihrer Befragung die Kontingenz des eigenen
Denkens und Handelns greifbar und somit Zugänge geschaffen werden können, die
\glqq andere, weniger einem, äußeren Zwang unterliegende, erstrebenswerte Selbst-
und Weltverhältnisse\grqq\footnotemark \footnotetext{Ebd., 89.}  ermöglichen.
\\

Während also in beiden Ansätzen
Krisenerfahrungen als Ausgang für Bildungsprozesse verstanden werden, da die
Kohärenz des Selbst sowie bisherige Deutungsmuster irritiert werden, erkennen
besonders Mecheril und Klingler in ihrem Beitrage das emanzipatorische
Potential, das dieser Irritation folgen kann, da so unter Umständen für das
Subjekt neue Beschreibungen und Analysen des Selbst- und der Welt erfahrbar
werden. 

Unklar bleibt für mich insbesondere bei Wimmers Ansatz, wie sein
Subjekt- und Bildungsbegriff zu Dimensionen von Verantwortung steht.

Welche
Verantwortung tragen jene, die die zweifelsfrei schmerzhaften Prozesse der
Selbstentfremdung begleiten, also Bildungsprozesse ermöglichen, die zu jenen,
von Wimmer beschriebenen tiefgreifenden Erschütterungen führen?\\
Die Frage der
Verantwortung geht jedoch noch über die Rahmung des Bildungsprozesses hinaus
und knüpft unmittelbar an das Subjektverständnis an, das hier von Wimmer
gezeichnet wird:
Ist das mit sich selbst nicht übereinstimmende, fragmentierte
Subjekt überhaupt in der Lage Verantwortung für sich, sein Denken und Tun zu
übernehmen bzw. kann es für letzteres zur Rechenschaft gezogen werden?\\
Zudem
bleibt aus meiner Sicht noch offen, von welchem Maß an Eigeninitiative bzw.
Bereitschaft Wimmer eigentlich ausgeht, sich in jene Bildungsprozesse zu
begeben, die das eigene Selbstverständnis und entsprechend auch
Handlungsvermögen in dem Maße irritieren. So scheint dieser Lesart die Annahme
vorauszugehen, Subjekte hätten zunächst den Schein eines kohärentes
Selbstbildes, welches erst in Folge des Bildungsprozesses irritiert wird.\\
Was
ermutigt Subjekte, sich jenen Schein nehmen zu lassen? Von wessen Befreiung ist
hier die Rede? Sind hier jene mit eingeschlossen, die niemals in den Genuss des
Scheins gelangen – weil sie das Trugbild der Kohärenz gar nicht erwerben
konnten oder wollten?

Es stellt sich also die Frage, von welchen
(welt)gesellschaftlichen Positionen heraus diese Bildungsprozesse angestrebt
werden, bzw. welches Subjekt hier imaginiert wird. Sollen alle gleichermaßen
irritiert werden oder gibt es Unterschiede, die berücksichtigt werden müssen?
Woran könnten die sich orientieren und welche Aufgabe erhält hier Universität?
\\

Die Universität erhält den Auftrag, so lassen sich Mecheril, Klingler und
Wimmer lesen, das Transzendente zu ermöglichen, das im von Handlungszwang
bestimmten Alltag selten einen Platz geniest. Wenn Bildungsprozesse als
Krisenerfahrungen konzeptioniert werden, wird es jedoch unabdingbar sich von
einem universellen Subjektverständnis zu lösen, dem eine universelle
Krisenerfahrung zu Grund liegt. 

Die Thematisierung von Herrschafts- und
Machtverhältnissen, auf die ich im Folgenden eingehe, stellt für mich hier
einen notwendigen und von den Autor\_innen offen beschrittenen Weg dar, um
keine beliebige, sondern eine kritische Universität mit einem kritischen
Bildungsverständnis zu fordern.

\subsubsection{Kritik an und durch die Universität}

Was kann nun mit kritischer Universität gemeint sein? 

Universität wird bei
Mecheril und Klingler in Anlehnung an Derrida \footnotemark
\footnotetext{Derrida, \textit{Die unbedingte Universität}, zitiert in: Mecheril und Klingler, ebd., 84. } zu einem Ort, an dem nichts
außer Frage steht und in dem der \glqq Begriff des Menschen seinen Vollzug
erfährt \grqq\footnotemark \footnotetext{Mecheril und Klingler, Ebd. }. 

Die Kritik, oder vielmehr die Idee der Kritik wird nun, so verstehe
ich ihr Anliegen, im doppelten Sinne gedacht: 
Zum einen wird die Universität zu
einem Ort, an dem Kritik geübt wird, indem \glqq andere epistemische Sätze\grqq \footnotemark \footnotetext{Ebd., 85.} über
die Welt gesagt werden, die \glqq irgendwie sinnvollere Welt- und
Selbstverhältnisse\grqq\footnotemark \footnotetext{Ebd.} versprechen. 
Zum anderen gilt es, jenen Auftrag der Kritik
kritisch in Bezug auf seine Mitwirkung in der Aufrechterhaltung von Herrschaft
zu befragen, die z.B. darin bestehen könnte, bestimmten \glqq Auffassungen,
Bilder[n] und Darstellungen des Menschen\grqq\footnotemark
\footnotetext{Ebd., 84.} Vorrang zu bieten.

Interessant ist an dieser Stelle ihre eigene Praxis der Kritik, die ich als Versuch verstehe,
eben jene Doppeldeutigkeit der Kritik ernst zu nehmen und die ich in einer
klaren Distanz zu dogmatischen Forderungen erkenne. Statt einer Festlegung
dessen, was Gegenstand kritischer Auseinandersetzung an der Universität sein
muss, erweist sich ihre Perspektive eher als eine Offenheit beispielsweise
gegenüber der Frage, \textbf{wie} \textit{irgendwie} sinnvollere Selbst- und Weltverhältnisse
aussehen könnten, oder \textbf{woran} sich \textit{andere} epistemische Sätze erkennen ließen. 
\\


Zunächst möchte ich mich jedoch, wie eben angekündigt, der Beziehung von
Kritik und Herrschaft widmen. Hierfür beziehe ich mich auf einen Kritikbegriff,
der auf Michel Foucault zurückgeht und im folgenden von Mecheril et al.
\footnotemark \footcitetext{mecherilmigration} im
Zusammenhang mit der Konturierung einer Migrationsforschung als Kritik
expliziert wird: 
\\ 

Das Grundmotiv der Kritik in jener Migrationsforschung
besteht darin aufzuzeigen, was Menschen von einer freieren und würdevolleren
Existenz beraubt. Damit stehen jene (diskursiven) Praxen im Zentrum der
Analyse, die an der Herstellung und Aufrechterhaltung von
Herrschaftsverhältnissen mitwirken.\footnotemark \footnotetext{Mecheril, et.
al, \glqq Enleitung\grqq, 34, 39, 45.}
Damit knüpfe ich an den bereits
skizzierten Hegemoniebegriff an und konkretisiere seine Bedeutung für Praxen
der Kritik. 

Herrschaft wird von den Autor\_innen als historisch gewachsene,
asymmetrische Beziehung zwischen Subjektpositionen verstanden, in der die
Möglichkeiten der Selbstbestimmung unterschiedlich verteilt sind. Herrschaft
wirke jedoch nicht ausschließlich repressiv, nicht nur beschneidend und
einschränkend auf die Beherrschten, sondern erweise sich als komplexes Gefüge,
das zumindest den Schein erweckt, \glqq funktional und bedeutsam \grqq \footnotemark \footnotetext{Ebd., 47.} sowie
alternativlos zu sein. Mecheril et al. Weisen hier auf die Notwendigkeit hin,
sich neben der destruktiven Form der Macht, auch ihrer produktiven Kraft zu
widmen. Herrschaft als komplexe Dynamik zu begreifen, ermögliche so, die
hegemoniale, subjektivierende Gewalt, die mit herrschaftsförmigen Praxen der
Unterdrückung einhergehen, zu benennen, ohne dadurch den Subjekten eine
ausweglose, eindeutige Position zuzuweisen.\footnotemark \footnotetext{Ebd., 34.}

Dieses Verständnis von Herrschaft
spiegelt sich auch im Kritikbegriff wider. Die Autor\_innen grenzen sich hierbei
explizit von orthodoxen Weltanschauungen ab, die wenigen privilegierten
Gelehrten die Fähigkeit zusprechen, zwischen authentischem und entfremdetem
Bewusstsein zu unterscheiden und damit Kritikfähigkeit nur einer Elite
vorbehalten. Mit dieser Haltung würde ein Dualismus bedient werden, der es
jenen Kritiker\_innen allein ermögliche, Herrschaftsdynamiken zu erkennen und zu
problematisieren. \footnotemark \footnotetext{Ebd., 39.} 
Stattdessen wird auf die Eingebundenheit der Kritiker\_innen
in die Verhältnisse, die sie zu kritisieren beanspruchen, verwiesen:\\
Jede
Kritik sei von ihrem Gegenstand abhängig, denn ohne Gegenstand gäbe es auch
keine Kritik. Es gelte dieses konstitutive Verhältnis stets neu zu bestimmen
und damit auch die notwendige Mitwirkung an Herrschaftsverhältnissen in
Anbetracht der eigenen Eingebundenheit anzuerkennen und angesichts der
\glqq subjektivierenden Effekte\grqq\footnotemark \footnotetext{Ebd., 34.} die von der Praxis der Kritik ausgehen, als
machtvoll einzuschätzen. Das Maß, an dem sich die Kritik orientiert, steht
damit so Mecheril et al. immer wieder auf dem Prüfstand und erfordert
Zurückhaltung gegenüber dogmatischen Gewissheiten und damit verbundenen
Absolutheitsansprüchen.\footnotemark \footnotetext{Ebd., 42.} \\
Trotz, oder vielleicht auch gerade auf Grund der
komplexen und von Widersprüchen geprägten Verhältnisse, sei Kritik von einem
epistemischem Engagement getragen, das davon ausgeht, dass das Erkennen von
Herrschaftsverhältnissen die Möglichkeit ihrer Veränderung birgt.\footnotemark
\footnotetext{Ebd., 45.}
\\

Das Subjekt, das Kritik übt, ist dementsprechend nicht nur vom Gegenstand,
sondern auch von der Wirksamkeit der praktizierten Kritik abhängig, die jedoch
nicht eindimensional von Subjekt ausgeht, sondern sich wechselseitig und damit
subjektivierend auf das Subjekt rückbezieht. 
Die Idee der Kritik wird in der
Postmoderne, das ist bereits in der Darstellung des Bildungsbegriffs
angeklungen, im Wesentlichen durch eine doppelte Kontingenz gezeichnet:\\
 Das
sich selbst transparente, stets intentional handelnde Subjekt wird von einer
Zeichnung des Subjekts abgelöst, das weder sich selbst repräsentieren, noch
durch andere repräsentiert werden kann, und sich einer
\glqq wirklichkeitskonstitutiven Kraft der Sprache und der diskursiven Verfasstheit
unserer Wirklichkeit \grqq\footnotemark \footnotetext{Wimmer, \glqq Die Agonalität
des Demokratischen,\grqq48. } gegenübersteht. Kritik wird damit zu einer Praxis, die
Paradoxien nicht als Denkfehler entlarvt, sondern als konstitutiv im Umgang mit
der Welt  anerkennt und es notwendig macht, die Positioniertheit von Subjekten
in Macht- und Herrschaftsverhältnissen als Ausgangspunkt von Kritik ernst zu
nehmen.  

Ein Verständnis, in dem weder die Grenzen des Selbst, noch die
Beschaffenheit der Welt vorbestimmt sind, sondern sich vielmehr durch z.B.
Praxen der Infragestellung ständig erweitern, ist somit immer aufgefordert, das
\textit{Be}stehende zu verlassen, um sich dem \textit{Ent}stehenden zu widmen, bzw. \glqq dass etwas
nicht Denkbares dennoch möglich und wirklich sein könnte, bedeutet, dass es –
für das Denken- Unmögliches möglicherweise gibt.\grqq\footnotemark
\footnotetext{Ebd., 49.}
\\

Doch wie muss das Denken beschaffen sein, um an die Grenzen seiner Selbst zu
stoßen bzw. diese zu überwinden? \\
Unterschiedliche Formen von Wissen und die
Handlungsräume, die sich in ihnen entfalten, werden hier zu wichtigen
Bezugspunkten.  

Dabei wird ein Verständnis von Kritik deutlich, das nicht auf
der Bereithaltung konkreter Alternativen basiert, sondern frei von Pragmatismus
und der Affirmation des Selbst eine Veränderung der Verhältnisse anstrebt.
Sowohl Wimmer als auch Mecheril und Klingler sind hier der Ansicht, dass die
Überschreitung des Bestehenden einem Antrieb folgt, der einer normativen Idee
dessen, was sein sollte und könnte, entspringt, und auf der Einsicht beruht,
dass hierfür die Grenzen des eigenen Denkens überschritten werden müssen.

Interessant ist hier insbesondere der wechselseitige Bezug von Bildung und
Kritik, der zwar nicht expliziert wird, sich aber vermuten lässt, insofern
Kritik jene Krisenerfahrung vorausgeht, die im Rahmen des Bildungsbegriffs
beschrieben wurde und die zugleich, auf einem kritischen, also normativ
bewegten und gesellschaftlich positionierten Verständnis von Machtverhältnissen
ruht. Bildung und Kritik sind, so lässt sich resümieren, dermaßen aufeinander
angewiesen, dass die  eine nicht ohne die andere existieren kann. Kritik wird
dogmatisch, wenn sie nicht an einen selbstreflexiven Bildungsprozess geknüpft
ist, und Selbstreflexivität wird beliebig, solang sie keine freiere und
würdevollere Existenz aller zum Ziel hat. 

\subsubsection{Erfahrung als Krise - Krise als Erfahrung}

Welche Aufgabe kommt Universität in eben jener Überschreitung der Selbst- und
Weltverhältnisse zu? Und was bedeutet dies dafür, welche Erfahrungen an der
Universität gemacht und geteilt werden (können)? \\
Universität, das habe ich
bereits mit Wimmer und Mecheril et al. herausgearbeitet, ist mehr als ihr
sichtbares Produkt in Form von Konferenzen und Publikationen. Die
Beschreibungen, Analysen und Interpretationen (später werde ich sie allesamt
Erzählungen nennen), die an der Universität entstehen und für die
(Fach)öffentlichkeit zugänglich gemacht werden, sind nur ein Aspekt der
Wissensproduktion. 

Wenn Universität als Ort der Bildung und der Kritik gedacht
werden soll, an dem Reflexivität im (wort)wörtlichen Sinn \textit{geübt} wird, dann wird
es notwendig, sich über die (sicht- und zählbaren) Produkte hinaus jenen
Erfahrungen zu widmen, die in diesem reflexiven \textit{Üben} aufgehen.  

Anstelle von
Ergebnissen möchte ich darum Prozesse in den Vordergrund rücken.\\  

Eine Theorie
beispielsweise darüber, wie sich bestimmte Geschlechterbilder auf
Beziehungsgewalt auswirken, kann auf ihre Bedeutung im Rahmen eines bestimmten
wissenschaftlichen Diskurses untersucht werden, in dem unterschiedliche
Perspektiven auf Geschlecht und Gewalt diskutiert werden. Dieselbe Theorie kann
aber auch, das wird in diesem Kapitel immer deutlicher, auf ihre
subjektivierende Wirkung hin untersucht werden. Oder anders gefragt:\\
Was macht
jene Erkenntnis mit einem Subjekt, das sich als vergeschlechtlicht begreift und
sich in Beziehungen befindet? Welche Prozesse der Selbstentfremdung werden
dadurch womöglich angeregt?  

Dynamiken der Aneignung oder Abweisung geraten
hier ebenso ins Blickfeld wie Momente der Krise, die jene Theorie unter
Umständen auf das Selbst- und Weltverständnis des Subjekts ausübt. \\
 Michel
Foucault beschreibt diese Wirkung während seines Denk- und Schreibprozesses in
seinem mehrbändigen Werk \glqq Sexualität und Wahrheit\grqq\footnotemark
\footnotetext{Das Werk umfasst insgesamt drei Bände: \glqq Der Wille zum Wissen
(1976)\grqq, \glqq Der Gebrauch der Lüste(1984)\grqq und \glqq Die Sorge um
sich (1984)\grqq.}: Rückblickend auf den
bereits veröffentlichten ersten Band \glqq Der Wille zum Wissen \grqq
\footnotemark \footcitetext{foucault} erklärt er im
zweiten Band \glqq Der Gebrauch der Lüste\grqq\footnotemark
\footnotetext{Michel Foucault, \textit{Der Gebrauch der Lüste.}} sein Motiv und legt dar, was ihm im
Prozess des Schreibens widerfahren ist. Erkenntnisprozesse versteht er immer als Beziehungsprozesse, in denen das Subjekt sich selbst bzw. seinen Vorannahmen begegnet und die von Neugier angetrieben werden. 

\begin{myenv}
  \textit{Jedoch \glqq [...] nicht diejenige [Neugier], die sich anzueignen
  sucht, was zu erkennen ist, sondern die, die es gestattet, sich von sich
selber zu lösen. Was sollte die Hartnäckigkeit des Wissens taugen, wenn sie nur
den Erwerb von Erkenntnissen brächte und nicht in gewisser Weise und so weit
wie möglich das Irregehen dessen, der erkennt?\grqq} \footnotemark
\footnotetext{Ebd., 15.}
\end{myenv}

Foucault unterscheidet also nicht zwischen einem erkennenden Subjekt und dem Objekt, das erkannt und dann angeeignet wird. Sein Fokus bleibt stattdessen am erkennenden Subjekt haften und beschreibt einen Prozess der Selbsterkennung, der jedoch nicht mit einer gesteigerten Kohärenz, sondern eher mit einer Irritation des Selbst einhergeht: 

\begin{myenv}
  \textit{ 
    \glqq [E]s ist sein Recht, zu erkunden, was in seinem eigenen Denken verändert
    werden kann. Indem er sich in einem ihm fremden Wissen versucht. Der
    'Versuch'- zu verstehen als eine verändernde Erprobung seiner selber und 
    nicht als vereinfachende Aneignung des anderen zu Zwecke der Kommunikation
    - ist der lebende Körper der Philosophie, sofern diese jetzt noch das ist,
    was sie einst war: Eine Askese, eine Übung seiner selber, im Denken.\grqq\footnotemark
  }\footnotetext{Ebd., 16.}

\end{myenv}

Die Unterscheidung in 'fremdes' und 'eigenes' 
Wissen kann zudem als Hinweis auf die Ordnung von Wissen verstanden werden. Formen und Inhalte von Wissensbeständen werden nicht als einander komplementierende Puzzlestücke verstanden, sondern befinden sich im Widerstreit mit den Selbst- und Weltverständnissen, auf die sie treffen.

Diesen Widerstreit beschreibt Foucault in einem Gespräch mit Ducio Trombadori als Grenzerfahrung:

\begin{myenv}
  \textit{\glqq Die Idee einer Grenzerfahrung, die das Subjekt von sich selbst
  losreißt […] hat mich dazu gebracht, meine Bücher, wie langweilig, wie
gelehrt sie auch sein mögen - stets als unmittelbare Erfahrungen zu verstehen,
die darauf zielen, mich von mir selbst loszureißen, mich daran zu hindern,
derselbe zu sein.\grqq\footnotemark \footcitetext{foucinterview}} 
\end{myenv} 

Erfahrung ist zugleich das, was die Veränderungen im Subjekt
auslöst, und das, was die Veränderung, den Vorgang, ein\_e andere\_r zu werden,
beschreibt. Das Subjekt, wie bereits mit Mecheril et al. und Wimmer
angeklungen, wird dabei nicht als mit sich identisch, sondern immer im Entzug,
und somit weder abbild- noch greifbar verstanden. Erfahrung beschreibt nun
diesen Prozess des Abgleichens und sich Enziehens und wird von Foucault als
Korrelation beschrieben, die \glqq in einer Kultur zwischen Wissensbereichen,
Normativitätstypen und Subjektivitätsformen\grqq\footnotemark
\footnotetext{Michel Foucault, \textit{Der Gebrauch der Lüste}, S. 10.} besteht. 
\\

Kritik- und
Bildungsprozesse, die nach Innen und Außen wirksam werden, legen nahe, Momente
des Befragens und der Irritation als Erkenntnisprozesse in ihrer
subjektivierenden Wirkung in den Blick zu nehmen und Erfahrung als ein
Beziehungsgeflecht zu begreifen, das sich irgendwo zwischen dem Intimen und dem
Sozialen seinen Ort sucht, um von dort aus zu vermitteln.

Dabei kann vorerst
nicht festgelegt werden, was das Intime vom Sozialen unterscheidet, und somit
auch nicht vorausgesehen werden, wo sich Wissen, Normen und
Subjektivitätsformen, wie sie von Foucault aufgezählt werden, verorten lassen.
Statt einer Dichotomie soll der Hinweis auf das Intime und Soziale eher als
Form der Rahmung verstanden werden, innerhalb derer sich mit Wissen, Normen und
Subjektivitätsformen auf verschiedene Weise auseinandergesetzt wird. Das Intime
und das Soziale bilden entsprechend nicht unterschiedliche Gegenstandsbereiche
ab, sondern zeigen eher verschiedene Modi der Beschäftigung auf, die ineinander
wirken und als Bezugspunkte in der Bestimmung von Erfahrung dienen können. 
 \\
 
Die Aufgabe von Universität, das ist deutlich geworden, verändert sich mit dem
hier erarbeiteten Verständnis dessen, wie Kritik und Bildung auf die Grenzen
des Denkbaren wirken: Universität, spiegelt sich nicht nur in den Texten wider,
die von wenigen geschrieben und vielen gelesen werden. Universität spiegelt
sich in den intimen und sozialen Erfahrungen von Ausschluss, Schmerz, von
Verletzung, aber auch Widerstand und Empowerment wider, darin, wie Menschen
angerufen werden, sich wiederfinden, sich erkennen oder unerkannt bleiben.
Während für Wimmer und Foucault die Verunsicherung des Selbst im Vordergrund
steht, das Losreißen von sich und die damit einhergehende Unfähigkeit über die
Grenzen des Selbst wachen zu können, wird bei Mecheril et al. das Potential der
Veränderung in diesen Momenten hervorgehoben und damit ein Ausbruch aus einer
Ordnung angedeutet, die sich zwar unter Umständen beruhigend, aber in erster
Linie beschränkend auf das Subjekt auswirkt. 

Subjektivierung, die Einwirkungen
des Sozialen in das Intime und deren Rückwirkung auf das Soziale wird, das ist
durch alle Autor\_innen deutlich geworden, zugänglich über Erfahrung. Dies macht
es unabdingbar, sich jenen Ort anzuschauen, an dem das legitimierte Wissen die
Lebens- und Erfahrungswelten trifft, um dort neu bestimmt zu werden.
