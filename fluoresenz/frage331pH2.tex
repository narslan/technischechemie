% !TEX encoding   = UTF-8
% !TEX program    = LuaLaTeX
% !TEX spellcheck = de_DE
\documentclass[a4paper]{article}
\usepackage{fontspec}
\usepackage[usenames,dvipsnames]{color}
\usepackage{siunitx}
\usepackage[german]{babel}

\usepackage{pgfplots}
\usepackage{pgfplotstable}
\pgfplotsset{compat=newest}
\usepgfplotslibrary{units}


\usepackage{luacode}
\definecolor{skyblue1}{rgb}{0.447,0.624,0.812}
\definecolor{scarletred1}{rgb}{0.937,0.161,0.161}
\definecolor{chameleon1}{rgb}{0.541,0.886,0.204}
%\usepackage{helvet}
\usepackage[landscape]{geometry}
\usepackage{unicode-math}

%\usepackage[landscape]{geometry}
%\setmainfont{Ubuntu}
\setmainfont{TeX Gyre Heros}

\setmathfont{TeX Gyre Heros}
\usepackage{fancyhdr}
\pagestyle{fancy}
\usepackage{subcaption}


\rhead{ Gruppe 6}
\lhead{\textbf{Fluoreszenz} Vergleich der Anregungs- und Emissionsspektren bei pH 2}
\setlength{\headheight}{15pt}
\pagenumbering{gobble}
\usetikzlibrary{pgfplots.groupplots}

\usepackage{caption}
\captionsetup[figure]{labelformat=empty}% redefines the caption setup of the figures environment in the beamer class.

%\usepackage{libertine}
%\usepackage{beramono}
%\setmainfont{Ubuntu}
\begin{luacode*}


function tabled()
local lines = {}

for line in io.lines("data/ph02emmissions_360_370.txt") do
  table.insert(lines, line)
end
local summex=0
local summey=0
local xwerte={}
local ywerte={}



for i, line in ipairs(lines) do
  if i>23 then
     local s, e = line:find("%s+", 1)
     local y=tonumber(line:sub(e))
     local x=tonumber(line:sub(1,e))
     table.insert(xwerte,x)
     table.insert(ywerte,y)
   end
end


    for i=1,#xwerte do
      tex.sprint(string.format("%d & %g \\\\",xwerte[i],ywerte[i]))
    end



end



function drawO()


local lines = {}

for line in io.lines("data/ph02anregung350_340.txt") do
  table.insert(lines, line)
end
local summex=0
local summey=0
local xwerte={}
local ywerte={}



for i, line in ipairs(lines) do
  if i>23 then
     local s, e = line:find("%s+", 1)
     local y=tonumber(line:sub(e))
     local x=tonumber(line:sub(1,e))
     table.insert(xwerte,x)
     table.insert(ywerte,y)
   end
end

    tex.sprint("\\addplot[".."no marks".."] coordinates{")

    for i=1,#xwerte do
      tex.sprint("("..xwerte[i]..","..ywerte[i]..")")
    end
    tex.sprint("};")
end



function drawP()


local lines = {}

for line in io.lines("data/ph02anregung420_410.txt") do
  table.insert(lines, line)
end
local summex=0
local summey=0
local xwerte={}
local ywerte={}



for i, line in ipairs(lines) do
  if i>23 then
     local s, e = line:find("%s+", 1)
     local y=tonumber(line:sub(e))
     local x=tonumber(line:sub(1,e))
     table.insert(xwerte,x)
     table.insert(ywerte,y)
   end
end

    tex.sprint("\\addplot[".."no marks".."] coordinates{")

    for i=1,#xwerte do
      tex.sprint("("..xwerte[i]..","..ywerte[i]..")")
    end
    tex.sprint("};")
end


function drawB()


local lines = {}

for line in io.lines("data/ph02emmissions_360_370.txt") do
  table.insert(lines, line)
end
local summex=0
local summey=0
local xwerte={}
local ywerte={}



for i, line in ipairs(lines) do
  if i>23 then
     local s, e = line:find("%s+", 1)
     local y=tonumber(line:sub(e))
     local x=tonumber(line:sub(1,e))
     table.insert(xwerte,x)
     table.insert(ywerte,y)
   end
end

    tex.sprint("\\addplot[".."no marks".."] coordinates{")

    for i=1,#xwerte do
      tex.sprint("("..xwerte[i]..","..ywerte[i]..")")
    end
    tex.sprint("};")
end



function drawC()


local lines = {}

for line in io.lines("data/ph02emmissions_320_330.txt") do
  table.insert(lines, line)
end
local summex=0
local summey=0
local xwerte={}
local ywerte={}



for i, line in ipairs(lines) do
  if i>23 then
     local s, e = line:find("%s+", 1)
     local y=tonumber(line:sub(e))
     local x=tonumber(line:sub(1,e))
     table.insert(xwerte,x)
     table.insert(ywerte,y)
   end
end

    tex.sprint("\\addplot[".."no marks".."] coordinates{")

    for i=1,#xwerte do
      tex.sprint("("..xwerte[i]..","..ywerte[i]..")")
    end
    tex.sprint("};")
end

\end{luacode*}

\newcommand\tabled{\directlua{tabled()}}
\newcommand\addplotb{\directlua{drawB()}}
\newcommand\addplotc{\directlua{drawC()}}
\newcommand\addploto{\directlua{drawO()}}
\newcommand\addplotp{\directlua{drawP()}}

\begin{document}
\pgfplotsset{
  standard/.style={
    axis x line=bottom,
    axis y line=middle,
     xlabel= Wellenlänge, % Set the labels
    ylabel= Intensität ,
    x unit= \si{\nano\meter},
    legend style={at={(0.1,-0.2)},anchor=north},
    y label style={at={(axis description cs:-.17,.5)},rotate=90,anchor=south},
    x tick label style={rotate=90,anchor=east},
   yticklabel style={/pgf/number format/sci},
scaled y ticks=false
  }
}
  \tikzset{
    every pin/.style={fill=scarletred1,rectangle,rounded corners=3pt,font=\tiny},
    small dot/.style={fill=chameleon1,circle,scale=0.3}
  }


% \begin{figure}

% \begin{subfigure}[b]{0.5\linewidth}

% \begin{tikzpicture}[transform shape]

% \begin{axis}[standard]
%     \addplotf
%   \end{axis}
% \end{tikzpicture}
% \caption{\large  Emissionsspektrum ph 6 360-370 \si{\nano\meter}}

% \end{subfigure}~



% \begin{subfigure}[b]{0.5\linewidth}

% \begin{tikzpicture}[transform shape]

% \begin{axis}[standard]
%     \addplotg
%   \end{axis}
% \end{tikzpicture}
% \caption{\large  Emissionsspektrum ph 6 320-330 \si{\nano\meter}}

% \end{subfigure}




% \begin{subfigure}[b]{0.5\linewidth}

% \begin{tikzpicture}[transform shape]

% \begin{axis}[standard]
%     \addplotr
%   \end{axis}
% \end{tikzpicture}
% \caption{\large  Anregungsspektrum ph 6 350-340 \si{\nano\meter}}

% \end{subfigure}~



% \begin{subfigure}[b]{0.5\linewidth}

% \begin{tikzpicture}[transform shape]

% \begin{axis}[standard]
%     \addplots
%   \end{axis}
% \end{tikzpicture}
% %\caption{\large  Anregungsspektrum ph 6 420-410 \si{\nano\meter}}

% \end{subfigure}



% \end{figure}

\begin{tikzpicture}
\begin{groupplot}[group style={group size=2 by 2,horizontal sep=60pt,vertical sep=50pt}]
\nextgroupplot[standard,title=Emissionsspektrum ph 2 360-370 \si{\nano\meter}]
    \addplotb
   % \addlegendentry{Emissionsspektrum ph 6 360-370 \si{\nano\meter}}
\nextgroupplot[standard,title=Emissionsspektrum ph 2 320-330 \si{\nano\meter}]
    \addplotc
    %    \addlegendentry{Emissionsspektrum ph 6 320-330 \si{\nano\meter}}
\nextgroupplot[standard,title=Anregungsspektrum ph 2 350-340 \si{\nano\meter}]
    \addploto
    \pgfplotsset{title style={at={(0.5,-0.4)}}}

     %\addlegendentry{Anregungsspektrum ph 6 350-340 \si{\nano\meter}}
    \nextgroupplot[standard,title=Anregungsspektrum ph 2 420-410 \si{\nano\meter}]
    \addplotp
        \pgfplotsset{title style={at={(0.5,-0.4)}}}

      %       \addlegendentry{Anregungsspektrum ph 6 420-410 \si{\nano\meter}}
\end{groupplot}
\end{tikzpicture}

\end{document}
