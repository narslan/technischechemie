\documentclass{article}

%encoding
%--------------------------------------

%----------------------------
\usepackage{microtype}
\usepackage{fourier}
\usepackage[utf8]{inputenc}
\usepackage[T1]{fontenc}
%units
%--------------------------------------
\usepackage{siunitx}

%drawing
%------------------------------------
\usepackage{tikz} % To generate the plot from csv
\usepackage{pgfplots}
\usepackage{pgfplotstable}
\usepackage{caption}
\usepackage{color, colortbl}
\usepackage{tabularx}
\usepackage{chemfig}
\definecolor{LightCyan}{rgb}{0.88,1,1}
\usetikzlibrary{datavisualization}
\pgfplotsset{compat=newest} % Allows to place the legend below plot
\usepgfplotslibrary{units} % Allows to enter the units nicely

\sisetup{
  round-mode          = places,
  round-precision     = 2,
}

    \schemestart
\chemname{\chemfig{H_2N-[:30](=[2]O)-[:-30]OH}}{\\ \\ $\beta$-Alanin }
\arrow{->[][][7mm]}
\chemname{\chemfig{ *4(-\chembelow{N}{H}-(=O)---)}}{\\ \\ $\beta$-Lactam}
\+{,,7pt}
\arrow(--[yshift=8mm]){0}[,0]
\chemfig{H_2O}
\schemestop
\chemnameinit{}


\begin{document}
%\begin{luacode*}
function string:split(sep)
        local sep, fields = sep or "%s", {}
        local pattern = string.format("([^%s]+)", sep)
        self:gsub(pattern, function(c) fields[#fields+1] = c end)
        return fields
end

function printHyperbola()
    local lines={}

    for line in io.lines("dampf.csv") do
            table.insert(lines, line)
    end
    tex.sprint("\\addplot[color=black] coordinates{")

    for i=2,#lines do
        local a=lines[i]:split()
        tex.sprint("("..a[1]..","..a[2]..")")
    end
     tex.sprint("};")
    tex.sprint("\\addplot[color=blue] coordinates{")

    for i=2,#lines do
        local a=lines[i]:split()
        tex.sprint("("..a[1]..","..a[3]..")")
    end
     tex.sprint("};")

end
\end{luacode*}
%\begin{figure}
\centering
     \begin{tikzpicture}
            \begin{axis}[standard,xlabel=Temperatur,ylabel=Druck]
                \directlua{printHyperbola()}
            \end{axis}
        \end{tikzpicture}
     \end{figure}
\setdoublesep{3pt}
\definesubmol{&}{-[,,,,draw=none]}
\chemfig{[:-90]\chemabove{N}{H}(-[:210](-[:150]R)=[:-90]O)-[:-30]*4(-(=O)-N*5(-(-(=[::-60]O)-[::+60]OH)-(-[::+0])(-[::-108])-S-)--)}

\end{document}
