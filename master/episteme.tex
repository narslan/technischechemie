\section{Episteme} \epigraph{\textit{ Eine von postkolonialer Theorie
inspirierte \glqq Pädagogik \grqq richtet ihren Blick dabei insbesondere auf die
gelernte Vergessenheit, auf die aktiv produzierten Amnesien und deren
Komplizenschaft mit dem imperialistischen Projekt. In den Hochschulen wie auch
in den katholischen Kindergärten kann \glqq Wissen \grqq sich immer auch
widerständig gegen die \glqq Institution \grqq selbst wenden. }}{María do Mar
Castro Varela \footnotemark} \footnotetext{María do Mar Castro Varela, \glqq
Verlernen und die Strategie des unsichtbaren Ausbesserns. Bildung und
Postkoloniale Kritik,\grqq download unter:
\url{http://www.igbildendekunst.at/bildpunkt/2007/widerstand-macht-wissen/varela},
am 26.11.2014.} 

Im letzten Kapitel habe ich eine Idee von Universität skizziert,
die Bildung und Kritik zur Aufgabe eines universitären Selbstverständnisses
erklärt, das an Reflexivität orientiert ist. Mit Mohanty und hooks argumentierte
ich anschließend für die Notwendigkeit, als Untersuchungsgegenstand konkrete,
bildungs-pädagogische Praxen in in den Blick zu nehmen. Eine machtkritische
Analyse fordert hier, Mecheril et al. folgend ein, Differenz, also Praxen des
Unterscheidens zu untersuchen, die in jenen Formen des 'sich bildens' und
'Kritik übens' subjektivierend wirken. Dazu haben hooks und Mohanty das Konzept
der Erfahrung, als Sprechen aus und über Erfahrung diskutiert und dessen
Bedeutung für emanzipatorische Prozesse an der Universität befragt. Hier weisen
sie auf die Notwendigkeit hin, Subjekt-Objekt Verhältnisse in
Wissensproduktionen zu historisieren und aufzubrechen.  Meine Auseinandersetzung
war dabei von der Frage geleitet, woran sich eine Reflexion, die sich mit der
Bedeutung von Erfahrung in Wissensproduktion und Bildungsprozessen aber auch
Praxen der Kritik beschäftigt, orientieren kann. Dabei konnte ich aufzeigen,
dass das Sprechen aus und über Erfahrung sowohl stabilisierend als auch
irritierend auf hegemoniale Ordnungen, darüber was von wem sagbar ist, wirkt.
Wann, von wem oder wogegen Kritik geübt wird und wessen Selbst- und
Weltverhältnisse in den benannten Bildungsprozessen zur Disposition stehen wurde
dabei insbesondere von Mohanty und hooks zum Thema gemacht. Sie markieren die
herrschenden Differenzverhältnisse in universitären Seminarkontexten als
sexistisch und rassistisch und Mohanty fordert in ihrem Text explizit die
Dekolonisierung von Universität. Damit macht sie deutlich, dass Rassismus und
Sexismus im kolonialen Verhältnis situiert sind und über eine historisierende
Analyse nachvollzogen werden müssen. Ich knüpfe an diese Forderung an, und
richte in diesem Kapitel einen dekolonialen Blick auf Episteme. Dabei
interessiere ich mich dafür, welchen Ort Erfahrung in der
