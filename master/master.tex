\documentclass[11pt]{article}

\usepackage{amsmath,mathtools} \usepackage[usenames,dvipsnames]{xcolor}
\usepackage{microtype} \usepackage[ngerman]{babel}
%\usepackage[utf8]{inputenc}
\usepackage[T1]{fontenc}

\usepackage{libertine}
% header
%\setlength{\headheight}{15pt} \setlength{\topmargin}{-15mm}
\usepackage{csquotes} \usepackage{geometry} \geometry{ a4paper,
total={170mm,257mm}, left=30mm, right=30mm, top=20mm, }

\renewcommand{\thesection}{}% Remove section references...
\renewcommand{\thesubsection}{\arabic{subsection}}%... from subsections

\usepackage{endnotes} \let\footnote\endnote

%\usepackage[style=verbose,backend=biber]{biblatex}
\usepackage[style=verbose,isbn=false,doi=false,backend=biber]{biblatex}
\addbibresource{books.bib}
%\renewcommand{\autocite}[1]{\footnote{\cite{#1}}}

\usepackage[onehalfspacing]{setspace}

\usepackage{epigraph} \renewcommand{\epigraphrule}{0pt}

\usepackage{changepage}   % for the adjustwidth environment
%\newenvironment{myenv}{\begin{adjustwidth}{2cm}{}\linespread{0.8}}{\end{adjustwidth}\linespread{1.25}}

\newenvironment{myenv}{\begin{adjustwidth}{2cm}{}\singlespacing\vspace{-0.3cm}}{\end{adjustwidth}\onehalfspacing}
\usepackage[hang]{footmisc}
\setlength\footnotemargin{10pt}
\usepackage{url}


\title{Erfahrung als Kritik. Kritik der Erfahrung?} 
\author{Birgit Manz} 
\usepackage{setspace}
%\linespread{1.25}
\begin{document}
\onehalfspacing
\setlength{\epigraphwidth}{0.7\textwidth}

%\tableofcontents

%\section{Prolog}
\epigraph{ We all need histories that no history book can
  tell, but they are not in the classroom, not the history classrooms, anyway.
  They are in the lessons we learn at home, in poetry and childhood games, in
  what is left of history when we close the history books with their verifiable
facts.}{Michel-Rolph Trouillot\footnotemark} \footnotetext{Michel-Rolph
Trouillot, \textit{Silencing the Past. Power and the Production of History} (Boston: Beacon Press, 1995).}

Was sind das für Geschichten, die wir alle brauchen, und die in keinem
Geschichtsbuch stehen? Was sind das für Geschichten, die nicht in den
Klassenzimmern, sondern in den Häusern der Schüler\_innen erzählt werden? Was
sind das für Geschichten, die sich uns erst eröffnen, wenn wir die Bücher
schließen und uns der Dichtung und dem Spiel zuwenden?

Die oben stehenden Zeilen bilden die letzten Worte eines Vorwortes, mit dem
Michel-Rolph Trouillot das Kapitel „The Haitian Revolution as a
Non-Event“\footnotemark\footnotetext{Trouillot, „The Haitian Revolution as a
Non-Event“ in \textit{Silencing the Past. Power and the Production of History} (Boston: Beacon Press, 1995).}
einleitet. In dem Vorwort teilt Trouillot mit den Leser\_innen zwei sehr
unterschiedliche Erfahrungen aus seiner Lehre an US-amerikanischen
Universitäten. Das eine Mal wird er von einer
Studentin\_*\footnotemark\footnotetext{Die Schreibweisen \textit{w}eiß; Schwarz, of
Color und der Gender Gap bzw. das * und ' weisen darauf hin, dass es sich hier
nicht um widerspruchsfreie Kategorien handelt denen ein ontologischer Zustand
vorausgeht. Viel eher sollen damit politische Positionen und Perspektiven
in rassistischen und sexistischen Verhältnissen zum Ausdruck kommen, die in
dieser Arbeit verhandelt werden. Die Schreibweise Schwarz und of Color, stellt
den Versuch dar, Selbstpositionierungen von 'Menschen' die Rassismus erfahren
zu berücksichtigen. Die Bezeichnung \textit{w}eiß mit dem kursiven \textit{w} soll das
Ungleichgewicht, dass in rassistischen Verhältnissen zumeist die von Rassismus
betroffenen markiert werden, die von Rassismus privilegiert und normalisierten
darum unbemerkt bleiben können entgegen wirken. Da es sich hier jedoch nicht um
eine Selbstbezeichnung im Sinne einer Wiederaneignung von Fremdbezeichnungen
geht, sondern um die Markierung von rassistischen Verhältnissen privilegiert zu
werden, wird hier eine andere Schreibweise als eben diskutierte Schreibweise
mit Großbuchstaben gewählt. Das Gender\_Gap zeigt den Zwischenraum auf, indem
sich alle 'Menschen' im Zweigeschlechterzwang verorten und das Sternchen weißt
daraufhin, dass die Leser\_in die Subjektkategorien Frau\_* wie auch andere
Subjektkategorien innerhalb einer Vielzahl von Differenzkategorien verorten
kann und sollte. Die Ein-Strich Anführungseichen betont einzelne Begriffe die
in dieser Arbeit auf ihren Normalitätsanspruch hin befragt werden. Dies
geschieht punktuell, an Stellen da mir die Betonung auf den
Konstruktionscharakter sinnvoll erscheint. Alle Schreibweisen sind im
strategischen Sinne zu verstehen, ich gehe nicht davon aus das die Frage nach
der Möglichkeit der Repräsentation von Subjekten durch Sprachordnungen damit
für alle Zeiten geklärt ist. Viel eher soll die Schreibweise mich und die
Leser\_innen dazu anregen, die sprachliche Ordnung als Werkzeug zu verstehen,
mit dem bestehende Bezeichnungspraktiken und damit auch Differenzpraktiken in
Frage gestellt werden können. } dazu aufgefordert, von den Kämpfen der
Sklav\_innen zu sprechen, anstatt nur \textit{w}eiße Autor\_innen zu
rezipieren:
\begin{myenv}
  \textit{„Mr. Trouillot, you make us read all those white scholars. What can
they know about slavery? Where were they when we were jumping off the boats?
When we chose death over misery and killed our own children to spare them from
a life of rape?“\footnotemark\footnotetext{Troillot, „The Haitian Revolution“, 70.}}
\end{myenv}
Das andere Mal und viele Jahre später fordert eine andere
Studentin\_* ihn dazu auf, endlich über die Schwarzen Millionäre zu sprechen:
„I'm tired to hear about that slavery stuff. Can we hear the stories of the
black millionaires“?\footnotemark\footnotetext{Ebd., 71.}

Troillot stellt den Leser\_innen beide Student\_innen als Schwarze Frauen\_* vor
und wünscht sich, die Zeit zurückdrehen und diese beiden Student\_innen
miteinander ins Gespräch bringen zu können. Worüber würden sie wohl sprechen?
Was verbindet und was trennt sie?

Die Worte der Student\_innen lösen vieles in mir aus. Nähe und Distanz,
Erinnerung und Sehnsucht, Unsicherheit und Mut.

Nähe, weil ich mit dieser
Geschichte die (nicht) erzählt wird verbunden bin. Es ist auch meine
Geschichte, weil Kolonialismus und Sklaverei auch eine Geschichte über
\textit{w}eiße ist.

Distanz, weil es nicht um mich, nicht um meine Gefühle
geht oder gehen sollte sondern um die Perspektiven und Forderungen derer, deren
Geschichte nicht erzählt wird.

Erinnerung, weil ich ähnliche
Auseinandersetzungen miterlebt habe und die Sehnsucht nach universitären Räumen
bleibt, in denen Subjekte mit all ihren Erfahrungen als Subjekte in diesen
Räumen vorkommen können, ohne auf diese Erfahrungen reduziert zu werden.
Unsicherheit, weil es unmöglich scheint, die Interventionen als Impuls
aufzugreifen ohne sich zugleich aneignend ihnen gegenüber zu verhalten.  Mut,
weil sie ein Zeichen dafür sind, dass es Widerstand gibt.

Die Gedanken die nun
folgen haben mehr mit mir zu tun als mit den Subjekten und dem Kontext aus dem
sie entstanden sind. Es sind meine Lesarten und sie sind damit keine absoluten
Deutungen sondern Versuche, jenen Interventionen aufzugreifen, um von ihnen zu
lernen.

Die Intervention der Schwarzen Frauen\_* in das Seminar, so unterschiedlich sie
auch sein mögen, verbinde ich mit einer geteilten Idee von Universität. Es sind
Forderungen nach einer Universität als Ort, an dem Geschichten erzählt werden
in denen sie, Schwarze Frauen\_*, sich wiederfinden. Geschichten, die ihnen
etwas darüber erzählen, woher sie kommen und was sie sein können. Und so
unterschiedlich ihre Vorstellungen davon auch sind, so ähnlich ist doch ihr
Verlangen, ihre eigene Geschichte zu hören oder vielleicht auch zu erzählen.

Die Interventionen dieser beiden Schwarzen Frauen\_* zeigen, dass Erfahrungen
Subjekte hervorbringen, die sich annehmend und abweisend gegenüber jenen
Erfahrungen verhalten können. Erfahrung geht nicht notwendig mit einem
unmittelbaren Erlebnis einher. Erfahrung schreibt sich über Generationen in die
Körper und Seelen der Menschen ein und sie hinterlässt Spuren.

%\section{Exposition}
\epigraph{
    What, after all, is opposed by the movements of slaves and their descendents?
slavery? capitalism? coerced industrialisation? racial terror?
or the ethnocentrism and European solipsism that these processes help to reproduce?
    }{Paul Gilroy\footnotemark} \footcitetext{gilroyatlantic}

\epigraph{ Wenn die Erkenntnis ein imperiales Instrument der Kolonialisierung
ist, dann ist Dekolonisierung der Erkenntnis eine der dringlichsten Aufgaben.
What, after all, is opposed by the movements of slaves and their descendents?
slavery? capitalism? coerced industrialisation? racial terror?  or the
ethnocentrism and European solipsism that these processes help to reproduce?
}{Aníbal Quijano\footnotemark}
\footnotetext{quijanocolonialidad}



\section{Vorwort}

\setlength{\epigraphwidth}{0.7\textwidth}

\epigraph{\textit{We all need histories that no history book can tell, but they
    are not in the class room, not the history classrooms, anyway.  They are in
    the lessons we learn at home, in poetry and childhood games, in what is left
    of history when we close the history books with their verifiable
facts.}}{Michel-Rolph Trouillot \footnotemark} \footcitetext{troillot} Was sind
das für Geschichten, die wir alle brauchen, und die in keinem Geschichtsbuch
stehen? Was sind das für Geschichten, die nicht in den Klassenzimmern, sondern
in den Häusern der Schüler\_innen erzählt werden? Was sind das für Geschichten,
die sich uns erst eröffnen, wenn wir die Bücher schließen und uns der Dichtung
und dem Spiel zuwenden?

\section{Universität}

Ausgang vorliegender Arbeit bilden spezifische Überlegungen über die Idee der
Universität. Bevor ich diese Überlegungen darstelle, die zugleich als
Perspektive auf die Universität verstanden werden können, möchte ich die
Kriterien offen legen, die mich in der Auswahl jener Perspektiven leiteten. In
meinen Recherchen bin ich auf einen vordergründig deutschsprachigen Diskurs
gestoßen, der aus unterschiedlichen Positionen und entsprechend verschiedenen
Beweggründen heraus die Idee der Universität befragt und auf die Probe stellt.
Hier kommen Studierende\footnotemark \footcitetext{lohman} und Lehrende
\footnotemark \footnotetext{ Die Markierung der Autor\_innen als Studierende
  und Lehrende soll darauf aufmerksam machen, dass die Autor\_innen im
  universitären Kontext durchaus unterschiedlich positioniert sind, aber
  allesamt im Bereich der Wissenschaften tätig sind. Stimmen von Angehörigen
  der Universität im Bereich der Verwaltung, Reinigung, etc. kommen hier nicht
  zu Wort.  } zu Wort, die sich alle im Bereich der Geistes- und
  Sozialwissenschaften verorten und die das Anliegen verbindet, die Universität
  einer Kritik zu unterziehen. Dieses geteilte Selbstverständnis, nämlich
  Kritik zu üben, zeichnet sich durch eine Aufmerksamkeit für die Diskrepanz
  zwischen dem Gewünschten und Tatsächlichen aus, also zwischen dem, wie etwas
  sein könnte, und dem, wie es erlebt und gelebt wird.\footnotemark
  \footnotetext{Mit dem Begriff der Kritik setze ich mich in dieser Arbeit zu
  einem späteren Zeitpunkt ausführlicher auseinander.} Der Text \glqq Die
  unbedingte Universität\grqq \footnotemark \footcitetext{derrida} von Jacques Derrida dient hierbei als fester
  Ausgangs- oder Bezugspunkt, anhand dessen die Kritik an der Universität aus
  eben jenen Blickwinkeln geübt und eine Krise der Universität konstatiert
  wird.  Bei näherer Betrachtung fällt jedoch auf, dass angesichts der
  unterschiedlichen Ausgangspositionen bzw. Blickwinkel von Krisen im Plural
  gesprochen werden muss, da hier durchaus verschiedene und nicht selten
  einander widerstreitende Vorstellungen zum Ausdruck kommen - sowohl im
  Hinblick auf das, was die Universität gegenwärtig ist, als auch was sie sein
  sollte. 

  Für Julian Nida-Rümelin\footnotemark \footcitetext{nida} besteht die Krise
  der Universität in ihrer Öffnung zur Massenuniversität. Die Idee der
  Universität ist für ihn untrennbar mit dem \glqq Humboldtschen Bildungsideal
  \grqq \footnotemark \footnotetext{Rümelins Rückgriff auf das Humboldsche
  Bildungsideal ist meiner Ansicht nach insofern widersprüchlich, als er zum
einen Bildung als Zweck an sich und nicht als Mittel begreift und damit Bildung
von Ausbildung abgrenzt. Zum Anderen schließt er in seiner Argumentation an
Ökonomietheorien an, die Berufsausbildungen als notwendig für die Ökonomie
begreifen und eine nur an Selbstzweck ausgerichtete Bildung für die breite
Bevölkerung problematisieren. Vgl.: ebd. } verknüpft und geht angesichts der
Forderung, universitäre Bildung für alle zu gewährleisten, verloren. Denn weder
brauche die Gesellschaft (und damit ist in erster Linie die Ökonomie gemeint)
eine vollakademisierte Bevölkerung, noch sei universitäre Bildung die richtige
Form der Bildung für alle. Um das Besondere an universitärer Bildung bewahren
zu können, müsse diese streng von der Idee der Berufsausbildung getrennt werden
und dürfe nur denen vorbehalten werden, die sich in jener intellektuellen
Tätigkeit, die für ihn Bildung kennzeichnet, zurecht finden. Lucyna Darowska
und Claudia Macholds \footnotemark \footcitetext{lucyna} Perspektive unterscheidet sich sowohl ontologisch als
auch normativ von den Ansichten Rümelins. Sie konstatieren nicht eine Öffnung
zur Massenuniversität, sondern fokussieren in ihrem Beitrag Praxen der
Grenzziehung, die zu Benachteiligung und Ausschluss führen und nach wie vor
konstitutiv für den universitären Raum gelten. Ganz im Gegensatz zu Rümelin
fordern sie dazu auf, eben jene Grenzziehungen zu problematisieren; und sie
entlarven die im Zuge der Internationalisierung vorangetriebene Pluralisierung
der Zugänge zur Universität als einseitige, auf ökonomische Verwertbarkeit
orientierte Fokussierung auf Differenz.  Während sowohl für Rümelin als auch
für Darowska und Machold die Frage, für wen die Universität offen sein sollte,
im Vordergrund steht, problematisiert Ingrid Lohman \footnotemark
\footcitetext{lohman} die neoliberale
Vereinnahmung der Universität durch Konzerne. Die Universität als Raum für
Kritik und den Gebrauch der öffentlichen Vernunft weicht ihrer Ansicht nach der
\glqq unternehmerischen Hochschule \grqq \footnotemark \footcitetext{lohmein}, die ihren Auftrag zunehmend darin sieht, dem
Modernisierungsbedarf europäischer Konzerne entgegen zu kommen, und ihre
Wissensproduktion entsprechend nach dem ökonomischen Nutzen ausrichtet. 

\subsection{Praxen der Überschreitung}

Die Perspektive, für die ich mich im Folgenden interessiere, unterscheidet sich
von den bisher dargestellten Perspektiven nicht nur darin, was hier
problematisiert wird, sondern auch, wie es problematisiert wird. Bisher
dargestellte Positionen gehen bei aller Unterschiedlichkeit (darin, was die
Universität ist und was sie sein sollte) davon aus, dass die Krise der
Universität temporär und damit überwindbar sei. Sie beziehen sich in ihren
Analysen auf gegenwärtige gesellschaftliche Veränderungsprozesse und erläutern
die Gefahren, die von diesen für ihre Idee der Universität ausgehen. Ihre
Forderungen sind entsprechend lösungsorientiert und erinnern in ihrer
Dringlichkeit an Rettungsversuche.

Die Perspektive auf Universität, die ich in
dieser Arbeit einnehmen möchte begreift Krise als konstitutives Moment von
Universität. Universität ist Krise. Eine Überwindung der Krise würde das Ende
der Universität bedeuten und darum orientiert sich diese Perspektive weniger an
einer Auflösung als an einem Umgang mit Krise. Bildung und Kritik bilden hier
gewissermaßen das Spannungsfeld, auf dem die Krise zur Praxis wird und damit
nicht nur den Kern universitären Tuns, sondern auch den Kern ihres an
Reflexivität orientierten Selbstverständnisses darstellt. Die Krise wird hier
nicht als Fehlentwicklung, sondern als Notwendigkeit verstanden, um
Bildungsprozesse und damit auch Praxen der Kritik zu ermöglichen.

Paul Mecheril und Birte Klingler verdeutlichen in dem Text \glqq Universität als
transgressive Lebensform. Anmerkungen, die gesellschaftliche Differenz- und
Ungleichheitsverhältnisse berücksichtigen \grqq \footnotemark
\footcitetext{metcheril} gleich zu Beginn, dass \glqq die Idee
der Universität weder präzise formuliert werden, noch sich der Wandlung
verschließen [kann].\grqq \footnotemark \footnotetext{Mecheril und Klingler,
\glqq Universität als transgressive Lebensform, \grqq 2.} Sie müsse viel eher mittels der Frage, was die
Universität ist, und was sie sein soll, ständig neu gestellt werden.

Damit wird deutlich, dass es ihnen nicht um die Suche oder gar die Verteidigung der
Idee der Universität geht. Ihr Anliegen besteht viel eher darin, Bezugspunkte
zu skizzieren, die für eine empirische und normative Auseinandersetzung mit
Universität dienlich sein können.

Dafür entwerfen sie die Universität als
einen Ort der Aufklärung, an dem \glqq irgendwie sinnvollere Welt- und
Selbstverhältnisse \grqq \footnotemark \footnotetext{Ebd. 85.} erprobt und damit Bildungsprozesse ermöglicht werden, in denen bisherige Deutungsmuster in Frage gestellt werden können. Bildung
verstehen sie dabei im emanzipatorischen Sinne als Ermächtigung und die
Universität als Zugang zu einer Lebensform, in der eine transgressive Praxis
der Überschreitung von (epistemischen) Grenzen ge- und erlebt werden kann. Vor
dem Hintergrund dieser normativen Skizze werfen die Autor\_innen anschließend
die Frage auf, wie unter gegebenen Verhältnissen, Bildung als Bürger\_innenrecht
eingefordert werden kann, durch das Praxen der Kritik möglich werden. Dabei
problematisieren sie die faktische Mitwirkung der Universität an der
Aufrechterhaltung von (welt)gesellschaftlichen Differenz- und
Ungleichheitsverhältnissen.

Michael Wimmers Beitrag \glqq Die Agonalität des
Demokratischen und die Aporetik der Bildung. Zwölf Thesen zum Verhältnis
zwischen Politik und Pädagogik \grqq \footnotemark \footcitetext{wimmer} knüpft
an dieser Stelle an und setzt sich mit dem Prozess der Entdemokratisierung und
der Verdrängung des Pädagogischen \footnotemark \footnotetext{Wimmer, \glqq Die
Agonalität des Demokratischen, \grqq 33.} auseinander. Eine an Bildung und Kritik
orientierte Universität weicht seiner Ansicht nach zunehmend \glqq einer an
ökonomischen Effizienz und Steigerung orientierten Steuerungslogik \grqq
\footnotemark \footnotetext{Ebd. 48.} mit dem
Effekt der Transformation von Bildung in Humankapital. In seiner durchgehend
Kapitalismus thematisierenden Analyse setzt er sich mit dem Verhältnis von
Politik zu Pädagogik und Bildung zu Demokratie auseinander und weist auf die
Gefahren hin, die in dem Diktat der Selbstoptimierung liegen.

Beide Beiträge skizzieren sowohl einen Bildungs- als auch einen Kritikbegriff, der sich durch
Reflexivität auszeichnet und damit das Spannungsfeld universitärer Krise
andeutet, das den Ausgangspunkt vorliegender Arbeit bildet. 

\subsubsection{Bildung an und für die Universität}

Die Auseinandersetzung darum, was Universität sein soll und was sie faktisch
ist, führt Wimmer zu der Erkenntnis, dass es niemals eine Universität geben
kann, die ihrem Ideal entspricht, da dies seiner Ansicht nach in einem
Widerspruch zu dem Anspruch steht, sich selbst in Frage zu
stellen.\footnotemark \footnotetext{Ebd. 48} Auch für
Mecheril und Klingler besteht das charakteristische Moment der Universität in
dem Versuch einen Ort zu bewahren, \glqq der sich entzieht, mitunter nicht
konservierbar ist.\grqq \footnotemark \footnotetext{Mecheril und Klingler,
\glqq Universität als transgressive Lebensform,\grqq 84.} Bildung bildet in beiden Beitragen einen zentralen
Bezugspunkt, anhand dem sie sich mit der Krise der Universität
auseinandersetzen.

Doch was ist mit Bildung, und insbesondere universitärer
Bildung gemeint? 

Für Wimmer erweist sich die viel praktizierte Gegenüberstellung des
neuhumanistischen Verständnis von Bildung mit einem ökonomistischen Verständnis
des unternehmerischen Selbst als nicht produktiv. Stattdessen ginge es darum,
\glqq den Bildungsbegriff auf das Differenzverhältnis von Singularität und
Gleichheit, Verantwortung und Gerechtigkeit zu gründen.\grqq \footnotemark
\footnotetext{Wimmer, \glqq Die Agonalität des Demokratischen,\grqq 35.} Bildung müsse dabei
in ein Verständnis von Demokratie eingebettet sein, das Demokratie als
\glqq Bewegungs- und Artikulationsform unaufhebbarer Konflikte, Antagonismen und
Differenzen \grqq versteht. Sie wird damit zu einer Praxis, die sich dem
Partikularen widmen muss, ohne dabei das Ganze aus den Augen zu verlieren. Nur
so sei es möglich, sich auf den Widerspruch als eine grundlegende Kategorie von
Bildung einzulassen.

Bildung und Demokratie, so führt Wimmer aus, sollten
hierbei nicht im instrumentellen Sinne der gegenseitigen Nutzbarmachung
verstanden werden, sondern vielmehr als theoretische Konzepte in ihrer
wechselseitigen Verschränkung theoretisiert werden. Eine an Demokratie
orientierte Gesellschaft müsse demnach dafür Sorgen, dass das Recht auf Bildung
allen offen stehe, da Bildung notwendige Voraussetzung für Beteiligung an
demokratischen Prozessen darstelle. Zugleich stelle die oben skizzierte Idee
der Demokratie als Form, die nicht die Auflösung, sondern die Bearbeitung von
Konflikten als Auftrag versteht, einen wichtigen, wenn nicht sogar notwendigen
Ausgangspunkt für Bildungsprozesse dar.\footnotemark \footnotetext{Ebd., 37.} 

Die Idee einer demokratischen Bildung,
als auch einer Demokratie, die auf Bildung aufbaut, muss sich nach Wimmer
demnach fragen, wie Bildung verstanden bzw. Bildungsräume gestaltet werden
können, damit Bildung ihrer Verschränkung mit einem demokratischen Anspruch,
gerecht werden kann. Hierin liegt jedoch gerade die Schwierigkeit, da, so
Wimmer, weder Bildung noch Demokratie einfach bestimmbare Begriffe sind. In
Anlehnung an Chantal Mouffes und Ernesto Lacleaus Idee des Politischen,
beschreibt er Demokratie und Bildung als leere Signifikanten, deren einzige
Bestimmbarkeit auf ihre Unbestimmbarkeit beruht, da sie \glqq auf irreduzible
Differenzen, konstitutive Spaltungen und unlösbare Konflikte im Sozialen wie
auch im Subjekt\grqq \footnotemark \footnotetext{Ebd., 40.} verweisen. 

Was heißt dies nun für eine Idee von Universität,
die die Ermöglichung von Bildung als eine ihrer zentralen Aufgaben versteht?
Für Wimmer muss die Universität in erster Linie ein Ort der Selbstbildung
werden, der die Krise des Selbst als Möglichkeit versteht, sich vom Anspruch
der Kohärenz zu lösen und stattdessen dem nicht greif- und abschließbaren
Selbst- und Weltverständnis zu widmen.  Denn Bildung als Selbstbildung muss
sich, so führt er fort, immer dem Selbst entbehren, das es zu bilden
beansprucht. Das Subjekt ist damit nicht nur von sich selbst, sondern von der
Welt getrennt, und damit fehlt der Bildung \glqq gerade der eigentliche Sinn eines
Gebildes als einer integrierten Ganzheit und Identität.\grqq \footnotemark \footnotetext{Ebd., 40.}  Die Fragmentierung
des Selbst stellt das Subjekt damit vor die Herausforderung sich in einer Welt
zu orientieren, über die es nicht verfügen kann, sondern die es in einen
unbestimmten Zwischenraum entlässt: 
\begin{myenv}
  \textit{ \glqq Je mehr Selbstbildung […] desto mehr
Weltbindung, was bedeuten würde, dass Bildung nur im Grenzbereich zwischen Ich
und Welt, Selbst und Anderem, Erfahrung und Widerfahrnis, agency und pathos
lokalisierbar und daher kein rein pädagogischer Begriff wäre.\grqq}
\footnotemark \footnotetext{Ebd.}  
\end{myenv}

Bildung
zeichnet sich damit erst durch eine 
\begin{myenv}
  \textit{\glqq [...]Erfahrung des Unmöglichen aus, der
Paradoxie der Fremderfahrung, die zugleich mit einer Transformation des Selbst-
und des Weltverhältnisses die Unmöglichkeit der Selbstpräsenz und völliger
Selbstverfügung erfahrbar macht und so weniger eine Steigerung der
Selbstkompetenz als viel mehr als eine zunehmende Vergegenwärtigung der
Selbstfremdheit zu verstehen wäre.\grqq}
\footnotemark \footnotetext{Ebd., 46}  
\end{myenv}

Universität als Ort der Bildung grenzt sich, so kann Wimmer gelesen werden, von
der klassischen Ausbildung bzw. Verwertbarkeit von Bildung ab und stellt in
Wimmers Anspruch eher einen Raum dar, an dem die Grenzen des Selbst- und
Welterkennens erfahrbar gemacht werden können. Die Auseinandersetzung mit dem
Selbst führt dabei jedoch nicht zu einer Manifestierung des Selbst, sondern zu
dessen Verunsicherung. Bildung zeigt damit die Grenzen der Selbsterfahrung auf,
da es die Erfassung der Nicht-Erfahrbarkeit des Selbst ist, die durch
Bildungsprozesse möglich wird. Die Erkenntnis, nicht mehr über die Grenzen des
Selbst verfügen zu können, erfordert damit die Bereitschaft, sich mit dem
Unmöglichen auseinanderzusetzen, nämlich sich des vermeintlich eigenen Selbst
zu entbehren.  

Auch für Mecheril und Klingler stellt Bildung eine
Grenzerfahrung dar. Bildung, so heißt es zunächst, ist eine Praxis des
Infragestellens.  \footnotemark \footnotetext{Mecheril und Klingler,
\glqq Universität als transgressive Lebensform,\grqq 85.} Dabei bilden Fragen nicht nur das Gerüst, durch das
Bildungsprozesse ermöglicht werden sollen, sondern auch die Praxis mittels
derer \glqq die Normalität des ordentlichen Lernens \grqq \footnotemark
\footnotetext{Ebd., 86.} überschritten werden könne.
Überschreitung meint hier sowohl die Veränderung bisheriger Deutungsgrundlagen,
also auch die Hinwendung zum Streit.\footnotemark \footnotetext{Hier wird der
Bezug zu Kollers Verständnis tranformatorischer Bildung deutlich: Koller
bezieht sich in seiner Theorie der transformatorischen Bildung auf die
Neubestimmung des Humboldschen Bildungsbegriff durch Rainer Kokemohr. Kokemohr
versteht unter Bildung eine \glqq Veränderung grundlegender Figuren des Welt- und
Selbstverhältnisses von Menschen \grqq, der in der Regel die Einsicht vorausgeht,
dass bisherige Vorstellungen des eigenen Verhältnisses zum Selbst und der Welt
nicht mehr ausreichen. Diese Einsicht, so Kokemohr tritt zumeist dann ein, wenn
Menschen mit Problemen konfrontiert werden. Diese Problemlagen, oder
Krisenerfahrungen, wie sie von Koller benannt werden, lösen demnach
Bildungsprozesse aus und führen zu einer Neubestimmung des Selbst in und zur
Welt. Vgl.: Hans-Christoph Koller, \textit{Bildung anders denken. Einführung in
die Theorie transformatorischer Bildungsprozesse}, Kohlhammer Verlag, Stuttgart, 2012, S. 16 } Nicht die Akkumulation von Wissen,
sondern die Auseinandersetzung mit Wissen und Nicht-Wissen kennzeichne damit
Bildungsprozesse, die es den Subjekten ermögliche, zwischen jenen Erklärungen,
die sich für ihre spezifischen Lebens- und Erfahrungskontexte bewähren bzw.
nicht bewähren, zu unterscheiden. \footnotemark \footnotetext{Mecheril und Klingler,
\glqq Universität als transgressive Lebensform,\grqq 88.} In diesem Sinne wird Bildung immer auch als
Praxis der Ermächtigung verstanden, da in der Auseinandersetzung mit
verschiedenen Formen des Wissens und ihrer Befragung die Kontingenz des eigenen
Denkens und Handelns greifbar und somit Zugänge geschaffen werden können, die
\glqq andere, weniger einem, äußeren Zwang unterliegende, erstrebenswerte Selbst-
und Weltverhältnisse \grqq \footnotemark \footnotetext{Ebd., 89.}  ermöglichen.

Während also in beiden Ansätzen
Krisenerfahrungen als Ausgang für Bildungsprozesse verstanden werden, da die
Kohärenz des Selbst sowie bisherige Deutungsmuster irritiert werden, erkennen
besonders Mecheril und Klingler in ihrem Beitrage das emanzipatorische
Potential, das dieser Irritation folgen kann, da so unter Umständen für das
Subjekt neue Beschreibungen und Analysen des Selbst- und der Welt erfahrbar
werden. 

Unklar bleibt für mich insbesondere bei Wimmers Ansatz, wie sein
Subjekt- und Bildungsbegriff zu Dimensionen von Verantwortung steht.

Welche
Verantwortung tragen jene, die die zweifelsfrei schmerzhaften Prozesse der
Selbstentfremdung begleiten, also Bildungsprozesse ermöglichen, die zu jenen,
von Wimmer beschriebenen tiefgreifenden Erschütterungen führen?

Die Frage der
Verantwortung geht jedoch noch über die Rahmung des Bildungsprozesses hinaus
und knüpft unmittelbar an das Subjektverständnis an, das hier von Wimmer
gezeichnet wird:\\
Ist das mit sich selbst nicht übereinstimmende, fragmentierte
Subjekt überhaupt in der Lage Verantwortung für sich, sein Denken und Tun zu
übernehmen bzw. kann es für letzteres zur Rechenschaft gezogen werden?

Zudem
bleibt aus meiner Sicht noch offen, von welchem Maß an Eigeninitiative bzw.
Bereitschaft Wimmer eigentlich ausgeht, sich in jene Bildungsprozesse zu
begeben, die das eigene Selbstverständnis und entsprechend auch
Handlungsvermögen in dem Maße irritieren. So scheint dieser Lesart die Annahme
vorauszugehen, Subjekte hätten zunächst den Schein eines kohärentes
Selbstbildes, welches erst in Folge des Bildungsprozesses irritiert wird.

Was
ermutigt Subjekte, sich jenen Schein nehmen zu lassen? Von wessen Befreiung ist
hier die Rede? Sind hier jene mit eingeschlossen, die niemals in den Genuss des
Scheins gelangen – weil sie das Trugbild der Kohärenz gar nicht erwerben
konnten oder wollten?

Es stellt sich also die Frage, von welchen
(welt)gesellschaftlichen Positionen heraus diese Bildungsprozesse angestrebt
werden, bzw. welches Subjekt hier imaginiert wird. Sollen alle gleichermaßen
irritiert werden oder gibt es Unterschiede, die berücksichtigt werden müssen?
Woran könnten die sich orientieren und welche Aufgabe erhält hier Universität?

Die Universität erhält den Auftrag, so lassen sich Mecheril, Klingler und
Wimmer lesen, das Transzendente zu ermöglichen, das im von Handlungszwang
bestimmten Alltag selten einen Platz geniest. Wenn Bildungsprozesse als
Krisenerfahrungen konzeptioniert werden, wird es jedoch unabdingbar sich von
einem universellen Subjektverständnis zu lösen, dem eine universelle
Krisenerfahrung zu Grund liegt. Die Thematisierung von Herrschafts- und
Machtverhältnissen, auf die ich im Folgenden eingehe, stellt für mich hier
einen notwendigen und von den Autor\_innen offen beschrittenen Weg dar, um
keine beliebige, sondern eine kritische Universität mit einem kritischen
Bildungsverständnis zu fordern.

\subsubsection{Kritik an und durch die Universität}

Was kann nun mit kritischer Universität gemeint sein? Universität wird bei
Mecheril und Klingler in Anlehnung an Derrida \footnotemark
\footnotetext{Derrida, \textit{Die unbedingte Universität}, zitiert in: Mecheril und Klingler, ebd., 84. } zu einem Ort, an dem nichts
außer Frage steht und in dem der \glqq Begriff des Menschen seinen Vollzug
erfährt \grqq \footnotemark \footnotetext{Mecheril und Klingler, Ebd. }. Die Kritik, oder vielmehr die Idee der Kritik wird nun, so verstehe
ich ihr Anliegen, im doppelten Sinne gedacht: Zum einen wird die Universität zu
einem Ort, an dem Kritik geübt wird, indem \glqq andere epistemische Sätze
\grqq  \footnotemark \footnotetext{Ebd., 85.} über
die Welt gesagt werden, die \glqq irgendwie sinnvollere Welt- und
Selbstverhältnisse \grqq \footnotemark \footnotetext{Ebd.} versprechen. Zum anderen gilt es, jenen Auftrag der Kritik
kritisch in Bezug auf seine Mitwirkung in der Aufrechterhaltung von Herrschaft
zu befragen, die z.B. darin bestehen könnte, bestimmten \glqq Auffassungen,
Bilder[n] und Darstellungen des Menschen \grqq \footnotemark
\footnotetext{Ebd., 84.} Vorrang zu bieten.

Interessant ist an dieser Stelle ihre eigene Praxis der Kritik, die ich als Versuch verstehe,
eben jene Doppeldeutigkeit der Kritik ernst zu nehmen und die ich in einer
klaren Distanz zu dogmatischen Forderungen erkenne. Statt einer Festlegung
dessen, was Gegenstand kritischer Auseinandersetzung an der Universität sein
muss, erweist sich ihre Perspektive eher als eine Offenheit beispielsweise
gegenüber der Frage, \textbf{wie} \textit{irgendwie} sinnvollere Selbst- und Weltverhältnisse
aussehen könnten, oder \textbf{woran} sich \textit{andere} epistemische Sätze erkennen ließen. 

Zunächst möchte ich mich jedoch wie eben angekündigt der zur Beziehung von
Kritik und Herrschaft widmen. Hierfür beziehe ich mich auf einen Kritikbegriff,
der auf Michel Foucault zurückgeht und im folgenden von Mecheril et al.
\footnotemark \footcitetext{mecherilmigration} im
Zusammenhang mit der Konturierung einer Migrationsforschung als Kritik
expliziert wird: \\ 
Das Grundmotiv der Kritik in jener Migrationsforschung
besteht darin aufzuzeigen, was Menschen von einer freieren und würdevolleren
Existenz beraubt. Damit stehen jene (diskursiven) Praxen im Zentrum der
Analyse, die an der Herstellung und Aufrechterhaltung von
Herrschaftsverhältnissen mitwirken.\footnotemark \footnotetext{Mecheril, et.
al, \glqq Enleitung \grqq, 34, 39, 45.}
Damit knüpfe ich an den bereits
skizzierten Hegemoniebegriff an und konkretisiere deine Bedeutung für Praxen
der Kritik. Herrschaft wird von den Autor\_innen als historisch gewachsene,
asymmetrische Beziehung zwischen Subjektpositionen verstanden, in der die
Möglichkeiten der Selbstbestimmung unterschiedlich verteilt sind. Herrschaft
wirke jedoch nicht ausschließlich repressiv, nicht nur beschneidend und
einschränkend auf die Beherrschten, sondern erweise sich als komplexes Gefüge,
das zumindest den Schein erweckt, \glqq funktional und bedeutsam \grqq 
\footnotemark \footnotetext{Ebd., 47.} sowie
alternativlos zu sein. Mecheril et al. Weisen hier auf die Notwendigkeit hin,
sich neben der destruktiven Form der Macht, auch ihrer produktiven Kraft zu
widmen. Herrschaft als komplexe Dynamik zu begreifen, ermögliche so, die
hegemoniale, subjektivierende Gewalt, die mit herrschaftsförmigen Praxen der
Unterdrückung einhergehen, zu benennen, ohne dadurch den Subjekten eine
ausweglose, eindeutige Position zuzuweisen.\footnotemark \footnotetext{Ebd., 34.}

Dieses Verständnis von Herrschaft
spiegelt sich auch im Kritikbegriff wider. Die Autor\_innen grenzen sich hierbei
explizit von orthodoxen Weltanschauungen ab, die wenigen privilegierten
Gelehrten die Fähigkeit zusprechen, zwischen authentischem und entfremdetem
Bewusstsein zu unterscheiden und damit Kritikfähigkeit nur einer Elite
vorbehalten. Mit dieser Haltung würde ein Dualismus bedient werden, der es
jenen Kritiker\_innen allein ermögliche, Herrschaftsdynamiken zu erkennen und zu
problematisieren. \footnotemark \footnotetext{Ebd., 39.} 
Stattdessen wird auf die Eingebundenheit der Kritiker\_innen
in die Verhältnisse, die sie zu kritisieren beanspruchen, verwiesen: Jede
Kritik sei von ihrem Gegenstand abhängig, denn ohne Gegenstand gäbe es auch
keine Kritik. Es gelte dieses konstitutive Verhältnis stets neu zu bestimmen
und damit auch die notwendige Mitwirkung an Herrschaftsverhältnissen in
Anbetracht der eigenen Eingebundenheit anzuerkennen und angesichts der
\glqq subjektivierenden Effekte \grqq \footnotemark \footnotetext{Ebd., 34.} die von der Praxis der Kritik ausgehen, als
machtvoll einzuschätzen. Das Maß, an dem sich die Kritik orientiert, steht
damit so Mecheril et al. immer wieder auf dem Prüfstand und erfordert
Zurückhaltung gegenüber dogmatischen Gewissheiten und damit verbundenen
Absolutheitsansprüchen.\footnotemark \footnotetext{Ebd., 42.} Trotz, oder vielleicht auch gerade auf Grund der
komplexen und von Widersprüchen geprägten Verhältnisse, sei Kritik von einem
epistemischem Engagement getragen, das davon ausgeht, dass das Erkennen von
Herrschaftsverhältnissen die Möglichkeit ihrer Veränderung birgt.\footnotemark
\footnotetext{Ebd., 45.}

Das Subjekt, das Kritik übt, ist dementsprechend nicht nur vom Gegenstand,
sondern auch von der Wirksamkeit der praktizierten Kritik abhängig, die jedoch
nicht eindimensional von Subjekt ausgeht, sondern sich wechselseitig und damit
subjektivierend auf das Subjekt rückbezieht. Die Idee der Kritik wird in der
Postmoderne, das ist bereits in der Darstellung des Bildungsbegriffs
angeklungen, im Wesentlichen durch eine doppelte Kontingenz gezeichnet: Das
sich selbst transparente, stets intentional handelnde Subjekt wird von einer
Zeichnung des Subjekts abgelöst, das weder sich selbst repräsentieren, noch
durch andere repräsentiert werden kann, und sich einer
\glqq wirklichkeitskonstitutiven Kraft der Sprache und der diskursiven Verfasstheit
unserer Wirklichkeit \grqq \footnotemark \footnotetext{Wimmer, \glqq Die Agonalität
des Demokratischen,\grqq 48. } gegenübersteht. Kritik wird damit zu einer Praxis, die
Paradoxien nicht als Denkfehler entlarvt, sondern als konstitutiv im Umgang mit
der Welt  anerkennt und es notwendig macht, die Positioniertheit von Subjekten
in Macht- und Herrschaftsverhältnissen als Ausgangspunkt von Kritik ernst zu
nehmen.  Ein Verständnis, in dem weder die Grenzen des Selbst, noch die
Beschaffenheit der Welt vorbestimmt sind, sondern sich vielmehr durch z.B.
Praxen der Infragestellung ständig erweitern, ist somit immer aufgefordert, das
Bestehende zu verlassen, um sich dem Entstehenden zu widmen, bzw. \glqq dass etwas
nicht Denkbares dennoch möglich und wirklich sein könnte, bedeutet, dass es –
für das Denken- Unmögliches möglicherweise gibt.\grqq  \footnotemark
\footnotetext{Ebd., 49.}

Doch wie muss das Denken beschaffen sein, um an die Grenzen seiner Selbst zu
stoßen bzw. diese zu überwinden? Unterschiedliche Formen von Wissen und die
Handlungsräume, die sich in ihnen entfalten, werden hier zu wichtigen
Bezugspunkten.  Dabei wird ein Verständnis von Kritik deutlich, das nicht auf
der Bereithaltung konkreter Alternativen basiert, sondern frei von Pragmatismus
und der Affirmation des Selbst eine Veränderung der Verhältnisse anstrebt.
Sowohl Wimmer als auch Mecheril und Klingler sind hier der Ansicht, dass die
Überschreitung des Bestehenden einem Antrieb folgt, der einer normativen Idee
dessen, was sein sollte und könnte, entspringt, und auf der Einsicht beruht,
dass hierfür die Grenzen des eigenen Denkens überschritten werden müssen.

Interessant ist hier insbesondere der wechselseitige Bezug von Bildung und
Kritik, der zwar nicht expliziert wird, sich aber vermuten lässt, insofern
Kritik jene Krisenerfahrung vorausgeht, die im Rahmen des Bildungsbegriffs
beschrieben wurde und die zugleich, auf einem kritischen, also normativ
bewegten und gesellschaftlich positionierten Verständnis von Machtverhältnissen
ruht. Bildung und Kritik sind, so lässt sich resümieren, dermaßen aufeinander
angewiesen, dass die  eine nicht ohne die andere existieren kann. Kritik wird
dogmatisch, wenn sie nicht an einen selbstreflexiven Bildungsprozess geknüpft
ist, und Selbstreflexivität wird beliebig, solang sie keine freiere und
würdevollere Existenz aller zum Ziel hat. 

\subsubsection{Erfahrung als Krise - Krise als Erfahrung}

Welche Aufgabe kommt Universität in eben jener Überschreitung der Selbst- und
Weltverhältnisse zu? Und was bedeutet dies dafür, welche Erfahrungen an der
Universität gemacht und geteilt werden (können)? 

Universität, das habe ich
bereits mit Wimmer und Mecheril et al. herausgearbeitet, ist mehr als ihr
sichtbares Produkt in Form von Konferenzen und Publikationen. Die
Beschreibungen, Analysen und Interpretationen (später werde ich sie allesamt
Erzählungen nennen), die an der Universität entstehen und für die
(Fach)öffentlichkeit zugänglich gemacht werden, sind nur ein Aspekt der
Wissensproduktion. 

Wenn Universität als Ort der Bildung und der Kritik gedacht
werden soll, an dem Reflexivität im (wort)wörtlichen Sinn \textit{geübt} wird, dann wird
es notwendig, sich über die (sicht- und zählbaren) Produkte hinaus jenen
Erfahrungen zu widmen, die in diesem reflexiven \textit{Üben} aufgehen.  

Anstelle von
Ergebnissen möchte ich darum Prozesse in den Vordergrund rücken.  

Eine Theorie
beispielsweise darüber, wie sich bestimmte Geschlechterbilder auf
Beziehungsgewalt auswirken, kann auf ihre Bedeutung im Rahmen eines bestimmten
wissenschaftlichen Diskurses untersucht werden, in dem unterschiedliche
Perspektiven auf Geschlecht und Gewalt diskutiert werden. Dieselbe Theorie kann
aber auch, das wird in diesem Kapitel immer deutlicher, auf ihre
subjektivierende Wirkung hin untersucht werden. Oder anders gefragt:\\
Was macht
jene Erkenntnis mit einem Subjekt, das sich als vergeschlechtlicht begreift und
sich in Beziehungen befindet? Welche Prozesse der Selbstentfremdung werden
dadurch womöglich angeregt?  

Dynamiken der Aneignung oder Abweisung geraten
hier ebenso ins Blickfeld wie Momente der Krise, die jene Theorie unter
Umständen auf das Selbst- und Weltverständnis des Subjekts ausübt.  Michel
Foucault beschreibt diese Wirkung während seines Denk- und Schreibprozesses in
seinem mehrbändigen Werk \glqq Sexualität und Wahrheit \grqq \footnotemark
\footnotetext{Das Werk umfasst insgesamt drei Bände: \glqq Der Wille zum Wissen
(1976) \grqq, \glqq Der Gebrauch der Lüste(1984) \grqq und \glqq Die Sorge um
sich (1984)\grqq.}: Rückblickend auf den
bereits veröffentlichten ersten Band \glqq Der Wille zum Wissen \grqq
\footnotemark \footcitetext{foucault} erklärt er im
zweiten Band \glqq Der Gebrauch der Lüste \grqq \footnotemark
\footnotetext{Michel Foucault, \textit{Der Gebrauch der Lüste.}} sein Motiv und legt dar, was ihm im
Prozess des Schreibens widerfahren ist. Erkenntnisprozesse versteht er immer als Beziehungsprozesse, in denen das Subjekt sich selbst bzw. seinen Vorannahmen begegnet und die von Neugier angetrieben werden. 

\begin{myenv}
  \textit{Jedoch \glqq [...] nicht diejenige [Neugier], die sich anzueignen
  sucht, was zu erkennen ist, sondern die, die es gestattet, sich von sich
selber zu lösen. Was sollte die Hartnäckigkeit des Wissens taugen, wenn sie nur
den Erwerb von Erkenntnissen brächte und nicht in gewisser Weise und so weit
wie möglich das Irregehen dessen, der erkennt?\grqq} \footnotemark
\footnotetext{Ebd., 15.}
\end{myenv}

Foucault unterscheidet also nicht zwischen einem erkennenden Subjekt und dem Objekt, das erkannt und dann angeeignet wird. Sein Fokus bleibt stattdessen am erkennenden Subjekt haften und beschreibt einen Prozess der Selbsterkennung, der jedoch nicht mit einer gesteigerten Kohärenz, sondern eher mit einer Irritation des Selbst einhergeht: 

\begin{myenv}
  \textit{ 
    \glqq [E]s ist sein Recht, zu erkunden, was in seinem eigenen Denken verändert
    werden kann. Indem er sich in einem ihm fremden Wissen versucht. Der
    'Versuch'- zu verstehen als eine verändernde Erprobung seiner selber und 
    nicht als vereinfachende Aneignung des anderen zu Zwecke der Kommunikation
    - ist der lebende Körper der Philosophie, sofern diese jetzt noch das ist,
    was sie einst war: Eine Askese, eine Übung seiner selber, im Denken. \grqq
  } \footnotemark
\footnotetext{Ebd., 16.}

\end{myenv}

Die Unterscheidung in 'fremdes' und 'eigenes' 
Wissen kann zudem als Hinweis auf die Ordnung von Wissen verstanden werden. Formen und Inhalte von Wissensbeständen werden nicht als einander komplementierende Puzzlestücke verstanden, sondern befinden sich im Widerstreit mit den Selbst- und Weltverständnissen, auf die sie treffen.

Diesen Widerstreit beschreibt Foucault in einem Gespräch mit Ducio Trombadori als Grenzerfahrung:

\begin{myenv}
  \textit{\glqq Die Idee einer Grenzerfahrung, die das Subjekt von sich selbst
  losreißt […] hat mich dazu gebracht, meine Bücher, wie langweilig, wie
gelehrt sie auch sein mögen - stets als unmittelbare Erfahrungen zu verstehen,
die darauf zielen, mich von mir selbst loszureißen, mich daran zu hindern,
derselbe zu sein. \footnotemark \footcitetext{foucinterview} \grqq } 
\end{myenv} 

Erfahrung ist zugleich das, was die Veränderungen im Subjekt
auslöst, und das, was die Veränderung, den Vorgang, ein\_e andere\_r zu werden,
beschreibt. Das Subjekt, wie bereits mit Mecheril et al. und Wimmer
angeklungen, wird dabei nicht als mit sich identisch, sondern immer im Entzug,
und somit weder abbild- noch greifbar verstanden. Erfahrung beschreibt nun
diesen Prozess des Abgleichens und sich Enziehens und wird von Foucault als
Korrelation beschrieben, die \glqq in einer Kultur zwischen Wissensbereichen,
Normativitätstypen und Subjektivitätsformen \grqq \footnotemark
\footnotetext{Michel Foucault, \textit{Der Gebrauch der Lüste}, S. 10.} besteht. 

Kritik- und
Bildungsprozesse, die nach Innen und Außen wirksam werden, legen nahe, Momente
des Befragens und der Irritation als Erkenntnisprozesse in ihrer
subjektivierenden Wirkung in den Blick zu nehmen und Erfahrung als ein
Beziehungsgeflecht zu begreifen, das sich irgendwo zwischen dem Intimen und dem
Sozialen seinen Ort sucht, um von dort aus zu vermitteln.

Dabei kann vorerst
nicht festgelegt werden, was das Intime vom Sozialen unterscheidet, und somit
auch nicht vorausgesehen werden, wo sich Wissen, Normen und
Subjektivitätsformen, wie sie von Foucault aufgezählt werden, verorten lassen.
Statt einer Dichotomie soll der Hinweis auf das Intime und Soziale eher als
Form der Rahmung verstanden werden, innerhalb derer sich mit Wissen, Normen und
Subjektivitätsformen auf verschiedene Weise auseinandergesetzt wird. Das Intime
und das Soziale bilden entsprechend nicht unterschiedliche Gegenstandsbereiche
ab, sondern zeigen eher verschiedene Modi der Beschäftigung auf, die ineinander
wirken und als Bezugspunkte in der Bestimmung von Erfahrung dienen können. 
 
Die Aufgabe von Universität, das ist deutlich geworden, verändert sich mit dem
hier erarbeiteten Verständnis dessen, wie Kritik und Bildung auf die Grenzen
des Denkbaren wirken: Universität, spiegelt sich nicht nur in den Texten wider,
die von wenigen geschrieben und vielen gelesen werden. Universität spiegelt
sich in den intimen und sozialen Erfahrungen von Ausschluss, Schmerz, von
Verletzung, aber auch Widerstand und Empowerment wider, darin, wie Menschen
angerufen werden, sich wiederfinden, sich erkennen oder unerkannt bleiben.
Während für Wimmer und Foucault die Verunsicherung des Selbst im Vordergrund
steht, das Losreißen von sich und die damit einhergehende Unfähigkeit über die
Grenzen des Selbst wachen zu können, wird bei Mecheril et al. das Potential der
Veränderung in diesen Momenten hervorgehoben und damit ein Ausbruch aus einer
Ordnung angedeutet, die sich zwar unter Umständen beruhigend, aber in erster
Linie beschränkend auf das Subjekt auswirkt. 

Subjektivierung, die Einwirkungen
des Sozialen in das Intime und deren Rückwirkung auf das Soziale wird, das ist
durch alle Autor\_innen deutlich geworden, zugänglich über Erfahrung. Dies macht
es unabdingbar, sich jenen Ort anzuschauen, an dem das legitimierte Wissen die
Lebens- und Erfahrungswelten trifft, um dort neu bestimmt zu werden.

\subsection{Erfahrung und Universität}
\epigraph{\textit{
   Feminists writing about race and racism have had a lot to 
say about scholarship, but perhaps our pedagogical and institutional practices
and their relation to scholarship have not been examined with quite the same
care and attention.     
    }}{Chandra Talpade Mohanty \footnotemark} \footnotetext{Mohanty, \glqq On
  Race and Voice,\grqq S. 183.} 

  Im Folgenden möchte ich die Auseinandersetzung damit, wie Bildung und Kritik
  an der Universität gedacht und praktiziert werden können, fortsetzen und im
  Hinblick auf die zuletzt von Foucault markierte Bedeutung von Erfahrung
  konkretisieren.  
  
  Was es heißen kann, Erfahrung im universitären Kontext,
  zugleich als vermittelnde und vermittelte Größe zu begreifen, werde ich im
  Kontext antisexistischer und antirassistischer Universitätspolitiken
  diskutieren. Anhand verschiedener Zu- und Umgänge mit Erfahrung, die von
  Chandra Talpade Mohanty und bell hooks analysiert werden, versuche ich damit
  auch meinem Anspruch nachzukommen, mich stärker den \textbf{Praxen} zu widmen, die in
  jenem Üben, das Denkbare zu überschreiten, aufgehen. An dieser Stelle wäre es
  durchaus auch möglich  erziehungswissenschaftlichen Auseinandersetzungen um
  die Bedeutung von Erfahrung für wissenschaftstheoretische Positionen und
  deren Anwendung für  Bildungsprozesse  aufzugreifen. Da sich die
  deutschsprachigen erziehungswissenschaftlichen Debatten1 jedoch stärker mit
  innerdisziplinären Fragen und weniger grundsätzlichen Einordnung in
  postkolonialen (welt)gesellschftlichen Verhältnissen befassen, ziehe ich
  ihnen die Überlegungen von  Mohanty und hooks für meine Arbeit vor.

%\subsection{Erfahrung und Universität}
\epigraph{ 
 Feminists writing about race and racism have had a lot to 
say about scholarship, but perhaps our pedagogical and institutional practices and their relation to scholarship have not been examined with quite the same care and attention. 
  }{Chandra Talpade Mohanty\footnotemark} \footnotetext{Mohanty, „On Race and Voice,“ 183.} 

Im Folgenden möchte ich die Auseinandersetzung damit, wie Bildung und Kritik
an der Universität gedacht und praktiziert werden können, fortsetzen und im
Hinblick auf die zuletzt von Foucault markierte Bedeutung von Erfahrung
konkretisieren.

Was es heißen kann, Erfahrung im universitären Kontext, zugleich als
vermittelnde und vermittelte Größe zu begreifen, werde ich im Kontext
antisexistischer und antirassistischer Universitätspolitiken diskutieren.
Anhand verschiedener Zu- und Umgänge mit Erfahrung, die von Chandra Talpade
Mohanty und bell hooks analysiert werden, versuche ich damit auch meinem
Anspruch nachzukommen, mich stärker den \textbf{Praxen} zu widmen, die in jenem Üben,
das Denkbare zu überschreiten, aufgehen. An dieser Stelle wäre es durchaus auch
möglich  erziehungswissenschaftlichen Auseinandersetzungen um die Bedeutung von
Erfahrung für wissenschaftstheoretische Positionen und deren Anwendung für
Bildungsprozesse  aufzugreifen. Da sich die deutschsprachigen
erziehungswissenschaftlichen Debatten\footnotemark \footnotetext{Vgl. Johannes
  Bilstein und Helga Peskoller, Herausgeber\_innen von \textit{Erfahrung,
  Erfahrungen}
(Wiesbaden: Chronos Verlag,[1968] 2013). Sowie Otto F. Bollnow „Der
Erfahrungsbegriff in der Pädagogik.“ In ebd. Und Helga Peskoller, Helga
„Erfahrung/en.“ im selbigen Band. Außerdem Susann Fegter und Nadine Rose.
„Herstellung von Legitimität. Zum Rekurs auf Erfahrung in der Lehre.“ In
Differenz unter Bedigungen von \textit{Differenz. Zu Spannungsverhältnissen
universitärer Lehre}, herausgegeben von Paul Mecheril,et al. (Wiesbaden:
Springer Verlag, 2013). } jedoch stärker mit innerdisziplinären Fragen und
weniger grundsätzlichen Einordnung in postkolonialen (welt)gesellschftlichen
Verhältnissen befassen, ziehe ich ihnen die Überlegungen von  Mohanty und hooks
für meine Arbeit vor.

Mohanty kritisiert, dass Feminist\_innen, die sich mit 'Race' und Rassismus
auseinandersetzen, ihre Analysen vielfach lediglich auf Publikationen beziehen
anstatt sich mit der Aneignung, Umdeutung und Herstellung dieser Erkenntnisse
in der Seminarpraxis auseinanderzusetzen. Mohantys Texte „On Race and
Voice“\footnotemark \footnotetext{Ebd.} und
„Locating the Politics of Experience“\footnotemark\footnotetext{Mohanty, „Feminist Encounters“.} greifen
die Frage auf, wie angesichts der Tatsache, dass sich die Universität nicht
außerhalb der Verhältnisse bewegt, die sie zu beschreiben beansprucht,
überhaupt kritisches Wissen über z.B. Rassismus entstehen kann und welche
Bedeutung dabei die Erfahrungen jener erhalten, die Rassismus tagtäglich
erleben – in und außerhalb der Universität. Dabei nehmen ihre Texte den
Charakter von Manifesten an, die eine Universität fordern, die sich nicht im
Diversity-Vokabular verliert, sondern Differenz als einen Angriff auf
Normalitätsvorstellungen begreift und einfordert.

Auch bell hooks beschäftigt sich in dem Kapitel „Essentialism and
Experience“\footnotemark\footcitetext{bellhooks}
mit der politischen Wirkung, die ein erfahrungsorientiertes
Sprechen\_Schreiben\_Lesen in universitären Kontexten entfalten kann. hooks
bezieht sich in dem benannten Kapitel auf einen Text von Diana Fuss mit dem
Titel „Essentially Speaking“\footnotemark\footcitetext[77]{bellhooks} und stellt ihm einen kritischen
Essentialismusbegriff entgegen.

Diana Fuss problematisiert in dem von hooks diskutierten Kapitel
erfahrungsbasierte Beiträge in universitären Seminarkontexten. Das Sprechen
über Erfahrung, so Fuss, sei automatisch mit einer Autorität besetzt, die
gerade von solchen Studierenden genutzt würde, die sie als marginalisiert
beschreibt. Letztere würden die Diskussion im Seminar durch den Rückgriff auf
eine essentialistische Auslegung ihres marginalisierten Standpunktes dominieren
und so andere Stimmen zum Schweigen bringen.\footnotemark\footnotetext{Diana Fuss in bell hooks, ebd., 81.}

Auch Mohanty bezieht zieht sich zunächst kritisch auf erfahrungsbasiertes
Sprechen. Sie merkt an, dass oftmals statt einer differenzierten
Auseinandersetzung darüber, wie sich politische Verhältnisse auf einer
strukturellen Ebene auf Differenzdynamiken z.B. im Seminar auswirken, eine
oberflächliche Gegenüberstellung von weiß und of Color bzw. Schwarz
positionierten Studierenden vorgenommen wird. Differenz werde hier nicht in
ihrer historischen Gewordenheit rekonstruiert, sondern vordergründig auf der
Ebene individueller, persönlicher Erfahrungen von z.B. Rassismus verstanden,
die von den vermeintlich nicht Betroffenen eine erhöhte Sensibilität oder
Feinfühlichkeit erfordere. Student\_innen of Color oder Schwarzen Studierenden
werde dadurch sowohl ein Expert\_innenstatus als auch eine erhöhte
Verletzbarkeit zugeschrieben, wohingegen weiß positionierte Studierende
lediglich als neutrale Beobachter\_innen fungieren. „In other words, white
students are constructed as marginal observers and students of color as the
real 'knowers' in such a liberal or left
classroom.“\footnotemark\footnotetext{Mohanty, „On Race and Voice,“ 194.}

Während sich also Fuss Kritik auf die vermeintlich unrechtmäßige Ermächtigung
Schwarzer Studierender und Studierender of Color bezieht, die ihrer Ansicht
nach durch erfahrungsbasiertes Argumentieren entsteht, problematisiert Mohanty
das Differenzverständnis, das jenem Sprechen zu Grunde liegt. So würden jene zu
Expert\_innen gemacht, die von Rassismus negativ betroffen sind, die Erfahrungen
von \textit{w}eiß Positionierten in rassistischen Verhältnissen und die damit
verbundenen Privilegien hingegen werden verschwiegen. Mohantys Darstellung
entlarvt die machtvolle Dynamik, die in diesen Zuschreibungen liegen. Der
Expert\_innenstatus wird per se allen sogenannten Betroffenen zugesprochen und
ihre Kompetenz ergo nicht als selbst erworbenes Wissen anerkannt, sondern als
natürliche Fähigkeit essentialisiert. Verletzlichkeit wird indessen nicht im
Zusammenhang mit kollektiven Traumata und Diskriminierungserfahrungen, sondern
als persönliche Schwäche im Sinne einer übertriebenen Selbstbezogenheit oder
mangelnden Souveränität gewertet. Zuschreibungen, das wird hier deutlich,
wählen mal kollektive und mal individuelle Erfahrungen als Ausgang, um Menschen
Fähigkeiten zu- und abzusprechen. 

Nicht ob, sondern in welcher Weise Erfahrung zum Thema gemacht wird, muss
stattdessen thematisiert werden. Mohanty kritisiert ein Verständnis von
Differenz, das sich nur auf die zu Anderen gemachten bezieht und das ich hier
als Betroffenheitszuschreibung markiert habe. Sie macht deutlich, dass die
politisch-historische Dimension, die hinter dem Zeichen der Differenz steht,
thematisiert werden muss.  Differenz wird von ihr nicht als Abweichung von
einer Norm, sondern als Infragestellung eben jener Norm verstanden:

\begin{myenv}
  \textit{„Difference seen as benign variation (diversity), for instance,
  rather than as conflict, struggle, or the threat of disruption, bypasses
power as well as history to suggest a harmonious, empty
pluralism.“\footnotemark\footnotetext{Ebd., 181.}}
\end{myenv}

Als problematisch an Fuss Argumentation lässt sich Mohanty folgend darum
herausstellen, dass sie nicht nach den Bedingungen fragt, die dem Sprechen über
eigene Erfahrungen, wie es u.a. auch von marginalisierten Sprecher\_innen
genutzt wird, unterliegen. Sie verkennt, dass der Sprachraum, der hier
Schwarzen Student\_innen und Studierenden of Color zugesprochen wird, oftmals
auf jene Themen begrenzt wird, in denen ihnen auf Grund ihrer zugeschriebenen
Erfahrung ein Stimme gewährt wird. Er unterliegt somit klaren Vorstellungen
davon, wie über das Thema gesprochen werden kann. 

\textit{W}eiße Dominanzverhältnisse, so lassen sich Mohanty und hooks hier lesen, werden
nicht durch verändertes Sprechen durchbrochen, wenn Zeitpunkt und Gegenstand
und damit auch Reichweite des Gesagten nach wie vor durch rassistische
Strukturen bedingt ist.

Darum wird es notwendig, die machtvollen rassistischen und sexistischen
Dynamiken, die auch in akademischen Kontexten wirken und darüber entscheiden,
wer zu welchem Thema und auf welche Art und Weise sprechen kann, in den Blick
zu nehmen und unter Umständen zu problematisieren. Dynamiken, die zwischen
Wissen und Erfahrung unterscheiden und bestimmten Gruppen nicht ermöglichen
ihre Erfahrungen als sogenannte Betroffene einzubringen, sondern dies geradezu
einfordern, müssen hier als hegemonial/paternalistische Gesten zurückgewiesen
werden.

Denn der Essentialismus liegt nicht bei denjenigen, die in diesen Momenten das
Wort ergreifen, sondern in den rassistischen und sexistischen Zuschreibungen,
die sie dazu auffordern als Expert\_innen 'ihrer' Gruppe persönliche Erfahrungen
zu teilen. Nicht selten, so argumentiert hooks weiter, werden an diesen Stellen
tatsächlich essentialistische Vorstellungen, wie sie innerhalb der
rassistischen Matrix entworfen werden,
bedient.\footnotemark\footcitetext[81]{bellhooks}

Der ausschließliche Fokus auf die Praxis der Marginalisierten, übersieht dabei,
so hooks, wie selbstverständlich und entsprechend unmarkiert, durch Rassismus
und Sexismus privilegierte Studierende ihre Erfahrung z.B. als weiße Männer\_*
zum Anlass nehmen, um selbstbewusst und unmarkiert ihre Sichtweisen zu
propagieren. Aus Erfahrung zu sprechen, oder Erfahrungen zum Gegenstand des
Beitrags zu machen, sind für hooks, so verstehe ich diese Beispiele, damit
nicht grundsätzlich verschieden. Doch während letzteres vielfach
problematisiert wird, kann ersteres unhinterfragt Hierarchien darüber, wer sich
zum Sprechen autorisiert fühlen darf, aufrecht
erhalten.\footnotemark\footnotetext{Ebd., 82.}

Der Terminus 'marginalisiert' umgeht dabei, ähnlich wie der Diskurs um die
Betroffenen, dass Rassismen und Sexismen auf alle Subjekte wirken. Ohne
Zentrierung, keine Marginalisierung –be\_troffen sind darum immer alle, nur
ge\_troffen werden alle unterschiedlich.  Es sind diese, subtilen und indirekten
Wege, über die essentialistische Selbstverständnisse von Dominanz und Intellekt
gerade bei den privilegierten Gruppen fungieren, und die von Fuss, so
argumentiert hooks, unberücksichtigt bleiben.  Essentialismus ist dabei stets
auf eine Projektionsfläche angewiesen, um das vermeintliche Wissen über 'die
Anderen' zu bestätigen. Mohanty beschreibt diese Projektionsfläche als Falle
der Repräsentation\footnotemark\footnotetext{Auf den Begriff der Repräsentation werde ich später noch ausführlicher eingehen.}. So könne sie, führt sie als Beispiel auf, nicht über
Verhältnisse z.B. der sogenannten 'Dritten Welt'\footnotemark\footnotetext{
Den Begriff  'Dritte Welt'  verstehe ich im Anschluss an Jürgen Dinkel als
„Ordnungsmuster“ das nicht auf realen Verhältnissen, sondern „ auf der
Phantasie der Sprechenden  [basiert] und [...] so koloniale Idealisierungen und
Stereotype in das postkkoloniale Zeitalter überführen [konnte].“  Dinkel
beschreibt sowohl aus historischer als auch semantischer Perspektive, wie das
das Konzept 'Dritte Welt' während des kalten Krieges, als Gegenpol zu den
einander konkurrierenden Lagern des  kapitalistischen Westen und
sozialistischen Osten konstruiert wurde. Seine Analyse verdeutlicht, wie
Prozesse der Aneignung und Umdeutung von Fremdzuschreibungen ( Der Begriff
wurde in Europa erfunden) auf globaler Ebene geführt wurden. Vgl.:
\url{https://docupedia.de/zg/Dritte_Welt}. 
} sprechen, ohne als Sprecherin \textbf{für} die 'Dritte Welt' missverstanden
zu werden.

\begin{myenv}
  \textit{
  „For I often come to embody the 'authentic' authority and experience for many
  of my students; indeed, they construct me as a native informant in the same
  way that left-liberal white students sometimes construct all people of color
  as the authentic voices of their people.“\footnotemark \footnotetext{Mohanty, „On Race and Voice,“ 194.}
    }
\end{myenv}

Die Zuschreibung, die Mohanty hier sowohl selbst erfährt als auch zwischen den
Studierenden beschreibt, basiert auf einer Vorstellung von Differenz a priori.
Differenz wird nach dieser Vorstellung nicht erst durch
Normalitätsvorstellungen hervorgebracht, sondern existiert unabhängig von
ihnen. Mohanty argumentiert nun, dass jene Naturalisierung von Differenz in
einer Kontinuität mir der kolonial-rassistischen Gewalt steht:

\begin{myenv}
  \textit{
    „In other words, I suggest that educational practices as they are shaped
  and reshaped at these sites cannot be analyzed as merely transmitting already
codified ideas of difference. These practices often produce, codify, and even
rewrite histories of race and colonialism in the name of
difference.“\footnotemark \footnotetext{Ebd., 184.}
    }
\end{myenv}

Betroffenheit, so lässt sich Mohanty hier lesen, wird in Bildungskontexten als
Mittel missbraucht, um kolonial-rassistische Strukturen fortzuschreiben und
dadurch an der Aufrechterhaltung von Herrschaftsverhältnissen mitzuwirken.
Erfahrung dient dabei als Anker, durch das sich das Betroffenheitsdenken
artikulieren und festsetzen kann.

Differenz und die in diesem Zusammenhang mächtigen Normalitätsvorstellungen
haben jedoch keine fixierten Bedeutungen. Die Bedeutungen werden immer wieder
neu verhandelt und können, so hooks, für die Stabilisierung von Dominanz- und
Herrschaftsverhältnissen, aber auch für ihre Infragestellung im Kontext einer
strategische Bündnispolitik nutzbar gemacht werden. Die Erfahrung von Differenz
kann somit immer Antrieb für Veränderungen strategisch genutzt werden und so
eine empowernde Wirkung entfalten:

\begin{myenv}
  \textit{„Identity politics emerges out of the struggles of oppressed or
    exploited groups to have a standpoint on which to critique dominant
    structures, a position that gives purpose and meaning to
    struggle.“\footnotemark \footcitetext[88]{bellhooks}
    }
\end{myenv}

Der entscheidende Unterschied ist hierbei, dass anders als bei zugeschriebener
Differenz, in strategische Identitätspolitiken die \textbf{eigene} Verortung in
Differenzverhältnissen als Ausgangspunkt dient und davon ausgegangen wird, dass
die damit einhergehenden Erfahrungen nicht individuell bedingt sind, sondern
einer Systematik unterliegen. Erfahrung spielt somit auch im emanzipatorischen
Umgang mit Differenz eine zentrale Rolle. Schließlich beabsichtigen
strategische Identitätspolitiken, die Systematik von zugeschriebener Differenz
aufzuzeigen, als ungerecht fort zuweisen und diejenigen, die davon profitieren,
in die Verantwortung zu ziehen. Die Thematisierung der kollektiven Dimension
persönlicher Erfahrungen eröffnet somit neue Interpretations- und
Handlungsräume, die in Bündnissen aufgehen und u. U. eine empowernde und
emanzipatorische Wirkung entfalten können. 

Wenn (Krisen-)Erfahrungen zum Gegenstand universitärer Bildung gemacht werden,
kann sich dies sowohl befreiend als auch beschränkend auf die Subjekte
auswirken. Die vorherrschende Ordnung, die ich mit Mohanty und hooks mit einer
rassismus- und sexismusthematisierenden Perspektive beschreibe situiert jene
(Krisen)Erfahrung hierbei in historisch gewachsenen, umkämpften
Differenzverhältnissen. Dies macht deutlich, dass Subjekte unterschiedliche
positioniert sind und ihre (Krisen-)Erfahrungen entsprechend unterschiedlich
stark Gefahr laufen, vereinnahmt zu werden. Bildungskontexte, die den Anspruch
haben, für politische Prozesse der Ermächtigung Räume zu schaffen, stehen damit
vor einer Herausforderung: Sie beabsichtigen Erfahrungen von Rassismus als
relevante Wissensquellen anzuerkennen, um die strukturelle Dimension von
Rassismus analysieren und seine Funktionsweisen verstehen zu können. Die
Verknüpfung von individuellen Erfahrungen mit Theorien über rassistische
Verhältnisse muss hierbei zugleich gegenüber einer wechselseitigen
Vereinnahmung achtsam sein. Denn es besteht immer die Möglichkeit, dass eine
konkrete Erfahrung notwendig für die Legitimierung einer Theorie, oder dass
eine bestimmte Theorie zur Voraussetzung für die Artikulation einer Erfahrung
erklärt wird. Hier ist es unabdingbar, dass der eingeführte Unterschied
zwischen Differenztheorie und Betroffenheitsdenken berücksichtigt wird.

Lernprozesse, in denen Interpretationsmöglichkeiten für eigene Erfahrungen
erarbeitet werden, oder Theorien angesichts von artikulierten Erfahrungen in
Frage gestellt werden, sind hier, so hooks,  stets auf tastende Bemühungen
angewiesen, die eine Offenheit sowohl gegenüber der Infragestellung etablierter
Theorien als auch gegenüber etablierten Interpretationsmustern von
individuellen Erfahrungen beabsichtigen.\footnotemark\footnotetext{hooks, ebd., 88ff.}

Für hooks steht darum statt einer orthodoxen Haltung für oder gegen die
Thematisierung von Erfahrung ein selbstreflexiver Umgang mit
erfahrungsbasiertem Sprechen im Vordergrund. Sie diktiert keine Regeln oder
Rezepte, sondern regt die Studierenden dazu an, sich über Anliegen und
Konsequenzen verschiedener Standpunkte zur Relevanz von erfahrungsbasierten
Beitragen im Seminar Gedanken zu machen. Eine Auseinandersetzung über Anliegen
und Effekte erfahrungsbasierter Beiträge müsse jedoch berücksichtigen, dass das
Sprechen über die Erfahrungen anderer, insbesondere unterdrückter Gruppen
systematisch praktiziert werde. Zunächst sei es darum unabdingbar, dass dieses
Ungleichgewicht erkannt und problematisiert werde.\footnotemark\footnotetext{Ebd., 89.} hooks verschiebt hier den
Fokus weg von der Frage, ob eine Ermächtigung aus einer marginalisierten
Position legitim ist, hin zu dem Umstand, dass sich universitäres Sprechen
vielfach auf die vermeintlich Anderen und wenig auf die eigene Position oder
Lebenswelt bezieht. 

\subsubsection{Dekoloniale Bildung}

Doch wie kann diese machtvolle Dynamik, die vermeintlich Anderen im Sprechen
über die vermeintlich Anderen zu erzeugen, durchbrochen werden? Eine
Voraussetzung, um neue Wege der Auseinandersetzung mit der Geschichte und
Gegenwart von Sexismus, Rassismus, Heterosexismus etc. bestreiten zu können,
muss Mohanty zufolge darin liegen, den Beitrag, den konventionelle
Wissensproduktionen und Pädagogiken an der Universität zur systematischen
Marginalisierung der Geschichte und der Erfahrungen von Menschen aus der sog.
'Dritten Welt' leisteten und leisten, zu erkennen und zu poblematisieren. Erst
diese Einsicht ermögliche eine ernsthafte und nachhaltige „dekolonisierung
unserer disziplinären und pädagogischen Praxen“\footnotemark\footnotetext{Mohanty, „On Race and Voice,“ 191.}. Dabei ergibt sich für Mohanty
folgende grundsätzliche Frage:

\begin{myenv}
  \textit{„The crucial question is how we teach about the West and its Others
    so that education becomes the practice of liberation. This question becomes
    all the more important in the context of the significance of education as a
    means of liberation and advancement for Third World and post colonial
    peoples and their/our historical belief in education as a crucial form of
    resistance to the colonization of hearts and minds.“\footnotemark
    \footnotetext{Ebd.}
    }
\end{myenv}

Bildung wird hier von Mohanty als Ort von Befreiung theoretisiert und damit zu
einem zentralen, auch geschichtlichen Schauplatz antikolonialen Widerstands
erklärt. Entsprechend stellt sie die Frage, wie jene Universität der Befreiung
geschaffen werden kann, also wie Lehr-Lernverhältnisse gestaltet werden können,
die der „Kolonisierung von Herz und Geist“\footnotemark\footnotetext{Ebd.} widerstehen.  Mit Mohanty wird
damit die eingangs geführte Auseinandersetzung um das emanzipatorische Moment
in Bildung in kolonialen Verhältnissen verortet. Ihre Verortung weist dabei
darauf hin, dass ein emanzipatorischer Anspruch an Bildung immer vor dem
Hintergrund der Geschichte von Bildung und damit der Geschichte von Kolonialismus geltend gemacht werden muss. Bildung  war und ist damit Teil hegemonialer Praxen von Ausschluss und Entmachtung ebenso wie von Widerstand und Aneignung.

Dekolonisierung ist ihrer Ansicht nach unmittelbar an die Möglichkeit der
Selbstdefinitionen geknüpft und findet in universitären Räumen statt, die das
Ziel haben historisch ausgeschlossenen Perspektiven sicht- und hörbar zu
machen.\footnotemark\footnotetext{Ebd., 184.} Dabei sind jene Räume stets damit beauftragt, sich gegenüber der Aneignung und Vereinnahmung durch apolitische Vielfaltbestrebungen zu wehren:
\begin{myenv} 
  \textit{„By their very location in the academy, fields such
    as women's studies are grounded in definitions of difference, difference
    that attempts to resist incorporation and appropriation by providing a
    space for historically silenced peoples to construct
    knowledges.“\footnotemark\footnotetext{Ebd.} } \end{myenv}

Mohanty betont immer wieder das, was an anderer Stelle in dieser Arbeit mit
der Überwindung von Subjekt-Objekt-Verhältnissen in der Wissensproduktion
diskutiert wird. Damit kristallisiert sich bereits hier die grundlegende
Frage, wem in Erkenntnisprozessen ein Subjektstatus zugesprochen wird und
auf welchen Ausschlüssen das Subjektverständnis basiert. Hiermit erweitert
sie die Diskussion der Fragmentierung des Selbst in Bildungsprozessen um
die Frage, wem in Bildungsprozessen überhaupt das Privileg zukommt, als
Subjekt angerufen zu werden und weist auf Praxen des Silencing hin.

Die Möglichkeit, individuelle Erfahrungen nicht nur zu artikulieren, sondern
sie als legitimen Wissensbestand, der sich durchaus auch im Dissens zu anderen
Wissensbeständen behaupten kann, gebrauchen zu können, kann hier als notwendige
Bedingung gesetzt werden, um Bildungsräume zu schaffen, in denen Lernende als
Subjekte agieren können: „The authorization of experience is thus a crucial
form of empowerment for students-a way for them to enter the classroom as
speaking subjects.“\footnotemark\footnotetext{Ebd., 193.} Bildung ist somit auf Subjekte angewiesen, die sich selbst
und damit ihre Stimme als notwendigen Teil von Bildungsprozessen begreifen.
Kritische Bildung verfolgt hierbei den Anspruch, jenen subjektiven Status nicht
als selbstverständlich anzunehmen, sondern die Mechanismen seiner Herstellung
in den Blick zu nehmen: 
\begin{myenv} 
  \textit{
„Without this analysis of culture and of experience in the classroom, there is no way to develop and nurture oppositional practices. After all, critical education concerns the production of subjectivities in relation to discourses of knowledge and power.“\footnotemark\footnotetext{Ebd., 196.} } \end{myenv}

Subjekte müssen also immer in ihrer Beziehung zu Diskursen, Wissen und Macht
untersucht werden. Subjektivierung, also der Prozess der Subjektwerdung, lässt
sich, so verstehe ich hier Mohanty, über eine Analyse von Erfahrung
untersuchen, die Erfahrung immer im Kontext historisch gewachsener Verhältnisse
verortet. Erst wenn Erfahrung zum Gegenstand der Analyse gemacht wird, kann
widerständiges Wissen entstehen. Dabei ist es unabdingbar die
(welt-)gesellschaftlichen Verhältnisse als Ausdruck und Teil ihrer Geschichte
zu begreifen. Sie müssen historisiert werden, um herrschende Differenzordnung
und ihre Wirkung auf Subjekte einbeziehen zu können. Mohantys Position knüpft
hier an die bereits explizierten Verständnisse von Bildung und Kritik an und
erweitert sie um die vielschichtigen und stets kontextgebundenen Bedeutungen,
die Erfahrungen in den Prozessen der Selbst- und Fremdzuschreibung von
Subjektpositionen einnehmen. Prozesse der Ermächtigung können hierbei nicht
losgelöst von Prozessen der Unterwerfung gedacht werden. 

Die kritische Auseinandersetzung mit Erfahrung kann dabei insbesondere im
Seminar, so führt sie fort, einen zentralen Stellenwert einnehmen. Erfahrung
müsse hier als Ausdruck von Erlebtem und von textueller und historischer
Repräsentation diskutiert werden. Nicht die Autorität der Erfahrung, sondern
eine Leidenschaft für Erfahrung und Erinnerung erlaube es, erfahrungs- und
analytisch orientierte Wissensbestände miteinander in Bezug zu
setzen.\footnotemark\footcitetext{bellhooks}

An dieser Stelle wird das eingangs erwähnte Verhältnis, das zwischen Erfahrung
als vermittelnder und vermittelter Größe besteht, konkreter. Mit Erfahrung als
vermittelnde Größe verstehe ich eben jenen Prozess der Krisenerfahrung, den
Mohanty hier als das Erleben beschreibt. Erfahrung als vermittelte Größe
fokussiert indessen den Prozess, in dem eigene oder Erfahrungen anderer als
Werkzeug für Bildungsprozesse zu sogenannten Vermittlern über gesellschaftliche
Verhältnisse werden, die es ermöglichen einen intersubjektiven Austausch
einzugehen. Während sich erstere also stärker auf das Erlebte und dessen intime
Verarbeitung beziehen, steht bei letzterem der Einzug dieses Erlebten in den
Diskurs und den damit möglicherweise einhergehenden Impulsen, die dies bei
Anderen in ihren Bildungsprozessen anstößt im Vordergrund.

Die Diskussion von Mohanty und hooks in Hinblick auf diese Differenzierung
zeigen jedoch eines: Eine Trennung von Erfahrung als Erlebtem und Erfahrung als
Erzähltem darf nicht darüber hinwegtäuschen, dass sowohl das Erleben als auch
das Erzählen von hegemonialen Strukturen nicht losgelöst sind, die Einfluss
darauf haben, was sag-,  denk- und damit auch erfahrbar ist. Und so wird es
dringend notwendig, dass „[...] within the classroom, [...] teachers and
students develop a critical analysis of how experience itself is named,
constructed, and legitimated in the academy.“\footnotemark\footnotetext{Mohanty, „On Race and Voice,“ 196.}

Dieser Schlussappell ist charakteristisch für Mohantys Anliegen, Lehrende und
Lernende in die Verantwortung zu ziehen und den Leser\_innen weniger Rezepte
als viel mehr Aufgaben mitzugeben. Sie fordert, eine kritische Analyse der
Verortung, Konstruktion und Legitimierung von Erfahrung in der Wissenschaft,
kurz, einen reflexiven Umgang mit Erfahrungen in universitärer
Wissensproduktion. Es geht weder Mohanty noch hooks darum die Leerstellen, die
durch hegemoniale Wissensstrukturen entstanden sind, nun mit Erfahrungen der
Marginalisierten zu füllen. Stattdessen weisen hooks und Mohanty darauf hin,
dass das Sprechen aus und über Erfahrung bereits ein zentraler Bestandteil in
der Auseinandersetzung mit Wissen einnimmt. Was fehlt ist jedoch eine reflexive
Auseinandersetzung damit, wie Erfahrung eigentlich beschaffen ist und welchen
Ort sie in der Produktion und Legitimation von Wissen erhält. 

Für mich stellt sich hier die Frage, ob der Erfahrungsbegriff den
Identitätsbegriff unter Umständen ablösen, und so einen kontingenten,
historisch verorteten und zugleich zutiefst subjektive Dimension dessen, was
Menschen verbindet und trennt, erahnen lässt.  Was, wenn Wissen nicht mehr an
Identitäten, sondern an Erfahrungen gekoppelt wird und wenn diese Erfahrungen
nicht als statisch, sondern wandelbare Interpretationen des Erlebten, ja
Widerfahrenen verstanden werden? 

%\section{Episteme} \epigraph{\textit{ Eine von postkolonialer Theorie
inspirierte 'Pädagogik' richtet ihren Blick dabei insbesondere auf die
gelernte Vergessenheit, auf die aktiv produzierten Amnesien und deren
Komplizenschaft mit dem imperialistischen Projekt. In den Hochschulen wie auch
in den katholischen Kindergärten kann 'Wissen' sich immer auch
widerständig gegen die 'Institution' selbst wenden. }}{María do Mar
Castro Varela\footnotemark} \footnotetext{María do Mar Castro Varela, \glqq
Verlernen und die Strategie des unsichtbaren Ausbesserns. Bildung und
Postkoloniale Kritik,\grqq download unter:
\url{http://www.igbildendekunst.at/bildpunkt/2007/widerstand-macht-wissen/varela},
am 26.11.2014.} 

Im letzten Kapitel habe ich eine Idee von Universität skizziert,
die Bildung und Kritik zur Aufgabe eines universitären Selbstverständnisses
erklärt, das an Reflexivität orientiert ist. Mit Mohanty und hooks argumentierte
ich anschließend für die Notwendigkeit, als Untersuchungsgegenstand konkrete,
bildungs-pädagogische Praxen in in den Blick zu nehmen. Eine machtkritische
Analyse fordert hier, Mecheril et al. folgend ein, Differenz, also Praxen des
Unterscheidens zu untersuchen, die in jenen Formen des 'sich bildens' und
'Kritik übens' subjektivierend wirken. Dazu haben hooks und Mohanty das Konzept
der Erfahrung, als Sprechen aus und über Erfahrung diskutiert und dessen
Bedeutung für emanzipatorische Prozesse an der Universität befragt. Hier weisen
sie auf die Notwendigkeit hin, Subjekt-Objekt Verhältnisse in
Wissensproduktionen zu historisieren und aufzubrechen.\\
Meine Auseinandersetzung
war dabei von der Frage geleitet, woran sich eine Reflexion, die sich mit der
Bedeutung von Erfahrung in Wissensproduktion und Bildungsprozessen aber auch
Praxen der Kritik beschäftigt, orientieren kann. Dabei konnte ich aufzeigen,
dass das Sprechen aus und über Erfahrung sowohl stabilisierend als auch
irritierend auf hegemoniale Ordnungen, darüber was von wem sagbar ist, wirkt.\\

\noindent Wann, von wem oder wogegen Kritik geübt wird und wessen Selbst- und
Weltverhältnisse in den benannten Bildungsprozessen zur Disposition stehen wurde
dabei insbesondere von Mohanty und hooks zum Thema gemacht. Sie markieren die
herrschenden Differenzverhältnisse in universitären Seminarkontexten als
sexistisch und rassistisch und Mohanty fordert in ihrem Text explizit die
Dekolonisierung von Universität.Damit macht sie deutlich, dass Rassismus und
Sexismus im kolonialen Verhältnis situiert sind und über eine historisierende
Analyse nachvollzogen werden müssen.\\
Ich knüpfe an diese Forderung an, und
richte in diesem Kapitel einen dekolonialen Blick auf Episteme. Dabei
interessiere ich mich dafür, welchen Ort Erfahrung in der
Legitimations\textbf{geschichte} von Wissen eingenommen hat und wie dies aus sexismus-
und rassismuskritischer Perspektive diskutiert wird. Von einer historisierenden
Analyse verspreche ich mir einen differenzierteren Zugang dazu, wie
Kolonialismus als (welt)gesellschaftliches Verhältnis in Wissensproduktionen
eingewirkt hat und von welchen Positionen Erfahrung in Wissensproduktionen
nutzbar gemacht werden konnte.\\
Dazu möchte ich in diesem Kapitel erkenntnistheoretische Überlegungen
anschließen, die sich einer rassismus- und sexismuskritischen
\textbf{Geschichts}schreibung widmen. Die Verbindung von  epistemologischer Praxis und
Geschichtsschreibung liegt hier nahe, da sich Erkenntnisse in die Narrative
über die Vergangenheit und Gegenwart einschreiben und so ihre diskursive Wirkung
entfalten können.\\

María do Mar Castro Varela versteht, wie im eingangs gewählten Zitat deutlich
wird, unter einer postkolonialen Pädagogik die Aufgabe, sich dem Vergessen
entgegenzustellen. 'Vergessen' beschreibt sie nicht wie gemeinhin üblich als
unbeabsichtigten, oder gar unvermeidbaren Nebeneffekt der Wissensaneignung,
sondern als aktives und gelerntes Verhalten. Dekoloniale Geschichtsschreibung
setzt hier an und untersucht die Bedingungen, unter denen das aktive Vergessen
zur gewöhnlichen Praxis wird, um so Möglichkeiten des Erinnerns aufzuzeigen. Das
Feld, auf dem jene Auseinandersetzungen um die Leerstellen des Wissens unter
einer historisierenden Perspektive geführt wird, bildet somit den Rahmen dieses
Kapitels.\\
Bevor ich mich wie nun angekündigt mit sexismus- und rassismuskritischer
Epistemologie beschäftige, möchte ich zunächst ganz allgemein skizzieren, was
unter Episteme verstanden werden kann.\\

Foucault unterscheidet zwei Interessensgebiete auf denen
Wissenschaftshistoriker\_innen, also jene Theoretiker\_innen, die sich mit der
Geschichte des Wissens auseinandersetzen, gemeinhin arbeiten. So würden sie sich
zum einen mit den öffentlichen, wissenschaftlichen Auseinandersetzungen
beschäftigen, indem sie die Errungenschaften spezifischer Disziplinen und damit
einhergehende Kontroversen nachzeichneten. Zum Anderen würden sie sich den
verborgenen oder impliziten Einflüssen zuwenden, um diese als Hindernisse,
Störungen und Verzerrungen, die die wissenschaftliche Qualität mindern,
zurückzuweisen.\footnotemark \footnotetext{Michel Foucault, \textit{Die Ordnung
der Dinge}, (Frankfurt am Main:Suhrkamp Verlag, [1966 frz] 1974).} \\
Epistemologie
wird hier also als eine historisierende Analyse des Wissens beschrieben, mit dem
Zweck letzteres zu prüfen und zu ordnen.  Dies knüpft jedoch nicht an die
eingangs formulierte Forderung von Castro Varela an, nach den Leerstellen und
Auslassungen zu forschen, die jene Wissenschaftsgeschichte begleiten, und u.U
sogar konstituieren. Stattdessen scheint es viel eher darum zu gehen, Paradigmen
zu skizzieren, um sie dann auf ihre wissenschaftliche Objektivität hin zu
prüfen.\\

\noindent Das Verständnis von Episteme, das ich dieser Arbeit zu Grunde legen möchte
grenzt sich entsprechend von jenem, beschriebenen Konzept ab und orientiert sich
an dem Episteme Begriff den Foucault in seinem erstmals 1966 in Frankreich
erschienenen Werk \glqq Die Ordnung der Dinge\grqq \footnotemark \footnotetext{Foucault, \textit{Die Ordnung der
Dinge.} } entwickelt:
Foucault kritisiert hier die eben beschriebenen, konventionellen Vorstellungen
der Wissenschaftshistoriker\_innen, nach denen das explizite dem impliziten
Wissen gegenübergestellt werden kann, um ersteres von letzterem zu bereinigen.
Sein Verständnis von Episteme grenzt sich von jenen dargestellten
Interessensgebieten ab, da ähnlich wie Castro Varela auch er sich insbesondere
für das interessiert, was gemeinhin als Störfaktor ausrangiert werden soll. \\
Das
Implizite, so vermutet er in seiner Studie, ist zwar außerhalb des Bewusstseins
des\_der Theoretiker\_in, greift jedoch als Gesetzmäßigkeit in den Aufbau
wissenschaftlicher Erkenntisgenerierung ein und bestimmt darüber \textit{wie}
gedacht wird.\footnotemark \footnotetext{Ebd., 11.} 
Es könne somit weder als komplementär, noch als verzichtbar
gegenüber dem ausgesprochenen Wissen verstanden werden.\footnotemark
\footnotetext{Ebd., 13.} 
 Das Unbewusste wird
also nicht analysiert um es auszulöschen, sondern um darüber in Erfahrung zu
bringen, wie das Denkbare vom Undenkbaren unterschieden werden kann, bzw. was
das Denkbare bedingt und das Undenkbare möglich macht. Leerstellen, so können
Foucault und Varela hier zusammengebracht werden, befinden sich also nicht
außerhalb, sondern innerhalb des Wissens. Sie greifen in das Wissen ein, machen
es erst möglich oder wie Foucault schreibt 'denkbar'.\\ 

\noindent Die Suche nach den
Leerstellen, so kann dies im Umkehrschluss interpretiert werden, dient nicht
einer Ergänzung des bereits ausgesprochenen, legitimierten Wissens. Die Annahme,
dass das nicht Denkbare das Denkbare bestimmt, lässt indes vermuten, dass die
Suche nach den Leerstellen mitsamt ihrer Markierung eine tiefgreifendere
Erschütterung hervorruft. Denn wenn Leerstellen das Fundament darstellen, auf
dem das Wissen ruht, können sie nicht einfach zum Gerüst werden, ohne das Gerüst
zum einstürzen zu bringen.\\
Episteme versteht Foucault entsprechend als machtvolles Gefüge, das darüber
bestimmt, welche von allen möglichen Aussagen als wissenschaftliche Aussagen
akzeptiert und Eingang in den wissenschaftlichen Diskursraum erhalten, um dort
diskutiert zu werden.\footnotemark \footnotetext{ Foucault, \textit{Dispositive
der Macht. Über Sexualität, Wissen und Wahrheit} (Berlin:Merve Verlag, 1978), 122. }\\

\noindent Die Befragung der Episteme richtet ihren Blick nun auf diejenigen Mechanismen,
die legitime von illegitimen Aussagen unterscheiden, um erstere sicht- und
letztere unsichtbar zu machen. So leitet er seine Studie mit der Frage ein:
\glqq Was ist eigentlich für uns unmöglich zu denken? Um welche Unmöglichkeit
handelt es sich? \grqq \footnotemark \footnotetext{Foucault, \textit{Die Ordnung
der Dinge}, 17.}\\

\noindent Um herausfinden zu können, was für 'uns' unmöglich zu denken ist, scheint es mir
ratsam, zunächst die Entstehung und Manifestierung dessen zu untersuchen, was
für 'uns' ganz selbstverständlich denkmöglich ist. Dabei lehne ich es in
Abgrenzung zu Foucault ab, von einem kollektiven 'wir' oder 'uns' auszugehen, da
dies mit einer sexismus- und rassismuskritischen Perspektive im Widerspruch
steht. Statt mich identitär zu verorten, präferiere ich die Konkretisierung
meiner Perspektive als Interesse für die geschichtlichen Möglichkeitsbedingungen
eines euro- und androzentrischen Wissens. \\
Dazu werde ich mich auf ein Kollektiv
beziehen, das ausgehend von der Idee der \glqq Verwestlichten Universität \grqq 
\footnotemark \footnotetext{Capucine Boidin et al. \glqq Introduction: From
University to Pluriversity: A Decolonial Approach to the Present Crisis  of
Western Universitiesn \grqq in Human Architecture: Journal of the Sociology of
Self Knowledge Vol.10 Issue 1 \textit{Decolonizing the University: Practicing
Pluriversity}, (2012).} den 
symbolischen Ort markiert, auf den ich mich beziehe.\\

\noindent In diesem Zusammenhang
möchte ich erstens klären, von welcher Universität in Zeit und Raum ich ausgehe,
zweitens markieren aus welchem Standpunkt ich versuche sie zu beschreiben und
drittens diskutieren wie die Geschichte dieser Universität (hier immer als
symbolischer Ort gedacht, vgl. voriges Kapitel) zum Gegenstand widerständiger
epistemologischer Praxen gemacht wird. Mein Interesse nach der Bedeutung von
Erfahrung in der Legitimationsgeschichte von Wissen erfordert also als ersten
Schritt, jene partikulare Wissensgeschichte,  mit der die Geschichte der
Universität verwoben ist, zu skizzieren.\\

\noindent Diese, eine Geschichte auf die ich mich beziehe und die notwendig eine Auswahl aus den vielfältigen Geschichten
darstellt, die über die Universität geschrieben und erzählt worden sind, lässt
sich u.A. dadurch kennzeichnen, dass sie einen Zusammenhang zwischen
Herrschafts- und Eroberungspraxen und der Erfindung von Identität aufzeigt. \\
Dies
macht sie zum Einen an den eingangs vorgestellten Kritik- und Bildungsbegriff
anschlussfähig und zum Anderen ermöglicht sie mir die Klärung eines für
sexismus- und rassismuskritische Theorien grundlegenden und zugleich umkämpften
Bezugspunkt: \\
Identität dient emanzipatorischen Bewegungen als Grundlage, von der
aus Unterdrückung und somit auch Herrschaft untersucht werden kann, da sie einen
zentralen Effekt von Differenzverhältnissen darstellt. Umkämpft hingegen ist der
strategische Umgang mit ihr, also wer sich unter welchen Voraussetzung ihrer
Kategorie bedient und damit in wessen Interessen handelt.\\ 

\noindent Eine kritische
Auseinandersetzung mit Identität erfordert nun, so Foucault, sich mit ihren
Entstehungsbedingungen auseinanderzusetzen, und er fragt: 
\begin{myenv} \textit{
\glqq    Von welchem historischen Apriori aus ist es möglich gewesen, das große
    Schachbrett der deutlichen Identitäten zu definieren, das sich auf dem
    verwirrten, undefinierten, gesichtslosen und gewissermaßen indifferenten
    Hintergrund der Unterschiede erstellt? […] Die Geschichte der Ordnung der
    Dinge wäre die Geschichte des Gleichen, (du meme), das für eine Zivilisation
    gleichzeitig dispers und verwandt ist, also durch Markierungen zu
    unterscheiden und in Identitäten aufzufassen ist.\grqq \footnotemark
  \footnotetext{Foucault, \textit{Die Ordnung der Dinge}, 27.} }
\end{myenv}
Unterschiede, so verstehe ich Foucault, erhalten erst durch ihre Definition ein
'Gesicht'. Sie werden geordnet um sich dann in Identitäten zu manifestieren. In
diesem Kapitel möchte ich nun jenes Wissenschaftsverständnis, das sich in der
Europäischen Moderne durchsetzte, und das, in unmittelbaren Zusammenhang mit der
von Foucault konstatierten Erfindung der Identität steht, untersuchen. Dazu
werde ich es anhand der Subjekt- und Erkenntnis-Theorien bei Descartes, Kant,
Hegel und Marx aus einer sexismus- und rassismuskritischen Perspektive
explizieren um daran anschließend die Konturen einer dekolonalen Epistemologie
aufzuzeigen.\\

\noindent Hierfür möchte ich mich zunächst auf Gayatri Chakravorty Spivak beziehen, da
sie die o.g. Philosophen\_*, in direkten Zusammenhang mit der Etablierung einer
\glqq menschlichen Norm \grqq \footnotemark \footnotetext{Barbara
Gabel-Kunningham et al. \glqq Vorwort der Übersetzer und Übersetzerinnen \grqq, In:
\textit{Spivak, Gayatri Chakrabarty: Kritik der Postkolonialen Vernunft. Hin zu einer
Geschichte der verrinnenden Gegenwart}, (Stuttgart: Kohlhammer Verlag),7.}
stellt, die unter Bezugnahme von \glqq inferiorisierenden Alteritätskonstrukten
\grqq das Mittel- und Westeuropäische als natürliche Identität und Denkstruktur
konstruierten.\\
Jene Denkansätze, müssen, so fordert sie, in ihrer unmittelbaren
Komplizenschaft mit dem kolonialen Projekt untersucht werden und ihre
Grundannahmen als kolonialpolitische Normierung dekonstruiert werden.
\footnotemark \footnotetext{Gabel-Kunningham, \glqq Vorwort \grqq, ebd.}
Barbara Gabel-Kühnung die mit ihren Kolleg\_innen Spivaks Monographie \glqq Die Postkoloniale Vernunft. Hin zu einer verrinnenden
Geschichte der Gegenwart \grqq \footnotemark \footnotetext{Gayatri Chakrabarty
  Spivak: \textit{Kritik der Postkolonialen Vernunft. Hin zu einer Geschichte
der verrinnenden  Gegenwart}, (Stuttgart: Kohlhammer Verlag)} vom Englischen ins Deutsche übersetzt hat schreiben dazu: 
\begin{myenv} 
  \textit{ \glqq Reale ethische
  Handlungsmöglichkeiten in einer globalen Welt sind nach Spivak nur möglich,
wenn die Geisteswissenschaften die weitreichenden kolonialen Komplizenschaften
'Europas' durch dekonstruktive Lektüren bewusst machen, Komplizenschaften die
in unsere Denktradition eingeschrieben sind, gerade deswegen 'natürlich'
erscheinen und insofern oft der Kritik entgehen \grqq \footnotemark
\footnotetext{Gabel-Kunningham, \glqq Vorwort,\grqq 14.}} 
\end{myenv} 
Sie führen fort, dass gerade in der allgemeingültig verstandenen Kategorien 'Mensch' die
\glqq einheimische Informantin \grqq verworfen wird:
\begin{myenv} \textit{
    \glqq Die einheimische Informantin ist in Spivaks Studie eine Leerstelle,
die in westliche ethnographische, philosophische, kulturelle, literarische und
historische Diskurse eingeschrieben wurde \grqq \footnotemark \footnotetext{Gabel-Kunningham, 15.}} 
\end{myenv}
Die Leerstelle, oder das was unmöglich ist zu denken, ist also untrennbar mit
der Kategorie 'Mensch' verbunden, bzw. in sie eingeschrieben. Eine Analyse der
Leerstellen erfordert nun also die Bedingungen zu untersuchen, die der
Nutzbarmachung der Kategorie 'Mensch' zu Grunde liegen. \\
Meine Auseinandersetzung mit
der Geschichte der Universität, oder dem Denken, das die Idee der Universität
ermöglicht, werde ich entsprechend mit einer Auseinandersetzung um die
Kategorie 'Mensch', und ihren Ausschlüssen beginnen.\\
Die Kategorie 'Mensch', so
meine Vermutung, hat ein Denken über 'Mensch' sein ermöglicht, das auf der
Verunmöglichung eines anderen 'Menschsein' beruht.\\

\noindent  Für meine Arbeit bedeutet
dies, dass ich mich mit einer (Idee von) Universität auseinandersetze, deren
Geschichte bis auf die Europäische Moderne zurückgeht und im Anschluss an
Spivak somit in einer unmittelbaren Kompliz\_innenschaft mit dem kolonialen
Projekt steht. Um diese Leerstelle beschreiben zu können, bedarf es zweifellos
der Perspektiven jener, die durch die Kategorie 'Mensch' kategorisch und
faktisch vom Status des Subjektes ausgeschlossen wurden. Das dekoloniale
Projekt stellt sich diesem schwierigen Anspruch der Repräsentation.

%\subsection{Hegemoniales Wissen} 
%\subsubsection{Perspektiven auf hegemoniales
%Wissen} \subsubsection{Entstehung und Durchsetzung hegemonialen Wissens}
%\subsubsection{Methodologie hegemonialen Wissens} \subsection{Widerständiges
%Wissen}

%\subsection{Hegemoniales Wissen}
\epigraph{\textit{ 
I need to understand how a place on the map is also place in history [...]
}}{Adrienne Rich \footnotemark} \footnotetext{
Adrienne Rich, \glqq Notes toward a Politics of Location,\grqq unbekannt, 1984.
} 

Wie ein bestimmtes Wissen hegemonial werden und damit andere Wissensformen an
den Rand drängen konnte, ist Gegenstand folgender Auseinandersetzung um die
Grenzen des Denkbaren an der Universität. Die Frage nach den Grenzen des
Denkbaren und danach, wie sie entstanden, verfestigt und gebrochen wurden
erfordert sich die \textbf{Geschichte} des Denkens an der Universität näher anzuschauen,
denn sie bildet den Ort von dem aus und über den ich spreche. Das heißt auch
die bereits angekündigt, konkreter darin zu werden, von welcher Universität in
Zeit und Raum ich ausgehe, aus welchem Standpunkt ich versuche sie zu
beschreiben und zu fragen, inwiefern die Geschichte dieser Universität zum
Gegenstand widerständiger epistemischer Praxen gemacht wird. Kein
deterministisches Geschichtsverständnis, sondern ein Verständnis von Gegenwart,
das sich der Kontingenz ihrer selbst über die Auseinandersetzung mit ihrer
Geschichte nähert, und dabei versucht die Ausschlüsse, die jene Gegenwart
mit\_hervorgebracht haben, nachzuvollziehen, motiviert mich zu diesem Vorhaben. 
\subsubsection{Perspektiven auf hegemoniales}

Die Perspektiven, aus denen heraus ich mich mit  hegemonialem Wissen
beschäftigen möchte, habe ich bisher als sexismus- und rassismuskritisch
beschrieben. Die epistemische Praxis, in der jene Perspektiven Anwendung
finden,  benenne ich im Anschluss an Mohantys Ausführungen im vorherigen
Kapitel als Dekolonisierung. Die Auseinandersetzung damit wofür diese
Begrifflichkeiten stehen (können), soll hierbei statt einer einmaligen
Einführung fortlaufend praktiziert werden. Die Anwendung sexismus- und
rassismuskritischer Ansätze bedeutet in diesem Sinne niemals nur die Verwendung
einer Perspektive, sondern immer auch die Befragung ihrer Vorannahmen,
Interessen und Vorgehensweisen, also einem learning on the way. Hierbei
interessiere ich mich insbesondere für die Verbindungslinien zum Postkolonialen
Feminismus, als eine kontextualiserte epistemologische Position, aus der heraus
Dekolonisierung praktiziert wird. Wie Erfahrung gedacht, nutzbar gemacht oder
auch verworfen wird, um eine Idee des 'Menschen' zu entwickeln, bildet hierbei
das Zentrum um das sich meine Diskussion kreist. Je nach theoretischer Schulung
wird sich auf den 'Menschen', das Subjekt oder auch die Identität bezogen.

Zunächst zu meiner Verwendung des Begriffspaars sexismus- und
rassismuskritisch: Jene Perspektiven verhalten sich, anders als das 'und'
vermuten lässt, alles andere als komplementär zueinander. Intersektionalität,
ein politischer und wissenschaftlicher Ansatz zeigt indessen auf, dass die
Differenzverhältnisse die unter diesen Perspektiven analysiert werden,
ineinandergreifen.

ntersektionalität beansprucht dabei die Wirkungsweisen von Geschlecht und
'Race' nicht isoliert von anderen Differenzen zu betrachten, und fordert eine
Analyse, in der verschiedene, ineinander wirkende Differenzverhältnisse
berücksichtigt werden. Stoller und Vetter machen hier darauf aufmerksam, dass
jene Differenzverhältnisse nicht als identitäre und damit feststehende
Kategorien missverstanden werden dürfen. Der Vorgriff auf Identität, der ihrer
Ansicht nach die traditionelle Philosophie kennzeichnet, führe unweigerlich
dazu, dass Alterität stets im Anderen, im Fremden gesucht werde und dabei
verkannt wird, dass die Differenz immer schon das Eigene konstituiert und ergo
auch im vermeintlich Unbekannten \glqq immer schon das Bekannte zu finden ist.\grqq
\footnotemark \footnotetext{Silvia Stoller und Helmuth Vetter, 
\glqq Einleitung, \grqq in \textit{Phänomenologie und Geschlechterdifferenz}, herausgegeben von ebd. (Wien: WUV Universitätsverlag, 1997), 8. }

Postkolonialer Feminismus kann nun als eine theoretische Position verstanden
werden, die jene Differenzverhältnisse im kolonialen Verhältnis situiert und in
ihrer Verzahnungen mit Wissensproduktion untersucht. Damit rücken
epistemologische Fragen in den Vordergrund, die feministische und postkoloniale
Perspektiven zusammen denken.

Feministische Epistemologie hinterfragt die Möglichkeit eines neutralen Wissens
a priori und geht davon aus, dass der Kontext bzw. der Wille aus dem heraus das
Wissen entsteht, maßgebend ist. Hinter jedem Wissen das produziert wird, steht
also nicht nur ein situierter Blickwinkel sondern auch ein spezifisches
Interesse, das den\_die, der\_die das Wissen erzeugt, leitet. Ein grundlegendes
Spannungsfeld, das mit dieser radikalen Befragung nicht nur des Erzeugten,
sondern auch des\_der Erzeugenden einhergeht, fragt nun nach dem Verhältnis von
Subjektivität zu Objektivität bzw. Relativität zu Universalität und den
Konsequenzen, die dies für die Relevanz, ja die Sprengkraft eines Wissens mit
sich bringt.\footnotemark \footnotetext{Linda M. Alcoff und Elizabeth Potter:
  \glqq Introduction. When Feminisms Intersect Epistemology,\grqq in
\textit{Feminist Epistemologies}. Herausgegeben von ebd. (London: Routledge, 1993), 1.} 
Die Politisierung von Wissensproduktionen als machtvolle Gefüge,
die immer auch von vergeschlechtlichten (sexed) Subjekten aus einem
spezifischen gesellschaftlichen Staus heraus gemacht werden, greift damit die
grundlegenden Strukturen der männlich dominierten Wissenschaft an, indem sie
das benennt, was bis dahin als unsichtbare Größe die Neutralität des Wissens
beanspruchen konnte. Das Zentrum wird in Frage gestellt.\footnotemark
\footnotetext{Alcoff und Potter, \glqq Introduction,\grqq 3.}

Postkolonialer Feminismus kann nun gewissermaßen als Korrektiv einer
feministischen Epistemologie gelesen werden, die die \glqq subjekttheoretische
Überlegungen [...] lediglich als dekontextualisierte erkenntnistheoretische
Fragestellungen[...] \grqq \footnotemark \footnotetext{Hito Steyerl und
Encarnacion Rodriguez, \glqq Einleitung, \grqq in \textit{Spricht die
Subalterne deutsch?  Migration und Postkoloniale Kritik}, herausgegeben von
ebd. (Münster:Unrast Verlag, 2003), 9.} betrachtet. Postkolonialer Feminismus
fordert dazu auf die kolonialen Diskurse in den Blick zu nehmen, die u.a. die
Kategorie betrachtet. Postkolonialer Feminismus fordert dazu auf die kolonialen
Diskurse in den Blick zu nehmen, die u.a. die Kategorie Geschlecht bestimmen
und \glqq analysiert die Gewalt des Westens \grqq \footnotemark
\footnotetext{Castro Varela, und Dhawan \glqq Postkolonialer Feminismus und die
  Kunst der Selbstkritik \grqq in \textit{Spricht die Subalterne deutsch? Migration und
  Postkoloniale Kritik}, herausgegeben von Hito Steyerl und Encarnacion Rodriguez
(Münster: Unrast Verlag, 2003), 271.} in der Herstellung der 'Anderen
Frauen\_*' über die sich die unmarkiert '\textit{w}eiße Frau\_*' konstituiert.
Castro Varela und Dahwan stellen Postkolonialen Feminismus hier als
\glqq widerständigen, antihegemonialen Gegendiskurs dar, der sich insbesondere gegen
epistemologische Ausschlussverfahren richtet.\grqq \footnotemark
\footnotetext{Castro Varela und Dhawan, \glqq Postkolonialer Feminismus \grqq, ebd.}

Meine Präferenz für die Position des Postkolonialen Feminismus liegt nun darin
begründet, dass sie, wie es der Anspruch auf Reflexivität vorsieht, sich nicht
außerhalb dessen versteht, was sie zu untersuchen bestrebt. Postkolonialer
Feminismus ist mit dem Gegenstand, nämlich kolonialer und patriarchaler
Strukturen der Europäischen Moderne verwoben und es ist eben jene
Auseinandersetzung mit der eigenen Involvierung, die diese Position für mich,
für ein an Reflexivität orientiertes Vorhaben, interessant macht. Die
Thematisierung von der sogenannten unvermeidbaren Komplizenschaft mit jenem,
das es zu überwinden gilt, macht in Gayatri Spivaks Sinne Mut, sich nicht
außerhalb, sondern innerhalb der Struktur zu begreifen, die eine kritisiert,
und mit denen sie doch \glqq aufs engste vertraut ist. \grqq \footnotemark 
\footnotetext{Spivak, zitiert in Castro Varela und Dhawan,
\textit{Postkolonialismus. Eine Kritische Einführung}, 202.}

Wie eingangs formuliert, gehe ich jedoch nicht davon aus, Lösungen zu finden.
Mich interessiert, welche Möglichkeiten Postkolonialer Feminismus für eine
Suche nach dem Nicht\_denkbaren an der Universität bietet: Welchen Beitrag
leisten Konzeptionen von Erfahrung(swisssen) für eine Kritik an hegemonialen
Wissensbeständen? Ist die Überschreitung des Mensch(lichen) möglich?

\subsubsection{Entstehung und Durchsetzung hegemonialen Wissens}

Nachdem ich nun in Ansätzen die Perspektive aus der ich auf hegemoniales Wissen
blicke vorgestellt habe, beschäftige ich mich nun aus jener Perspektive mit der
Konkretisierung des Ortes über den ich spreche und der bisher mit 'der
Universität' zeit- und räumlich unbestimmt blieb. Dazu beziehe ich mich mich
auf wissenschaftliche Beiträge einer Konferenz, die vor nicht allzulanger Zeit
in Paris stattfand und den Titel: \glqq Quelles universités et quels
universalismes demain en Europe? Un dialogue aves les Amériques \grqq
\footnotemark \footnotetext{Boidin et al., \glqq From University to
Plurisversity \grqq 2012.} trug. Während der Titel er auf das zukünftige
verweist und nach den Universitäten und Universalismen von Morgen in Europa
fragt, bestand der Ausgangspunkt dieser Konferenz in der Dekolonisierung der
bisherigen 'Westernized University' mitsamt ihren eurozentrischen
Wissensstrukturen. Die Konferenz proklamierte dabei das Ziel einer dekolonialen
Intervention in akademische Wissensproduktion und verwestlichte
Universitätsstrukturen. Die Organisator\_innen gingen, ähnlich wie die in dieser
Arbeit eingangs vorgestellten Perspektiven, davon aus, dass eine
kapitalismuskritische Analyse nicht ausreicht, um die Ursachen der Krise der
Universität zu bestimmen. Kritik an neoliberalen Vereinnahmungen wie sie
beispielsweise in Zusammenhang mit Bologna diskutiert werden \footnotemark
\footnotetext{Vgl. Lohman et al. Herausgerber\_innen, \textit{Schöne Neue
Bildung.}} greifen ihrer
Ansicht nach nicht tief genug, da sie die Bedeutung rassistischer und
sexistischer Praxen der Dehumanisierung/Entmenschlichung als konstitutiven
Bestandteil in der Geschichte der Universität verkennen. Meine Frage von
welcher Universität in Zeit und Raum ich ausgehe, wird durch ihre
Konferenzbeitrage dabei um zwei Dimensionen konkretisiert: Zum einen fassen die
Autor\_innen Uni-versität in Abgrenzung zur Pluri-versität und beziehen sich
damit auf den auf Universalität ausgerichteten Wissensanspruch der Eurpäischen
Moderne als Entstehungskontext von Uni-versität. Zum Anderen beschreiben sie
die Bedingungen unter denen Uni-versität entstanden ist: Europas Rolle in
Epistemiziden und ihre Mündung in solipsistischen und dualistischen
Denkstrukturen. Wie bereits im Zusammenhang mit der Konturierung des
Postkolonialen Feminismus erwähnt, wird Wissensproduktion auch hier in einem
durch koloniale Herrschaft strukturierten Raum verortet. Diesen zu explizieren,
ist nun das Anliegen von Ramón Grosfoguel, einem der Vortragenden, und wird im
folgenden erörtert. \footnotemark \footnotetext{ Ramón Grosfoguel, \glqq The
Structure of Knowledge in Westernized Universities: Epistemic Racism/Sexism and
the Four Genocides/Epistemicides of the Long 16th Century, \grqq in \textit{Human
Architecture: Journal of the Sociology of Self-Knowledge}: Vol. 11: Iss. 1,
Article 8., (2013): VIII.} Dabei steht zunächst der Zusammenhang, der zwischen
den sogenannten (kolonialen) Epistemiziden und einer solipsistischen und
dualistischen Denkstruktur hergestellt wird, im Vordergrund.  

Grosfoguel bezieht sich in seinen Ausführungen zum Epistemizid der Europäischen
Moderne auf vier Genozide bzw. Epistemizide die allesamt im 17. Jahrhundert
verübt wurden. Diese Epistemizide richteten sich, so Grosfoguel, gegen die
muslimische und jüdische Bevölkerung der spanisch\_iberischen Halbinsel
Al-Andalus, gegen die indigene Bevölkerung der Amerikas und Asiens, gegen die
Bevölkerung Westafrikas, die versklavt und verschleppt wurde und gegen
europäischen Frauen\_*, die als Hexen verfolgt wurden. Grosfoguel betont in
seiner Darstellung die Notwendigkeit, die Zusammenhänge dieser, meist als
voneinander getrennt untersuchten Phänomene, zu markieren und die rassistische
Logik darin als konstitutiv für die Etablierung einer westeuropäischen
Vorherrschaft und ihres Selbstverständnisses zu begreifen: 

\begin{myenv}
  \glqq \textit{The attempt here is to see them [the epistemicides] as interlinked,
 inter-related to each other and as constitutive of the modern/colonial world’s
 epistemic structures. These four genocides were at the same time forms of
 epistemicide that are constitutive of Western men epistemic privilege. To
 sustain this argument we need to not only go over the history but also explain
 how and when racism emerged.} \footnotemark \footnotetext{Grosfoguel \glqq The
 Structure of Knowledge \grqq, 77.} \grqq 
\end{myenv}

 Seine historische Darstellung ist entsprechend durch das Anliegen geprägt, die
 Denkstrukturen, die o.g. Epistemizide ermöglichten und legitimierten,
 aufzudecken. \footnotemark \footnotetext{Vorab möchte ich dabei bereits
 einschränkend darauf hinweisen, dass Grosfoguel keine differenzierte
 historische Rekonstruktion der angesprochenen Genozide auflistet. Statt einer
 umfangreichen Deskription kennzeichnet sich seine Auseinandersetzung eher
 durch eine punktuelle Analyse in der insbesondere Verbindungslinien innerhalb
 der Genozide hervorgehoben werden. Seine Analyse ist jedoch insbesondere darum
 interessant, weil er die koloniale Eroberungspolitik mit der rassistischen und
 sexistischen Gewaltausübung auf dem europäischen Festland in Verbindung bringt
 bzw. letztere als konstitutiv für die koloniale Gewaltherrschaft expliziert.}

Grosfoguel skizziert dazu zunächst die historischen Entwicklungen im heutigen
Spanien um Fünfzehnhundert. Die Eroberung der iberischen Halbinsel Al-Andaluz
durch die spanische Monarachie, die mit einer gewaltvollen Unterwerfung der
dort lebenden muslimischen Bevölkerung einherging, verfolgte, so Grosfoguel,
die Etablierung einer 'spanisch-christlichen' Vorherrschaft über die dort
lebenden Menschen. Dabei hätte die Idee der Einheit von Staat und Bevölkerung,
im Sinne einer direkten Korrespondenz von subjektiver und kollektiver
Identität, eine zentrale Rolle gespielt. \footnotemark
\footnotetext{Grosfoguel, \glqq The Structure of Knowledge,\grqq 79.} Die mit
der gewaltsamen Eroberung der iberischen Halbinsel einsetzende
Zwangsassimilierung der muslimischen Bevölkerung, käme einem kulturellen
Genozid gleich, da hier eine kulturelle Selbstbestimmung, sowie
intergenerationale Weitergabe kultureller Vorstellungen und Praktiken massiv
unterbunden wurde. \footnotemark \footnotetext{Ebd.}

Theoretisch fasst Grosfoguel diesen Epistemizid in Abgrenzung zu Rassismus als
religiöse Diskriminierung, da im Unterschied zu Rassismus die Menschlichkeit
der unterdrückten Bevölkerung nicht in Frage gestellt worden sei.\footnotemark
\footnotetext{Ebd.} An dieser Stelle verweist er auf den direkten Zusammenhang, der zwischen der Eroberung
der iberischen Halbinsel und dem kolonialen Projekt bestand: Die sogenannnte
'Indian Enterprise' die Christopher Columbus dem König und der Königin des
castilianischen Königreiches vorstellte, sei nur unter der Voraussetzung einer
vollständigen Eroberung der iberischen Halbinsel, bewilligt worden.

Die direkte Abfolge der Eroberung Al- Andalus mit der Eroberung der Amerikas,
ist dabei laut Grosfoguel bisher nur wenig erforscht.\footnotemark
\footnotetext{Ebd.} Im Weiteren zeigt er
auf, wie sowohl die Eroberung von Al-Andalus, als auch die Eroberung der
Amerikas durch einen epistemischen Genozid getragen wurden, der die
vollständige Zerstörung der Wissensarchive wie beispielsweise Bibliotheken und
Schriftsysteme zum Ziel hatte.\footnotemark
\footnotetext{Ebd., 80}  Der Epistemizid wurde, so Grosfoguel dabei
durch die evangelikale Missionierung sowohl in Al-Andalus als auch in der
Amerikas vorangetrieben, die die Vernichtung von insbesondere spirituellen
Formen des Wissens vorantrieb. Mit der Kolonialisierung der Amerikas habe
hierbei erstmals auch eine rasisstische Logik eingesetzt: \glqq But with the
colonization of the Americas, these old medieval discriminatory religious
discourses mutated rapidly, transforming into modern racial domination. \grqq
\footnotemark \footnotetext{Ebd.,82}

Religiosität kam in dieser rassistischen Logik, so argumentiert Grosfoguel
weiter, eine zentrale Bedeutung zu. Denn die mit der Kolonialisierung
einhergehende rassistische Unterteilung in Menschen, denen Menschlichkeit zu-
bzw. abgesprochen wurde, sei unmittelbar an die Praktizierung von Religion
gebunden worden. Diejenigen mit der falschen Religion (muslimische und jüdische
Bevölkerung der iberischen Halbinsel) konnten missioniert werden, denjenigen
ohne Religion (indigene Bevölkerung der Amerikas) wurde das Menschsein an sich
abgesprochen.\footnotemark \footnotetext{Ebd.} 
Diese Unterteilung hätte jedoch niemals absoluten Charakter
gehabt sondern sei immer auch von Stimmen innerhalb der christlichen Kirche
infrage gestellt worden.

Diese 'kritischen' Stimmen argumentierten, dass die Indigenen zwar keine
Religion, aber dennoch eine Seele hätten und aus ihrem 'Barbarismus' mittels
einer Missionierung befreit werden könnten. \footnotemark \footnotetext{Ebd.}
Eine Versklavung wäre damit eine
Sünde und würde von Gott bestraft werden. Der Disput innerhalb der christlichen
Kirche, ist für Grosfoguel das erste schriftliche Anzeichen nicht nur von
Rassismus, sondern damit unmittelbar verknüpft, von der Erfindung der
Identität: \glqq This debate was the first racist debate in world history and
'Indian' as an identity was the first modern identity. \grqq \footnotemark
\footnotetext{Grosfoguel schreibt hierzu genauer: \glqq [...] at the time, the
debate about having a soul or not was already a racist debate in the sense used
by scientific racism in the 19th century. The theological debate of the 16th
century about having a soul or not had the same connotation of the 19th century
scientificist debates about having the human biological constitution or not.
Both were debates about the humanity or animality of the others articulated by
the institutional racist discourse of states such as the Castilian Christian
monarchy in the 16th century or Western European imperial nation-states in the
19th century. These institutional racist logics of 'not having a soul' in the
16th century or 'not having the human biology' in the 19th century became the
organizing principle of the international division of labor and capitalist
accumulation at a worldscale. \grqq ebd.} Grosfoguel skizziert an dieser Stelle
die Folgen des Disputes zwischen Bartolomé und Sepulveda für die
Eroberungspolitik und Versklavung der indigenen Bevölkerung der Amerikas und
der Versklavung und Verschleppung der afrikanischen Bevölkerung in die
Amerikas, sowie den Effekt, den die rassistischen Epistemizide auf die Politik
gegenüber den ursprünglich sich muslimisch-jüdisch identifizierenden und später
christlich konvertierten Menschen z.B. aus der iberischen Halbinsel
hatte.\footnotemark \footnotetext{Ebd., 85.} Die von Spivak und Foucault
problematisierte Kategorie 'Mensch' wird von Grosfoguel hier in ihrem
kolonialen Enstehungszusammenhang expliziert. Er macht deutlich, dass die
Entscheidungsmacht darin, wem das Menschsein zu- oder abgesprochen wurde, bei
den kolonialen Herrschern bzw. den von ihnen beauftragten christlichen
'Gelehrten' lag. Überraschend ist für mich an dieser Stelle, dass seiner
Ansicht nach die Erfindung von Identität, also die Fixierung des Menschlichen
auf konkrete Wesensmerkmale in der Fremdzuschreibung fungierte. Indem er die
\glqq Indian Identity \grqq als erste moderne Form der Identität beschreibt, zeigt er
auf, dass diejenigen, die sich selbst unmissverständlich als Menschen
begriffen, die Idee der Identität jenen aufzwangen, die sich aus ihrer Sicht
das Menschsein erst noch, durch die Annahme der (richtigen) Religion, verdienen
mussten. 

Die Auslöschung von Wissen, das hemonial gemachtes Wissen verunsichern konnte,
wurde ebenfalls durch 'Hexenverbrennung'  versucht. Die Verbrennung von
Frauen\_*, die durch die Inquisition in Europa als Hexen verfolgt wurden wird
von Grosfoguel in Anlehnung an die marxistisch-feministische Theoretikerin
Silvia Federici, mit der frühkapitalistischen Expansion in Zusammenhang
gebracht. Federici argumentiert
\begin{myenv}
  \textit{\glqq the African enslavement in the Americas with the witch hunt of
    Women in Europe as two sides of the same coin: capital accumulation at a
    world-scale in need of incorporating labor to the capitalist accumulation
  process  \grqq \footnotemark \footnotetext{Federici in Grosfoguel, ebd., 86.} }
\end{myenv}

Der Epistemizid richtete sich gegen tausende Frauen\_*, deren \glqq Autonomie,
Führungskompetenzen und Wissen \grqq \footnotemark \footnotetext{Ebd.} 
eine Gefahr für die transnationale Etablierung
eines kapitalistischen Klassensystems darstellte.

m Gegensatz zu dem Epistemizid an Indigenen der Amerikas und der muslimischen
und jüdischen Bevölkerung der iberischen Halbinsel, konnte die Inquisition
keine Bücher verbrennen, da die Weitergabe von Wissen in der Regel durch
mündliche Überlieferung stattfand. \glqq The 'books' were the women’s bodies and,
thus, similar to the Andalusian and Indigenous 'books' their bodies were burned
alive \grqq \footnotemark \footnotetext{Grosfoguel, ebd.}

Zuletzt diskutiert Grosfoguel die Auswirkungen der Genozide auf die
Monopolisierung des Wissens auf den europäischen Mann\_* nördlich der Pyrenäen,
infolgedessen die 'Westernized University' bis heute ihren universellen
Anspruch verteidigt. Dem entgegen führt er die widerständigen
Wissensproduktionen an, die sich zwar nicht vollkommen außerhalb, sondern mit
dem Epistemizid der Europäische Moderne verwoben, aber dennoch eine relative
Außenposition bewahrend, bis heute überlebt haben.\footnotemark \footnotetext{Ebd., 89.}

Wissen, das wird mit Grosfoguel deutlich, das bis heute seine Legitimität
bewahren konnte, hat eine hegemoniale Stellung durch die gewaltvolle Zerstörung
konkurrierender Wissenssysteme erhalten können und entwickelte sich analog zu
der imperialen und kolonialen Etablierung eines transnationalen Kapitalismus,
mitsamt seiner rassistischen und patriarchalen Strukturlogik. 
Mit Grosfoguel wird die Zeit und der Raum der Universität von der ich ausgehe um die
gewaltsame Geschichte europäischer Herrschaft und der der damit einhergehenden
Monopolisierung von Wissen konkreter.

Das Wissen, das an der Universität als legitimiertes Wissen entstehen konnte,
ist in Zeiten und an Orten entstanden, an denen nicht nur Menschen unterworfen
und ermordet wurden, die dem weißen europäischen Mann\_* und dessen Bild vom
Menschen nicht entsprachen, sondern mit ihnen ihre Archive und damit ihre
Bilder und Vorstellungen vom Mensch-sein. Mit den Worten von Encarnacíon
Gutiérrez Rodrigues: \glqq Die Expansion Europas verfolgte nicht nur die Ausbeutung
und Aneignung von Arbeit, Ressourcen und Land, sondern auch eine politische und
kulturelle Unterwerfung der kolonisierten Menschen. \grqq \footnotemark
\footnotetext{Encarnacíon G. Rodriguez \glqq Repräsentationen, Subalternität und
  postkoloniale Kritik,\grqq in  \textit{Spricht die Subalterne deutsch? Migration und
  Postkoloniale Kritik}, herausgegeben von Hito Steyerl und Encarnacion Rodriguez,
(Münster: Unrast, 2003), 19.}
\subsubsection{Methodologie hegemonialen Wissens}
Mit Grosfoguel habe ich die geschichtlichen Möglichkeitsbedingungen euro- und androzentrischen Wissens nachvollzogen. Denn die eben beschriebenen geopolitischen, ökonomischen, historischen und kulturellen Prozesse haben nicht nur andere Wissenssysteme zerstört, sondern durch diese Zerstörung auch eine spezifische Erfahrung ermöglicht, die zur Grundlage für Wissensproduktion wurde. Rodriguez formuliert dies folgendermaßen:
\begin{myenv}
  \textit{
  \glqq
[....] neben der territorialen Annektierung, der Ausbeutung von Ressourcen und des Genozids an der indigenen Bevölkerung ging Kolonialismus mit einer >Neu-Schreibung< der Kolonien einher, die auf der Grundlage des Erfahrungshintergrundes und der Wissenstradition der Kolonisatoren stattfand
\grqq } \footnotemark \footnotetext{ Rodriguez, \glqq Repräsentation, Subalternität und postkoloniale Kritik, \grqq 21.}
\end{myenv}

Wie aus dieser Erfahrung der Kolonisatoren eine Wissenschaft mitsamt einer eine
Methodologie gewachsen ist, möchte ich im folgenden mit Enrique Dussel
erläutern. Seine Ansicht, nach der erst die koloniale Expansion und Herrschaft
das europäischen Selbstverständnis mit all seiner Allmacht- und
Allwissensfantasie ermöglichte, wird auch von weiteren Theoretiker\_innen
geteilt. Todorov und Spivak sowie weitaus früher die Négritude Bewegung haben
argumentiert, dass das europäische (Wissenschafts)verständnis auf Kolonisierung
aufbaut: Erst durch die \glqq Schaffung […] der >Neuen Welt< als
Erkenntnisobjekt wurde die >Kolonialmacht< im Namen des autonomen Subjekts als
regierendes und wissendes Subjekt geschaffen.\grqq Die koloniale Expansion
hat also, so lässt sich mit Rodriguez vermuten, ein kollektives Subjekt
geschaffen das sich durch die Erfahrung der Kolonisierung konstituiert und
dabei all jene mit einschließt, die sich der Kolonialmacht von nah und fern
zugehörig fühlen.  Worin der Zusammenhang zwischen kolonialer Expansion und dem
europäischen Wissenschafts- und Subjektverständnis dabei genau besteht, möchte
ich nun mit Dussel aufzeigen. Hierzu beziehe ich mich auch auf die Arbeiten von
Alcoff, die sich intensiv mir Dussels Positionen auseinandergesetzt hat. 

Ein Subjekt, so die These von Dussel, das sich selbst als Zentrum der Welt
versteht, kann nur vor dem Hintergrund einer Eroberungsgeschichte dieser Welt
entstanden sein. Nur wer sich als Eroberer dieser Welt versteht, kann den
Anspruch erheben, durch seinen Standort und seine Perspektive dermaßen über sie
verfügen zu können. Erst die Erfahrung der Eroberung der Welt ermöglicht also
ein Denken, das sich im Zentrum dieser Welt verortet. Die von Grosfoguel
beschriebenen Epistemizide haben in Westeuropa ein kollektives Subjekt
hervorgebracht, das von Dussel als \glqq masterful ego \grqq \footnotemark
\footnotetext{Alcoff, \glqq Enrique Dussels Transmodernism,\grqq in
  \textit{Transmodernity, Journal of Peripheral Cultural Production of the
Luso-Hispanic World}, Vol 1, Iss. 3, (2012):62.} bezeichnet wird. Dieser
\glqq masterful ego \grqq zeichnet sich durch eine Verweigerung gegenüber einer
kritischen Befragung der Vorannahmen und Kontextbedingungen, die sein Denken
bestimmen, aus. Dass über die konkreten Vernichtung von Bevölkerungsgruppen und
ihren Wissensbeständen hinaus, auch die Erinnerung an diese Gewaltausübung
verweigert wird, ermöglicht die Etablierung einer solipsistischen und
dualistischen Wissenschaft, in der 'die Anderen' im 'Eigenen' vereinnahmt
werden:




%\subsection{Widerständiges Wissen}
\epigraph{\textit{ 
Jede (Identitäts-) Politik wird von einem Begehren, von einer besetzen
Orientierung, dem Schatten- und Wunschbild eines Anderen, der destillierten
Erfahrung einer aufgeschobenen Erfahrung getragen, die Gegenwarten erträglich
macht und transformative Handlungen weg von der Gegenwart mobilisiert, und
zugleich muss die Politik sich selbst als verhaltenes und gedrosseltes Begehren
artikulieren.
  }}{Paul Mecheril \footnotemark} \footnotetext{Mecheril, \textit{Politik
der Unreinheit. Ein Essay über die Hybridität} (Wien: Passagen-Verlag, 2003), 14.} 

Die Perspektive, mit der Grosfoguel, Dussel und Mignolo euro- und
androzentrische Wissensproduktionen untersuchen, beschreiben sie selbst als
dekolonial. Ein zentraler Aspekt dekolonialer Intervention ist dabei die
Überlegung \glqq dass auch die Erkenntnis ein Instrument der Kolonialisierung war
und dass die Dekolonisierung daher Wissen und Sein, dh. Subjektivität
impliziert \grqq. \footnotemark \footnotetext{Mignolo, \textit{Epistemischer
Ungehorsam}, 47.}

Indem Erkenntnis hier als subjektivierender Prozess verstanden wird, der auch
die kolonisierten Subjekte betrifft, verdeutlicht Mignolo, dass die Erfahrung
der Kolonisierung sich nicht nur in die kolonisierenden, sondern auch in die
kolonisierten Subjekte einschreibt und ihre Subjektverständnisse bestimmt.
Erkenntnis wird damit in der dekolonialen Perspektive auch als Ausdruck einer
Unterwerfung unter bestehenden Wissensregimen darstellt und ergo als Mittel
der Fügsamkeit in kolonialen Verhältnisse entlarvt. 
\\

Die eingeführte Kontextualisierung, die Wissensproduktionen aus einer
dekolonialen Perspektive situiert, macht an dieser Stelle deutlich, dass der
auf Emanzipation bzw. Befreiung zielende Bildungs- und Kritikbegriff, der die
Idee von Universität bestimmt, sein Versprechen unter Umständen nur schwer
einlösen kann. Denn für diejenigen, die sich in ihren Erkenntnissen nur auf
ein System berufen dürfen, das sie unterdrückt, bleibt jede Hoffnung auf eine
Befreiung von eben jenem System vergebens. \\
Das oben stehende Zitat von Mecheril
wirft jedoch noch einen anderen Blick auf Möglichkeiten, sich zu den
Verhältnissen zu verorten. Identitätspolitiken, greifen Sprache als
Wissenssysteme gleichzeitig auf und an. Die Erkenntnis, das Erfahrungen niemals
authentisch sondern immer schon destilliert, also bereits durch ein System
gewandert sind, das die Erfahrungen spaltet um aus ihr das gewünschte zu
extrahieren, bedeutet nicht, dass sie nicht für Überlebensstrategien genutzt
werden können. Es bedeutet dass die Prozesse der Destillation in den Blick
genommem werden müssen um ihre Logik zu hinterfragen.
\\

Der Postkoloniale Feminismus knüpft hieran an. Der Hinweis, Erkenntnis als
Ergebnis kolonialer Einschreibung in Wissen und Sein ernst zu nehmen, führt
hier nicht dazu, sich von Erkenntnispolitik abzuwenden. Wie bereits mit Castro
Varela und Dhawan erläutert, steckt im Postkolonialer Feminismus trotz dieser
komplexen Ausgangslage der Anspruch nach Wegen zu suchen wie die
Wissensproduktion auf andere, umfassendere und damit weniger auschließende
Subjekt- und Erkenntnistheorien aufgebaut sein kann. \\
Mein Interesse liegt nun
darin, jene Komplexität der erkenntnistheoretischen Auseinandersetzung um die
Bedeutung von Erfahrungen für die Theoriegenerierung Schritt für Schritt zu
entfalten und dabei postkolonial\_feministische Perspektiven und Widerstreits in
den Blick nehmen. Im nächsten Schritt folgende Perspektiven habe ich jedoch
nicht aus dem emanzipatorisch funkelnden Teich des Postkolonialen Feminismus
herausgefischt. Wie so oft verhält es sich mit Etiketten und Dingen nicht ganz
so widerspruchsfrei.
\\

Das Etikett Postkolonialer Feminismus hat in erster Linie \textit{mich} angesprochen, da
es gegenwärtige Verhältnisse postkolonial situiert und theoretische Positionen,
die sich mit feministischer Kritik auseinandersetzen in diesen Verhältnissen
verortet. Die Texte auf die ich mich im folgenden beziehe habe ich jedoch nach
einem anderen Kriterium ausgewählt, da es den glitzernden Teich in der Form
bekanntlich nicht gibt - er wird viel eher durch Arbeiten wie diese immer
wieder neu hergestellt.\\
 Das Kriterium, das für mich in der Auswahl relevant
war, ist die Thematisierung von Erfahrung als Modi und Gegenstand von
feministischer Kritik. Eine postkoloniale Situierung findet dabei nur
sporadisch statt. Die Verzerrungen und Ausblendungen, die dies mit sich bringt
aufzuspüren und zu fragen, inwiefern jene Denkansätze für eine dekoloniale
Strategie fruchtbar sein können, sind darum Fragen die ich im Folgenden stellen
werde.

\subsubsection{Das Epistemische Privileg, Differenz und Erfahrung}

Ausgehend von der marxistischen These, nach der die materiellen Umstände in
denen ein Mensch sich befindet sein Bewusstsein bestimmen, entwickelt Nancy
Hartsock eine feministische Standpunktepistemologie. Während bei Marx, wie eben
beschrieben, das sogenannte Arbeiterbewusstsein durch die Erfahrungen des
weißen Arbeiters im Produktionsprozess entsteht, muss, so Hartsock, für eine
Analyse des Patriarchats der Kontext der Produktion um Reproduktion erweitert
werden. So würden all jene Tätigkeiten miteinbezogen werden, die außerhalb der
Lohnarbeit für letztere die Basis schaffen und in der Regel von Frauen\_*
ausgeübt werden. Die spezifische Position, die Frauen\_* innerhalb der
Reproduktionsarbeit einnehmen, gehe mit einer ebenso spezifischen Erfahrung der
patriarchalen Strukturen einher. Erfahrungen, die Frauen\_* dazu befähigen, eben
jene Strukturen zu erkennen und zu analysieren, welche sie unterdrücken. Erst
die tatsächliche Erfahrung reproduktiver Arbeit und das Bewusstsein, das
Hartsock zufolge hierbei entsteht, ermöglichen demnach eine fundierte,
situierte Erkenntnisproduktion die patriarchale Strukturen erkennen und in
Frage stellen kann.\footnotemark \footnotetext{ Nancy Hartsock, \glqq The
Feminist Standpoint: Developing the Ground for a Specifically Feminist
Historical Materialism \grqq , zitiert in \textit{Waldraut Ernst, Diskurspiratinnen.
Wie feministische Erkenntnisprozesse die Wirklichkeit verändern}. (Wien: Milena
Verlag, 1999), 67.} 

Ähnlich wie Hartsock geht auch Patricia Hill Collings von einem weiblichen
Bewusstsein aus, erweitert diese jedoch noch um das spezifische Bewusstsein
Schwarzer Frauen\_*. Während bei Hartsock die Positionierung von Frauen\_* in
rassistischen Verhältnissen unberücksichtigt bleibt, beansprucht Hill Collins
eine intersektionale Analyse. Dafür befasst sie sich mit der Entstehung und
Bedeutung des Black Feminist Thought, eben auch für epistemologische Fragen und
setzt sie sich explizit mit der Bedeutung der Erfahrung Schwarzer Frauen\_* für
die Wissensproduktion auseinander. Sie entwickelt hierbei das Konzept von
Weisheit das sie unmittelbar an spezifische Erfahrungen von multipler
Unterdrückung knüpft.\footnotemark \footcitetext{hillcollins} 

Für hooks stellte die Anerkennung von Erfahrung, insbesondere in ihren
universitären Anfängen, eine wichtigen Ansatz dar, um die Differenz, die sie
als Schwarze Frau\_* z.B. in weiß dominierten Auseinandersetzungen mit
Feminismus erfuhr, ernst zu nehmen. Gerade weil es zu diesem Zeitpunkt anders
als heute keinen breiten Fundus an wissenschaftlichen Beiträgen über
rassistische Ausschlüsse in der weiß dominierten feministischen Theoriebildung
gab, also weil ihre Erfahrung noch nicht analytisch und wissenschaftlich
Ausdruck fanden, war hooks darauf angewiesen, ihre Erfahrungen von
Nicht-Zugehörigkeit nicht abzutun sondern als Wissensquelle ernst zu
nehmen.\footnotemark \footnotetext{hooks, \textit{Teaching to Transgress}, 90.}
Der Einzug von Critical Race Studies oder Gender Studies in die Universitäten,
hat Räume für diese von hooks beschriebenen Erfahrungen geschaffen und damit
die Frage, wie und von wem Wissen produziert wird, neu zur Disposition
gestellt.\footnotemark \footnotetext{Mohanty, \glqq On Race and Voice,\grqq
185.} 
\\
Mohantys Forderung nach universitären Räumen, (vgl. Kapitel: Universität)
in denen marginalisierte Erfahrungen artikuliert werden können, liest sich
gewissermaßen als praktische Konsequenz aus oben formulierten Standpunkten.
Diese Räume zeichnen sich durch den Anspruch einer analytischen Offenheit aus,
die die Gleichzeitigkeit verschiedener Machtverhältnisse in ihrer komplexen
Verwobenheit artikulierbar machen. Die Kategorie Frau\_* wird damit zu einem
Ort, der nicht nur Geschlechterverhältnisse, sondern gleichermaßen Class und
Race als Strukturkategorien mit einbezieht. Räume in denen dies möglich wird
sind, so Mohanty, jedoch stets umkämpft, da sie sich permanent gegen eine
Vereinnahmung durch apolitische und ahistorische Diversitätsdiskurse zur Wehr
setzen müssen. Denn Differenz, so führt sie aus, ist nur dann produktiv wenn
die hegemonialen Vorstellungen darüber, wer was sein kann, irritiert werden,
anstatt sich in komplementärer Manier identitären Kategorien
anzuschließen.\footnotemark \footnotetext{Ebd., 184.} 

Anders als Hartsock und Collins geht Mohanty jedoch nicht davon aus, dass
Frauen\_* allein durch ihre Position innerhalb des Geschlechterverhältnisses
bereits eine kritische und entsprechend erkenntnistheoretisch priviligierte
Sichtweise einnehmen. Feminismus ist ihrer Ansicht nach mehr als ein
natürlicher Effekt der mit dem Frau\_* sein einhergeht. Hier beruft sie sich auf
King nach der Lesbe zu sein nicht davor schützt heteronormative Strukturen und
Wissensbestände aufrecht zu erhalten. \\
Eine kritische politische Haltung ergibt
sich ihnen nach nicht auf Grund einer gesellschaftlich marginalisierten
Position oder Praxis allein, sondern nur durch eine bewusste Reflexion
innerhalb derer die Mechanismen, die zur eigenen Marginalisierung führen
erkannt und bewertet werden.\footnotemark \footnotetext{Mohanty, \glqq Feminist
Encounters. Locating the Politics of Experience, \grqq 177.}

Auch Bat-Ami Bar On kritisiert die Annahme eines epistemischen Privilegs, das
marginalisierten Gruppen per se zukommt. Dies würde die prekären Verhältnisse,
die an den Rändern der Gesellschaft herrschen, romantisieren und dabei
verkennen, dass der Rand sich nicht außerhalb der dominanten Diskurse befindet.
Unterdrückt zu sein bedeute nicht, von den Denkweisen befreit zu sein, die eben
jene Unterdrückung ermöglichen. \footnotemark \footnotetext{Bat-Ami Bar On,
 \textit{Marginality and Epistemic Privilege}. Zitiert in \textit{Ernst, Waldraut:
Diskurspiratinnen. Wie feministische Erkenntnisprozesse die Wirklichkeit
verändern}. (Wien: Milena Verlag, 1999), 72.}

Von der Notwendigkeit einer universellen, weiblichen Unterdrückungserfahrung
für feministische Kritik auszugehen, wird auch von Waltraud Ernst
problematisiert, weil sie diejenige Struktur, die es zu überwinden gilt, als
Bedingung voraussetzt. Das Zweigeschlechtermodell werde reproduziert, da
Frauen\_* nur qua ihrer untergeordneten Rolle den Status der Kritiker\_innen
genießen dürfen. \\
Statt die Erfahrung von Unterdrückung in den Vordergrund zu
stellen, müssten, so Ernst, viel eher Momente des Widerstandes und der
unerwarteten Ausbrüche aus als weiblich erklärten Rollenverständnissen in den
Mittelpunkt der Analyse rücken. Erkenntnis dürfe nicht an die Position gebunden
sein, die das Patriarchat den Frauen\_* zuschreibt, sondern entstehe in den
Momenten der Befreiung.\footnotemark \footnotetext{Ernst,
\textit{Diskurspiratinnen}, 73.} 
\\

Dekolonisierung verlangt hier eine
Aufmerksamkeit für die asymmetrischen Machtverhältnissen, in die Sprechpraxen
an der Universität eingebunden bzw. durch sie wirksam werden. Theorien, die auf
einer abstrakteren Ebene funktionieren als es persönliche Erfahrungen tun,
werden dadurch, das betont Mohanty, keineswegs obsolet. Sie werden vielmehr in
einen Dialog mit konkreten Alltagserfahrungen gesetzt und dadurch ergänzt oder
auch in Frage gestellt\footnotemark \footnotetext{Mohanty, \glqq On Race and
Voice \grqq, 184.}. Die Universität muss entsprechend als ein Ort verstanden
werden, an dem einander widerstreitende Positionen und daraus folgende
Interessen aufeinander treffen und Mohanty plädiert dafür, dass diese Kämpfe
gefochten werden und nicht in einem leeren Pluralismus versiegen.\\

Erfahrung in Wissensproduktion wurde in bisherigen Beiträgen stets als
kollektive Erfahrung konzipiert - ihr wird ein Ort zugewiesen, und zwar in den
meisten Fällen, ein Ort in Geschlechterverhältnissen. Dass diese
Geschlechterverhältnisse auch rassifiziert sind, wird zumeist ignoriert und
damit auch die Sensibilität für die Differenzen zwischen Frauen\_*. Die
Berücksichtigung differenter Erfahrungen wird nur aus Schwarzen und of Color
Perspektiven thematisiert, wie Collins, Mohanty oder Hooks zeigen.

Umstritten ist auch welche Erfahrung eigentlich emanzipatorisches Potential
hat. Hierbei spielen Unterdrückungserfahrungen ebenso eine Rolle wie Erfahrung
von Widerstand und Ausbruch aus vorgeschriebenen Wegen. Insgesamt wird jedoch
deutlich, dass Erfahrung einen enorm wichtigen Bestandteil in dem Versuch
bildet, sowohl die Leerstelle, die von hegemonialer Philosophie ausgeht zu
benennen, also auch, um aus dieser Leerstelle heraus eine
erkenntnistheoretische Position zu beziehen und zu legitimieren.
\\

Im letzten Kapitel habe ich mit Dussel und Grosfoguel aufgezeigt, dass die
Erfahrung von Herrschaftsausübung den europäischen weißen männlichen
Philosophen eine Selbstverständnis ermöglichte das ihre Methodologie bestimmt.
Es wurde deutlich dass in der emanzipatorischen Epistemologie, diese Erfahrung
von Ausschluss als Ausgangspunkt gewählt wird, um die eigene Stimme bzw. das
Interesse zu markieren. \\
Der Unterschied ist nun, dass sich in der
postkolonial\_feministischen Epistemologie auf diesen marginalen Standpunkt
explizit bezogen wird und entsprechend sich auch ihre Legitimation hieraus
ableitet. In der hegemonialen Epistemologie hingegen wird der Anspruch zwar
auch aus der Erfahrung heraus entwickelt, wie mit Dussel und Grosfoguel
aufgezeigt, diese Erfahrung wird jedoch nicht reflektiert und entscheidend:
nicht benannt.
\\
           
Bisher habe ich mich mit Perspektiven beschäftigt, in denen Erfahrung als
Standpunkt theoretisiert und problematisiert wird. Eine Beobachtung, die
Canning an anderer Stelle\footnotemark \footnotetext{Kathleen Canning,
 \glqq Problematische Dichotomien. Erfahrung zwischen Narrativität und
 Materialität, \grqq
in Erfahrung: Alles nur Diskurs? Zur Verwendung des Erfahrungsbegriffes in der
Geschlechtergeschichte, herausgegeben von Marguérite Bos, Bettina Vincenz und
Tanja Wirz, (Zürich: Chronos Verlag, 2004), 38.} macht, lässt sich dabei auch
auf die bisher dargestellten Positionen übertragen.\\
 Die Thematisierung von
Erfahrung, in Erkenntnisproduktion, so Canning, beschränkt sich oftmals auf
die Erfahrungen des forschenden Subjekts, das auf diese Weise seinen Zugang zum
Forschungsthema transparent macht. Was hier jedoch unberücksichtigt bleibt, und
auch in meiner bisherigen Analyse kaum Ausdruck fand, ist die Erfahrung des
historischen Subjekts. Canning führt damit die Erfahrung der\_des Anderen ein.
Ob er\_sie vergangen oder anwesend ist spielt für mich hierbei eine zweitrangige
Rolle. Für mich ist die Frage interessant, wie sich ein reflexiver Umgang in
dem Wissen gestalten kann, dass die Erkenntnisse der beforschten Subjekte aus
einem spezifischen Standpunkt bzw. einer konkreten Erfahrung heraus entstehen
und nun von einem anderen Standpunkt heraus analysiert bzw. nutzbar gemacht
werden.          
\\

Joan Scott hat sich sehr explizit mit Erfahrung als Gegenstand von
Wissensgenerierung auseinandergesetzt. Als Historikerin merkt sie an, dass
Sehen, bzw. die Sichtbarkeit von etwas vielfach als einziger oder zumindest
vielversprechendster Zugang zu Erkenntnis gehandelt wird. Im Sehen verberge
sich damit der eigentliche Ursprung des Wissens, wohingegen die Aufgabe des
Schreibens lediglich als Ausführung, oder Reproduktion dieser Erkenntnis
verstanden werde. Diese Hierarchisierung kritisiert Scott in ihrem Essay
\glqq Experience \grqq \footnotemark \footnotetext{Joan Scott: \glqq Experience
 \grqq in
\textit{Feminists Theorize the Political}, herausgegeben von Judith Butler und Joan Scott. ( New York: Routledge, 1992).}, auf das ich im folgenden näher eingehen werde.

Die Praxis des Sehens beschreibt Scott als Form der unmittelbaren Erfahrung.
Insbesondere in den Geschichtswissenschaften habe die Sichtbarmachung von
Erfahrungen (anderer) zu einer willkommenen \glqq Pluralisierung von
Geschichten \grqq \footnotemark \footnotetext{Scott, \glqq Experience \grqq,
24.} geführt. Forschungen, die auf Erfahrung (anderer) basieren, würden somit
nicht nur zu einer Diversifizierung dominanter Narrative führen, sondern diese
auch um bis dato unberücksichtigte Subjektpositionen erweitern. 
\footnotemark \footnotetext{Ebd.} So werde mit
der Dezentrierung des hegemonialen Narratives durch die Inanspruchnahme von
Erfahrungswissen, zugleich der weiße Mann\_* als einzig möglicher Ursprung von
Erkenntnis in Frage gestellt. \\
Nun gilt es aber angesichts des emanzipatorischen Potentials, das
erfahrungsbasierte Forschungen folglich beanspruchen, vorsichtig zu sein. Auf
die Geschichtswissenschaft bezogen merkt Scott an, dass der Hype um Erfahrung
einer notwendigen und zwar fundamentalen Kritik am geschichtswissenschaftlichen
Methodenverständnis im Wege steht: \\
Die Hinwendung zu den Erfahrungen von bis
dato als unwesentlich und unsichtbar gehandelten Subjekten diene der
Geschichtswissenschaft nämlich lediglich als willkommene Ergänzung bereits
existierender Erkenntnisinstrumente. Die Legitimität dieser Erkenntnisquelle
werde dabei über die \glqq Autorität der Erfahrung \grqq \footnotemark
\footnotetext{Ebd.} behauptet. Die Erfahrung als \glqq authentisch, wahrer
Bericht dessen, was ein Mensch erlebt hat, dient entsprechend als Ursprung für
Erklärungen \grqq. \footnotemark \footnotetext{Ebd.} 


Scott kritisiert diese Handhabung von Erfahrungswissen auf zwei Ebenen.\\
 Zum
einen hinterfragt sie den erkenntnistheoretischen Wert, der in jenen
Erfahrungsberichten liegt. Zum anderen kritisiert sie die Fokussierung auf die
sogenannte Identität der Betroffenen bzw. Ermächtigten und stellt dem ihr
Interesse an der Herstellung von Identität entgegen.\\
Scotts Problematisierung
von Erfahrungswissen zeigt sich in ihrem Verständnis von Erfahrung als
Narrativ.\\
 Erfahrung, und dies wird in späteren Ausführungen deutlicher, ist für
Scott in erster Linie das, was als Erfahrungswissen Eingang in den Diskurs
erhält, also eine Erzählung. Erfahrungs-Erzählungen dürfen, so argumentiert sie
nun, nicht als Beweis für oder gegen eine theoretische Position gedeutet
werden. Stattdessen müsse die Logik, nach der Erkenntnis erst durch eine vorher
festgelegte Beweisführung legitimiert werden muss, in Frage gestellt werden.
Die komplexe Verwobenheit und Abhängigkeit, die in dem Verhältnis von Narrativ
und Beweis liegt, dürfe nicht umgangen werden sondern müsse eigentlicher
Gegenstand der Kritik werden. An diese grundsätzliche Kritik am
Legitimationszwang von Erkenntnis durch Beweise, schließt sie die die
Problematik der identitären Aufladung von Erfahrung an.


Die Identität jener, die ihre Erfahrungen teilen, werden, so kritisiert Scott,
als gegeben vorausgesetzt und ihre Erzählungen als Intervention in das
dominante Narrativ gewertet. Dies führe zu einer Platzierung der Subjekte
außerhalb von Diskursen und einer damit einhergehenden Individualisierung ihrer
Erfahrung. Scott bezieht sich hier vermutlich auf die Versuche, jenen, die z.B.
von Missbeständen betroffen sind, und entsprechend als Betroffene über jene
Missstände berichten, Gehör zu verschaffen. \\
Auch wenn von einer allgemeinen,
also keiner individuellen Erfahrung ausgegangen wird, und Erfahrung damit als
kollektives identitätsstiftendes Moment verstanden wird, bleibt der Fokus, so
verstehe ich Scott, individualisiert, weil eben jene Prozesse des 'kollektiv
werdens' nicht untersucht werden.\\
Problematisch für Scott ist also nicht, dass
Betroffen aus ihrer Perspektive heraus ihre Erfahrungen teilen, sondern
problematisch ist, dass sie als Betroffene wahrgenommen und ihre Erfahrungen
als Betroffenheitserfahrungen verstanden werden, ohne den diskursiven Rahmen,
der diese Artikulation bestimmt näher zu betrachten und unter Umständen zu
kritisieren. Wenn nun aber davon ausgegangen wird, das Subjekte nicht zufällig
auf eine bestimmte oder vermeintliche Erfahrung reduziert werden, bzw. dass
jene Zuschreibung bereits Teil der Erfahrungen und ihre
Artikulationsmöglichkeit ist, werden andere Fragen interessant: 

\begin{myenv}
 \textit{\glqq Questions about the constructed nature of experience, about how
 subjects are constituted as different in the first place, about how one's
vision is structured, about language (or discourse) and history – [...]\grqq}
\footnotemark \footnotetext{Ebd.,25.}
\end{myenv}

Scott interessiert sich, wie hier deutlich wird also weniger für die Erfahrung
der vermeintlich Anderen, sondern vielmehr dafür wie diese Anderen zu Anderen
gemacht werden und wie dies mit ihrer Sicht auf die Welt verbunden ist.
Erfahrung soll entsprechend nicht als Beweis für einen Unterschied, sondern zum
Medium werden, um die Herstellung diesen Unterschiedes zu untersuchen:
\begin{myenv}
 \textit{\glqq 
  The evidence of experience then becomes evidence for the fact of difference, rather than a way of exploring how difference is established, how it operates, how and in what way it constitutes subjects who see and act in the world. 
  \grqq}
\footnotemark \footnotetext{Ebd.}
\end{myenv}

Letztlich geht es Scott also darum, die Herstellung von Differenz zu
untersuchen. Ein Unterfangen, das Differenz nun schon als natürlich gegeben
voraussetzt, verliert das Potential, das Verhältnis von Subjekt und Diskurs in
den Blick zu nehmen um die Momente der wechselseitigen Hervorbringung
untersuchen zu können. \\
Erfahrung ist für Scott somit nur eine von vielen
Erzählungen, die niemals als Beweis, sondern lediglich als Gegenstand für eine
kritische Hinterfragung der darin zu Grunde liegenden Annahmen etc. genutzt
werden kann. Die Pluralisierung von Erzählungen und damit einhergehende
Sichtbarmachungen marginalisierter Realitäten gibt noch keinen Aufschluss über
ihre Verortung in den kontingenten Strukturen, wie beispielsweise der
Zweigeschlechtlichkeit: \glqq We learn about the existence of difference, but we do
not learn anything about how this difference is being constituted and in
relation to what.\grqq \footnotemark \footnotetext{Ebd.}
  
Stattdessen fordert Scott dazu auf, einen historisierenden Blick auf jene Diskurse zu werfen, die Subjekte positionieren und somit Erfahrungen erst erzeugen:

\begin{myenv}
 \textit{\glqq 
It is not individuals who have experience, but subjects who are constituted through experience. Experience in this definition then becomes not the origin of our explanation […] but rather that which we seek to explain, that about which knowledge is produced.
  \grqq}
\footnotemark \footnotetext{Ebd.}
\end{myenv}

Diese Vorstellungen, nach der Erfahrung dem Diskurs nicht vorausgeht, sondern
durch ihn ermöglicht wird, stellt Wissenschaftler\_innen damit vor die Aufgabe,
jene Erfahrungen zu historisieren und damit auch die Identitäten, die so
erzeugt werden als geschichtlich gewordene Subjektpositionen nachzuvollziehen. 

\begin{myenv}
 \textit{\glqq 
 An jet it is precicely the questions precluded – questions about discourse, difference, and subjectivity, as well as about what counts as experience and who gets to make that determination – that would enable ust o historicize experience, to reflect critically on the history we write about it, rather than premise out history upon it. 
 \grqq}
\footnotemark \footnotetext{Ebd., 33.}
\end{myenv}

Scott fordert also dazu auf, einen Schritt zurückzugehen und Erfahrung nicht
als leichte Antwort auf auf schwierige Fragen auszunutzen.
\\
 Um Machtverhältnisse
zu untersuchen, müssen, so verstehe ich Scott, jene Verhältnisse in denen
Erfahrungen entstehen und als Erfahrungen markiert und gehört werden, als
machtvolle Verhältnisse untersucht werden. Nicht die Erfahrung an sich, sondern
der Umgang der Wissenschaft mit Erfahrungen und damit, wer wessen Erfahrung
Gültigkeit verleiht, muss damit im Zentrum stehen, um herrschende Verhältnisse
unterwandern zu können.
\\

Linda Martin Alcoff schließt sich Scotts Erkenntnisinteresse an, und versteht
wie Scott unter Feminismus im weitesten Sinne eine Ideologiekritik.\\
Ideologien
können im Anschluss an Althusser als Praxen der Legitimierung von Herrschaft
bezeichnet werden, die es vermögen, selbst denjenigen, die unter dieser
Herrschaft leiden, die erlittene Ungerechtigkeit als unveränderlich, u.U. sogar
gerecht zu inszenieren (vgl. Herrschaft, voriges Kapitel). \\
Als Phänomenologin\_* fragt sie
nun in Anbetracht der Tatsache, dass sich phänomenologische Analysen auf
leibliche Erfahrungen stützen, wie die Mechanismen dieser ideologischen Praxis
anhand von Erfahrungen untersucht werden können, die sich folglich nicht
außerhalb sondern innerhalb des ideologischen Apparates befinden und seinen
Praktiken entsprechend nicht nur ausgesetzt sind, sondern in ihnen mitwirken.\\

Sie greift damit Scotts Bedenken an der Möglichkeit einer auf Erfahrung
basierenden Erkenntnis auf, kommt jedoch zu einem anderen Schluss:\\
Anders als
Scott weist Alcoff die methodologische Bezugnahme von Erfahrung nicht als
unbrauchbar zurück. Sie plädiert stattdessen für einen phänomenologischen
Feminismus, der sich zwar kritisch auf phänomenologische Grundkategorien
bezieht ihren Erfahrungsbegriff jedoch nicht zurückweist sondern für
feministische Theoriebildung fruchtbar macht.

Ihre Argumentation folgt dabei
einer geneologischen Herangehensweise, das heiß, sie leitet die Notwendigkeit,
sich mit dem phänomenologischen Konzept der Erfahrung auseinander zusetzen,
anhand einer historisierenden Perspektive auf die ursprünglichen Anliegen des
Feminismus her. \\
Dazu bezieht sie sich auf das in der europäischen Philosophie
grundlegende Problem, dass Vernunft und Erkenntnis als vergeschlechtlichte
Kategorien konzipiert werden. Dies würde Frauen\_* von den Möglichkeiten an
Wissensproduktion bzw. Erkenntnis teilzuhaben ausschließen: \\
Vernunft, so
schreibt sie, wird durch ihren Gegensatz zur weiblichen Leiblichkeit definiert
und durch die Vorherrschaft über sie manifestiert. \footnotemark
\footnotetext{Alcoff, \glqq Phänomenologie, Poststrukturalismus und
 feministische Theorie. Zum Begriff der Erfahrung, \grqq in Phänomenologie und
Geschlechterdifferenz, herausgegeben von Silvia Stoller und Helmuth Vetter,
(Wien: WUV Universitätsverlag, 1997), 229.} \\
Irrationalität, Intuition und
Emotionalität werden hier als weibliche und zugleich der männlichen Vernunft
gegenüber abgewertete Begriffe genutzt, um den wie auch schon von Grosfoguel
beschriebenen Geist- Körper Dualismus aufrecht zu erhalten, bzw. das darin
enthaltenen Machtverhältnis zu legitimieren: Der Geist muss über den Körper
herrschen, \glqq wenn der Mensch Erkenntnis erlangen will.\grqq \footnotemark
\footnotetext{Alcoff, \glqq Zum Begriff der Erfahrung, \grqq Ebd. }

Alcoff verdeutlicht die Tragweite dieses Denkens anhand eines Zitates von Rousseau:
\begin{myenv}
 \textit{
 \glqq Der Mann ist nur hie und da Mann, die Frau aber ist immer eine Frau
 […]. Alles erinnert sie an ihr Geschlecht. So empfiehlt er: Ziehe die Meinung
 einer Frau [nur] in Fragen des Körpers und allem, was die Sinne betrifft, zu
 Rate. Bei Fragen der Moral und allem, was den Verstand betrifft, so höre auf
 die Meinung der Männer \grqq \footnotemark \footnotetext{Rousseau in Alcoff,
 ebd., 229.}} und sie resümiert: \textit{ \glqq Die männliche Vernunft wurde
 durch diese wertenden Hierarchie von Körper und Geist paradoxerweise sowohl
 gestützt als auch verdeckt. Indem der empfindende Leib vom erkennenden Geist
 getrennt wurde und nur als roher Empfänger von Wahrnehmungsbilder diente,
 schienen körperliche Unterschiede keine Rolle in der Ausformung von Vernunft
 spielen zu können. Der Körper und die ihm zugeschriebenen Gelüste,
 Leidenschaften etc. mussten alle bezwungen werden, um dem Geiste die
Möglichkeit nach Erkenntnis zu gewähren.\grqq \footnotemark \footnotetext{
Alcoff, Ebd., 230.} } \end{myenv}


Die Vernunft wurde somit zum körperlosen, und damit universell gültigen Maß
erkoren, und all diejenigen, denen diese Vernunft abgesprochen wurde, z.B.
Frauen\_*, wie hier mit Rousseau verdeutlicht wird, vom Wissenschaftsbetrieb
ausgeschlossen. \\
Alcoffs Analyse liest sich somit analog zu Grosfoguels Analyse
des dualistischen Wissenschaftsverständnisses mit dem zentralen Unterschied,
dass sie zumindest hier, nicht auf den Ausschluss rassifizierter Subjekte Bezug
nimmt. Auch das Rousseau mit großer Wahrscheinlichkeit von \textit{w}eißen Männern und
Frauen ausgeht, wird von Alcoff nicht berücksichtigt. 

Auch Louise Lopman kommt in
ihren Auseinandersetzungen mit dem Dualismus in der Soziologie zu einem
ähnlich, weißen Ergebnis: Sie zeigt auf, dass es männliche Perspektiven auf
Frauen\_* sind, die die Theorien von Sokrates bis hin zu Rousseau etc. prägen.
Frauen\_* würden hier stets dem männlichen Blick und entsprechendem Maßstab
untergeordnet und dienten lediglich als Objekte der Analyse. Es fehlen, so
resümiert sie, Selbstdefinitionen von und durch Frauen\_*. \footnotemark
\footcitetext{lopman} \\
Dass sowohl die
Männer\_* als auch Frauen\_* in Bezug auf ihre Position in rassistischen
Verhältnissen unmarkiert bleiben, könnte nun, angesichts dessen, dass niemals
alle Differenzverhältnisse mitgedacht werden (können), verharmlost werden. Es
ist immer einfach, anzumerken, woran eine\_r nicht gedacht hat.

Mir geht es hier jedoch nicht nur um das Aufzeigen von Lücken. Das gravierenende an
der Dethematisierung von Rassismus liegt meiner Ansicht nach darin, dass hier
die Entstehung des dualistischen Wissenssystems im kolonialen Verhältnis
unberücksichtigt bleibt. Die Auseinandersetzungen ignorieren damit nicht
irgendein Herrschaftsverhältnis, sondern dasjenige, auf das sich der
Kolonialismus in erste Linie stützte: Rassismus.

Eine feministische Kritik mit
dem Ziel, Frauen\_* eine gleichberechtigte Rolle zu ermöglichen, kann den
Autor\_innen nach jedoch nur gelingen, wenn sie jenen Dualismus aufbricht.
Alcoff beruft sich hierbei auf Genevieve Loyd, die davor warnt, die Lösung
einzig und allein im Anspruch auf Vernunft zu behaupten ohne die gefährliche
Gegenüberstellung von Körper und Geist zu überwinden. Denn dies führe zu nichts
anderem als dem eigenen Ausschluss: \glqq Wenn die Frau es nun wagt, in die Sphäre
der Vernunft einzudringen so löscht sie sich zugleich selbst aus, da die
Vernunft auf dem Ausschluss des Weiblichen basiert \grqq \footnotemark
\footnotetext{Genieve Loyd zitiert in Alcoff, \glqq Zum Begriff der
Erfahrung,\grqq 230.} Für Alcoff besteht also
die Aneignung einer Kategorie, wie der Vernunft, nicht in einer Ermächtigung
über sondern in einer Affirmation eben jenes Dualismus, der über die Bestimmung
dessen, was als Quelle von Wissen Geltung beanspruchen darf, entscheidet. Statt
einer Affirmation steht der Feminismus jedoch vor der Aufgabe, einen
Wahrheitsbegriff zu entwickeln, der nicht vom Körper losgelöst ist, sondern
\glqq
die intensive libidinöse Kraft, auf welche sie sich stützt, respektiert und
deren Spuren trägt \grqq \footnotemark \footnotetext{Ros Braidotti in Alcoff,
Ebd., 231.}

So kommt Alcoff zu dem Ergebnis, dass \glqq wenn Frauen\_* eine
epistemische Glaubwürdigkeit und Autorität besitzen sollen, […] wir die Rolle
der leiblichen Erfahrung in der Entwicklung von Wissen neu erwägen \grqq
\footnotemark \footnotetext{Alcoff, Ebd.} müssen.
\\

Doch was hat es mit dem Leib, bzw. der leiblichen Erfahrung eigentlich auf
sich? Bzw. wie kann leibliche Erfahrung zur Entwicklung von Wissen genutzt
werden? Und welche Körper werden hier imaginiert?

\subsubsection{Feminismus/ Körper/ Soma/ Erfahrung}

In ihrem Buch „Claiming Reality. Phenomenology and Women's Experience“,
diskutiert Lopman die Vorzüge der phänomenologischen Soziologie für
feministische Forschung und bezieht sich im Anschluss an Alcoff auf die in
soziologischen Forschungen gängige Ignoranz gegenüber den sogenannten
\glqq Mind/Body Experiences \grqq also Geist/Körpererfahrungen. Insbesondere jene
Frauen\_*spezifischen Erfahrungen, wie \glqq Schwangerschaft, Geburt, Menstruation
und Menopause [...]\grqq \footnotemark \footnotetext{Lopman, \textit{Claiming
Reality}, VIII.} blieben hier, so ihre Beobachtung, unberücksichtigt.

Für Lopman ergibt sich hieraus die Notwendigkeit, eine Methode zu entwickeln,
die gegenüber dem \glqq intuitiven und empirischen Wissen \grqq \footnotemark
\footnotetext{Ebd.} der je individuellen
Erfahrungen von Frauen\_* aufmerksam ist.\\
 So hätte die feministische Forschung
nicht allein die Aufgabe, Kritik an z.B. den sexistischen Strukturen
gesellschaftlicher Institutionen zu richten. Sie sei darüber hinaus gefordert,
\glqq den partriarchalen Blick auf unsere Erfahrung der sozialen Welt, auf
fundamentale und tiefgreifende Weise zu transformieren \grqq . Dafür, so fügt sie
hinzu, reicht ein bloßer Standpunkt nicht aus, viel eher braucht es eine
Theorie und eine Methode.\\
 Es folgt eine Übersicht über die unterschiedlichen
Ausgangslagen und Stoßrichungen feministischer Interventionen innerhalb der
klassischen Soziologie in den 70iger und 80iger Jahren. Ein wesentlicher Fokus
der Kritik stellt ihrer Ansicht dabei die geteilte Beobachtung dar, dass
männliche Erfahrungen und daraus entstehende Perspektiven auf Wirklichkeit die
Theorietradition innerhalb der Soziologie bestimmen, sodass Frauen\_* nur durch
den männlichen Blick als Objekte vorkommen. Die feministische Strategie
entgegnet dem, so Lopman, indem sie die weibliche Erfahrung als Ausgangspunkt
von Wissensproduktion stellt. Diese sollte jedoch nicht komplementär zu
bisherigen Erkenntnissen verstanden werden, sondern letztere in ihrem
Erkenntnisanspruch radikal in Frage stellen. Hierbei werden von ihr sowohl
Methodik als auch Gegenstand herkömmlicher soziologischer Theorien kritisiert,
und ein feministischen Projekt entworfen, in dem sowohl der Körper, als auch
ein spezifisch weiblicher subjektiver Zugang zu Wirklichkeit in den Vordergrund
rückt. Damit verschiebt sich die Aufmerksamkeitsrichtung von einem Sprechen und
Forschen über Frauen\_* hin zu einer Forschung aus der Perspektive von
Frauen\_*. \footnotemark \footnotetext{Ebd., vgl. 1-4.}
\\

Es fällt auf, dass Körper-Geist Dualismus zu durchbrechen, keine leichte
Aufgabe ist. Während dem Poststrukturalismus Affirmation vorgeworfen wird, weil
er wie androzentrische Theorien das körperliche negiert und als Konstrukt
ablehnt, kann der phänomenologischen Position von Lopman, angefangen bei ihren
Beispielen der genuin weiblichen Erfahrung von Schwangerschaft etc. über die
weiblichen Zugang zu Wirklichkeit eine essentialistische, cis-normative und
dualistische Vorstellung von weiblich vs. männlicher Erfahrung unterstellt
werden. \\
Beide Positionen scheitern daran, Geist und Körper zusammen zu denken
und so den Dualismus tatsächlich aufzubrechen. Körper wird hierbei weder von
den poststrukturalistischen noch von den phänomenologischen Theoretiker\_innen
als rassifiziert gedacht. Die Dethematisierung davon, wie sich Rassismus in
Körper einschreibt und dass eine historisierende Perspektive sich mit den
kolonialen Konstruktionen weißer und Schwarzer Körper auseinandersetzen müsste,
zeigt auf, dass der postkoloniale Diskurs von den \textit{w}eiß dominierten
feministischen Auseinandersetzungen um Erfahrung und Körper ignoriert bleibt. 
\\

Barabara Duden grenzt sich von jenen phänomenologischen Konzeption des Leibes
durch den von ihr genutzten Begriff des Soma ab, knüpft jedoch diskursiv an den
von Alcoff argumentierten Standpunkt an, nach der die Erfahrung des leiblichen,
bzw. des Soma einen notwendige Dimension für die feministische
Erkenntnistheorie beinhaltet. Dabei führt sie auch ein \textit{w}eiß dominiertes Körper
bzw. Soma Verständnis fort. Zugleich kann ihre Theorie des Soma auch für
postkoloniale Perspektiven fruchtbar gemacht werden, da die diskursive
Dimension von Soma zumindest Imganinationsmöglichkeiten für nicht-\textit{w}eiße Körper
eröffnet, selbst wenn das nicht explizit benannt wird.
\\

Die Auseinandersetzung mit Erfahrung erfordert, so argumentiert Duden, das
Verhältnis zwischen dem Leiblichen und dem Diskursiven zu untersuchen auch um
das asymmetrische Verhältnis in dem sowohl auf quantitativer Ebene (\glqq da immer
weniger vom Gewussten […] begriffen oder gar betastet [werden] kann […] und
somit das leibhaft erfahrene Wissen der Fülle an Informationen und Kenntnissen
weichen muss \grqq) \footnotemark \footnotetext{Barbara Duden, \glqq
Somatisches Wissen, Erfahrungswissen und 'diskursive' Gewissheiten.
Überlegungen zum Erfahrungsbegriff aus der Sicht einer Körper-Historikerin.
\grqq in
Erfahrung: Alles nur Diskurs? Zur Verwendung des Erfahrungsbegriffes in der
Geschlechtergeschichte. Herausgegeben Marguérite Bos,et al. (Zürich: Chronos
Verlag, 2004), 27.} als auch auf qualitativer Ebene (medizinisches Wissen vs.
Sinnliche Wahrnehmung des eigenen Körpers) zwischen leiblichem und diskursiven
Wissen gewertet wird, zu berücksichtigen.

Die Gegenüberstellung von leiblichem und diskursivem Wissen, ermöglicht ihr die
Selbsterfahrung des eigenen Körpers als Wissen zu theoretisieren, das außerhalb
von kategorialen Rastern existiert und damit einen eigenständigen
Wissensbestand bildet. \\
Um nicht, in Anschluss an Lopman ein essentialisierenden
Körperbegriff zu reproduzieren grenzt sie sich jedoch von dem
phänomenologischen Begriff des Leibes ab und nutzt stattdessen den Begriff
Soma, der den Körper als \glqq epochenspezifische Erfahrung \grqq \footnotemark
\footnotetext{Vgl. Duden, \glqq Somatisches Wissen \grqq. } konzipiert und somit
etwas Bezeichnet, das niemals immer schon da ist sondern sich im Werden
befindet: „Mit Soma versuche ich, etwas immer Geschichtliches zu bezeichnen und
so den Fehlschluss zu vermeiden, es ginge mir um eine natürlich gegebene oder
gar 'authentische' Befindlichkeit. Interessant ist an dieser Stelle ihr
Geschichtsverständnis, das sie Prodi entlehnt: \glqq [...] Geschichte ist die
Disziplin, die im Licht von heute die Vergangenheit die in uns ist, sucht \grqq
\footnotemark \footnotetext{Paolo Prado in Duden, ebd., 33.}

Mit diesem, verkörperten Geschichtsbegriff macht Duden nun darauf aufmerksam,
dass es \glqq leibliche Erfahrung gab, die nicht das Resultat kategorialer
Konstruktion war, und die doch in ihrer historischen Prägung untersucht werden
kann.\grqq \footnotemark \footnotetext{Duden, Ebd., 25.} Der, u.A. aus poststrukturalistische Perspektive formulierte Zweifel
\glqq an der sinnlichen Erfahrung spiegelt [für Duden] eine epistemische
Bodenlosigkeit sondergleichen [wieder]: Den Verlust der untrüglichen
Gewissheit, leibhaftig die Quelle der eigenen Aussage zu sein. \grqq
\footnotemark \footnotetext{Ebd., 26.} 
\\
Die leibhafte
Erfahrung wird damit für Duden zu unumstößlichem Wissen, ohne dass sie die
historischen Bedingungen, die diese Erfahrung erst ermöglichen, als unwichtig
zurückweist. \\
In gewisser Weise, beschreibt Duden einen Zwischenraum, zwischen
dem diskursiv bestimmten und dem Ursprünglichen, das sich der kategorialen
Macht entzieht. Es ist weder natürlich und immer schon da, noch ist es
ausschließlich durch das Soziale konstruiert und damit abhängig von dem bereits
Gewussten und Artikulierten. 
\\

Jener Zwischenraum birgt jedoch stets Gefahr vereinnahmt zu werden, selbst von
jenen die ihn im Sinne der Emanzipation aufspüren und sich gegen die
hegemoniale Ordnung wenden. Cathleen Canning \footnotemark \footnotetext{
Cathleen Canning, \glqq Feminist Discourse after the Linguistic Turn: Historicizing
Discourse and Experience, \grqq in Signs, Vol. 19 Nr. 2, (1994):370.} führt hier an, dass die
Neuschreibung von Subjektivität als Ort von Diskontinuität und Konflikt
zunächst zwar, wie eben beschrieben, einen emanzipatorischen Raum öffnen kann
in dem weibliche Subjektivitäten sich durch die Artikulation von weiblichen
Erfahrungen einen Platz erkämpften und so die Gleichsetzeung von \textit{human} with
\textit{male} korigieren. Postkoloniale Feminist\_innen fragen nun so Canning,
von wessen Befreiung hier die Rede ist. Sie stellen somit nicht nur die
Geschichte des human as male sondern auch die Geschichte der \textit{woman} as
\textit{white} in
Frage:
\begin{myenv}
 \textit{
 \glqq[...]'feminist dream of a common naming of experience' was illusory,
 totalizing, and racist'. As feminists of color rewrote histories of slavery,
 colonialism, and feminism from their oppositionall ocations,they also
 contested their own colonization in the discourses of Western feminist
 humanism. \grqq \footnotemark \footnotetext{ Cathleen Canning, \glqq Historicizing
 Discourse and Experience \grqq, Ebd.}
 }
\end{myenv}

In diesem Sinne möchte ich mit Mohanty und ihren Reflexionen aus einer
postkolonialen Perspektive das Potential der Selbsterfahrung aus der Sicht von
Frauen of Color und Schwarzer Frauen fortführen.

Widerständige, und auf vielfältige Art und Weise artikulierte
Selbstdefinitionen aus Selbsterfahrung ermöglichen, so argumentiert Mohanty,
jene Leerstellen zu füllen, für die es im dominanten Diskurs keinen Ort gibt.
Indem die Selbstbeschreibungen mit den Kategorien brechen die von der
hegemonialen Ordnung für die Marginalisierten vorhergesehen werden,
transformieren sie diese bis zur Unkenntlichkeit und schaffen damit
Bezugspunkte für die Artikulation von Erfahrung die sich in ihrer Differenz
gegenüber der dominanten Erfahrung behaupten. \footnotemark
\footnotetext{Mohanty, \glqq On Race and Voice,\grqq ebd. }

Es ist diese Differenz, nicht die Differenz der Diversity Konzepte, sondern die
Differenz die aneckt, die das bestehende irritiert und seiner vermeintlichen
Vollständigkeit beraubt die für Mohanty das Potential hat, einer
Homogenisierung der Marginalisierten entgegenzuwirken. Differenz muss demnach
sowohl als Erfahrung von Differenz innerhalb des vermeintlichen Kollektives als
auch in Abgrenzung zu der Norm gedacht werden. Es ist eine Differenz für die es
keine Sprecher\_in geben kann, sondern die sich über die vielfältigen sich
einander widersprechender Stimmen \footnotemark \footnotetext{Al-Samaray,
\glqq Inspired Topography, \grqq 120.} auszeichnet über die widerspenstige,
uneindeutige Erfahrungen artikuliert werden und die damit jedem Versuch der
Essentialisierung der\_des Anderen entgegensteht. \\
Positionierung wird damit zu
einer Strategie, die sich gegen das Dominante behauptet und sich in der
Behauptung unmittelbar der Einordnung durch den dominanten Diskurs entzieht.
Bündnisse sind darum immer als strategische Zusammenschlüsse zu verstehen, die
die überlebensnotwendig werden, um sich in der hegemonialen Ordnung behaupten
zu können. \glqq You don't go into coalition because you like it. The only reason
you would consider tring to team up with somebody who could possibly kill you,
is because that's the only way you can figure, you can stay alive. \grqq
\footnotemark \footnotetext{Bernice Johnson Reagon zitiert in Mohanty, Ort unbekannt.}

Dies zeigt, dass der weiß dominierte feministische Diskurs durchaus auch für
den Postkolonialismus eine Plattform bildet, solange die Antwort auf die Frage
was mit weibliche Erfahrungen gemeint wird fortlaufend gestellt wird und offen
bleibt für die vielfältigen Positionen aus denen Frauen\_* heraus Erfahrungen
in strukturellen, historisch gewachsenen Gewaltverhältnissen machen.

\subsubsection{Grenzen widerständigen Wissens}

Sowohl in der phänomenologischen Perspektive, in der Erfahrung als Quelle der
Erkenntnis verstanden und ihre subversives Potential auch auf Diskurse
einwirken zu können beansprucht wird \footnotemark \footnotetext{Alcoff, \glqq Zum
Begriff der Erfahrung \grqq.} als auch in poststrukturalistischen
Perspektiven die Erfahrung stärker als Produkt von Diskursen begreift
\footnotemark \footnotetext{Scott \glqq Experience, \grqq.} nimmt
die frage nach den Möglichkeiten Standorte zu beschreiben, und damit situiertes
Wissen auch als solches zu markieren eine große Bedeutung ein.
Standortgebundene Erkenntnisproduktion kommt nicht umhin, der Theoretisierung
von Erfahrung im Sinne der Geschichte des Blickwinkels von dem aus Wissen
produziert wird, einen Wert zuzuschreiben. Während die phänomenologische
Perspektive hier in der unterschiedlichen Erfahrung von Wirklichkeit das
Potential sieht, dominante Perspektiven auszuhebeln, und ihre Partikularität
zu entlarven verhält sich der Poststrukturalismus zögerlicher. Die Struktur von
Diskursen kann demnach nicht über die Analyse von Erfahrungen allein aufgedeckt
werden, da letztere erst durch die Diskurse hindurch entstehen, artikulierbar
werden und entsprechend nicht außerhalb machtvoller Normen darüber was wie
sagbar ist, denkbar sind.
\\

Es wird deutlich dass poststrukturalistische und phänomenologische Positionen
bestreben, den Körper-Geist Dualismus zu überwinden und dadurch auch die
jahrhundertelange Tradition von einem Forschenden Subjekt und einem beforschten
Objekt aufzubrechen. \\
Während in der Phänomenologie dazu der Erfahrungsbegriff
produktiv diskutiert wird und wie bei Duden erkennbar, ein
anti-essentialistisches Körperverständnis entwickelt wird, bleibt im
Poststrukturalismus die Vorsicht bestehen, sich auf eine Kategorie zu stürzen
die dermaßen ideologisch aufgeladen und damit nicht fruchtbar für eine Analyse
eben jener Ideologie zu sein scheint. \\
Gerade im Poststrukturalismus wird daran
gezweifelt, ob Objekte von Forschung einfach zu Subjekten transformiert werden
können indem ihre Erfahrungen als Wissen geltend gemacht werden. Erkenntnis ist
hier nicht automatisch an eine soziale Position gebunden, sondern an einen
Prozess der Untersuchung, des Schreiben und Denkens. Also an eine Praktik die
weniger aus einer Erfahrung resultiert als an einem analytischen Blick auf
Verhältnisse und einer Reflexion der eigenen Verwobenheit in ihnen. \\
Erfahrung
wird jedoch auch im Poststrukturalismus nicht vollkommen negiert. Es wird, das
konnte mit Scott gezeigt werden weniger als Quelle von Erkenntnis, sondern als
Medium für Erkenntnis konzipiert.
\\

Für fast alle diskutierten Positionen kann jedoch festgestellt werden, dass
eine Situierung im postkolonialen Verhältnis ausbleibt. Die Kritik am Dualismus
und der Versuch Subjekt-Objekt Verhältnisse zu überwinden wurde weder in den
poststrukturalistischen noch in den phänomenologischen Positionen als Teil
einer Dekolonisierung \glqq of hearts and minds \grqq begriffen.\\
Dass Körper und Geist
nicht nur vergeschlechtlichte, sondern auch rassifizierte Kategorien sind, die
sich durch das koloniale Verhältnis hindurch formen konnten, wird erst von
Canning in Bezug auf Mohanty und Haraway deutlich. Mohanty beschreibt hier
anschaulich, wie notwendig eine postkoloniale Aneignung des Erfahrungsbegriffs
für eine emanzipatorische Epistemologie ist die hier Erfahrung immer als
Erfahrung von (rassifizierter) Differenz denkt.
\\

Im folgenden möchte ich mit dieser Position weiterdenken und dabei Erfahrung in
Differenzverhältnissen als Ausgangspunkt und Gegenstand von Gegenerzählung diskutieren.

%\section{Erzählung}
\epigraph{
The Black Atlantic developed from my uneven attempts to show these students, that the experience of black people were part of the abstract modernity they found so puzzling and to produce as evidence some of the things that black intellectuals had said – sometimes as defenders of the West, sometimes as its sharpest critics- about their sense of embeddedness in the modern world.
}{Paul Gilroy\footnotemark} \footnotetext{Gilroy,\textit{The Black Atlantic.
Modernity and Double Consciousness},IV.}

Die Unterteilung der Welt in Zentrum und Peripherie wird von postkolonialen
Theoretiker\_innen als eine der maßgebendsten Praktiken der sogenannten
Europäischen Moderne gefasst. \\
Maßgebend in zweierlei Hinsicht:\\
Mit Blick auf
die gewaltvolle Herrschaft der kolonisierenden Länder über die kolonisierten
Länder und mit Blick auf die mit dieser Herrschaftsform einhergehende
dualistische Wissensproduktion, die sich gewiss nicht ohne
Widerstand\footnotemark \footnotetext{Widerstand als allgegenwärtiges Moment
des kolonialen Verhältnisses – bereits widerständige Lesart, da das Phantasma
der Fügsamkeit der Kolonisierten in Ausbeutungsverhältnisse dadurch gebrochen
wird. Vgl.: Troillot, \textit{Silencing The Past. Power and the Production of
History}.} aber
dennoch mit enormer Kraft in das Denken der kolonisierten wie auch
kolonisierenden Gesellschaften und ihren Subjekten einschrieb und
schreibt.\footnotemark \footnotetext{Frantz Fanon, \textit{Schwarze Haut, Weiße
Masken}, Frankfurt: Suhrkamp, 1992 und Smith, \textit{Decolonizing
Methodologies}.} 
\\
Im letzten Kapitel konnte ich aufzeigen, dass hierbei
insbesondere die Erfindung des 'Menschen' von einer tiefen Spaltung gezeichnet
ist, in der sich vermeintlich Subjekte und Objekte gegenüberstehen. Die Idee
vom Menschen ist damit stets an einen Spiegel gebunden, der einer die eigene
Negation vorhält und erst so die Möglichkeit der Selbstkonstitution eröffnet.\\
Diese Spaltung, das hat die Auseinandersetzung mit dem Erfahrungsbegriff in der
emanzipatorischen Epistemologie gezeigt, eröffnet den dermaßen verschieden
positionierten Subjekten jedoch auch äußerst verschiedene Spiegelbilder oder
anders, Erfahrungsräume die unterschiedliches Wissen hervorbringen. Subjekte
können sich der Spaltung zwar nicht entziehen, sie können ihre Erfahrungen
jedoch im Verhältnis zu dieser bestimmen, reflektieren und so Wissen erzeugen
das andere Spiegel- und Selbstbilder ermöglicht. \\
Dazu bedarf es nicht nur eines
kritischen Blickes darauf wie sich Dualismen in das eigene Denken einschreiben,
sondern auch, wie jene Dualismen historisch gewachsen sind und so das Denken
mitsamt seinen vermeintlichen Grenzen formen konnten. 
\\
Dieser Blick in die
Geschichte des Denkens muss, das habe ich im letzten Kapitel aufgezeigt, die
Gewalt der kolonialen Eroberung fokussieren aus der jene Spaltung entspringt.
Das Wissen, das als legitimiertes Wissen aus dieser Gewalt heraus entstanden
ist und die Spaltung des Menschen in Subjekte und Objekte immerwieder von Neuem
vollzieht, schreibt sich dabei auch in die Geschichten ein, die von und über den Menschen erzählt werden.
\\

Im Verlaufe dieses Kapitels möchte ich auf jene Perspektiven Bezug nehmen,
welche diese Praktik der Unterteilung in Zentrum und Peripherie als Erzählung
theoretisieren. \\
Die im vorherigen Kapitel explizierte andro- und eurozentrische
Wissenschaftspraxis, ermöglicht, so die These, eine Erzählung in der sich
Europa nicht nur als Mittelpunkt der Welt, sondern darüber hinaus als
Ursprungsort eines Denkens inszeniert, welches tatsächlich Ausdruck einer
gewaltvollen Begegnungs- und Beziehungsgeschichte war und ist. 

Meine
Bezugnahme von Repräsentations- und Erzähltheorie zeigt hier auf, dass es sich
bei der Europäischen Moderne nicht um irgendeine Erzählung handelt, sondern in
direkter Abhängigkeit zur hegemonialen Episteme um eine hegemoniale Erzählung.
Dabei wird die Europäische Moderne auf der Ebene ihrer diskursiven Selbst- und
Außenkonstruktion untersucht und der Zusammenhang bzw. die Verschränkung von
Erzählung und Wissensproduktion verdeutlicht. Im Vordergrund steht dabei für
mich nach wie vor die Frage nach der Erfahrung: \\
Welchen Ort erhält Erfahrung?
Wie wird Erfahrung gedacht? Kann Erfahrung für eine dekoloniale Intervention
nutzbar gemacht werden?
\\

Gilroy greift die Bedeutung von Erfahrung auf, wenn er wie im oben stehenden
Zitat zu seinen Schwarzen Student\_innen spricht, und diese auffordert, sich als
Teil der \glqq abstrakten Moderne \grqq\footnotemark \footnotetext{Gilroy,
\textit{The Black Atlantic.}} zu begreifen die ihnen bisweilen \glqq rätselhaft
\grqq\footnotemark \footnotetext{Ebd.}
erscheint. Letzteres verdeutlicht die narrative Macht, die dem kolonialen
Projekt entspringt. Wie sonst kann den Studierenden etwas abstrakt und
rätselhaft erscheinen, das in vielfacher Hinsicht von ihren Erfahrungen und den
Erfahrungen ihrer Vorfahr\_innen konstituiert ist? 

Rätsel haben bei aller Unterschiedlichkeit gemein, dass sie etwas verbergen.
Die Erzählung über die sogenannte Europäische Moderne, das konnte ich in den
letzten Kapiteln herausstellen, verbirgt nicht nur die Gewalt aus der sie
entsprungen ist, sie verbirgt zudem, dass sie nur auf Grund dieser Gewalt
möglich werden konnte. So ist es nicht verwunderlich, dass einer\_m etwas
rätselhaft erscheint, wenn dieses etwas um eine geteilte, gewaltvolle Erfahrung
kreist, die weder beschrieben noch benannt wird. Und es ist noch weniger
verwunderlich, dass diese Ent\_nennung, das Misstrauen jener weckt, die in
dieser Erzählung kategorisch ausgeschlossen werden, wie von Gilroy beschrieben.\\
Wenn im folgenden die Europäische Moderne, die viel genauer die von Europa
vereinnahmte Moderne genannt werden müsste, als Erzählung theoretisiert wird,
dient dies nicht einer Leugnung der stattgefundenen Gewalt. Eine Erzählung
findet nicht anstatt einer Wirklichkeit statt, sie ist viel eher ein Werkzeug,
das aus der Gewalt entspringt und sie zugleich legitimiert und die darum für
eine Untersuchung interessant ist.\footnotemark \footnotetext{Vgl. Bristol,
\textit{Plantation Pedagogies.}} 
\\

Gilroys Forderung, die tatsächliche Eingebundenheit der Ausgeschlossenen
anzuerkennen, ist Teil eines postkolonialen und feministischen
Theorieprojektes, das es sich zur Aufgabe macht, von und zu dem Außen zu
sprechen, auf das sich Europa in seinem Bestreben, das Zentrum der Welt zu
sein, heimlich bezieht. \\
Dadurch wird die Spaltung, durch die sich die
Europäische Moderne konstituiert, unterwandert und die Europäische Moderne als
Prozess der wechselseitigen Hervorbringung von Zentrum und seinem Außen
theoretisiert. 

Die Spaltung wurde insbesondere in der feministischen
Epistemologie als Spaltung im Geschlechterverhältnis begriffen.
Dekolonisierung, das Vorhaben die  Europäische Moderne zu Dezentrieren bzw.
Europa zu \glqq provinzialisieren \grqq\footnotemark
\footnotetext{Chakrabarty, \textit{Europa als Provinz.}} problematisiert jedoch nicht nur die Logik der
Zweigeschlechtlichkeit sondern greift darüber hinaus Rassismus als inhärenten
Bestandteil der Europäischen Moderne auf und an. Postkolonialer Feminismus
versucht hier den Dualismus von Körper- Geist, Subjekt-Objekt zu überwinden in
dem es aufzeigt, wie vermeintlich Gegensätzliches zusammen gedacht werden kann
und so Erfahrungen vom Menschsein in das Wissen eingeschrieben werden können,
die im dominierenden Diskurs negiert wurden. \\
Die kontroverse Auseinandersetzung
um Erfahrung in der Wissensproduktion kann hier als ein konkreter Versuch
verstanden werden, emanzipatorische Epistemologien aufzubauen.
\\

Theoretiker\_innen haben in ihrem Versuch, kulturelle Phänomene
nachzuvollziehen, jene Phänomene auf verschiedene Art und Weise
konzeptionalisiert.\\
 Was heißt es nun, das soziale Phänomen, der Europäischen
Moderne als Erzählung zu theoretisieren?\\

In dieser Arbeit, das ist bereits vielfach angeklungen, dominiert eine
diskursive Ebene als theoretischer Zugang. Dies birgt zugleich den Anspruch,
dass das Materielle, Verkörperte etc. auch in dem Diskursiven Ausdruck gewinnen
bzw. durch das Diskursive zugänglich gemacht werden kann. Problematisch ist
hierbei, dass eine Sprache in der Narrativ, Erzählung, Diskurs, Repräsentation
etc. zu den Kerntermini werden es unter Umständen nicht schafft, die Präsenz
der Gewalt, also ihre Gegenwärtigkeit in dem Leben der Menschen, zu
artikulieren.\\

Die Auseinandersetzung mit Erfahrung kann hier jedoch als Brücke zwischen dem
Materiellen und Diskursiven verstanden werden, da Erfahrungen zum Beispiel im
Anschluss an Duden\footnotemark \footnotetext{Vgl. Duden, \glqq Somatisches
Wissen\grqq. }, als Ausdruck somatischer Erlebnisse verstanden werden
können, die durch ihre Artikulation Einzug in das Diskursive erhalten bzw.
darin intervenieren. Insbesondere Dudens Konzept scheint sich daher für eine
postkoloniale Kritik am Dualismus zu eignen. Dabei muss, wie schon im letzten
Kapitel herausgestellt, stets achtsam damit umgegangen werden, welches Subjekt
imaginiert und damit auch welchen Erfahrungen Raum gegeben wird um die
ausschließende Logik hegemonialen Wissens nicht fortzuschreiben. 
\\

Bevor ich mich, wie eben angekündigt, mit der Bedeutung von Erfahrung für die
Praxis des (Gegen)Erzählens beschäftige, möchte ich mich zunächst noch in
allgemeinerer Form dem Begriff der Erzählung widmen.\\
Die Disziplin, die ich hierfür heranziehe ist die Narratologie. Dabei
beschränke ich mich auf Auseinandersetzungen im deutschsprachigen Raum und gehe
nicht auf den, mit dem postmodernen Aufschwung einher gegangenen, und
insbesondere durch Jean-François Lyotard bekannt gewordenen Begriff des Grand
Récit, der Großen Erzählung ein, da es sich hier um ein zwar interessantes,
aber sicherlich den Rahmen dieser Arbeit sprengendes Diskursfeld handelt.
\\

Die Narratologie ist auch ohne die französischen, postmodernen Intellektuellen
ein weites Feld. Deren Aufschwung und Niedergang in Mitten der Disziplinen wird
dabei recht unterschiedlich erzählt. \\
Insbesondere die sogenannten Feinde der
Narratologie könnten gegensätzlicher nicht sein: \\
So wird einmal die Hegemonie
der Naturwissenschaften\footnotemark \footnotetext{Achim Saupe und Felix
Wiedemann, \glqq Narration und Narratologie. Erzähltheorien in der
Geschichtswissenschaft,\grqq Version 1.0, in \textit{Docupedia-Zeitgeschichte}, (2015), 1.} angeführt, ein anderes Mal dem Poststrukturalismus die
Schuld für das sinkende Interesse an der Narratologie in die Schuhe
geschoben\footnotemark \footnotetext{Ansgar Nünning, \glqq Wie Erzählungen Kulturen
erzeugen. Prämissen, Konzepte und Perspektiven für eine kulturwissenschaftliche
Narratologie \grqq in \textit{Kultur-Wissen-Narration. Perspektiven transdisziplinärer
Erzählforschung}. Herausgegeben von Andrea Strohmaier,  (Bielefeld. Transkript,
2013), 16.}. \\
Gemeinsam ist jenen Erzählungen jedoch, ihre geteilte Erleichterung
gegenüber dem Come Back der Narratologie in den vielfältigsten Disziplinen. Die
Spanne reicht von literaturwissenschaftlichen Fragestellungen nach der
Erzählstruktur, über psychologische Untersuchungen der narrativen Identität, zu
geschichtswissenschaftlichen Auseinandersetzungen um das Spannungsfeld von
Fiktionalität und Faktizität angesichts der narrativen Aufbereitung von
historischen Ereignissen. Auch wenn es nicht wenig unterhaltsam ist, sich mit
den verschiedenen Niedergangs- und Behauptungsgeschichten der Narratologie zu
befassen, muss an dieser Stelle davon abgesehen werden. Stattdessen möchte ich
mich mit einem Teilbereich der Narratologie befassen, nämlich der
kulturwissenschaftlichen Erzählforschung. \\
Grund für meine Präferenz der
Kulturwissenschaften ist ihre Konzeption der sogenannten
\glqq Wirklichkeitserzählung \grqq\footnotemark \footnotetext{Nünning, 
\glqq Wie Erzählungen Kulturen erzeugen,\grqq 27.}. \\
Anders als andere Zugänge gerät hier das
Wechselspiel, in dem sich das Grand Narrative mit den Partikularen Geschichten
befindet, in den Fokus. Ein zentraler Vertreter der kulturwissenschaftlichen 
Erzählforschung ist Ansgar Nünning.

 Nünning plädiert für eine Synthese kultur-
und erzählwissenschaftlicher Forschungsansätze.\footnotemark \footnotetext{Ebd. 28.}
Ausgangspunkt dieser Forderung
bilden dabei seine Auffassung, dass sich Kultur u.a. über Praxen des Erzählens
konstituiert bzw. dass Erzählungen immer als kulturelle und damit wandelbare,
an den zeit- und räumlichen Kontext gebundene Phänomene untersucht werden
müssen. \\
Eine kulturwissenschaftliche Narratologie ermöglicht es in diesem Sinne
eine in der Vergangenheit vielfach ahistorisch praktizierte Narratologie für
die Bedeutung des Erzählkontextes sensibel zu machen. Für die
Kulturwissenschaften hingegen ergebe sich durch die Synthese weniger eine
Erweiterung ihrer theoretischen Perspektive, als vielmehr ein Impuls für einen
äußerst relevanten, vielfach jedoch ignorierten Gegenstandsbereich: \\
Erzählungen
werden hier, so Nünnung, nicht auf Prosa reduziert, viel eher interessiert sich
die kulturwissenschaftliche Narratologie für sogenannte
'Wirklichkeitserzählungen' und versteht darunter jene über das literarische
Erzählens hinausgehende Praxen der Bedeutungskonstruktion. Erzählungen werden
hier durch zwei wesentliche Kennzeichen gefasst. \\
Zum einen sind sie stets
Ausdruck \glqq spezifischer Verknüpfungen \grqq\footnotemark
\footnotetext{Saupe und Wiedermann, \glqq Narration und Narratologie,\grqq 2.}, zum anderen werden sie von einer
\glqq temporalen Struktur \grqq\footnotemark \footnotetext{Ebd.}  geformt, sodass sie als \glqq zeitlich strukturierte
Repräsentation von Ereignissequenzen\grqq\footnotemark \footnotetext{Ebd.} begriffen werden können. So könne die
Erzählung auch vom Diskursbegriff abgegrenzt werden, der sich stärker auf die
Ordnungsfunktion von Sprache beziehe und damit synchron angelegt sei, wogegen
sich der Erzählbegriff eher auf linear und temporale Dimensionen
beziehe.\footnotemark \footnotetext{Nünning, \glqq Wie Erzählungen Kulturen
erzeugen,\grqq 28.}
\\
Erzählungen bilden damit eine Form des materialisierten, bzw. performierten
Ausdrucks gesellschaftlicher Ordnungen:
\begin{myenv}
  \textit{
  \glqq Zweifelsohne sind es Erzählungen die kollektiven, nationalen
  Gedächtnissen zugrunde liegen und Politiken der Identität bzw. Differenz
  konstituieren. Kulturen sind immer auch als Erzählgemeinschaften anzusehen,
  die sich gerade im Hinblick auf ihr narratives Reservoir unterscheiden \grqq\footnotemark \footnotetext{Wolfang Müller-Funk, \textit{Die Kulturen und ihre
Narrative. Eine Einführung}, (Wien: Springer Verlag, 2008), zitiert  in Nünning, ebd., 29.}
  } \end{myenv}

  Wolfgang Müller-Funk greift hier die politische Dimension kultureller
  Praktiken auf und weist daraufhin, dass jene Ordnungen von
  Auseinandersetzungen um Identität und Differenz geprägt sind, die als Kämpfe
  um Zugehörigkeit zu eben jenen kulturellen Gemeinschaften aufgefasst werden
  können. \\
  Die Frage der Macht bahnt sich auf diese Weise ihren Weg in die
  Narratologie und nimmt zugleich Gebrauch von deren formalistischen
  Ausprägungen. Das heißt, dass der Blick auf die Form für die Narratologie
  konstitutiv ist, da er die Möglichkeit bietet, über direkte Erzählbotschaften
  hinaus Aufschlüsse über die ideologische Dimension des Erzählten zu erhalten.
  Frederic Jamesons ideology of the form die davon ausgeht, dass die Art und
  Weise wie erzählt wird mehr über das Motiv der Erzählung aussagt, als eine
  bloße Aufmerksamkeit für den Inhalt es vermag, gilt als einer ihrer
  prominentesten Vertreter. 
  \\
  An dieser Stelle wäre eine tiefer gehende
  Auseinandersetzung mit den Methoden der Narratologie bzw. ihrem
  Formverständnis möglich, für meine Arbeit jedoch nicht fruchtbar. Da meine
  Arbeit keine Anwendung dieser Methode beabsichtigt, sondern sich vielmehr der
  Narratologie bedient um den Begriff der Erzählung, wie er im postkolonialen
  Feminismus Verwendung findet, zu schärfen.

  \subsection{Ich-Erzählung als Gegenerzählung}

  Ich knüpfe nun wieder an die feministische Auseinandersetzung um den
  Erfahrungsbegriff in der Erkenntnisproduktion an, konkretisiere die
  Diskussion allerdings auf eine postkoloniale Praxis: die Ich-Erzählung als
  Gegenerzählung. Dazu setze ich mich mit den Möglichkeiten von Repräsentation
  und Dialog auseinander. Kann eine\_r sich selbst erzählen? Was braucht sie\_er
  dafür? Ab wann ist eine Ich-Erzählung eine Gegenerzählung?

  \subsubsection{Repräsentation}

  Eine Auseinandersetzung mit den Möglichkeiten und Grenzen von Repräsentation
  durchzieht diese Arbeit wie ein roter Faden. Bereits im Bildungsbegriff, der
  Wimmer folgend, auf einem Subjektverständnis ruht das niemals in der Lage ist
  sich selbst zu repräsentieren, da es immer auf die Entbehrung eines
  kohärenten Selbst zu Gunsten eines Zwischenraums zwischen Ich und Welt
  zurückgeworfen wird, wird die Problematik der Repräsentation aufgegriffen.
  Diese Problematik wird mit  Mohanty in Anbetracht eines Sprechens über bzw.
  für die 'Dritte Welt' weitergedacht, wenn sie die Herausforderungen
  beschreibt, in ihren Darstellungen nicht zugleich als Vertreterin eben jener
  Dargestellten missverstanden zu werden. Während es also bei Wimmer zunächst
  um die Repräsentation des Selbst geht, beschreibt Mohanty die Schwierigkeit
  andere zu repräsentieren.
  \\
  
  Auch die hegemoniale Episteme, die daraufhin von
  mir erörtert wird, beruht letztlich auf einem falschen Versprechen der
  Repräsentation: Sie gibt vor für alle zu sprechen,  in Wirklichkeit
  verkörpert sie jedoch nur die Interessen einer Minderheit.  Die Einordnung
  diesen Wissenssystems als hegemonial, beschreibt darüber hinaus die
  Schwierigkeit sich ihm zu widersetzen und damit aus dem herrschenden
  Repräsentationsregime auszubrechen, da es so stark auf  Zustimmung aufgebaut
  und in der Lage ist auch die Versuche des Widerstands für sich zu
  vereinnahmen.
  \\

  Versuche, der hegemonialen Idee des  Menschen eine anderes Fundament
  entgegenzustellen,  kommen hier nicht ohne Widerspruch aus. Emanzipatorische
  Strategien fallen oftmals auf eine von identitären Kategorien durchdrungene
  Sprache zurück – zu Lasten einer Ideen von Erfahrungen als Erfahrung von
  Differenz. Der Repräsentationsanspruch der mit identitären Kategorien, wie
  der Kategorie Frau\_* einhergeht wird hier von jenen Frauen\_* zurückgewiesen,
  die sich in den proklamierten 'weiblichen' Erfahrungen nicht wiedererkennen
  und sich diesem kategorialen Ausschluss widersetzen.

  Die Gegengeschichtsschreibung, oder Gegenerzählung kann nun als Versuch
  beschrieben werden, einen Zwischenraum zu öffnen der dem dissonanten Stimmen
  kein vorgegeben identitären Kategorien überstülpt. Gegenerzählungen  können
  auf diese Weise neue Wege der Repräsentation erkämpfen, die sich dem
  vereinnahmenden Praxen des Verstehens widersetzen und Erfahrung immer als
  Erfahrung von Differenz verstehen. Im Sinne eines strategischen
  Essentialismus, kann sich aus dieser Differenz heraus behauptet werden ohne
  auf die Differenz reduziert zu werden.
  \\
 
  Laurette Bristol geht auf dieses emanzipatorische Potential von Erfahrung als
  Wissen in der Praxis des Erzählens  in  \glqq Plantation Pedagogy. A Postcolonial
  and Global Perspective \grqq\footnotemark \footnotetext{Bristol,
  \textit{Plantation Pedagogy.}} ein.

  Geschichten, bzw. das Erzählen von Geschichten, so schreibt Bristol in ihrer
  Monographie können als Praktiken verstanden werden, die die Kraft haben,
  herrschende Repräsentationsregime zu unterwandern. Sie werden damit zu
  Werkzeugen für eine Pädagogik, die es den Erzähler\_innen und Zuhörer\_innen
  ermöglicht, wenn auch nur für einen Moment, aus den “ravages of
  existence”\footnotemark \footnotetext{Ebd., 1.}
  auszubrechen und an an einen anderen Ort zu gelangen. \\
  Dieser andere Ort birgt
  zugleich ein Entkommen, eine Distanz zum Selbst  als auch einen Standpunkt,
  von dem aus eine andere Perspektive auf das Selbst in der Welt möglich ist.\\
  
  Damit schließt Bristols Konzept der Erzählung an den Bildungsbegriff der im
  ersten Kapitel dargelegt wurde an und beschreibt das emanzipatorische
  Potential, welches im Erzählen von Geschichten liegt:
  \begin{myenv}

\textit{\glqq To tell a story [she claims, is] to seize a political space
  created through talk, to represent self, to claim his/her authority on
  the topic of a discussion and to share his/ her interpretation of social
existence with the community.\grqq\footnotemark \footnotetext{Ebd.}} 
\end{myenv}

Interessant sind hierbei ihre Verwendung der Verben to \textit{seize},
\textit{claim} und \textit{share}.
Das Erzählen von Geschichten beschreibt sie hier als Praxis, in der eine
bestimmte Perspektive zugleich beansprucht, behauptet und geteilt wird. Das
Subjekt, das Autorität über seinen Zugang zur Wirklichkeit beansprucht, wird
dabei nicht als Individuum allein, sondern in der Verbindung zur Gemeinschaft
begriffen mit der es seine Behauptung, die immer als Interpretation verstanden wird, teilt.\\

Der Versuch, der hegemonialen Erzählung etwas entgegenzusetzen, kann jedoch nur
gelingen, wenn das Subjekt-, Geschichts- und Wissenschaftsverständnis, das sich
in der Europäischen Moderne etablierte, überwunden wird. Denn „even marginal
experiences narratives risk repeating the biases and exclusions of received
narrative logics“\footnotemark \footcitetext[S. 142]{sstone}, wenn sie von einem autonomen Subjekt ausgehen, lineares
Geschichtsverständnis bedienen oder Wissenschaft außerhalb von
gesellschaftlichen Verhältnissen begreifen, so Shari Stone-Mediatore in ihrer
Monographie „Reading across Borders. Storytelling and Knowledges of
Resistance“\footnotemark \footnotetext{Stone-Mediatore, \textit{Reading across
Borders}.}. Um Räume für eine Gegengeschichtsschreibung zu schaffen, ist es
also dringend notwendig, Strategien zu entwickeln mit denen alternative, den
dominanten Vorstellungen widersprechende Subjekt-, Geschichts- und
Wissenschaftsverständnisse ermöglicht werden.
\\

Gloria Anzalduás und Domitila Barrios de Chúngaras Arbeiten liefern für
Stone-Mediatore wichtige Impulse und ich möchte deren Perspektiven an dieser
Stelle als Einstieg nutzen, um konkreter darin zu werden, woran sich Praxen der
Gegenerzählung orientieren können: 
\\
Anzalduá, so Stone-Mediatore, beschreibt
ihre Arbeit als Versuch, persönlichen Erfahrungen Raum zu geben bei einer
gleichzeitigen Notwendigkeit, neue, eigene Kategorien zu schaffen. Poesie,
Autobiographie und Geschichte würden es Anzalduá ermöglichen sich mit jenen
Erfahrungen auseinanderzusetzen, die die Spannkraft besitzen, herrschende
Bilder von Frauen\_* of Color zu unterwandern.\footnotemark \footnotetext{Anzalduá in Stone-Medatore, ebd. 143.} \\
Auch Barrios de Chúngara sehe in
erfahrungsorientiertem Schreiben die Möglichkeit, zwar keine Revolution, aber
immerhin „Sand ins Getriebe zu streuen“ \footnotemark \footnotetext{Barrios de Chúngara in Stone-Mediatore, ebd.}, da die Schilderung eigener
Erfahrungen eine Möglichkeit für andere böte, sich neu zu orientieren. Stone
Mediatore betont an dieser Stelle, dass sowohl Anzalduá als auch Barrios de
Chúngara insbesondere die Erfahrungen von Widerstand in neue, unorthodoxe
Formen bringen. \\
Ihre Erzählungen seien Ausdruck spezifischer Erfahrungen die
sie an den sozial/geografischen Kontext binden, ohne dabei ein
essentialistisches Verständnis des Kontextes zu bedienen. Damit würden sie sich
gegen eine Aufwertung abgewerteter Subjektpositionen wehren und stattdessen die
Logik angreifen, in welcher der Kontext die Handlungsmöglichkeiten der Subjekte
determiniert.\footnotemark \footnotetext{Stone- Mediatore, 144.} Stone-Mediatore beschreibt dies folgendermaßen:
\begin{myenv}
    \textit{„[...] they each use their writing to explore the forces that
    condition their experience and to renarrate their identities in ways that
  help them to confront those forces more effectively. In the process, they
also recast 'identity' as a historically rooted yet also strategic
category.“\footnotemark \footnotetext{Ebd.}}
\end{myenv}


SEITE FEHLT


Es geht also vielmehr darum, die Grenzen der Beschreibbarkeit, mit denen die
eigene Situation sichtbar gemacht werden kann, zu verschieben, also eine
Sprache zu entwickeln die dem Anspruch nach Repräsentation der eignen
Erfahrungen ein wenig mehr gerecht wird, als es herrschende Erzählparadigmen
und ihre Konzepte vermögen.
\subsubsection{Dialog}

Mit der Frage, wie unterschiedliche Ausdrucksformen von Wissen miteinander
in Bezug gesetzt werden können, setzt sich Lorraine Code auseinander. Sie
interessiert sich dafür,  wie Wissen über Menschen und die  Situationen in
denen sie sich befinden dialogisch entstehen könnte.\\
 Hier richtet sie ihr
Augenmerk auf das Potential, das in sogenannten \textit{First-Person-Accounts} also
Ich-Erzählungen liegt, in denen von eigenen Erfahrungen berichtet wird. Diese
Form der Beschreibung von Wirklichkeit erhält ihrer Ansicht nach in der
\textit{Malestream Epistemology} kaum Bedeutung.\footnotemark \footnotetext{Lorraine
Code, „Experience, Knowledge and Responsibility“ , in A. Garry and M. Pearsall, Hrsg., Women, Knowledge and Reality (Boston: Unwin Hyman, 1989), 168.} Auch Code setzt sich mit der Bedeutung von Erfahrung auseinander. Sie versteht Erfahrung in Abgrenzung zu theoretischen Positionen. 


Die Begegnung mit und Einbeziehung von Erfahrungen, die über Ich-Erzählungen
artikuliert werden kann hierbei, so Code,  der Formalität und Isoliertheit
abstrakter Theoriegebilde eine lebendige, lebensweltorientierte Erkenntnis
gegenüberstellen. Persönliche Erfahrungen sind für Code keineswegs wahrer oder
authentischer als wissenschaftliche Texte. Ihre Affinität für Ich-Erzählungen
liegt hingegen in der enormen Mannigfaltigkeit der Perspektiven die durch die
Praxis des Erzählens zum Ausdruck kommen. \\
Ziel der erkenntnistheoretischen
Nutzbarmachung von Erfahrung sieht sie entsprechend nicht in der Ergänzung
bereits bestehender Theorien durch Ich-Erzählungen. Viel eher soll die
Einbindung unterschiedlicher Wissensformen die Brüche und Leerstellen aufzeigen,
die ihrer Ansicht nach einen unumgehbaren Bestandteil jeder Theorie beinhalten
und zu einem Dialog zwischen theorie- und erfahungsbasiertem Wissen anregen.

Um unterschiedliche Formen des Wissens in einem gleichberechtigten Dialog
miteinander sprechen zu lassen, müsse jedoch die Suche nach dem 'reinen Wissen'
als unerreichbares und ohnehin nicht erstrebenswertes Ziel aufgegeben
werden.\footnotemark \footnotetext{Code, „Experience, Knowledge and Responsibility,“ 157.}\\
Stattdessen gilt es, so Code, Epistemologie und Ethik als gemeinsame und gleichsam bedeutende Aspekte der Wissensproduktion anzuerkennen. Verantwortungsbewusstes und an Erfahrung orientiertes wissenschaftliches Tun werde dadurch zu einer Grundprämisse.

Code macht hier auch deutlich, dass sie ihrem Plädoyer keine verallgemeinerbare
Erfahrung des 'Frau\_* seins' zu Grunde legt. Eine solche Annahme verdecke die
vielfachen Unterschiede, die zwischen Frauen\_* existierten. Stattdessen
verwendet sie stets den Plural und spricht von Erfahrung\textbf{en} von Frauen\_* um der
Heterogenität von Lebensumständen in denen Frauen\_* entsprechend
unterschiedliche Erfahrungen machen, gerecht zu werden.
\\

Ihr Ansatz, sowohl den Erfahrungen als auch der Verantwortung eine wesentliche
Bedeutung in der Erkenntnisproduktion und ihrer Bewertung zuzuschreiben, kann
dabei weder als komplementär zur traditionellen Erkenntnistheorie, noch als von
ihr vollständig losgelöst verstanden werden. Sie begreift ihre Arbeit viel eher
als eine permanente Auseinandersetzung mit der malestream tradition deren
Lücken, Ausschlüsse und unbefragten Vorannahmen sie durch eine kritische
Bezugnahme weiblicher Erfahrungen zur Diskussion stellen möchte.\footnotemark
\footnotetext{Ebd., 158.}

Code schreibt weiblicher Erfahrung für feministische Erkenntnistheorie zwar
einen zentralen Stellenwert zu, warnt aber davor, essentialisierenden
Universalismen auf der einen Seite, oder subjektivistischen Relativismen auf der
anderen Seite zuzuspielen. Hingegen müsse die klassische Gegenüberstellung und
das damit einhergehende Ausschlussprinzip zwischen Subjektivität und
Objektivität überwunden werden.\\
 Denn da Frauen\_* per se ein subjektiver
Standort in Abgrenzung zum objektiven wissenschaftlichen Wissen zugeschrieben
werde, führe dies zu dem fatalen Schluss, dass weibliches Wissen nicht
gleichzeitig wissenschaftliches Wissen sein könne.\footnotemark
\footnotetext{Ernst, \textit{Diskurspiratinnen}, 87.} 

Feministische Standpunktepistemologie, in die ich bereits mit Hartsock eingeführt haben, kehren dieses Prinzip zwar erfolgreich um, indem sie gerade dem weiblichen Wissen einen objektiveren Zugang zu Wirklichkeit zuschreiben, bleiben aber in der binären Logik von Objektivität und Subjektivität verhaftet, die es zu überwinden gelte.

An dieser Stelle führt Code in die Praxis des Dialogs ein und löst damit auch die Dichotomie, nach der zwischen dem Subjekt das Wissen generiert und das Objekt, über das Wissen hergestellt wird, unterschieden werden kann, auf.
Stattdessen werden Subjekt und Objekt gewissermaßen austauschbare Kategorien, da
es fortan stärker um den Dialog zwischen unterschiedlichen
Abstraktionsdimensionen geht, indem verschiedene Formen des Wissens miteinander
in Beziehung gesetzt werden.\footnotemark \footnotetext{Ebd., 88.} 

Ein solcher Dialog setzte jedoch ein gewisses Maß an Bescheidenheit, sich den
Grenzen des eigenen Wissens bzw. der individuellen Perspektive bewusst zu sein,
voraus. Nur ein von wechselseitigem Interesse bestimmter Austausch könne einen
Dialog ermöglichen, der tatsächlich verschiedene Formen des Wissens zueinander
in Beziehung setzt ohne dabei in klassische Hierarchisierungen zu verfallen.\\
Code stellt an dieser Stelle das Konzept der Freundschaft als interessante
Allegorie zur Diskussion. So sind es ihrer Ansicht nach, und hier nähert sich
ihre Position der von Mohanty bedeutend an, gerade die Getrenntheit, Distanz und
die Anerkennung von Differenz die den Dialog in Freundschaften prägt und die
unabdingbare Voraussetzungen für die Auseinandersetzungen um epistemologische
Fragen darstellt.\footnotemark \footnotetext{Ebd., 89.}

Code unterscheidet in ihrer Argumentation zwischen wissenschaftlich
abgeleiteter, bzw. interpretierter Erfahrung und alltäglicher Erfahrung. Das
Abstraktionsniveau soll jedoch für die Nutzung der Erfahrungen für die
Erkenntnisgenerierung nicht bewertet werden. Vielmehr ist die Einsicht, dass
gerade der Verwobenheit unterschiedlicher Erkenntnisprozesse Raum gegeben werden
muss, zentral für das Zustandekommens eines Dialoges.\footnotemark
\footnotetext{Ebd., 92.}

Erzählungen, so kann Code hier verstanden werden, bedienen unterschiedliche
Abstraktionsniveaus, die insbesondere in ihren Verknüpfungen bzw.
wechselseitigen Bezugnahmen interessant werden.

Ihre Perspekive ist darum interessant, weil sie die poststrukturalistische
Präferenz des abstrakten Denkens und die phänomenologische Affinität für das
unmittelbare als gleichberechtigte Formen des Wissens anerkennt und für ihre
Synthese plädiert. \\
Das sogenannte 'gegen den Strich lesen' von ideologisch
aufgeladenen Erfahrungen, dem vermeintlich eine theoretische Schulung
vorausgehen muss, wird insbesondere auf Grund der Hierarchisierung von Erfahrung
und Theorie, die diesem Ansatz vorausgeht, von ihr in Frage gestellt. Nicht erst
durch die abstrakte Analyse von  Erfahrungen werden Erkenntnisse gewonnen, viel
eher so lässt sich Code resümieren, sind Erkenntnisse das Ergebnis dialogischer
Praxen in denen Wissen mit Wissen in den Austausch geht wenn Erfahrung auf
theoretische Positionen trifft.

\subsection{Transmoderne als Gegenerzählung}

Ein anderer Ansatz des Gegenerzählens stellt das Konzept der Transmoderne von
Enrique Dussel dar. Die Gegenerzählung versucht hier nicht als partikulare
Ich-Erzählung das hegemoniale Narrativ der Europäischen Moderne zu unterwandern,
sondern als eine Metaerzählung, die über die europäische Moderne hinausgeht,
letztere in ihrer Macht einzuschränken. So wird die Europäische Moderne zur
partikularen Erzählung innerhalb einer Transmoderne, in der viele einander
widerstreitende Erzählungen nebeneinander existieren.

Für Dussel stellt der Partikularismus, den ich eben unter den Aspekten
Repräsentation und Dialog vorgestellt habe, keine ausreichende Kritik an der
Europäischen Moderne dar. \\
Seine Position knüpft zwar teilweise an die von Code
und Bristol vorgestellten Strategien an, stellt diese jedoch unter ein anderes
Vorzeichen. Seine Vermutung ist, dass sich Formen des Partikularismus zu schnell
relational zur Europäischen Moderne positionieren und damit leicht von ihr
vereinnahmt werden können. Alcoff beschreibt seine Position hier folgendermaßen:
Die Gefährlichkeit der Europäischen Moderne liege ihm nach nicht in ihrem
Anspruch, global gültige Aussagen zu treffen, sondern in der wissenschaftlichen
Praxis, mit der sie ihre Erkenntnisse gewinnt und legitimiert.\footnotemark
\footnotetext{ Alcoff, „Dussels Transmodernism,“ 62.} Dussel greift somit nicht
nur das gegenwärtige hegemoniale Wissenssystem der Europäischen Moderne, sondern
auch ihr Entstehungsnarrativ und damit einhergehendes Selbstverständnis an.
\\

Er macht darauf aufmerksam, dass die Alternative weder in einem puren
Relativismus, noch in einer radikalen Umkehrung bzw. Neubestimmung des Zentrums
liegen kann. Wir sind viel eher gefordert, die eindimensionale und damit
notwendig gewaltvolle Setzung dessen, von wo aus über wen Wissen produziert
werden kann, durch eine Praxis des Dialoges zu unterwandern, die verschiedene
epistemische Gemeinschaften verbindet und so dem Universalismus der Moderne ein
Pluriversalismus der Transmoderne gegenüberstellt. Das Konzept der Transmoderne
knüpft an die Kritik an der hegemonialen Erzählpraxis der Europäischen Moderne
an, und stellt ihr die Konzepte der Verortung, Exteriorität und Autonomie
entgegen. \\
Die Praxis des Dialoges nimmt hierbei einen zentralen Stellenwert ein
um jenen Konzepten in ihrer wechselseitigen Bedingtheit Wirkungskraft zu
verschaffen. Gegenerzählung verlangt dabei eine situierte, im Dialog mit anderen
Wissensbeständen praktizierte Wissensproduktion, die ein reflexiven Umgang mit
denjenigen Kategorien vornimmt, die sie konstituieren.
\\

Die Dramaturgie Dussels Ausführungen erhält neben einer chronologischen auch
eine analytische Ordnung. Auf beide möchte ich im Folgenden eingehen und mich
auf diese Weise dem Konzept der Transmoderne, wie es von Dussel entwickelt
worden ist, nähern. So ist es nicht überraschend, dass Dussel in seine
Ausführungen auf die Geschichte seiner intellektuellen Auseinandersetzung, als
Schüler\_* und später Student\_* zurückblickt und damit deutlich macht, dass sich
seine Philosophie unmittelbar aus dem eigenen Erleben und seiner Position in
(welt-)gesellschaftlichen Verhältnissen entfaltet:\footnotemark \footnotetext{
    Dussel „Transmodernity and Interculturality. An Interpretation from the
    Philosophy of Liberation,“ \textit{Transmodernity: Journal of Peripheral Cultural
    Production of the Luso-Hispanic World}, Vol. 1 Nr. 3, (2012).}
\\

„Was ist eigentlich Lateinamerika?“ Mit dieser Frage wird Dussel bei seiner
Ankunft in Europa konfrontiert und sie bestimmt seine ersten Texte, die er
während seiner Zeit als Student\_* an europäischen Universitäten publiziert.
Dabei wird ihm die Frage nicht etwa von außen gestellt, viel eher beginnt seine
Auseinandersetzung mit der Geschichte Lateinamerikas auf Grund der
Selbsterkenntnis bzw. Fremdzuschreibung, dass er, entgegen seiner bisherigen
Annahmen, selbst kein Europäer\_* sei:

\begin{myenv}
    \textit{ „With my trip to Europe – in my case, crossing the Atlantic by ship
    in 1957 – we discovered ourselves to be “Latin Americans,” or at least no
    longer “Europeans,” from the moment that we disembarked in Lisbon or
    Barcelona. The differences were obvious and could not be concealed.
    Consequently, the problem of culture—humanistically, philosophically, and
    existencially—was an obsession for me: “Who are we culturally? What is our
    historical identity?” This was not a question of the possibility of
    describing this “identity” objectively; it was something prior. It was the
    existential anguish of knowing oneself.“\footnotemark \footnotetext{Dussel,
  „Transmodernity and Interculturality,“ 28.}}
\end{myenv}

Dussels autobiographischer Einstieg kann an dieser Stelle bereits als Praxis der
Vorortung gelesen werden. Im Mittelpunkt steht dabei eine Reflexion über die
Wissensbestände, die zu Beginn seiner intellektuellen Auseinandersetzung als
legitimes Wissen an ihn herangetragen wurden und entsprechend sein
Selbstverständnis prägten. \\
Dass die Legitimierung von Wissensbeständen immer
auch mit der Delegitimierung anderer Wissensbestände einhergeht, wird deutlich,
wenn er\_* wie eben zitiert, über die Bedeutung indigenen Wissens während seines
Studiums spricht.

Sein Interesse, den lateinamerikanischen Kontinent bzw. dessen Kultur zu
verorten setzte er in den folgenden Jahren mit dem Versuch fort, die
„historische Identität“\footnotemark \footnotetext{Ebd., 29.} Lateinamerikas innerhalb der Weltgeschichte zu
rekonstruieren. Sein zu Beginn statisch und an national-staatlichen Kategorien
geprägtes Verständnis von Kultur wird dabei schon nach kurzer Zeit verworfen und
weicht einem Verständnis von Kultur, das sich durch Hybridität und stetige
Veränderung auszeichnet. Die Schwierigkeit, die lateinamerikanische Kultur zu
beschreiben darf jedoch, so schreibt er nach seiner Rückkehr nach Lateinamerika,
durch ein essentialismus-kritischen Ansatz nicht in der Leugnung ihrer Existenz
münden. Viel eher muss die Existenz gegenüber der Ignoranz eurozentristischer
Philosophien verteidigt werden.\\
 Der Austausch zwischen Theoretiker\_innen aus
Asien, Afrika und Lateinamerika, den er in den folgenden Jahren etabliert,
gleicht dementsprechend einer Suche nach einem kulturellen Selbstverständnis
jener Teile der Welt, die in der Sprache des hegemonialen Europas als Peripherie
bezeichnet werden. Statt einer Ergänzung der hegemonialen Geschichtsschreibung
durch eine Rekonstruktion der lateinamerikanischen Geschichte, forderte Dussel
eine Rekonstruktion und damit einhergehende Neuschreibung der \textit{mythical
narratives} der Europäischen Moderne und setzte damit in seiner Analyse bei den
Auslassungen und Verzerrungen jener westeuropäischen Philosophien an, die im
klassischen Verständnis als allumfassend und universell gültig dargestellt wird
und auf die ich in dieser Arbeit bereits ausführlich eingegangen
bin.\footnotemark \footnotetext{Ebd., 30.}
\\

Nun reicht es nicht aus, und dies kann als Dussels zentrale Kritik an vielen
postmodernen Theoretiker\_innen verstanden werden, sich von dieser Meta-Erzählung
abzuwenden um sich voll und ganz dem Partikularen zu widmen. Die Gefahr, die von
der Europäischen Moderne und ihrer Wissensproduktion ausgeht, liegt Dussel nach,
nicht in dem Versuch globale Zusammenhänge in Form einer Metaerzählung
aufzuzeigen und zu theoretisieren. Viel eher muss die Logik von Linearität und
Fortschritt, mit Europa an dessen Spitze aufgebrochen werden um eine radikale
Neuschreibung der Geschichte zu ermöglichen.

Eine solche Neuschreibung der Geschichte muss die Setzung von Zentrum und
Peripherie aufbrechen und dadurch auch die jahrhundertelange Degradierung und
Herabsetzung der\_des Anderen durch Europa ein Ende bereiten. Der Beginn dieser
Geschichte würde nicht mit Europa als Nabel der Welt gemacht werden, sondern auf
die vielfältigen komplexen Beziehungen Bezug nehmen, die die Selbst(er)findung
von Europa ermöglichten:

\begin{myenv}
    \textit{„Transmodernity displaces the linear and geographically enclosed timeline of
    Europe’s myth of autogenesis with a planetary spatialization that includes
    principal players from all parts of the globe.“\footnotemark}
    \footnotetext{Alcoff, „Dussels Transmodernism“, 63.} 
\end{myenv}

Die Idee der Transmoderne kann entsprechend als eine Erzählung verstanden
werden, die die Europäische Moderne nicht als Ausgang oder Maßstab nimmt,
sondern viel eher als Bestandteil eines komplexen Gefüges, partikularer
Geschichten versteht, die in permanenter Aushandlung miteinander entstanden sind
und fortgeschrieben werden. Die Unterteilung in Zentrum und Peripherie wird
damit obsolet, viel eher versteht sich die Transmoderne als ein Raum in dem
multiple Modernen nebeneinander in Solidarität existieren können, ohne dass sie
durch eine einzige Erzählung vereinnahmt, bzw. ihr unter oder übergeordnet
werden.\footnotemark \footnotetext{Ebd.}
\\

Die Transmoderne ist damit nicht Endzustand, sondern viel eher ein Prozess, der
auf einem egalitären, sogenannten \textit{transversal} Dialog aufbaut bzw. diesen
ermöglicht. Denn um an dem Dialog als der multiplen Modernen teilnehmen zu
können, muss sich die Europäische Moderne als partikular begreifen lernen und
damit auch die Geschichte, die sie bis dahin an der Spitze des Fortschritts und
der Rationalität vermutete umschreiben:

\begin{myenv} 
    \textit{„If the modern understands itself, as it so often does, as the
    unique moment of self-conscious reflexivity, with epistemic rigor and a
    capacity to escape conventions of doxa from pre-rational eras, it is not
clear how to achieve a meaningful solidarity.“\footnotemark \footnotetext{Ebd., 64.}}
\end{myenv}

Dussel macht hierbei auch darauf aufmerksam, dass es nicht darum gehen sollte
einen neuen Nullpunkt zu finden, von dem aus alles von Neuem beginnen kann und
der einen egalitären Dialog unterschiedlicher Wissensbestände ermöglichen würde.
Das koloniale Verhältnis wirkt fort und macht ein symmetrisches Verhältnis des
Dialoges unmöglich. Für Dussel besteht die Radikalität eher darin, die Sphäre zu
verlassen, in der eine spezifische Form moderner Rationalität die
Legitimitätsgrundlage für Wissen darstellt. Er fordert die Regeln darüber zu
verändern, was als legitimes Wissen gilt bzw. wer darüber entscheiden kann.

Hierzu stellt Dussel drei Kriterien auf: \\
So soll erstens Ort (\textit{Location}) von dem
aus gesprochen wird maßgebend für den Zugang zur Legitimität darstellen. Ideen,
die ungebunden ihres Entstehungskontextes postuliert werden, werden hiermit
disqualifiziert.\\
 Zweitens weist Dussel darauf hin, dass kritisches Denken gerade
in den marginalisierten Räumen gesucht werden müsse, und stellt in diesem
Zusammenhang sein Verständnis von Exteriorität vor. \\
Zuletzt, so fordert er, muss
der Anspruch auf Autonomie, und damit die Möglichkeit sich frei für und gegen
Dialoge zu entscheiden, anerkannt werden.

Der Diaolog wird damit zur konstitutiven Praxis einer dezentrierten
Epistemologie, die es den Teilnehmer\_innen ermöglicht durch wechselseitige
Bezüge eigene Perspektiven neu zu denken ohne sich an nur einem, vermeintlich
universellen Maßstab, zu messen.

\subsubsection{Verortung}

Die Theoretisierung von Kultur wird vor dem Hintergrund der machtvollen
epistemischen Hierarchien, die mit der Kolonialisierung einhergehen, zu einem
Projekt in dem die ökonomische Dimension mitgedacht und Kulturen immer auch als
Ausdruck eines kolonialen Verhältnisses verstanden werden, das von Dominanz und
Ausbeutung geprägt ist.\footnotemark \footnotetext{ Dussel, „Transmodernity and
Interculturality,“ 32.} Eine Analyse der Europäischen Moderne bzw. ihre
Gegengeschichtsschreibung muss entsprechend immer in ein Dialog eingebettet
sein, der Wissensbestände als spezifische Ausdrücke ihrer
historisch-geografischen Verortung in global-politischen Verhältnissen versteht.

Dussel verdeutlicht an dieser Stelle, dass sich das Denken von Zentrum und
Peripherie mit der gewaltvollen Eroberung durch die Kolonialmächte in das Denken
der Kolonialisierten eingeschrieben hat. Der Dialog, den er zwischen den
Theoretikern aus Asien, Afrika und Lateinamerika praktiziert, ist entsprechend
auch geprägt von neokolonialen Verbindungen. \\
So unterscheidet Dussel zwischen
jenen, die sich mit den Eliten des Westens verbinden, deren Perspektiven
verinnerlichen und somit koloniale Denkmuster fortschreiben und jenen, die auf
ein ursprüngliches, vorkoloniales Außen rekurrieren. Hier wird das
anti-essentialistische Verständnis von Verortung deutlich. Eine Position im
postkolonialen Kontext\footnotemark \footnotetext{Mit postkolonialem Kontext sind an dieser Stelle jene Territorien gemeint, die unter kolonialer Herrschaft waren oder sind. In theoretischen Auseinandersetzungen mit Kolonialismus und Postkolonialismus wird jedoch darauf hingewiesen, dass insbesondere auch die Kolonialmächte und ihre Gesellschaften als postkoloniale Kontexte gedacht werden müssen, da sich das koloniale Projekt hier ebenso in das Selbstverständnis eingeschrieben hat.} ist demnach nicht per se von hegemonialen Diskursen
verschont, sondern reproduziert diese unter Umständen bewusst oder unterbewusst.
\\

An dieser Stelle fragt sich Dussel, wie eine dezentrale, plurale
Erkenntnisproduktion aussehen kann, die sich nicht in einem puren Relativismus
verliert. Denn globale Strukturen der Ausbeutung und des Widerstandes, wie sie
durch den Kolonialismus entstanden sind, verlangen, so Dussel auch eine Analyse
und Kritik auf globalem Level. \\
Wie kann nun vor dem Hintergrund dieser
Überlegungen ein pluraler Dialog entstehen, der nicht von universellen,
normativen Kriterien bestimmt ist, die eben gerade der Idee des Dialoges
entgegenstehen würden?

Hier stellt sich die Frage, wer am Diskurs auf welche Weise teilnehmen kann bzw.
gehört werden kann und wie Kriterien aufgestellt werden können, die einen
offenen und damit für alle zugänglichen Diskurs ermöglichen.
\\

Die Forderung nach Verortung und damit nach kontextgebundener Wissensproduktion
darf hier  jedoch, so Alcoff in Anlehnung an Dussel nicht die Notwendigkeit
ersetzen, globale Zusammenhänge zu benennen und zu analysieren. Der Fetisch des
Lokalen, verkennt, dass das Lokale immer erst durch dominante Logiken hindurch
intelligibel wird:

\begin{myenv}
    \textit{„When we make cultures or knowledges irreducibly local, we truly
    risk ahistorical reifications. We risk losing the sight of how our
    represenstations of the local practices or knowledges my be constituted
    through imperial sign systems, or in other words, mistaking the local as a
    solipsistic spontaneous emergence, rather than implicated – at least in its
    representations and how it is understood- within a larger colonial
    semiosis.\footnotemark \footnotetext{ Alcoff, „Dussels Transmodernism,“ 65.}}
\end{myenv}

Das Lokale ist demnach immer bereits schon ein Konstrukt einer höheren, unter
Umständen hegemonialen Ordnung, in der das Lokale aufgeht. Zugleich verweist
Alcoff auf die Möglichkeit, dass es auch außerhalb der Ebene der Repräsentation
lokales Wissen bzw. lokale Kulturen gibt und verweist damit bereits auf die Idee
der Exteriorität.

Einem Relativismus, in dem das Lokale damit auf das Lokale reduziert wird, kann
nur mit einer 'provisorischen Metaerzählung der globalen Geschichte' begegnet
werden, die die wechselseitigen Abhängigkeiten und Bezugnahmen anerkennt und
damit auch Referenzpunkte und Rahmensetzungen für lokale Narrative bietet, die
ihre Einbettung in größere Zusammenhänge ermöglicht.\\
 Relationalität wird damit
als notwendiger Bestandteil von einer Idee der Vernunft konzipiert, und grenzt
sich von einem transzendenten Vernunftbegriff ab. Nur so kann die konstitutive
Bedeutung des Kolonialismus im Zusammenspiel mit und Aufeinanderwirken von
verschiedenen Akteur\_innen berücksichtigt werden ohne dabei Europa und die
Europäische Moderne als zentralen Bezugspunkt für die gesamte Geschichte in den
Vordergrund zu rücken.
\\

 Alcoff weist an dieser Stelle darauf hin, dass ein
reflexives Bewusstsein einen Prozess beinhaltet, der prinzipiell jedem offen
steht, ganz egal von wo aus er\_sie spricht. Das gleiche gelte jedoch auch für
blinden Dogmatismus und vorsätzliche Ignoranz, von der niemand per se geschützt
sei.\footnotemark \footnotetext{Ebd.}

\subsubsection{Exteriorität}

Der Versuch Lateinamerika in globalen Zusammenhängen zu situieren wurde nach und
nach durch eine Kritik an der „standart vision“\footnotemark \footnotetext{Dussel, „Transmodernity and Interculturality,“ 36.} dieser Globalgeschichte
abgelöst. Nicht nur die Darstellung und damit immer auch Herstellung der
vermeintlichen Peripherie, sondern auch die Selbstverortung- und Beschreibung
Europas wird als verzerrt entlarvt (vgl. Solipsismus).\footnotemark
\footnotetext{Ebd.}\\
 Eine postmoderne Analyse
dieser Moderne kann darum, so argumentiert Dussel, nicht aus diesem
vermeintlichen Zentrum geschehen, sondern muss aus der Exteriorität heraus
praktiziert werden. Entgegen der vielfachen Annahmen, dass die durch
industrielle Revolution, koloniale Eroberung und hegemonialen Stellung Europas
im global- politischen Kontext, sich auch dessen Kultur in imperialer Praxis in alle Sphären einschreiben konnte, stellt Dussel die These der Exteriorität:
\begin{myenv} 
    \textit{„These cultures have been partly colonized, but most of the
    structure of their values has been excluded—disdained, negated and ignored—
    rather than annihilated. The economic and political system has been
    dominated in order to exert colonial power and to accumulate massive riches,
    but those cultures were deemed to be unworthy, insignificant, unimportant,
    and useless. This disdain, however, has allowed them to survive in silence,
in the shadows, simultaneously scorned by their own modernized and westernized
elites.“\footnotemark \footnotetext{Ebd., 42.}} \end{myenv}

Für Dussel ist durch das Selbst- und Weltverständnis der Europäischen Moderne,
das alle anderen kulturellen Praxen und Ausdrucksformen als Minderwertig
begreift, eine Situation entstanden in der es manchen 'Kulturen' gelingen konnte
sich subversiv und im Schatten zu behaupten. Dussel sieht in dieser Exteriorität
das Potential, auf Basis einer völlig unvergleichlichen Erfahrung,
Lösungsvorschläge zu entwickeln, die für die Europäische Moderne niemals
erreicht werden können. Eine transmoderne Kultur sieht entsprechend auch vor,
die Weltgeschichte neu zu schreiben und dabei die Perspektiven und kulturellen
Entwicklungen der Exteriorität zu berücksichtigen:

\begin{myenv} \textit{„'Trans-modernity' points towards all of those aspects
    that are situated 'beyond' (and also 'prior to') the structures valorized by
    modern European/North American culture, and which are present in the great
    non-European cultures and have begun to move toward a pluriversal
utopia.“\footnotemark \footnotetext{Ebd., 43.}} \end{myenv}

Statt von einer in sich isolierten Exteriorität auszugehen, plädiert Dussel für
die Zwischenräume, die Grenzsubjekte, die zwischen dem hegemonialen Zentrum und
der Exteriorität übersetzen können und sich jeweils das Beste, was diese zu
bieten haben für sich nutzen lernen. Kritisches Denken entspringt demnach in dem
Vertrauen, sich auf die eigene Kultur berufen zu können und der gleichzeitigen
Offenheit, die intellektuellen Errungenschaften anderer Kulturen, die für das
eigene Vorhaben Nützlichkeit versprechen, sehen und anwenden zu
können.\footnotemark \footnotetext{Ebd., 47.}\\
 Kultur verstehe ich hierbei immer
in dem zuvor dargelegten Verständnis von Hybridität, und nicht als monolithisch.
Die Kontingenz die in kulturellen Selbstverständnissen enthalten ist, wird mit
dem Konzept des Dialoges aufgegriffen, da hier eine stetige Veränderung des
Selbst- und Weltverhältisses zur Praxis wird.

Dussel weist an dieser Stelle darauf hin, dass sein Verständnis von Dialog die
Überschreitung der eigenen Perspektive zum Ziel hat, ohne jedoch darin die
Bedeutung der eigenen Wertigkeit zu verkennen. Viel eher gilt letzteres als
Voraussetzung um sich mit dem eigenen und dem anderen kritisch
auseinanderzusetzen, ohne dabei zu schnellen Schlussfolgerungen zu kommen, die
eines von beiden obsolet erscheinen lassen:
\begin{myenv}
    \textit{„It is not a dialogue among those who merely defend their culture
    from its enemies, but rather among those who recreate it, departing from the
    critical assumptions found in their own cultural tradition and in that of
    globalizing Modernity.“\footnotemark \footnotetext{Ebd., 48.}}
\end{myenv}

\subsubsection{Autonomie}

Inwiefern Dussel einen Dialog zwischen hegemonialem Zentrum und der Exteriorität
für möglich hält, bzw. ob er diesen überhaupt befürwortet kann ich seinen
folgenden Ausführungen nicht entnehmen. \\
Zum einen schildert er die
Notwendigkeit, das asymmetrische Verhältnis das zwischen 'dem Westen' und 'dem
Außen' liegt, zu benennen. Eine Asymmetrie, die jedoch aus einer radikalen
eurozentristischen Perspektive nicht erkannt werden kann, weil die Anerkennung
der zu Anderen gemachten, als Subjekte, nicht gegeben ist. In ihrer
eurozentristischen Sprache,  „[...] there can be no cultural dialogue with
China, India, the Islamic world, Mexico, etc., because they are neither
enlightened nor primitive cultures. They are 'no man's land'.”\footnotemark
\footnotetext{Ebd., 40.} Zum anderen argumentiert er, dass selbst wenn die
Asymmetrie anerkannt wird und damit auch einer Anerkennung der Anderen als
Subjekte möglich wird, dies nicht bedeutet, dass die Strukturen überwunden
werden können, die dieser Asymmetrie unterliegen.\\
 Solange die Bedingungen des
Dialoges vom hegemonialen Zentrum aus gesetzt werden, wird, so Dussel sein
Ausgang immer auch jenen dienen, die ihre ökonomischen Interessen unter dem
Vorwand kultureller Verständigung durchsetzen wollen.\footnotemark
\footnotetext{Ebd.} Letztlich scheint es, dass Dussel Verständnis von Dialog die
Sprecher\_innen des Westens erst einmal nicht umfasst. Viel eher geht es ihm
darum, zunächst einen Dialog innerhalb der Exteriorität bzw. ihrer Grenzen zu
schaffen, um im Anschluss hieraus einen Dialog mit der Europäischen Moderne
einzugehen:
\begin{myenv}
    \textit{But, additionally, this is not even the dialogue between the critics of the metropolitan “core” and the critics of the cultural “periphery.” It is more than anything a dialogue between the “critics of the periphery,” it must be an intercultural South-South dialogue before can become a South-North dialogue.“\footnotemark \footnotetext{Ebd., 48.}}
\end{myenv}

Dieser Dialog, der zunächst in der Exteriorität stattfindet, ist damit weder
Teil der Moderne noch der Postmoderne. Viel eher einspringt er aus der
Exteriorität, dem Außen der Moderne, bzw. ihrer Grenzgebiete. Ein Ort, der sich
nicht durch die Negativität zur Moderne auszeichnet, sondern in seiner
Differenz: „This exteriority is not pure negativity. It is the positivity rooted
in a tradition distinct from the Modern.“\footnotemark \footnotetext{Ebd., 50.}
\\

Dussels Positionen verknüpfen die verschiedenen Standpunkte und Strategien mit
denen ich mich in dieser Arbeit beschäftigt habe auf interessante Weise. \\
Die
Auseinandersetzung mit der hegemonialen Episteme und die Kritik an dem
solipsistischen und dualistischen Methoden, auf die meine Arbeit aufbaut, ist
maßgeblich von ihm bestimmt. Alcoff und Grosfoguel beziehen sich in ihren
Analysen vielfach auf Dussel und es ist darum nicht außergewöhnlich, dass sein
Konzept der Transmoderne so nahtlos an die Kritik an der hegemonialen Episteme
anschließt.\\
 Der Weg, den ich in dieser Arbeit in der Zwischenzeit gegangen bin
hat jedoch epistemische Gemeinschaften und Projekte miteinbezogen, die nicht
unmittelbar mit Dussels Positionen in Verbindung stehen. Wie lässt sich also die
Auseinandersetzung mit dem Erfahrungsbegriff in der feministischen Epistemologie
oder die Strategie der Ich-Erzählung als postkoloniale Intervention mit Dussels
Forderungen nach einer Transmoderne verbinden?

Dussels Forderungen nach Verortung, Exteriorität und Autonomie, lassen sich, so
meine Argumentation, erst durch eine Reflexion über Erfahrung nachvollziehen.
Keines dieser Konzepte wird greifbar, wenn ihm nicht ein Konzept der Erfahrung
als Erfahrung von Differenz zu Grunde gelegt wird.

Verortung kann auf diese Weise als Ort in der Geschichte von Erfahrungen
nachvollzogen werden die Menschen aus einer bestimmen Position heraus machen
oder gemacht haben und geht damit weit über einen Punkt auf der Karte hinaus.\\

Kultur als Ausdruck kolonialer Verhältnisse zu verstehen wird konkret, wenn dem
ein Verständnis von Erfahrungsgemeinschaften zu Grunde liegt, die ihre
erkenntnistheoretische Position aus den Erfahrungen von Widerstand kolonialer
Gewalt entwickeln oder wie in der hegemonialen Episteme der Erfahrung von
Herrschaft. Erst durch Erfahrung wird Erinnerung möglich und Verortung damit zu
einer immer auch biografisch situierten Praxis, die die Subjekte an die
Geschichte bindet. \\
Für Dussel geht mit der Verortung in einer ehemals
kolonisierten Gesellschaft nicht an sich schon eine emanzipatorische Position
einher. Diskursangebote spielen für ihn eine zentrale Rolle in der Formung eines
Standortes. Neokoloniale Perspektiven greifen hier in Kontexte ein und werden
von Subjekten aufgenommen die durch koloniale Lesarten hegemoniale Logiken
fortschreiben. \\
Dussel vertritt hier die Ansicht, dass die Erfahrung von
Unterdrückung und Widerstand nicht per se eine emanzipatorische Position
schafft. Hegemoniales Wissen und damit einhergehende Perspektiven können also
auch Erfahrungen vereinnahmen und für die Stabilisierung von Herrschaft
beanspruchen. Verortung unterscheidet sich für ihn von dem Konzept der
Lokalität, weil es auf eine transmoderne Ordnung angewiesen ist, die Erfahrungen
in global-politischen Verhältnisse situiert. Eine situierte Praxis ist also
gleichermaßen auf Erfahrung, also auch auf ein transmodernes Verständnis von
weltgesellschaftlichen Verhältnissen angewiesen die jene Erfahrungen in einer
Geschichte verorten, die nicht von Europa aus gedacht wird.
\\

Exteriorität verstehe ich in diesem Sinne auch als symbolischen Ort, an dem
Erfahrungen die epistemische Herrschaft hegemonialen Wissens \textit{über}leben und sich
so zu Erinnerungen formen können die sich der Kolonisierung widersetzt haben.
Ich-Erzählungen verknüpfen diese Erinnerungen mit gegenwärtigen Erfahrungen und
stellen so eine Verbindung mit Wissensarchiven her, die die koloniale Herrschaft
überlebt haben. Auch für sie bedarf es aber einen Erfahrungsbegriff, der sich
wie in dieser Arbeit herausgearbeitet als Zwischenraum zwischen Subjekt und
Diskurs denken lässt und der entsprechend nicht dermaßen von der gewaltvollen
epistemischen Macht hegemonial gewordenen Wissens bestimmt ist. 

Die Exteriorität
wird damit zu einem Erfahrungsraum, der das politische Moment aus der Geschichte
heraus entfaltet und nicht auf Interpretationsangebote poststrukturalistischer
Theoretiker\_innen angewiesen ist. Zugleich plädiert Dussel aber auch für
Grenzsubjekte, die als Übersetzer\_innen fungieren und so Verbindungen zu anderen
Wissensgemeinschaften aufbauen. Die Kritik an neokolonialen Abhängigkeiten
scheint also Verbindungen nicht per se auszuschließen solange sie fruchtbar für
die Auseinandersetzung mit dem bisher erfahrenen sind, und letztere nicht
negieren.  
\\
Das Recht auf Autonomie wird durch einen Blick auf Erfahrung als intimes Erleben
und Verarbeiten nachvollziehbar. Autonomie stellt es Subjekten frei, ihre
Erfahrungen zu teilen oder für sich zu behalten und die Räume selbst zu wählen,
in denen sie ihre Erfahrungen teilen möchten und können. Autonomie versteht
Dussel jedoch nicht nur als emanzipatorische Praxis, sondern auch als notwendige
Konsequenz aus der Missachtung, mit der die europäische Moderne anderen
Wissensgemeinschaften gegenübertritt. Die Vormachtstellung der hegemonialen
Episteme ist also, so lässt sich Dussel hier lesen, immer auch mit einer
Selbstisolation verbunden, die wiederum, so Dussel, für andere Wissensgemeinschaften einen Raum der Autonomie öffnet.

Wissen als Wissen aus Erfahrung wird hier mit Dussels Forderung nach Autonomie zu etwas, über dessen Grenzen das Subjekt selbst wacht.
\\
Die Auseinandersetzung mit Erfahrung bzw. Ich-Erzählungen als widerständige
Praxen lassen sich mit Dussels Standpunkten sinngebend verknüpfen. Das Konzept
der Transmoderne weist umgekehrt aber auch auf die Leerstellen und Grenzen der
feministischen Auseinandersetzung mit Erfahrung hin. So zeigt Dussel auf, dass
Dialoge nicht, wie bei Code angeklungen, einfach zwischen verschiedenen
Wissensgemeinschaften bzw. Wissensformen praktiziert werden können. Viel eher
muss die Geschichte des Kolonialismus als konstitutiver Faktor in
Wissensgeschichten berücksichtigt werden um die Grenzen, die dies für einen
Dialog aus unterschiedlichen Positionen bedeutet, anerkennen zu können. Es gibt
keinen Nullpunkt, von dem aus Subjekte sich auf Augenhöhe begegnen können. Der
Kolonialismus hat Spuren hinterlassen und diese Spuren führen nicht notwendig
zueinander, sondern auch voneinander weg. Auch die Ich-Erzählungen die als
postkoloniale Intervention fungieren sollen, müssen, Dussel folgend, achtsam
sein, da die Berufung auf die Partikularität immer nur angesichts des
hegemonialen Zentrums funktioniert und so Logiken fortschreibt, die in der
solipsistischen und dualistischen Struktur angelegt sind.

%\section{Resumé}

Meine Arbeit beginnt mit der Beschreibung einer Idee von Universität. Diese Idee
begreift die Aufgabe von Bildung und Kritik an der Universität in der
Überschreitung epistemischer Grenzen. Dabei dient die Überschreitung, im Sinne
der Herrschaftskritik, zwar letztlich einer Befreiung des Subjekts, versetzt es
jedoch auf diesem Weg zugleich in eine Krise, die ihm den Schein der Kohärenz
nimmt. Während sich die von mir diskutierten Autor\_innen einig in ihrer
poststrukturalistischen Grundausrichtung sind und somit eine Affinität für das
krisenhafte kontingenter Strukturen teilen, erweisen sich ihre Interpretationen
dieser Grenzerfahrungen des Selbst jedoch als äußerst unterschiedlich.
Bildungsprozesse werde hier als Verunsicherung und völligen Verlust des Selbst
über Ermächtigungsprozesse für ein freieres, würdevolleres Selbst bis hin zu
Prozessen der Wiederaneignung von Subjektpositionen aus silenced Positionen,
sehr unterschiedlich konzipiert. An dieser Stelle führe ich erstmals in den
Erfahrungsbegriff – hier als Krisen- und Grenzerfahrung des Subjekts markiert –
ein. Erfahrung, das wird an dieser Stelle bereits deutlich, vermag den
Zwischenraum zu beschreiben, der zwischen Subjekt und Diskurs in eben jenen
Praxen des 'Sichbildens' und 'Kritikübens' entsteht. Mein Interesse an Praxen
der Bildung und Kritik, kristallisiert sich nahezu in der Auseinandersetzung mit
Erfahrung, da Erkenntnisse einerseits aus Erfahrung, insbesondere
Krisenerfahrungen hervorgehen – andererseits werden durch Erkenntnisse, im Sinne
eines veränderten Blick auf das Selbst in Verhältnissen auch neue Erfahrungen
möglich. Erfahrungen sind damit zugleich Voraussetzung für und Folge von Kritik. 

Diesem anfänglichen Interesse an Erfahrung folgt nun der Versuch einer ersten
Kontextualisierung von Erfahrung in Wissensproduktionen. Die Verhältnisse, in
denen Erfahrungen an der Universität gemacht und geteilt werden können, geraten
hierbei in den Vordergrund. Ich stelle Fragen nach dem Umgang mit Erfahrung im
universitären Kontext, also wie Erfahrung im Ringen um Erkenntnis genutzt wird
und welche Effekte dies für unterschiedlich positionierte Subjekte und ihre
Erkenntnisse hat. Dabei wird sichtbar, dass die Auseinandersetzung mit Erfahrung
in Räumen stattfindet, in denen die Fäden kolonialer Gewalt nach wie vor
gespannt sind. Hier wird legitimes von illegitimem Wissen getrennt und die
historisch gewachsenen Machtverhältnisse greifen in die Wissensordnungen, die
darüber bestimmen, was von wem sagbar ist, ein. Mein Fokus liegt dabei auf
rassistischen und sexistischen Normen und ihren Effekten für unterschiedlich
positionierte Subjekte. Die gegebenen Verhältnisse müssen ergo in zweierlei
Hinsicht problematisiert werden: Zum einen, weil sie reglementierend, also
einschneidend auf die Artikulation von Erfahrung der Subjekte wirken und somit
Subjekten in unterschiedlichem Maße Artikulationsräume und damit auch Räume der
Bildung und Kritik verschaffen. Zum anderen, weil dieser Prozess der
Beschneidung, auf die Ordnung zurückgeworfen, sie in ihrer Allgemeingültigkeit
in Frage stellt, da sie eben nicht für alle gleichermaßen gültig ist. Der
anfangs beschriebene Bildungs- und Kritikbegriff wird hier insofern korrigiert,
als nicht mehr von allgemeinen Krisenerfahrungen ausgegangen werden kann,
sondern viel eher zu vermuten ist, dass sich die Auslöser und Effekte von Krisen
je nach Passungsverhältnis der Subjekte zu der Wissensordnung sehr
unterschiedlich gestalten.

Das Eingreifen der Wissensordnung ist jedoch kein einseitiger Prozess, der
ausschließlich reglementierend auf die Subjekte und die Möglichkeit der
Artikulation ihrer Erfahrungen wirkt. Viel eher scheint in der
Auseinandersetzung mit Erfahrung die Möglichkeit einer Rückwirkung auf diese
Ordnung zu bestehen. Das sprechen aus und über Erfahrung wird also nicht nur
durch jene historisch gewachsene Wissensordnung bestimmt, es greift
gleichermaßen in sie ein und birgt damit das Potential von Veränderung. Die
Veränderung, oder auch Überschreitung der Ordnung, die das legitime von
illegitimem Wissen unterscheidet, kann im Anschluss an postkoloniale
Theoretiker\_innen als Auftrag dekolonialer Bildung verstanden werden. Hieraus
ergibt sich auch die zentrale Fragestellung: Welche Bedeutung kann Erfahrung für
Praktiken des Gegenerzählens an der Universität einnehmen, um im Sinne einer
dekolonialen Bildung das Denkbare zu überschreiten?\\

\textbf{\large Hegemoniale Episteme}\\

Um besser verstehen zu können, wie jene Wissensordnung, die Allgemeingültigkeit
verspricht, aber nicht hält, überschritten werden kann, und welche Bedeutung
Erfahrung hierbei einnimmt, ist es zunächst jedoch Notwendig sich mit ihr
intensiver zu befassen. Im zweiten Teil meiner Arbeit beschäftige ich mich mit
der Entstehung dieser Wissensordnung, ihren Annahmen und Methoden. Auf diese
Weise möchte ich mehr darüber herausfinden, was sie so ausschließend macht und
wie sie überwunden werden kann. Meine Auseinandersetzung ist dabei von der Frage
geleitet, auf welcher Idee vom 'Menschsein' diese Wissensordnung fußt und welche
Erfahrungen vom 'Menschsein' dadurch ausgeschlossen werden. Hierbei kann ich
aufzeigen, dass die gewaltvolle Auslöschung von Wissensarchiven, die bereits mit
der Vorbereitung der kolonialen Expansion auf dem europäischen Festland begann
und im kolonialen Projekt mündete, dem bis heute vorherrschenden Wissen seine
hegemoniale Stellung verschafft hat. Die kritische Analyse der Zerstörung von
Wissenssystemen einerseits und der Durchsetzung hegemonialen Wissens
andererseits stellt dabei heraus, dass die Erfahrung kolonialer Herrschaft sich
tief in das Grundverständnis dieser Wissensordnung eingeschrieben hat, die
folglich einen hegemonialen Status beanspruchte. Dabei muss notwendigerweise von
Erfahrungen im Plural gesprochen werden, da sich die Erfahrung der Herrschaft
nur in Verbindung mit der Erfahrung des Beherrschtseins – also Erfahrungen von
Widerstand und Unterdrückung – denken lässt. Es ist jene doppelte Erfahrung, die
die hegemoniale Ordnung konstituiert und die durch ihr solipsistisches und
dualistisches Grundverständnis negiert wird. Denn dieses Grundverständnis birgt
die Vorstellung eines von seiner Umwelt isolierten Subjekts, das durch innere
Monologe, also ohne jegliche Begegnung, Erkenntnis gewinnt.

Das Bild vom
'Menschen' in dieser Wissensordnung ist das von einem 'Menschen', der denkt –
nicht erfährt – er ist völlig losgelöst von jeglicher Erfahrung und folglich
auch von denjenigen 'Menschen', mit denen er auf Grund jener Erfahrung zutiefst
verbunden ist. Mit der Erfahrung des Tötens geht immer auch die Erfahrung des
Sterbens einher. Das Selbstverständnis einer Ordnung, die zu einem Zeitpunkt
entsteht, an dem 85 Prozent der Welt unter europäisch kolonialer Herrschaft
steht, ist jedoch zutiefst durch die Erfahrung des Herrschens und
Beherrschtseins ebenso wie vom Widerstand und der Ausblendung dieses
Widerstandes bestimmt.

Wissen, das sich widerständig gegen diese Ordnung wendet, so lässt sich meine
Auseinandersetzung mit der hegemonialen Episteme resümieren, ist auf ein Bild
des 'Menschen' angewiesen, das sich über Erfahrung konstituiert. Indem es
Erfahrung als konstitutiv für sein Subjektverständnis und damit auch als
notwendigen Teil seines Erkennens von Welt begreift, wendet sich dieses Wissen
nicht nur gegen die Negation der Geschichte, sondern auch gegen die Negation der
Subjekte, deren Erfahrungen in dieser Geschichte miteinander in Verbindung
stehen.\\ 

\textbf{\large Widerständige Episteme}\\

Doch wie kann Erfahrung zum Gegenstand von Wissensproduktion gemacht werden? Um
wessen Erfahrung und welche Erfahrung muss es sich handeln? Werden durch die
Einbindung von Erfahrung automatisch Solipsismus und Dualismus überwunden?

Im
ersten Teil meiner Arbeit ist bereits deutlich geworden, dass die Thematisierung
von Erfahrung sich sowohl stabilisierend als auch irritierend auf die
hegemoniale Wissensordnung auswirken kann. Die Leerstelle, die ich in meiner
Auseinandersetzung mit hegemonialem Wissen aufzeigen konnte, kann also nicht
einfach durch die Platzierung von Erfahrung jener gefüllt werden, die in dieser
Ordnung bisher keinen Platz finden. Viel eher ist es notwendig, sich näher
anzusehen, welche Ort Erfahrung in jenen epistemischen Projekten erhält, die
einen widerständigen oder emanzipatorischen Anspruch gegenüber hegemonialem
Wissen hegen. Meine Auseinandersetzung ist entsprechend von der Frage geleitet,
wie Erfahrung hier erkenntnistheoretisch genutzt wird und welches Versprechen
davon für eine Kritik oder Überwindung der hegemonialen Wissensordnung ausgeht.
In der feministischen Epistemologie hat in den 1990er-Jahren ein breiter Diskurs
zu diesem Thema, vorwiegend in US-Amerika, stattgefunden. Die Notwendigkeit
Wissen auf Erfahrung zu gründen wird hier jedoch nicht, wie in meiner Arbeit,
als notwendige dekoloniale Strategie begriffen. Die Erfahrung von Rassismus bzw.
eine Perspektive auf Kolonialgeschichte als Erfahrungsraum, der in der
hegemonialen Wissensordnung ausgeklammert wird, erhält in dieser Diskussion (für
mich überraschend) fast gar keine Beachtung. Der Grund, warum ich mich trotzdem
mit diesem Diskurs beschäftige, liegt darin, dass hier eine Frage im Zentrum
steht, die auch für dekoloniale Wissensproduktionen zu den Kernfragen gehört:

Ist es möglich, die Erfahrung, die Subjekte innerhalb der hegemonialen
Wissensordnung machen, gegen eben jene Wissensordnung zu wenden?

Dieser Frage
liegt die Annahme zu Grunde, dass Erfahrungen nicht von den vorherrschenden
Diskursen losgelöst sind, sondern unter Umständen durch diese erst ermöglicht
werden. Wenn aber nur erfahrbar ist, was in der hegemonialen Ordnung bereits
denkbar ist, wie kann dann diese Erfahrung das Denkbare überschreiten?

Die
feministischen Antworten auf diese Frage werden von mir auf zwei Ebenen
diskutiert. Zum einen geht es um die Frage, wie sich die eigene Lebenserfahrung
und damit – in direktem Zusammenhang gedacht – die gesellschaftliche Position
als Perspektive in Wissensproduktion einschreibt. Es wird kontrovers diskutiert,
ob die Erfahrung von Marginalisierung bereits per se einen erkenntnistheoretisch
privilegierten Standpunkt eröffnet oder ob es dafür notwendig ist, jene
Erfahrung erst gegen den Strich zu lesen – sich also der hegemonialen
Interpretation zu widersetzen. Zum anderen wird Erfahrung als Gegenstand von
Wissensproduktion und hier insbesondere als Erfahrung der beforschten Subjekte
in ihrem erkenntnistheoretischen Gehalt diskutiert. An dieser Stelle geht es um
die Übersetzung von Erfahrung der Beforschten durch eine\_n Forscher\_in, auch
hier mit dem Anliegen, für Erfahrung von und aus Marginalisierung in der
Wissensproduktion einen Ort zu schaffen. Der Paternalismus, der hier schon
anmutet, und sich in dem paradoxen Anspruch der Repräsentation versteckt, wird
hier zum Thema gemacht. Eine Position als marginalisiert zu beschreiben, oder
anders formuliert, von Betroffenen auszugehen, setzt, so die
poststrukturalistische Kritik, Identitäten voraus, die tatsächlich aber durch
eben jene Zuschreibung erst hergestellt werden. Die hegemoniale Wissensordnung
kann jedoch nicht überwunden werden, wenn ihre Identitätskategorien ungefragt
übernommen werden. Subjekt- Objekt-Verhältnisse werden durch diese wohlwollenden
Gesten nicht aufgebrochen sondern zementiert. 

Phänomenologische Positionen hingegen erkennen im Rückgriff auf Erfahrung die
Möglichkeit, identitäre Kategorien zu überwinden. Erfahrung eröffnet ihrer
Ansicht nach einen Zwischenraum, der das kategoriale Raster des Körper-Geist
Dualismus irritieren, sogar aus ihm ausbrechen kann. Denn ihr Fokus richtet sich
nicht an vorhergesehene Subjektpositionen und deren zugeschriebene Erfahrungen.
In der Phänomenologie werden Erfahrungen indessen als Körperempfinden
konzeptionalisiert. Der Erkenntnisanspruch, den sie hieraus entwickelt, greift
somit das Wissensmonopol Vernunft/Ratio/Geist an. Ein essentialismuskritischer
Körperbegriff wird hier mit dem Konzept des Soma als historisch verorteter und
somit diskursiv gebundener Körper theoretisiert.

Phänomenologische Positionen hingegen erkennen im Rückgriff auf Erfahrung die
Möglichkeit, identitäre Kategorien zu überwinden. Erfahrung eröffnet ihrer
Ansicht nach einen Zwischenraum, der das kategoriale Raster des Körper-Geist
Dualismus irritieren, sogar aus ihm ausbrechen kann. Denn ihr Fokus richtet sich
nicht an vorhergesehene Subjektpositionen und deren zugeschriebene Erfahrungen.
In der Phänomenologie werden Erfahrungen indessen als Körperempfinden
konzeptionalisiert. Der Erkenntnisanspruch, den sie hieraus entwickelt, greift
somit das Wissensmonopol Vernunft/Ratio/Geist an. Ein essentialismuskritischer
Körperbegriff wird hier mit dem Konzept des Soma als historisch verorteter und
somit diskursiv gebundener Körper theoretisiert.

Die Frage nach der Bedeutung von Erfahrung für Dekolonisierung ist hier noch
lange nicht beantwortet, da der dekoloniale Bezug in der feministischen
Diskussion fehlt. Das komplexe Feld, das durch die erkenntnistheoretische
Diskussion um die Bedeutung von Erfahrung in Wissensproduktionen eröffnet wurde,
ermöglicht mir jedoch  eine Vielzahl an Anknüpfungspunkten für die dekoloniale
Forderungen einer an Befreiung orientierten Bildung an der und für die
Universität.

Dabei geht es mir nicht darum die Grenzen zu verschärfen, die phänomenologische
und poststrukturalistische Perspektiven voneinander trennen bzw. diese den
postkolonialen Theorien gegenüberzustellen. Dies ist weder analytisch sinnvoll
noch ethisch verantwortbar, da diese Strömungen in einem permanenten
Wechselspiel miteinander entstanden sind. Insbesondere in der feministischen
Auseinandersetzung, in der Erfahrung als entweder prä- oder postdiskursiv
gedacht wird, geht es implizit nicht so sehr um die Schärfung des
Erfahrungsbegriffs als um die Verteidigung spezifischer Subjektverständnisse.
Das spricht jedoch nicht dagegen, sich mit Erfahrung auseinanderzusetzen. Diese
Erkenntnis zeigt hingegen, wie notwendig die Auseinandersetzung mit dem
Erfahrungsbegriff bleibt, da sie so eng mit dem hegemonialen Subjektbegriff
verwoben ist, dass sich auch Versuche neuer Subjektkonstruktionen auf Erfahrung,
nur eben anders,  beziehen müssen.\\

\textbf{\large Erzählung}\\

Mein Interesse, die verschiedenen bisher besprochenen epistemischen Projekte
zusammenzuführen, wird zu einer Kernaufgabe des letzten Kapitels. Als Konsequenz
aus den Erkenntnissen, insbesondere der Auseinandersetzung mit der hegemonialen
Episteme und den erkenntnistheoretischen Positionen innerhalb der feministischen
Diskussion,  nehme ich hierfür eine Praxis in den Blick, die verspricht, dem
theoretischen Anspruch den ich bis dahin entwickelt habe, gerecht zu werden. Der
Reiz, sich mit dem Konzept der Erzählung zu beschäftigen, liegt nämlich darin,
dass sie zugleich einen Blick auf und Strategien in Verhältnisse bereithält. Zum
einen ermöglicht mir die Erzähltheorie, die epistemischen Praxen, seien sie
hegemonial oder widerständig, als Erzählungen zu theoretisieren, die
Erzählgemeinschaften miteinander verbinden und so geteilte und weitergegebene
Erfahrung als Motor für Kollektivierungsprozesse einbezieht. Sie ermöglicht mir
also einen Blick auf gesellschaftliche Praxen der Bedeutungskonstruktion, in der
Erfahrungen zu Erinnerungen werden und über Erzählungen Einzug in den Diskurs
erhalten. Zum anderen werden insbesondere in der postkolonialen Theorie
Erzählungen als Strategien verstanden, durch die Verhältnisse bestätigt aber
auch irritiert werden können. Hier setzte ich mich mit zwei einander auf den
ersten Blick sehr gegensätzlichen Strategien auseinander. Die Ich-Erzählung kann
gewissermaßen als Guerilla-Strategie interpretiert werden, die sich gegen die
Gegenüberstellung von Theorie und Erfahrung wendet und fordert letztere als
gleichberechtigte Wissensbestände miteinander in Dialog zu bringen. Die
Transmoderne hingegen begreift sich als Metaerzählung einer globalen Moderne, in
der verschiedene Wissens- oder Erzählgemeinschaften nebeneinander existieren.

In den Ich-Erzählungen stehen Praxen des Dialogs und der Repräsentation im
Zentrum. Die unterschiedlichen Ebenen der Abstraktion, die Erzählungen
kennzeichnen, sind auf einander angewiesen und dürfen nicht in ein
hierarchisches Verhältnis zueinander gesetzt werden, so die Forderungen. Dafür
ist es allerdings notwendig, dass diese Guerilla-Erzählungen die dominante Logik
dessen, wie Subjekte imaginiert werden, in welchem Verhältnis sie zur Geschichte
stehen, ja überhaupt woraus Wissen besteht, unterwandern. So können neue Formen
für die Auseinandersetzung mit den eigenen Erfahrungen gesucht werden, die es
ermöglichen aus einschneidenden Interpretationsrastern auszubrechen und so
andere, Subjekt-, Geschichts- und Wissensverständnisse zu entwickeln. Das
interessante an diesem Ansatz ist, dass, anders als in der feministischen
Diskussion um den erkenntnistheoretischen Gehalt von Erfahrung, nicht zwischen
emanzipatorischem versus affirmativem Wissen unterschieden wird. Nicht die
Wertung, wie kritisch sich Wissen verhält, sondern der Grad der Abstraktion von
Wissen steht hier im Vordergrund. Dabei kann nicht zwischen den
unterschiedlichen Abstraktionsniveaus hierarchisiert werden. Die Kunst liegt
indessen darin, jene verschiedenen Sprach- und Erzählweisen in einen Dialog zu
bringen um so abstrakte Vorstellung mit konkreten Erfahrungen in Widerstreit zu
setzen. Weder das eine noch das andere darf hierbei leitend werden, viel eher
geht es um einen Austausch, in dem nicht nur verschiedene Perspektiven sondern
auch verschiedene Erzählpraxen sich miteinander verbinden.

Die transmoderne Erzählung ist nicht weniger radikal, setzt jedoch an einem
anderen Punkt an. Anstatt die Erzählung der Europäische Moderne von unten zu
unterwandern, stellt die Transmoderne die Europäische Moderne in den Schatten
einer globalen Moderne und situiert sie auf diese Weise als partikulare
Geschichte. Das Fundament, auf dem die hegemoniale Episteme ruht, wird so
erschüttert, da in der Transmoderne verschiedene Modernen nebeneinander
existieren und so die Einteilung in Zentrum und Peripherie obsolet werden. Ob es
möglich ist, unter diesen neuen Verhältnissen Bündnisse einzugehen und Wissen
auszutauschen ist jedoch davon abhängig, ob eurozentristische Logiken überwunden
werden und so Verortung, Exteriorität und Autonomie, als Schlüsselkonzepte der
Transmoderne, Einzug erhalten. Das Konzept der Transmoderne bremst die Hoffnung,
die mit einem Dialog verschiedener Erzählungen einhergehen könnte, wieder aus,
da ein Dialog unterschiedlicher Erzählgemeinschaften bzw. Erzählformen nicht
unter herrschenden Bedingungen stattfinden kann. Solange die hegemoniale
Episteme ihren Anspruch bewahrt, Zentrum allen Wissens und Denkens zu sein,
verschließt sie sich anderen Perspektiven oder eignet sie sich an. Die neue,
transmoderne Ordnung entsteht demnach nicht durch einen Dialog, sondern dadurch,
dass Räume der Autonomie in der Exteriorität erschlossen werden. Die
Exteriorität wird hier als Ort gedacht, der in der hegemonialen Sprache als
Außen markiert wird und ihren Wirkungsweisen darum nicht dermaßen ausgesetzt
ist. Hier, in dieser Abgeschiedenheit liegt entsprechend auch die Hoffnung für
eine Neuordnung, in der verschiedene Wissensformen in Solidarität miteinander
existieren können. Erst wenn sich die Exteriorität ihre gleichwertige Position
im Diskurs erkämpfen und die Europäische Moderne ihre hegemoniale Position  als
Verlust erkennen und überwinden kann, können verschiedene Formen des Wissens und
damit Erfahrungsgemeinschaften miteinander in einen Dialog treten, in dem neue
Ordnungen von Wissen und ihren Repräsentationen entstehen. 

Ein reflexiver Umgang mit Erfahrung als konstitutiver Bestandteil jeden Wissens,
muss dabei immer Teil einer Gegenerzählung sein. Das Subjekt, das erzählt,
erzählt nicht nur über Erfahrungen, es erzählt vielmehr gegen die Negierung von
Erfahrungen, und damit gegen die Illusion, dass Wissen sich von seiner
Entstehungsgeschichte loslösen kann. Die feministische Diskussion um den
erkenntnistheoretischen Gehalt von Erfahrung wird ohne die Forderungen nach
einer Transmoderne zu einem leeren Versprechen. Eingebettet in eine zeit- und
räumlich situierte Praxis werden die Reflexionen jedoch zu einem wichtigen
Bestandteil. Es sollte also nicht darum gehen, Erfahrung als emanzipatorischen
Wunderstab oder ideologisch aufgeladenen Windbeutel zu stilisieren. Viel eher
besteht die Aufgabe darin, Verhältnisse zu schaffen, in denen Erfahrungen
artikuliert und reflektiert werden können, die sich sperrig zu der hegemonialen
Ordnung verhalten.  Dekoloniale Bildung kann somit als eine Praxis verstanden
werden, die jenen Erfahrungen einen angemessenen, das heißt, selbstbestimmtem
Ort bietet,  die sich in ihren Hin- und Wegbewegungen von der imperialen
Vernunft bis heute behaupten.



%\section{Epilog}
Was sind das für Geschichten, die wir alle brauchen, und die in
keinem Geschichtsbuch stehen? Was sind das für Geschichten, die nicht in den
Klassenzimmern, sondern in den Häusern der Schüler\_innen erzählt werden? Was
sind das für Geschichten, die sich uns erst eröffnen, wenn wir die Bücher
schließen und uns der Dichtung und dem Spiel zuwenden?

Meine Auseinandersetzungen mit den unerzählten Geschichten, den Leerstellen, die
das hegemoniale Denken an der Universität konstituieren hat mich zu den
Erfahrungen der Subjekte geführt, die in diesem hegemonialen Denken an den Rand
gedrängt werden. Dass diese Erfahrungen in der hegemonialen Sprache
unbeschreibbar bleiben und somit auch hier nicht erzählt werden  können, zeigt,
dass die Räume

\begin{myenv}
\textit{Das Schwarze Subjekt ist eine Spiegelung und, sobald es sich selbst erfährt und
es erfährt sich immerzu selbst, eine Selbst-Spiegelung; es ist eine Spiegelung
von Spiegelungen von Spiegelungen. Es kommen Sätze vor, Erinnerungen,
vergebliche Versuche, es kommt eine Bitterkeit vor, die nur aus der Erfahrung
rassistischer Herabwürdigung resultieren kann, deren Subtilität die Sprachen
verschlägt, deren Unverfrorenheit Wut gefriert. Es kommen Sätze über Sätze,
Erinnerungen über Erinnerungen, Sprachlosigkeiten über Sprachlosigkeiten vor,
die sich zu einem Leben aufschichten, das an den Beulen, die es sich in der
Praxis der Gebundenheit zufügt, Befreiung ebenso wie Resignation erfährt, das
gewusst wird, das unerreichbar ist, das sich dem Lebenden entzieht, und ihn als
Schwarzes Subjekt in die Welt einführt, ihn in der Welt bestätigt.\footnotemark
\footnotetext{Mecheril, „Der doppelte Mangel, der das Schwarze Subjekt hervorbringt“ in Mythen, Masken und Subjekte, herausgegeben von Maureen M. Eggers ( unrast Verlag: Münster, 2005), 78.}}
\end{myenv}

Subjekte sind mehr und gleichzeitig weniger als die Summe ihrer Erfahrungen.
Weniger, weil den individuellen Erfahrungen stets ein eine Geschichte zu Grunde
liegt, durch die hindurch die Erfahrung sichtbar wird, erinnert werden kann, in
Sätze geformt und sich in den Körper einschreiben kann. Und sie sind mehr, weil
die Erfahrungen, als gesellschaftlich informierte Zugänge, die zwischen dem
Innen und Außen vermitteln, niemals beständig und klar, sich niemals lückenlos
und ohne Widerstände in Körper und Geist einfügen können. Sie widerfahren viel
eher einem Subjekt, das sich aneignend oder abweisend gegenüber ihnen verhält,
lernt die Erfahrung zu nutzen, sie zu transformieren. Ein Subjekt, das
Strategien entwickelt um sich die Erfahrungen zu eigen oder sie sie zu einem
anderen zu machen. Eines, das um die kontingente Lesbarkeit der eigenen
Erfahrungen weiß, das mehr ist als ihre Summe, es ist das, was zwischen Erlebtem
und Diskurs vermittelt.
\\

Diese Arbeit kann unter anderem zeigen, dass ein kritischer Erfahrungsbegriff
einen reflexiven Umgang mit erfahrungsorientiertem Sprechen ermöglichen kann in
dem ein Subjektverständnis gelebt und praktiziert wird, das sich nicht nur als
Spiegel von Diskursen und auch nicht als davon unabhängig begreift. Eines um
diese Pole weiß und das sich darum kritisch mit dem Absolutheitsanspruch der
eigenen Erfahrungen auseinandersetzen kann, ohne darin zu verfallen, die eigenen
Erfahrungen zu negieren.
\\

Ein Subjekt, das sich bejaht und zugleich aber auch Erfahrung als das erkennt,
was ihm eine partikulare Perspektive ermöglicht. So kann ein Dialog unter
Bedingungen von Differenz jenseits der Identität erprobt werden, durch den die
Erzählenden zugleich Teil der Geschichte und Möglichkeit ihrer Veränderung
werden. Veränderung schließt dabei immer auch das Selbst mit ein.

%\printbibliography

\begin{thebibliography}{100} % 100 is a random guess of the total number of
%references
  \bibitem{adichie}Adichie, \emph{Chimamanda N. The Danger of a Single Story.} TED Talk, gefilmt von TED Global, 2009. 
\bibitem{alsamarai} Al Samarai, Nicola L. „Inspired Topography: Über/Lebensräume, Heim-Suchungen und die
Verortung der Erfahrung in Schwarzen deutschen Kultur- und Wissenstraditionen.“
In \emph{Mythen, Masken Subjekte. Kritische Weißseinsforschung in Deutschland}, herausgegeben von Maureen M.Eggers, Grada Kilomba, Peggy Piesche, Susan Arndt, Münster:Unrast Verlag, 2005.
\bibitem{alcoff} Alcoff, Linda M. und Elizabeth Potter. „Introduktion: When Feminisms Intersect Epistemology. In
\emph{Feminist Epistemologies}, herausgegeben von Linda M. Alcoff und Elizabeth Potter, London: Routledge, 1993.
\bibitem{alcolo} Alcoff, Linda M. „Phenomenology, Post-structuralism, and Feminist Theory on the Concept of
Experience.“ In \emph{Feminist Phenomenology}, herausgegeben von Linda Fisher und Lester Embree, Dordrecht: Kluwer Academic Publishers, 2000.
\bibitem{alcollin} Alcoff, Linda M. „Enrique Dussels Transmodernism.“ \emph{Transmodernity, Journal of Peripheral
Cultural Production of the Luso-Hispanic World}, Vol 1, Iss. 3 (2012).
\bibitem{onami} Bar On, Bat Ami. „Marginality and Epistemic Privilege.“ In
\emph{Feminist Epistemologies},
herausgegeben von Linda M. Alcoff und Elizabeth Potter, London: Routledge, 1993.
\bibitem{bilstein} Bilstein, Johannes, und  Helga Peskoller, Herausgeber\_innen von
\emph{Erfahrung, Erfahrungen}, Wiesbaden: Chronos Verlag, [1968] 2013.
\bibitem{bodin} Bodin, Capucine, James Cohen und Ramón Grosgoguel. „Introduction: From University to
Pluriversity: A Decolonial Approach to the Present Crisis of Western
Universities.“ \emph{Human Architecture: Journal of the Sociology of Self
Knowledge} Vol.10 Issue 1 (2012).
\bibitem{bollnow} Bollnow, Otto F. „Der Erfahrungsbegriff in der Pädagogik.“ In
\emph{Erfahrung, Erfahrungen},
herausgegeben von Johannes Bilstein und Helga Peskoller, Wiesbaden: Chronos Verlag, [1968] 2013.
\bibitem{bristol} Bristol, Laurette S. M. \emph{Plantation Pedagogy. A
Postcolonial and Global Perspektive}. New York:
Peter Lang, 2012.
\bibitem{canning} Canning, Kathleen. „Feminist Discourse after the Linguistic Turn.Historicizing Discourse and
Experience.“ \emph{Signs}, Vol. 19, No. 2 (1994): 368-404
\bibitem{cannkathleen} Canning, Kathleen. „Problematische Dichotomien. Erfahrung zwischen Narrativität und
Materialität.“ In \emph{Erfahrung: Alles nur Diskurs? Zur Verwendung des
  Erfahrungsbegriffes in der Geschlechtergeschichte. Beiträge zur 11.
Schweizerischen HistorikerInnentagung}, herausgegeben von Marguérite Bos, Bettina Vincenz und Tanja Wirz, Zürich: Chronos Verlag, 2004.
\bibitem{catrov} Castro Varela, María do Mar, und Nikita Dhawan. \emph{Postkoloniale Theorie. Eine kritische
Einführung}. Bielefeld: transkript, 2015.
\bibitem{castrova} Castro Varela, María do Mar, und Nikita Dhawan. „Postkolonialer Feminismus und die Kunst der
Selbstkritik.“ In \emph{Spricht die Subalterne deutsch? Migration und
Postkoloniale Kritik},
herausgegeben von Hito Steyerl und Encarnacion Rodriguez, Münster: Unrast Verlag, 2003.
\bibitem{castromar} Castro Varela, María do Mar: „Verlernen und die Strategie des unsichtbaren Ausbesserns. Bildung 
und Postkoloniale Kritik.“ \emph{Bildkunst. Zeitschrift der IG Bildende Kunst},
download unter:
\url{http://www.igbildendekunst.at/bildpunkt/2007/widerstand-macht-wissen/varela}, am 26.11.2014
\bibitem{code} Code, Lorraine. „Experience, Knowledge, and Responsibility.“ In
\emph{Feminist Perspectives} in
Philosophy, herausgegeben von Morwenna Griffith und Margaret Withford, London: The Macmillan Press, 1988. 
\bibitem{darowska} Darowska, Lucyna und Claudia Machhold. Herausgeber\_innen.
\emph{Hochschule als transkultureller
Raum? Kultur, Bildung und Differenz in der Universität}. Bielefeld: transkript, 2011. 
\bibitem{lauretis} De Lauretis, Teresa. „Eccentric Subjects: Feminist Theory
and Historical Concousness.“ \emph{Feminist Studies}, Vol 16, No. 1 (1990): 115-150 
\bibitem{derrida} Derrida, Jacques. \emph{Die unbedingte Universität}. Frankfurt am Main: Suhrkamp, 2001.
\bibitem{dhawan} Dhawan Nikita. „Zwischen Empire und Empower. Dekolonisierung und Demokratisierung“ 
\emph{Femina Politica. Zeitschrift für Feministische Politikwissenschaft}. (2009).
\bibitem{dinkel} Dinkel, Jürgen. „'Dritte Welt.' Geschichte und Semantiken.“Version 1.0 Dokupedia Zeitgeschichte,
     2014. 
\bibitem{duden} Duden, Barbara. „Somatisches Wissen, Erfahrungswissen und 'diskursive' Gewissheiten.
Überlegungen zum Erfahrungsbegriff aus der Sicht einer Körper-Historikerin. In
\emph{Erfahrung: Alles nur Diskurs? Zur Verwendung des Erfahrungsbegriffes in
  der Geschlechtergeschichte. Beiträge zur 11. Schweizerischen
HistorikerInnentagung}, herausgegeben von Marguérite Bos, Bettina Vincenz und Tanja Wirz,  Zürich: Chronos Verlag, 2004.
\bibitem{dussel} Dussel, Enrique. „Transmodernity and Interculturality. An Interpretation from the Philosophy of
Liberation, \emph{Transmodernity: Journal of Peripheral Cultural Production
of the Luso-Hispanic World}, Vol. 1 Nr. 3, (2012).
\bibitem{eggers} Eggers, Maureen M., Grada Kilomba, Peggy Piesche und Susan
Arndt  et al. Herausgeberinnen. \emph{Mythen, Masken Subjekte. Kritische
Weißseinsforschung in Deutschland}. Münster: Unrast Verlag, 2005.
\bibitem{waldraut} Ernst, Waldraut. \emph{Diskurspiratinnen. Wie Feministische
  Erkenntnisprozesse die Wirklichkeit verändern}.Wien: Milena Verlag, 1999.
  \bibitem{fanon} Fanon, Frantz. \emph{Schwarze Haut, Weiße Masken}. Frankfurt am Main: Suhrkamp, 1992. 
  \bibitem{linda} Fisher, Linda. „Feminist Phenomenology.“ In \emph{Feminist   Phenomenology}, herausgegeben von ebd. und Lester Embree, Dordrecht: Kluwer Academic Publishers, 2000.
  \bibitem{linfisher} Fisher, Linda. „Phenomenology and Feminism. Perspectives
  on their Relation. In \emph{Feminist Phenomenology}, herausgegeben von ebd. und Lester Embree,  Dordrecht: Kluwer Academic Publishers, 2000. 
  \bibitem{foucault} Foucault, Michel. \emph{Die Ordnung der Dinge}. Frankfurt am Main: Suhrkamp Verlag, [1966 frz] 1974.
  \bibitem{foucdisp} Foucault, Michel. \emph{Dispositive der Macht. Über
  Sexualität, Wissen und Wahrheit}. Berlin: Merve Verlag, 1978.
  \bibitem{foucmensch} Foucault, Michel. \emph{Der Mensch ist ein
  Erfahrungstier.Gespräch mit Ducio Trombardio}, Frankfurt am Main: Suhrkamp Verlag, [1980] (1996).
  \bibitem{foucaultsex} Foucault, Michel. \emph{Sexualität und Wahrheit. Bd. 2
  Der Gebrauch der Lüste}.  Frankfurt am Main: Suhrkamp Verlag, [frz. Original 1984] 1986.
  \bibitem{kunningham} Gabel-Kunningham, Barbara et al. Vorwort der Übersetzer
  und Übersetzerinnen für \emph{Kritik der Postkolonialen Vernunft. Hin zu
  einer Geschichte der verrinnenden Gegenwart},von Gayatri C. Spivak, Stuttgart: Kohlhammer Verlag, 2013
  \bibitem{gilroy} Gilroy, Paul. \emph{The Black Atlantic. Modernity and Double Conscousness}. London: Verso, [1993] 1999.
  \bibitem{giuliani} Giuliani, Regula. „Körpergeschichte zwischen Modellbildung und haptischer Hexis. Thomas
  Laqueur und Barbara Duden. In \emph{Phänomenologie und Geschlechterdifferenz}, herausgegeben von Silvia Stoller und Helmuth Vetter, Wien: WUV Universitätsverlag, 1997.
  \bibitem{grosfou} Grosfoguel, Ramón. „Decolonizing Western Uni-versalisms:
  Decolonial Pluri-versalisms from AiméCésaire to the Zapatistas.“ \emph{Transmodernity, Journal of Peripheral Cultural Production of the Luso-Hispanic World}, Vol 1, Iss. 3, (2012). 
  \bibitem{grosfugel} Grosfoguel, Ramón. „The Structure of Knowledge in Westernized Universities: Epistemic
Racism/Sexism and the Four Genocides/Epistemicides of the Long 16th Century.“
\emph{Human Architecture: Journal of the Sociology of Self-Knowledge:} Vol. 11: Iss. 1, (2013). 
\bibitem{harding} Harding, Sandra. „Rethinking Standpoint Epistemology. 'What
Is Strong Objectivity' ?“ In \emph{Feminist Epistemologies}, herausgegeben von Linda M. Alcoff und Elizabeth Potter,  London: Routledge, 1993.
\bibitem{collins} Hill Collins, Patricia. \emph{Black Feminist Thought.
Knowledge, Consciousness and the Politics of Empowerment}. New York: Routledge, 2000.
\bibitem{hooksb} hooks, bell. \emph{Talking Back. Thinking feminist, thinking black}. Boston: South End Press, 1989
\bibitem{hookbell} hooks, bell. \emph{Teaching to Transgress. Education as the Practice of Freedom}. New York: Routledge, 1994.
\bibitem{laundry} Laundry, Donna, und Gerald MacLean. Einleitung zu \emph{The Spivak Reader. Selected Works of
Gayatri Chakravorty Spivak.} Herausgegeben von ebd., New York: Routledge, 1996.
\bibitem{levesque} Levesque Lopman, Louise. \emph{Claiming Reality. Phenomenology and
Women's Experience}. Totowa, New Jersey, Rowman &  Littlefield Publishers, 1988
\bibitem{lloyd} Lloyd, Genevieve. Das Patriarchat der Vernunft. \emph{„Männlich“ und „weiblich“ in der westlichen
Philosphie}. Bielefeld: Daedalus Verlag,1985. 
\bibitem{lohman} Lohman, Ingrid,  Sinah Mielich, Florian Muhl, Karl-Josef Pazzini, Laura Rieger und Eva Wilhelm. 
\emph{Schöne neue Bidlung? Zur Kritik der Universität der Gegenwart}. Bielefeld: transkript, 2011.
\bibitem{clintock} McClintock, Anne, Aamir Mufti und Ella Shohat,
Herausgerber\_innen. \emph{Dangerous Liaisons.
Gender, Nation and Postcolonial Perspectives}. Minneapolis: University of Minnesota Press, 1997.
\bibitem{mecholal} Mecheril. Paul, O Thomas-Olalde, Claus Melter, Susanne Arens und Elisabeth Romaner,
Herausgerber\_innen. \emph{Migrationsforschung als Kritik? Konturen einer
Forschungsperspektive}. Wiesbaden: Springer Verlag, 2013.
\bibitem{mechpaul} Mecheril, Paul. „Der doppelte Mangel, der das Schwarze Subjekt hervorbringt.“ In \textit{Mythen, Masken Subjekte. Kritische Weißseinsforschung in Deutschland}, herausgegeben von Maureen M.Eggers, Grada Kilomba, Peggy Piesche, Susan Arndt, Münster:Unrast Verlag, 2005.
\bibitem{mechpol} Mecheril, Paul. \emph{Politik der Unreinheit. Ein Essay über die Hybridität}. Wien: Passagen-Verlag, 2003
\bibitem{mechbirte} Mecheril, Paul und  Birte Klingler. „Universität als
transgressive Lebensform. Anmerkungen, die gesellschaftliche Differenz- und
Ungleichverhältnisse berücksichten.“ In \emph{Hochschule als transkultureller
Raum? Kultur, Bildung und Differenz in der Universität}, herausgegeben von L. Darowska et al. Bielefeld: transkript, 2011. 
\bibitem{mignolo} Mignolo, Walter D. \emph{Epistemischer Ungehorsam. Rhetorik
der Moderne, Logik der Kolonialität und Grammatik der Dekolonialität}. Wien: Turia + Kant, 2006.
\bibitem{mohanty} Mohanty, Chandra T. „Feminist Encounters. Locating the
Politics of Experience.“ In \emph{Destabilizing
Theory. Contemporary Feminist Debates}, herausgegeben von Michèle Marett und Anne Phillips, Cambridge: Polity Press, 1992. 
\bibitem{mohchan} Mohanty, Chandra T. „On Race and Voice.Challenges for Liberal
Education in the 1990s.“ \emph{Cultural Critique}, No. 14, (1989-1990): 179-208
\bibitem{narayan} Narayan, Uma, und Sandra  Harding. \emph{Introduction to
Decentering the Center}, herausgegeben von ebd., Indiana University Press: Bloomington, 2000
\bibitem{nederveen} Nederveen Pieterse, Jan und Bhikhu Parek. \emph{The Decolonization of Imagination. Culture,
Knowledge and Power}. London:  Zed Books. 1995.
\bibitem{ruemelin} Nida-Rümelin, Julian. \emph{Der Akademisierungswahn. Zur Krise beruflicher und akademischer Bildung}. Hamburg: Ed. Körber Stiftung, 2014.
\bibitem{nuenning} Nünning, Ansgar. „Wie Erzählungen Kulturen erzeugen. Prämissen, Konzepte und Perspektiven für
eine kulturwissenschaftliche Narratologie.“ In Kultur-Wissen-Narration. Perspektiven transdisziplinärer Erzählforschung, herausgegeben von Andrea Strohmaier,  Bielefeld: transkript, 2013.
\end{thebibliography}
\end{document}
