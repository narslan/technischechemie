\documentclass[12pt]{article}
\usepackage{amsmath,mathtools}
\usepackage[usenames,dvipsnames]{xcolor}
\usepackage[german]{babel}
\usepackage[utf8]{inputenc}
\usepackage[T1]{fontenc}
\usepackage{textcomp}
%\usepackage{libertine}
%\usepackage{helvet}
\usepackage{lscape}

\usepackage{microtype}
% Minion and Myriad fonts
\usepackage[minionint,mathlf]{MinionPro}
\renewcommand{\sfdefault}{Myriad-LF}
\usepackage{siunitx}
\usepackage{fancyhdr}
\usepackage{sectsty}
\usepackage{setspace}
\usepackage{booktabs} % To thicken table lines
\usepackage[version=4]{mhchem}
\usepackage[draft]{graphicx}
\usepackage[labelfont=bf]{caption}
\usepackage{subcaption}
\usepackage{chemstyle}
\usepackage{tabularx}
\usepackage[framemethod=tikz]{mdframed}
\usepackage{pgfplots}
\usepackage{pgfplotstable}
\pgfplotsset{compat=newest}
\usepgfplotslibrary{units}
\mdfdefinestyle{mystyle}{%
  innerleftmargin=10,
  innerrightmargin=10,
  outerlinewidth=3pt,
  topline=false,
  rightline=false,
  leftline=false,
  bottomline=false,
  skipabove=\topsep,
  skipbelow=\topsep
}
%\usepackage[compatibility=4.7,language=german]{chemmacros}
\usepackage[backend=biber,sorting=none,autocite = superscript,natbib=true]{biblatex} \addbibresource{books.bib}

\renewcommand{\familydefault}{\sfdefault}
\sisetup{detect-all}
\usepackage{chngcntr}
\counterwithin{table}{section}
\counterwithin{figure}{section}

\usepackage{titlesec}

\titleformat*{\section}{\large\bfseries}
\titleformat*{\subsection}{\normalsize\bfseries}

\usepackage{geometry}
 \geometry{
 a4paper,
 left=20mm,
 top=30mm,
 right=20mm
 }

\begin{document}

\begin{onehalfspace}

\section{Geometrieoptimierung und Frequenzberechnung an Dichlorethen}
In dieser Aufgabe wird die Bedeutung der Geometrieoptimierung näher untermauert.
Dafür werden die Schwingungsfrequenzen von \textit{cis-} und \textit{trans}-1,2-Dichlorethen auf HF-Niveau und MP2-Niveau berechnet.
Die angewandten quantumchemischen Methoden werden im Zusammenhang von Elektronenkorrelation verglichen.\\
\textbf{Auswertung 1 }
\begin{figure}[!hptb]
    \caption{Die optimierten Geometrien}
    \begin{subfigure}[b]{0.4\textwidth}
       \fbox{  \includegraphics[width=\textwidth]{data/cis_darstellung.png}}
                \subcaption{\textit{cis}-1,2-Dichlorethen }
    \end{subfigure}
    ~ %add desired spacing between images, e. g. ~, \quad, \qquad, \hfill etc.
      %(or a blank line to force the subfigure onto a new line)
    \begin{subfigure}[b]{0.4\textwidth}
       \fbox{  \includegraphics[width=\textwidth]{data/trans_darstellung.png}}
        \subcaption{ \textit{trans}-1,2-Dichlorethen }
    \end{subfigure}
    \label{figure:opt}
\end{figure}


\begin{table}[!htpb]

\caption{Gesamtenergien und Symmetrien von \textit{cis-} und \textit{trans}-1,2-Dichlorethen}
\begin{tabular}{llllc}
\toprule
Molekül & Methode &   Basissatz & Ges. En. \si{\hartree} & Symm. \\
\midrule
cis-1,2-Dichlorethen   & RHF& 6-311G(d,p)& -995.89694703 &$C_ {2v}$  \\
trans-1,2-Dichlorethen & RHF& 6-311G(d,p)& -995.89737964 &$C_ {2h}$ \\
cis-1,2-Dichlorethen   & MP2& 6-311G(d,p)& -996.45258595  &$C_ {2v}$  \\
trans-1,2-Dichlorethen & MP2& 6-311G(d,p)& -996.45188 &$C_ {2h}$    \\
\bottomrule
\label{table:energie}
\end{tabular}
\end{table}
 %\num{4.32e-4}
 %\num{7.05e-4}





\begin{table}[!htpb]

\caption{theoretische und experimentelle Schwingungsfrequenzen mit irreduziblen Darstellungen für \textit{cis}-1,2-Dichlorethen}
\begin{tabular}{llllll}
\multicolumn{2}{c}{RHF/6-311G(d,p)}&\multicolumn{2}{c}{MP2/6-311G(d,p)}&\multicolumn{2}{c}{Experimentell\cite{cisvib}} \\
\midrule
Irrep &  $\tilde{\nu}$ \si{\per\centi\meter} & Irrep &   $\tilde{\nu}$ \si{\per\centi\meter} & Irrep &  $\tilde{\nu}$\si{\per\centi\meter} \\
\midrule
$A _1$ & 183  &$A _1$  & 173    & $A_1$&   173 \\
$A _2$ & 458  &$A _2$  & 411    & $A_2$&    406 \\
$B _2$ & 614  &$B _2$  & 584    & $B_1$&     571  \\
$A _1$ & 759  &$A _1$  & 714    & $B_2$& 697  \\
$B _1$ & 803  &$B _1$  & 747    & $A_1$& 711  \\
$B _2$ & 920  &$B _2$  & 888    & $B_1$& 857 \\
$A _2$ & 1051 &$A _2$  & 897    & $A_2$& 876 \\
$A _1$ & 1324 &$A _1$  & 1238   & $A _1$& 1179\\
$B _2$ & 1439 &$B _2$  & 1341   & $B _1$ &1303 \\
$A _1$ & 1818 &$A _1$  & 1648   & $A _1$ &1587 \\
$B _2$ & 3373 &$B _2$  & 3249   & $B _1$ & 3072\\
$A _1$ & 3397 &$A _1$  & 3270   & $A _1$ & 3077\\
\bottomrule
\end{tabular}
\label{tab:cisvergleich}

\end{table}





\begin{table}[!htpb]

\caption{Schwingungsintensitäten mit irreduziblen Darstellungen und Schwingungsmodi für
\textit{cis}-1,2-Dichlorethen}
\begin{tabular}{llll}
\midrule
Irrep & IR-Int(RHF)/rel & Raman-Int(RHF)/rel & Schwingungsmodus~\cite{cisvib} \\
\midrule
$A _1$ & 0.41 & 2.160   & CCCl Deformationss.\\
$A _2$ & 0 & 4.86       & Torsion\\
$B _2$ & 7.93 & 6.74    &  CCCl Deformationss.\\
$A _1$ & 27.36 & 12.33  & C-H Deformationss.\\
$B _1$ & 73.62 S & 1.13   & C-Cl Strecks.\\
$B _2$ & 106.94 & 0.01  & C-Cl Strecks.\\
$A _2$ & 0 & 4.54       & C-H Deformationss.\\
$A _1$ & 0.14 & 20.88   & C-H Deformationss.\\
$B _2$ & 36.02 & 1.05   & C-H Deformationss.\\
$A _1$ & 34.83 & 59.81  & C-C Strecks.\\
$B _2$ & 15.14 & 53.92  & C-H Strecks.\\
$A _1$ & 2.93 & 141.85  & C-H Strecks.\\
\bottomrule
\end{tabular}
\label{tab:cisschwings}

\end{table}



\begin{table}[!htpb]

\caption{theoretische und experimentelle Schwingungsfrequenzen mit irreduziblen Darstellungen für \textit{trans}-1,2-Dichlorethen}
\begin{tabular}{llllll}
\multicolumn{2}{c}{RHF/6-311G(d,p)}&\multicolumn{2}{c}{MP2/6-311G(d,p)}&\multicolumn{2}{c}{Experimentell}~\cite{transvib} \\
\midrule
Irrep &  $\tilde{\nu}$ \si{\per\centi\meter} & Irrep &   $\tilde{\nu}$ \si{\per\centi\meter} & Irrep &  $\tilde{\nu}$\si{\per\centi\meter} \\
\midrule
$A _u$ & 233  & $A _u$ & 215  & $A_u$ & 227\\
$B _u$ & 257  & $B _u$ & 243  & $B_u$ & 250\\
$A _g$ & 376  & $A _g$ & 360  & $A_g$ & 350\\
$B _u$ & 875  & $B _g$ & 753  & $B_g$ & 763\\
$B _g$ & 912  & $B _u$ & 869  & $B_u$ & 828\\
$A _g$ & 920  & $A _g$ & 892  & $A_g$ & 846\\
$A _u$ & 1049 & $A _u$ & 939  & $A_u$ & 900\\
$B _u$ & 1333 & $B _u$ & 1258 & $B_u$ & 1200\\
$A _g$ & 1417 & $A _g$ & 1325 & $A_g$ & 1274\\
$A _g$ & 1815 & $A _u$ & 1645 & $A_g$ & 1578\\
$B _u$ & 3393 & $B _u$ & 3263 & $A_g$ & 3073\\
$A _g$ & 3400 & $A _g$ & 3267 & $B_u$ & 3090\\
\bottomrule
\end{tabular}
\label{tab:transvergleich}

\end{table}


\begin{table}[!htpb]
\caption{Schwingungsintensitäten mit irreduziblen Darstellungen und Schwingungsmodi für \textit{trans}-1,2-Dichlorethen }
\begin{tabular}{llll}
\midrule
Irrep & IR-Int(RHF)/rel & Raman-Int(RHF)/rel & Schwingungsmodus~\cite{transvib}   \\
\midrule
$A _u$ & 0.73 w& 0 & Torsion\\
$B _u$ & 4.58 w& 0 & CCCl Deformationss.\\
$A _g$ & 0.0 & 9.82 w&CCCl Deformationss.\\
$B _u$ & 156.36 s& 0 & C-H Deformationss.  \\
$B _g$ & 0 & 6.61 w & C-Cl Strecks.\\
$A _g$ & 0 & 9.56 w & C-Cl Strecks.\\
$A _u$  & 82.03 m & 0 & C-H Deformation \\
$B _u$ & 27.08 w& 0 & C-H Deformation\\
$A _g$ & 0 & 24.55 w& C-H Deformation\\
$A _g$ & 0 & 49.17 m& C-C Strecks.\\
$B _u$ & 17.72 w& 0 & C-H Streck. \\
$A _g$ & 0 & 115.68 s& C-H Strecks.\\
\bottomrule
\end{tabular}
\end{table}

\textbf{Auswertung 2}

\begin{figure}[!hptb]
    \centering
    \begin{subfigure}[b]{0.4\textwidth}
         \fbox{   \includegraphics[width=\textwidth,height=5cm]{data/cis_ir_spektrum.png}}
        \subcaption{auf RHF/6-311G(d,p) Niveau}
    \end{subfigure}
    ~ %add desired spacing between images, e. g. ~, \quad, \qquad, \hfill etc.
      %(or a blank line to force the subfigure onto a new line)
    \begin{subfigure}[b]{0.4\textwidth}
         \fbox{   \includegraphics[width=\textwidth,height=5cm]{data/cis_ir_mp2.png}}
        \subcaption{auf MP2/6-311G(d,p) Niveau}
    \end{subfigure}
    \caption{Die berechneten IR-Spektren von \textit{cis}-1,2-Dichlorethen}
    \label{figure:vergleichmethodec}
\end{figure}

\begin{figure}[!hptb]

\begin{tikzpicture}[font=\sffamily]
\begin{axis}[xlabel=Wellenzahl, ylabel=Intensität,  enlargelimits=true,axis x line=middle,
    axis y line=middle,
   y label style={at={(axis description cs:-0.15,.5)},anchor=south,rotate=90},
    x tick label style={rotate=90,anchor=east},
    x label style={at={(axis description cs:0.5,-.3)},anchor=south},
     xmin=0
    ]
\addplot[blue] table[x=frequency ,y=intensity] {data/cisex.txt};
\end{axis}
\end{tikzpicture}
    \caption{Das experimentelle IR-Spektrum von \textit{cis}-1,2-Dichlorethen \cite{cisir} }
\label{figure:vergleichcis}
\end{figure}

\begin{figure}[!hptb]
    \centering
    \begin{subfigure}[b]{0.4\textwidth}
         \fbox{   \includegraphics[width=\textwidth,height=5cm]{data/trans_ir_spektrum.png}}
        \subcaption{auf RHF/6-311G(d,p) Niveau}
    \end{subfigure}
    ~ %add desired spacing between images, e. g. ~, \quad, \qquad, \hfill etc.
      %(or a blank line to force the subfigure onto a new line)
    \begin{subfigure}[b]{0.4\textwidth}
         \fbox{   \includegraphics[width=\textwidth,height=5cm]{data/trans_ir_mp2.png}}
                \subcaption{auf MP2/6-311G(d,p) Niveau}
    \end{subfigure}
        \caption{Die berechneten IR-Spektren von \textit{trans}-1,2-Dichlorethen}
\label{figure:vergleichmethodet}

\end{figure}

\begin{figure}[!hptb]

\begin{tikzpicture}[font=\sffamily]
\begin{axis}[xlabel=Wellenzahl, ylabel=Intensität,  enlargelimits=true,axis x line=middle,
    axis y line=middle,
   y label style={at={(axis description cs:-0.15,.5)},anchor=south,rotate=90},
    x tick label style={rotate=90,anchor=east},
    x label style={at={(axis description cs:0.5,-.3)},anchor=south},
     xmin=0
    ]
\addplot[blue] table[x=frequency ,y=intensity] {data/trans.txt};
\end{axis}
\end{tikzpicture}
    \caption{Das experimentelle IR-Spektrum von \textit{trans}-1,2-Dichlorethen~\cite{transir}}
\label{figure:vergleichtrans}
\end{figure}


\begin{figure}[!hptb]
    \centering
    \begin{subfigure}[b]{0.4\textwidth}
          \fbox{  \includegraphics[width=\textwidth,height=5cm]{data/cis_raman_spektrum.png}}
        \subcaption{ \textit{cis}-1,2-Dichlorethen }
    \end{subfigure}
    ~ %add desired spacing between images, e. g. ~, \quad, \qquad, \hfill etc.
      %(or a blank line to force the subfigure onto a new line)
    \begin{subfigure}[b]{0.4\textwidth}
         \fbox{   \includegraphics[width=\textwidth,height=5cm]{data/trans_raman_spektrum.png}}
        \subcaption{\textit{trans}-1,2-Dichlorethen}
    \end{subfigure}
            \caption{Raman-Spektren auf RHF/6-311G(d,p) Niveau}
\end{figure}

.\\
\textbf{Auswertung 3}\\
Die MP2-Methode ist für die Berechnung von Schwingungsfrequenzen besser geeignet als die RHF-Methode,
da die Spektren der MP2-Methode näher an den experimentellen Spektren sind.
Dies liegt daran, dass die RHF-Methode die Elektronenkorrelation (die Wechselwirkung der Elektronen) vernachlässigt,
die MP2 Methode diese jedoch berücksichtigt.\\
\textbf{Auswertung 4}\\
Mit Elektronenkorrelation ist gemeint, dass die Coulombabstoßung $\dfrac{e^2}{\epsilon r_{12}}$ zwischen den beiden Elektronen einer Bindung dafür sorgt, dass sich
die Elektronen nicht zu nahe kommen. Ist das eine Elektron momentan in der Umgebung des einen Kerns,
so ist die Wahrscheinlichkeit gross, dass sich das andere Elektron in der Umgebung des anderen Kerns aufhält. Die Elektronen bewegen sich derart \glqq
korreliert\grqq, dass sie sich nicht nahe kommen. Die Elektronenkorrelation wirkt sich auf die Geometrie bzw. den
Energiezustand einer Geometrie aus.\cite{coulson}  \\
\textbf{Auswertung 5}\\
Beim \textit{trans}-1,2-Dichlorethen fällt auf, dass es entweder nur ramanaktiv oder infrarotaktiv ist.
Das \textit{cis}-1,2-Dichlorethen hingegen
 ist bei beiden aktiv. Dies liegt daran, dass das \textit{trans}-1,2-Dichlorethen ein
  Inversionszentrum hat und das \textit{cis}-1,2-Dichlorethen nicht.
Das Inversionszentrum verbietet bei einer symmetrischen Schwingung die
Infrarotaktivität und bei einer asymmetrischen Schwingung die Ramanaktivität.\\
\textbf{Auswertung 6}\\

Anhand der Werte für die Hartree-Energien aus ~\ref{table:energie} lässt sich sagen, dass
bei der Verwendung der RHF Methode für das \textit{trans}-1,2-Dichlorethen eine
kleinere Hartree-Energie ermittelt wurde als für das cis-1,2-Dichlorethen.
Dementsprechend ist unter Verwendung der RHF Methode das cis-Isomer der
Verbindung die stabilere. Betrachtet man jedoch die Werte, die anhand der MP2
Methode berechnet wurden, so lässt sich feststellen, dass hierbei das \textit{cis}-Isomer
eine kleinere Hartree-Energie aufweist als das trans-Isomer. Der Grund dafür ist, dass die MP2-Methode die destabilisierende Elektronenabstoßung
 zwischen Elektronen der vicinalen Cl-Atomen von \textit{cis}-1,2-Dichlorethen in Betracht zieht.

\printbibliography

\end{onehalfspace}
\end{document}