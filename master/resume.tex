\section{Resumé}

Meine Arbeit beginnt mit der Beschreibung einer Idee von Universität. Diese Idee
begreift die Aufgabe von Bildung und Kritik an der Universität in der
Überschreitung epistemischer Grenzen. Dabei dient die Überschreitung, im Sinne
der Herrschaftskritik, zwar letztlich einer Befreiung des Subjekts, versetzt es
jedoch auf diesem Weg zugleich in eine Krise, die ihm den Schein der Kohärenz
nimmt. Während sich die von mir diskutierten Autor\_innen einig in ihrer
poststrukturalistischen Grundausrichtung sind und somit eine Affinität für das
krisenhafte kontingenter Strukturen teilen, erweisen sich ihre Interpretationen
dieser Grenzerfahrungen des Selbst jedoch als äußerst unterschiedlich.
Bildungsprozesse werde hier als Verunsicherung und völligen Verlust des Selbst
über Ermächtigungsprozesse für ein freieres, würdevolleres Selbst bis hin zu
Prozessen der Wiederaneignung von Subjektpositionen aus silenced Positionen,
sehr unterschiedlich konzipiert. An dieser Stelle führe ich erstmals in den
Erfahrungsbegriff – hier als Krisen- und Grenzerfahrung des Subjekts markiert –
ein. Erfahrung, das wird an dieser Stelle bereits deutlich, vermag den
Zwischenraum zu beschreiben, der zwischen Subjekt und Diskurs in eben jenen
Praxen des 'Sichbildens' und 'Kritikübens' entsteht. Mein Interesse an Praxen
der Bildung und Kritik, kristallisiert sich nahezu in der Auseinandersetzung mit
Erfahrung, da Erkenntnisse einerseits aus Erfahrung, insbesondere
Krisenerfahrungen hervorgehen – andererseits werden durch Erkenntnisse, im Sinne
eines veränderten Blick auf das Selbst in Verhältnissen auch neue Erfahrungen
möglich. Erfahrungen sind damit zugleich Voraussetzung für und Folge von Kritik. 

Diesem anfänglichen Interesse an Erfahrung folgt nun der Versuch einer ersten
Kontextualisierung von Erfahrung in Wissensproduktionen. Die Verhältnisse, in
denen Erfahrungen an der Universität gemacht und geteilt werden können, geraten
hierbei in den Vordergrund. Ich stelle Fragen nach dem Umgang mit Erfahrung im
universitären Kontext, also wie Erfahrung im Ringen um Erkenntnis genutzt wird
und welche Effekte dies für unterschiedlich positionierte Subjekte und ihre
Erkenntnisse hat. Dabei wird sichtbar, dass die Auseinandersetzung mit Erfahrung
in Räumen stattfindet, in denen die Fäden kolonialer Gewalt nach wie vor
gespannt sind. Hier wird legitimes von illegitimem Wissen getrennt und die
historisch gewachsenen Machtverhältnisse greifen in die Wissensordnungen, die
darüber bestimmen, was von wem sagbar ist, ein. Mein Fokus liegt dabei auf
rassistischen und sexistischen Normen und ihren Effekten für unterschiedlich
positionierte Subjekte. Die gegebenen Verhältnisse müssen ergo in zweierlei
Hinsicht problematisiert werden: Zum einen, weil sie reglementierend, also
einschneidend auf die Artikulation von Erfahrung der Subjekte wirken und somit
Subjekten in unterschiedlichem Maße Artikulationsräume und damit auch Räume der
Bildung und Kritik verschaffen. Zum anderen, weil dieser Prozess der
Beschneidung, auf die Ordnung zurückgeworfen, sie in ihrer Allgemeingültigkeit
in Frage stellt, da sie eben nicht für alle gleichermaßen gültig ist. Der
anfangs beschriebene Bildungs- und Kritikbegriff wird hier insofern korrigiert,
als nicht mehr von allgemeinen Krisenerfahrungen ausgegangen werden kann,
sondern viel eher zu vermuten ist, dass sich die Auslöser und Effekte von Krisen
je nach Passungsverhältnis der Subjekte zu der Wissensordnung sehr
unterschiedlich gestalten.

Das Eingreifen der Wissensordnung ist jedoch kein einseitiger Prozess, der
ausschließlich reglementierend auf die Subjekte und die Möglichkeit der
Artikulation ihrer Erfahrungen wirkt. Viel eher scheint in der
Auseinandersetzung mit Erfahrung die Möglichkeit einer Rückwirkung auf diese
Ordnung zu bestehen. Das sprechen aus und über Erfahrung wird also nicht nur
durch jene historisch gewachsene Wissensordnung bestimmt, es greift
gleichermaßen in sie ein und birgt damit das Potential von Veränderung. Die
Veränderung, oder auch Überschreitung der Ordnung, die das legitime von
illegitimem Wissen unterscheidet, kann im Anschluss an postkoloniale
Theoretiker\_innen als Auftrag dekolonialer Bildung verstanden werden. Hieraus
ergibt sich auch die zentrale Fragestellung: Welche Bedeutung kann Erfahrung für
Praktiken des Gegenerzählens an der Universität einnehmen, um im Sinne einer
dekolonialen Bildung das Denkbare zu überschreiten?\\ 

\textbf{\large Hegemoniale Episteme}\\

Um besser verstehen zu können, wie jene Wissensordnung, die Allgemeingültigkeit
verspricht, aber nicht hält, überschritten werden kann, und welche Bedeutung
Erfahrung hierbei einnimmt, ist es zunächst jedoch Notwendig sich mit ihr
intensiver zu befassen. Im zweiten Teil meiner Arbeit beschäftige ich mich mit
der Entstehung dieser Wissensordnung, ihren Annahmen und Methoden. Auf diese
Weise möchte ich mehr darüber herausfinden, was sie so ausschließend macht und
wie sie überwunden werden kann. Meine Auseinandersetzung ist dabei von der Frage
geleitet, auf welcher Idee vom 'Menschsein' diese Wissensordnung fußt und welche
Erfahrungen vom 'Menschsein' dadurch ausgeschlossen werden. Hierbei kann ich
aufzeigen, dass die gewaltvolle Auslöschung von Wissensarchiven, die bereits mit
der Vorbereitung der kolonialen Expansion auf dem europäischen Festland begann
und im kolonialen Projekt mündete, dem bis heute vorherrschenden Wissen seine
hegemoniale Stellung verschafft hat. Die kritische Analyse der Zerstörung von
Wissenssystemen einerseits und der Durchsetzung hegemonialen Wissens
andererseits stellt dabei heraus, dass die Erfahrung kolonialer Herrschaft sich
tief in das Grundverständnis dieser Wissensordnung eingeschrieben hat, die
folglich einen hegemonialen Status beanspruchte. Dabei muss notwendigerweise von
Erfahrungen im Plural gesprochen werden, da sich die Erfahrung der Herrschaft
nur in Verbindung mit der Erfahrung des Beherrschtseins – also Erfahrungen von
Widerstand und Unterdrückung – denken lässt. Es ist jene doppelte Erfahrung, die
die hegemoniale Ordnung konstituiert und die durch ihr solipsistisches und
dualistisches Grundverständnis negiert wird. Denn dieses Grundverständnis birgt
die Vorstellung eines von seiner Umwelt isolierten Subjekts, das durch innere
Monologe, also ohne jegliche Begegnung, Erkenntnis gewinnt.

Das Bild vom
'Menschen' in dieser Wissensordnung ist das von einem 'Menschen', der denkt –
nicht erfährt – er ist völlig losgelöst von jeglicher Erfahrung und folglich
auch von denjenigen 'Menschen', mit denen er auf Grund jener Erfahrung zutiefst
verbunden ist. Mit der Erfahrung des Tötens geht immer auch die Erfahrung des
Sterbens einher. Das Selbstverständnis einer Ordnung, die zu einem Zeitpunkt
entsteht, an dem 85 Prozent der Welt unter europäisch kolonialer Herrschaft
steht, ist jedoch zutiefst durch die Erfahrung des Herrschens und
Beherrschtseins ebenso wie vom Widerstand und der Ausblendung dieses
Widerstandes bestimmt.

Wissen, das sich widerständig gegen diese Ordnung wendet, so lässt sich meine
Auseinandersetzung mit der hegemonialen Episteme resümieren, ist auf ein Bild
des 'Menschen' angewiesen, das sich über Erfahrung konstituiert. Indem es
Erfahrung als konstitutiv für sein Subjektverständnis und damit auch als
notwendigen Teil seines Erkennens von Welt begreift, wendet sich dieses Wissen
nicht nur gegen die Negation der Geschichte, sondern auch gegen die Negation der
Subjekte, deren Erfahrungen in dieser Geschichte miteinander in Verbindung
stehen.\\ 

\textbf{\large Widerständige Episteme}\\

Doch wie kann Erfahrung zum Gegenstand von Wissensproduktion gemacht werden? Um
wessen Erfahrung und welche Erfahrung muss es sich handeln? Werden durch die
Einbindung von Erfahrung automatisch Solipsismus und Dualismus überwunden?

Im
ersten Teil meiner Arbeit ist bereits deutlich geworden, dass die Thematisierung
von Erfahrung sich sowohl stabilisierend als auch irritierend auf die
hegemoniale Wissensordnung auswirken kann. Die Leerstelle, die ich in meiner
Auseinandersetzung mit hegemonialem Wissen aufzeigen konnte, kann also nicht
einfach durch die Platzierung von Erfahrung jener gefüllt werden, die in dieser
Ordnung bisher keinen Platz finden. Viel eher ist es notwendig, sich näher
anzusehen, welche Ort Erfahrung in jenen epistemischen Projekten erhält, die
einen widerständigen oder emanzipatorischen Anspruch gegenüber hegemonialem
Wissen hegen. Meine Auseinandersetzung ist entsprechend von der Frage geleitet,
wie Erfahrung hier erkenntnistheoretisch genutzt wird und welches Versprechen
davon für eine Kritik oder Überwindung der hegemonialen Wissensordnung ausgeht.
In der feministischen Epistemologie hat in den 1990er-Jahren ein breiter Diskurs
zu diesem Thema, vorwiegend in US-Amerika, stattgefunden. Die Notwendigkeit
Wissen auf Erfahrung zu gründen wird hier jedoch nicht, wie in meiner Arbeit,
als notwendige dekoloniale Strategie begriffen. Die Erfahrung von Rassismus bzw.
eine Perspektive auf Kolonialgeschichte als Erfahrungsraum, der in der
hegemonialen Wissensordnung ausgeklammert wird, erhält in dieser Diskussion (für
mich überraschend) fast gar keine Beachtung. Der Grund, warum ich mich trotzdem
mit diesem Diskurs beschäftige, liegt darin, dass hier eine Frage im Zentrum
steht, die auch für dekoloniale Wissensproduktionen zu den Kernfragen gehört:

Ist es möglich, die Erfahrung, die Subjekte innerhalb der hegemonialen
Wissensordnung machen, gegen eben jene Wissensordnung zu wenden?

Dieser Frage
liegt die Annahme zu Grunde, dass Erfahrungen nicht von den vorherrschenden
Diskursen losgelöst sind, sondern unter Umständen durch diese erst ermöglicht
werden. Wenn aber nur erfahrbar ist, was in der hegemonialen Ordnung bereits
denkbar ist, wie kann dann diese Erfahrung das Denkbare überschreiten?

Die
feministischen Antworten auf diese Frage werden von mir auf zwei Ebenen
diskutiert. Zum einen geht es um die Frage, wie sich die eigene Lebenserfahrung
und damit – in direktem Zusammenhang gedacht – die gesellschaftliche Position
als Perspektive in Wissensproduktion einschreibt. Es wird kontrovers diskutiert,
ob die Erfahrung von Marginalisierung bereits per se einen erkenntnistheoretisch
privilegierten Standpunkt eröffnet oder ob es dafür notwendig ist, jene
Erfahrung erst gegen den Strich zu lesen – sich also der hegemonialen
Interpretation zu widersetzen. Zum anderen wird Erfahrung als Gegenstand von
Wissensproduktion und hier insbesondere als Erfahrung der beforschten Subjekte
in ihrem erkenntnistheoretischen Gehalt diskutiert. An dieser Stelle geht es um
die Übersetzung von Erfahrung der Beforschten durch eine\_n Forscher\_in, auch
hier mit dem Anliegen, für Erfahrung von und aus Marginalisierung in der
Wissensproduktion einen Ort zu schaffen. Der Paternalismus, der hier schon
anmutet, und sich in dem paradoxen Anspruch der Repräsentation versteckt, wird
hier zum Thema gemacht. Eine Position als marginalisiert zu beschreiben, oder
anders formuliert, von Betroffenen auszugehen, setzt, so die
poststrukturalistische Kritik, Identitäten voraus, die tatsächlich aber durch
eben jene Zuschreibung erst hergestellt werden. Die hegemoniale Wissensordnung
kann jedoch nicht überwunden werden, wenn ihre Identitätskategorien ungefragt
übernommen werden. Subjekt- Objekt-Verhältnisse werden durch diese wohlwollenden
Gesten nicht aufgebrochen sondern zementiert. 

Phänomenologische Positionen hingegen erkennen im Rückgriff auf Erfahrung die
Möglichkeit, identitäre Kategorien zu überwinden. Erfahrung eröffnet ihrer
Ansicht nach einen Zwischenraum, der das kategoriale Raster des Körper-Geist
Dualismus irritieren, sogar aus ihm ausbrechen kann. Denn ihr Fokus richtet sich
nicht an vorhergesehene Subjektpositionen und deren zugeschriebene Erfahrungen.
In der Phänomenologie werden Erfahrungen indessen als Körperempfinden
konzeptionalisiert. Der Erkenntnisanspruch, den sie hieraus entwickelt, greift
somit das Wissensmonopol Vernunft/Ratio/Geist an. Ein essentialismuskritischer
Körperbegriff wird hier mit dem Konzept des Soma als historisch verorteter und
somit diskursiv gebundener Körper theoretisiert.

Phänomenologische Positionen hingegen erkennen im Rückgriff auf Erfahrung die
Möglichkeit, identitäre Kategorien zu überwinden. Erfahrung eröffnet ihrer
Ansicht nach einen Zwischenraum, der das kategoriale Raster des Körper-Geist
Dualismus irritieren, sogar aus ihm ausbrechen kann. Denn ihr Fokus richtet sich
nicht an vorhergesehene Subjektpositionen und deren zugeschriebene Erfahrungen.
In der Phänomenologie werden Erfahrungen indessen als Körperempfinden
konzeptionalisiert. Der Erkenntnisanspruch, den sie hieraus entwickelt, greift
somit das Wissensmonopol Vernunft/Ratio/Geist an. Ein essentialismuskritischer
Körperbegriff wird hier mit dem Konzept des Soma als historisch verorteter und
somit diskursiv gebundener Körper theoretisiert.

Die Frage nach der Bedeutung von Erfahrung für Dekolonisierung ist hier noch
lange nicht beantwortet, da der dekoloniale Bezug in der feministischen
Diskussion fehlt. Das komplexe Feld, das durch die erkenntnistheoretische
Diskussion um die Bedeutung von Erfahrung in Wissensproduktionen eröffnet wurde,
ermöglicht mir jedoch  eine Vielzahl an Anknüpfungspunkten für die dekoloniale
Forderungen einer an Befreiung orientierten Bildung an der und für die
Universität.

Dabei geht es mir nicht darum die Grenzen zu verschärfen, die phänomenologische
und poststrukturalistische Perspektiven voneinander trennen bzw. diese den
postkolonialen Theorien gegenüberzustellen. Dies ist weder analytisch sinnvoll
noch ethisch verantwortbar, da diese Strömungen in einem permanenten
Wechselspiel miteinander entstanden sind. Insbesondere in der feministischen
Auseinandersetzung, in der Erfahrung als entweder prä- oder postdiskursiv
gedacht wird, geht es implizit nicht so sehr um die Schärfung des
Erfahrungsbegriffs als um die Verteidigung spezifischer Subjektverständnisse.
Das spricht jedoch nicht dagegen, sich mit Erfahrung auseinanderzusetzen. Diese
Erkenntnis zeigt hingegen, wie notwendig die Auseinandersetzung mit dem
Erfahrungsbegriff bleibt, da sie so eng mit dem hegemonialen Subjektbegriff
verwoben ist, dass sich auch Versuche neuer Subjektkonstruktionen auf Erfahrung,
nur eben anders,  beziehen müssen.


\textbf{\large Erzählung}\\

Mein Interesse, die verschiedenen bisher besprochenen epistemischen Projekte
zusammenzuführen, wird zu einer Kernaufgabe des letzten Kapitels. Als Konsequenz
aus den Erkenntnissen, insbesondere der Auseinandersetzung mit der hegemonialen
Episteme und den erkenntnistheoretischen Positionen innerhalb der feministischen
Diskussion,  nehme ich hierfür eine Praxis in den Blick, die verspricht, dem
theoretischen Anspruch den ich bis dahin entwickelt habe, gerecht zu werden. Der
Reiz, sich mit dem Konzept der Erzählung zu beschäftigen, liegt nämlich darin,
dass sie zugleich einen Blick auf und Strategien in Verhältnisse bereithält. Zum
einen ermöglicht mir die Erzähltheorie, die epistemischen Praxen, seien sie
hegemonial oder widerständig, als Erzählungen zu theoretisieren, die
Erzählgemeinschaften miteinander verbinden und so geteilte und weitergegebene
Erfahrung als Motor für Kollektivierungsprozesse einbezieht. Sie ermöglicht mir
also einen Blick auf gesellschaftliche Praxen der Bedeutungskonstruktion, in der
Erfahrungen zu Erinnerungen werden und über Erzählungen Einzug in den Diskurs
erhalten. Zum anderen werden insbesondere in der postkolonialen Theorie
Erzählungen als Strategien verstanden, durch die Verhältnisse bestätigt aber
auch irritiert werden können. Hier setzte ich mich mit zwei einander auf den
ersten Blick sehr gegensätzlichen Strategien auseinander. Die Ich-Erzählung kann
gewissermaßen als Guerilla-Strategie interpretiert werden, die sich gegen die
Gegenüberstellung von Theorie und Erfahrung wendet und fordert letztere als
gleichberechtigte Wissensbestände miteinander in Dialog zu bringen. Die
Transmoderne hingegen begreift sich als Metaerzählung einer globalen Moderne, in
der verschiedene Wissens- oder Erzählgemeinschaften nebeneinander existieren.

In den Ich-Erzählungen stehen Praxen des Dialogs und der Repräsentation im
Zentrum. Die unterschiedlichen Ebenen der Abstraktion, die Erzählungen
kennzeichnen, sind auf einander angewiesen und dürfen nicht in ein
hierarchisches Verhältnis zueinander gesetzt werden, so die Forderungen. Dafür
ist es allerdings notwendig, dass diese Guerilla-Erzählungen die dominante Logik
dessen, wie Subjekte imaginiert werden, in welchem Verhältnis sie zur Geschichte
stehen, ja überhaupt woraus Wissen besteht, unterwandern. So können neue Formen
für die Auseinandersetzung mit den eigenen Erfahrungen gesucht werden, die es
ermöglichen aus einschneidenden Interpretationsrastern auszubrechen und so
andere, Subjekt-, Geschichts- und Wissensverständnisse zu entwickeln. Das
interessante an diesem Ansatz ist, dass, anders als in der feministischen
Diskussion um den erkenntnistheoretischen Gehalt von Erfahrung, nicht zwischen
emanzipatorischem versus affirmativem Wissen unterschieden wird. Nicht die
Wertung, wie kritisch sich Wissen verhält, sondern der Grad der Abstraktion von
Wissen steht hier im Vordergrund. Dabei kann nicht zwischen den
unterschiedlichen Abstraktionsniveaus hierarchisiert werden. Die Kunst liegt
indessen darin, jene verschiedenen Sprach- und Erzählweisen in einen Dialog zu
bringen um so abstrakte Vorstellung mit konkreten Erfahrungen in Widerstreit zu
setzen. Weder das eine noch das andere darf hierbei leitend werden, viel eher
geht es um einen Austausch, in dem nicht nur verschiedene Perspektiven sondern
auch verschiedene Erzählpraxen sich miteinander verbinden.

Die transmoderne Erzählung ist nicht weniger radikal, setzt jedoch an einem
anderen Punkt an. Anstatt die Erzählung der Europäische Moderne von unten zu
unterwandern, stellt die Transmoderne die Europäische Moderne in den Schatten
einer globalen Moderne und situiert sie auf diese Weise als partikulare
Geschichte. Das Fundament, auf dem die hegemoniale Episteme ruht, wird so
erschüttert, da in der Transmoderne verschiedene Modernen nebeneinander
existieren und so die Einteilung in Zentrum und Peripherie obsolet werden. Ob es
möglich ist, unter diesen neuen Verhältnissen Bündnisse einzugehen und Wissen
auszutauschen ist jedoch davon abhängig, ob eurozentristische Logiken überwunden
werden und so Verortung, Exteriorität und Autonomie, als Schlüsselkonzepte der
Transmoderne, Einzug erhalten. Das Konzept der Transmoderne bremst die Hoffnung,
die mit einem Dialog verschiedener Erzählungen einhergehen könnte, wieder aus,
da ein Dialog unterschiedlicher Erzählgemeinschaften bzw. Erzählformen nicht
unter herrschenden Bedingungen stattfinden kann. Solange die hegemoniale
Episteme ihren Anspruch bewahrt, Zentrum allen Wissens und Denkens zu sein,
verschließt sie sich anderen Perspektiven oder eignet sie sich an. Die neue,
transmoderne Ordnung entsteht demnach nicht durch einen Dialog, sondern dadurch,
dass Räume der Autonomie in der Exteriorität erschlossen werden. Die
Exteriorität wird hier als Ort gedacht, der in der hegemonialen Sprache als
Außen markiert wird und ihren Wirkungsweisen darum nicht dermaßen ausgesetzt
ist. Hier, in dieser Abgeschiedenheit liegt entsprechend auch die Hoffnung für
eine Neuordnung, in der verschiedene Wissensformen in Solidarität miteinander
existieren können. Erst wenn sich die Exteriorität ihre gleichwertige Position
im Diskurs erkämpfen und die Europäische Moderne ihre hegemoniale Position  als
Verlust erkennen und überwinden kann, können verschiedene Formen des Wissens und
damit Erfahrungsgemeinschaften miteinander in einen Dialog treten, in dem neue
Ordnungen von Wissen und ihren Repräsentationen entstehen. 

Ein reflexiver Umgang mit Erfahrung als konstitutiver Bestandteil jeden Wissens,
muss dabei immer Teil einer Gegenerzählung sein. Das Subjekt, das erzählt,
erzählt nicht nur über Erfahrungen, es erzählt vielmehr gegen die Negierung von
Erfahrungen, und damit gegen die Illusion, dass Wissen sich von seiner
Entstehungsgeschichte loslösen kann. Die feministische Diskussion um den
erkenntnistheoretischen Gehalt von Erfahrung wird ohne die Forderungen nach
einer Transmoderne zu einem leeren Versprechen. Eingebettet in eine zeit- und
räumlich situierte Praxis werden die Reflexionen jedoch zu einem wichtigen
Bestandteil. Es sollte also nicht darum gehen, Erfahrung als emanzipatorischen
Wunderstab oder ideologisch aufgeladenen Windbeutel zu stilisieren. Viel eher
besteht die Aufgabe darin, Verhältnisse zu schaffen, in denen Erfahrungen
artikuliert und reflektiert werden können, die sich sperrig zu der hegemonialen
Ordnung verhalten.  Dekoloniale Bildung kann somit als eine Praxis verstanden
werden, die jenen Erfahrungen einen angemessenen, das heißt, selbstbestimmtem
Ort bietet,  die sich in ihren Hin- und Wegbewegungen von der imperialen
Vernunft bis heute behaupten.


