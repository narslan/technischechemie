\documentclass{article}
\usepackage{amsmath}


%encoding
%--------------------------------------

%----------------------------
\usepackage{microtype}
\usepackage{fourier}
\usepackage[utf8]{inputenc}
\usepackage[T1]{fontenc}
%units
%--------------------------------------
\usepackage{siunitx}
\usepackage{verbatim}
%drawing
%------------------------------------
\usepackage{tikz} % To generate the plot from csv
\usepackage{pgfplots}
\usepackage{pgfplotstable}
\usepackage{caption}
\usepackage{color, colortbl}
\usepackage{tabularx}
\usepackage{setspace}
\definecolor{LightCyan}{rgb}{0.88,1,1}
\usetikzlibrary{datavisualization}
\pgfplotsset{compat=newest} % Allows to place the legend below plot
\usepgfplotslibrary{units} % Allows to enter the units nicely
\usepackage{overcite}
\renewcommand\citeform[1]{[#1]}

\sisetup{
  round-mode          = places,
  round-precision     = 2,
  inter-unit-product =\ensuremath{{}\cdot{}}
}




\usetikzlibrary{arrows, positioning, calc, datavisualization}

\pgfplotsset{
  standard/.style={
    axis x line=bottom,
    axis y line=middle,
  }
}
\usepackage{url}

\begin{document}
%\begin{luacode*}
function string:split(sep)
        local sep, fields = sep or "%s", {}
        local pattern = string.format("([^%s]+)", sep)
        self:gsub(pattern, function(c) fields[#fields+1] = c end)
        return fields
end

function printHyperbola()
    local lines={}

    for line in io.lines("dampf.csv") do
            table.insert(lines, line)
    end
    tex.sprint("\\addplot[color=black] coordinates{")

    for i=2,#lines do
        local a=lines[i]:split()
        tex.sprint("("..a[1]..","..a[2]..")")
    end
     tex.sprint("};")
    tex.sprint("\\addplot[color=blue] coordinates{")

    for i=2,#lines do
        local a=lines[i]:split()
        tex.sprint("("..a[1]..","..a[3]..")")
    end
     tex.sprint("};")

end
\end{luacode*}
%\begin{figure}
\centering
     \begin{tikzpicture}
            \begin{axis}[standard,xlabel=Temperatur,ylabel=Druck]
                \directlua{printHyperbola()}
            \end{axis}
        \end{tikzpicture}
     \end{figure}

\noindent
\begin{onehalfspace}

\section{Ziel des Versuches}
In einer Hydrogencarbonat/Carbonat-Pufferlösung soll die Absorptionsrate für die $CO_2$ -
Absorption gemessen werden. Durch Arsenoxid wird die Reaktion katalysiert. Die
Phasengrenzfläche $A$ und die Geschwindigkeitskonstante $k_kat$ der Reaktion werden ermittelt.
\section{Versuchsdurchführung}
Zuerst wurde eine Stammlösung von 150 mg As 4 O 6 in 300 mL 0.5 M K 2 CO 3 -Lösung hergestellt.
Daraus wurden drei weitere Lösungen á 400 mL angesetzt. Die Zusammensetzungen der
Lösungen befinden sich in Tabelle 1.
Die Pufferlösung wurde in eine Rührzelle überführt und für ca. 8 Minuten evakuiert.
Anschließend wurde unter Rühren Kohlenstoffdioxid über einen Gasballon zugeführt. Über
einen Seifenblasenströmungsmesser wurde die Zufuhrgeschwindigkeit für 20 Minuten
gemessen.
\section{Messwerte}
\text{Konzentration $c(As_4O_6) = 0$ \si{\milli\mol\per\liter}}
% subsection subsection_name (end)
\begin{table}[ht!]
  \centering
 \begin{tabularx}{\textwidth}{XXXX}
$c_{kat}$ [\si{\milli\mol\per\liter}] & $V_{Stamm}$ [\si{\milli\liter}]  & $V_{K_2O_3}$ [\si{\milli\liter}] & $V_{NaHCO_3}$ [\si{\milli\liter}]  \\
\hline
\rowcolor{LightCyan}
0  & 0 & 200 & 200\\
132.6  & 100 & 100 & 200\\
\rowcolor{LightCyan}
265.3 & 200 & 0 & 200\\
\end{tabularx}
  \caption{Herstellungen der Messlösungen}
\end{table}
Die Pufferlösung wurde in eine Rührzelle überführt und für ca. 8 Minuten evakuiert.
Anschließend wurde unter Rühren Kohlenstoffdioxid über einen Gasballon zugeführt. Über
einen Seifenblasenströmungsmesser wurde die Zufuhrgeschwindigkeit für 20 Minuten
gemessen.
\section{Auswertung}
Für $c(As_4O_6)= 0$ :
\begin{table}[ht!]
  \centering
 \begin{tabularx}{\textwidth}{XX}
$T_{kat}$ [\si{\second}] & $V_{Stamm}$ [\si{\milli\liter}]    \\
\hline
\rowcolor{LightCyan}
 200 & 200\\
 100 & 200\\
\rowcolor{LightCyan}
 0 & 200\\
\end{tabularx}
  \caption{Herstellungen der Messlösungen}
\end{table}

\begin{thebibliography}{}
\bibitem{crc1}
D. R. Lide, \textit{CRC Handbook of Chemistry and Physics}, 88. Aufl., CRC Press, New York \textbf{2007}, S. 888.
\bibitem{crc2}
D. R. Lide, \textit{CRC Handbook of Chemistry and Physics}, 88. Aufl., CRC Press, New York \textbf{2007}, S. 1069.
\bibitem{crc3}
\url{http://webbook.nist.gov/chemistry/fluid/}
\bibitem{skript1}
A. Brehm, \textit{Praktikumskript Wärmeübertragung}, Oldenburg, \textbf{2014},  S. 14.
\end{thebibliography}
\end{onehalfspace}

\end{document}
