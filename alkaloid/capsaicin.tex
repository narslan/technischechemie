
\begin{frame}[t]{Capsaicin}
  
\chemfig[][scale=0.5]{
     -[:270]% 2
               (
         -[:330]% 3
               )
     -[:210]% 4
     =[:150]% 5
     -[:210]% 6
     -[:150]% 7
     -[:210]% 8
     -[:150]% 9
     -[:210]% 10
               (
         =[:270]O% 11
               )
     -[:150]\chemabove{N}{H}% 12
     -[:210]% 13
     -[:150]% 14
    =_[:210]% 15
     -[:150]% 16
               (
         -[:210]O% 21
         -[:270]% 22
               )
     =_[:90]% 17
               (
     -[:150,,,2]HO% 20
               )
      -[:30]% 18
    =_[:330]% 19
               (
         -[:270]% -> 14
       )}
       \begin{itemize}
         \item Capsaicinoide sind farblos und sehr stabil. 
         \item Capsaicin bindet an den TRP-Kanal TRPV1, der auch durch eine Erhöhung der Temperatur aktiviert wird. Von diesem Umstand leitet sich der Ausdruck „brennen“ ab.
       \end{itemize}
       \url{https://de.wikipedia.org/wiki/Capsaicin}
\end{frame}
