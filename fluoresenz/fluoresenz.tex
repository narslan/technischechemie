\documentclass[12pt]{article}
\usepackage{amsmath,mathtools}
\usepackage[usenames,dvipsnames]{xcolor}
\usepackage[bitstream-charter]{mathdesign}
\usepackage{microtype}
\usepackage[utf8]{inputenc}
\usepackage[T1]{fontenc}
\usepackage{libertine}
\usepackage{graphicx}
\usepackage{siunitx}
\usepackage[german]{babel}
\usepackage{comment}
\usepackage{nicefrac}
\usepackage{booktabs}
\usepackage{float}
\usepackage{tikz}
\usetikzlibrary{arrows,chains,matrix,positioning,scopes,patterns,decorations.pathmorphing,shapes}
\usepackage{pgfplots}
\pgfplotsset{compat=newest}
\usepackage[version=4]{mhchem}
\usepackage[backend=biber,sorting=none,autocite = superscript,natbib=true]{biblatex} \addbibresource{books.bib}

\floatstyle{plaintop}
\restylefloat{table}
\usepackage[justification=justified,singlelinecheck=false]{caption}
% header
\usepackage{fancyhdr}
\fancyhf{}
\rhead{02.12.2016}
\lhead{Nevroz Arslan, Justin König Gruppe 6}
\setlength{\headheight}{15pt}
\rfoot{\thepage }
\lfoot{Versuch 1: Fluoreszenzspektroskopie}
\pagestyle{fancy}
%\renewcommand{\thesection}{\arabic{section}}% Remove section references...
%\renewcommand{\thesubsection}{\arabic{subsection}}%... from subsections
%\usepackage[backend=biber,sorting=none]{biblatex} \addbibresource{books.bib}

\usepackage{url}
\usepackage{csquotes}
\sisetup{inter-unit-product=\ensuremath{{}\cdot{}}}
\renewcommand{\arraystretch}{1.5}
\definecolor{skyblue1}{RGB}{135, 206, 250}
\definecolor{flame}{RGB}{226, 88, 34}
\definecolor{scarletred1}{RGB}{252, 40, 71}
\definecolor{cyanblau}{RGB}{0, 158, 224}
\definecolor{charcoal}{rgb}{0.21, 0.27, 0.31}

\begin{document}

\section{Ziel des Versuches}
Der Versuch befasst sich mit spektralen Untersuchungen im UV-Bereich. Dafür werden die Fluoreszenzspektren und Anregungsspektren von 2-Naphthol und 2-Naphtolat in Wasser aufgenommen.
\section {Theorie}
\subsection{Absorption und Emission\supercite{og}}
\glqq Eine chemische Verbindung erscheint dann farbig, wenn sie aus dem sichtbaren Teil des Spektrums einen Wellenlängenbereich selektiv absorbiert.
Im Allgemeinen sind gesättigte organische Verbindungen für unser Auge farblos, da sich ihr Absorptionsbereich außerhalb der Sichtbaren im fernen Ultraviolett
befindet. Durch Einführung von $\pi$-Bindungssystemen wie \ce{C=O}, \ce{C=N}, \ce{N=O} u.a., besonders wenn diese miteinander in Konjugation stehen, verschieben sich
die Absorptionsbanden immer mehr zum langwelligeren, sichtbaren Teil des Spektrums.
Für die Farbigkeit ist vor allem die energetisch hochliegenden und damit leichter anregbaren $\pi$ Elektronen verantwortlich.
Je stärker die $\pi$-Elektronen delokasiert sind, umso langwelliger ist das Licht, welches die Verbindung absorbiert.
Man bezeichnet eine Gruppierung in einem Molekül, die Strahlung im UV/Vis Bereich absorbiert, als \textbf{Chromophore} (gr. \textit{phoron} = Träger). Die Anhäufung von Chromophoren,
besonders in konjugierten Systemen, ruft eine Farbvertiefung (\textbf{Bathochromie}, Rotverschiebung) hervor, d.h. eine Verschiebung der Absorptionsmaxima nach längeren Wellen.
Die auxochromen (gr. \textit{auxesis}= Zunahme) Gruppen verschieben die Absorptionsmaxima noch längeren Wellen. Eine typische, auxochrome Gruppe ist die Hydroxgruppe, \ce{-OH}.
Demgegenüber können Substituenten, die über einen \ce{-I}-Effekt verfügen, einen hypsochromen Effekt (Blauverschiebung) (\textbf{Hypsochromie}) auslösen.\grqq

\subsection{Physikalische Vorgänge der Fluoreszenz\supercite{harris}}
Wir nehmen an, dass das Molekül M Licht absorbiert und in den angeregten Zustand \ce{M^*}
überführt wird:
\begin{align*}
 \text{Absorption:} &= \ce{M + h \cdot v -> M^*}\\
 \text{Geschwindigkeit:} &= - \frac{d[M^*]}{dt} = k _{Abs.} [M]
\end{align*}
Die Geschwindigkeitskonstante, $k_{Abs.}$ , hängt von der Intensität der Bestrahlung und
dem Extinktionskoeffizienten ($\epsilon$) von $M$ ab. Je intensiver die Einstrahlung und je besser es
absorbiert wird, desto schneller wird $M^*$ gebildet.

Nach der Absorption kann $M^*$ ein Photon emittieren und in den Grundzustand zurückkehren:
\begin{align*}
 \text{Emmision:} &= \ce{M^* + -> M +  h \cdot v}\\
 \text{Geschwindigkeit:} &= - \frac{d[M^*]}{dt} = k _{E} [M^*]
\end{align*}

Alternativ kann das angeregte Molekül die Energie durch Wärme verlieren:
\begin{align*}
 \text{Desaktivierung:} &= \ce{M^* -> M +  W\ddot{a}rme}\\
 \text{Geschwindigkeit:} &= - \frac{d[M^*]}{dt} = k _{D} [M^*]
\end{align*}


Eine weitere Möglichkeit ist die Übertragung der Energie des angeregten Moleküls auf ein
anderes Molekül (Quencher, Q) durch Fluoreszenzlöschung ( Quenching ). Dabei wird
der Quencher angeregt:
\begin{align*}
 \text{Quenching:} &= \ce{M^* + Q -> M +  Q^*}\\
 \text{Geschwindigkeit:} &= - \frac{d[M^*]}{dt} = k _{Q} [M^*] [Q]
\end{align*}

Aus den beschriebenen Einzelprozessen resultiert, dass sich die Geschwindigkeit der Deaktivierung von $M^*$ aus der Summe der Geschwindigkeiten von Emission, Deaktivierung
und Löschung ergibt:
\begin{align*}
 \text{Geschwindigkeit des Verschwindens von } M^* &=  k _{E} [M^*] + k _{D} [M^*] + k _{Q} [M^*] [Q]\\
\end{align*}
Setzt man Bildungs- und Abbaugeschwindigkeit von M* gleich, erhält man:
\begin{align*}
 k_{Abs} [M] &=  k _{E} [M^*] + k _{D} [M^*] + k _{Q} [M^*] [Q]\\
\end{align*}
\subsection{Quantenausbeute\supercite{harris}} % (fold)
\label{ssub:quantenausbeute}

% subsubsection quantenausbeute (end)
\grqq Die Quantenausbeute für einen photophysikalischen Prozess ist der Anteil der absor-
bierten Photonen, der die gewünschten Emissionsprozesse hervorruft. Wenn das für jedes
absorbierte Photon zutrifft, wäre die Quantenausbeute Eins.\grqq
Die Quantenausbeute für die Emission von $M^*$ ist die Geschwindigkeit der Emission
dividiert durch die Geschwindigkeit der Absorption. Bei Abwesenheit des Quenchers :
\begin{align*}
  \Phi _0 = \frac{\text{pro Sekunde Emittierte Photonen}}{\text{pro Sekunde absorbierte Photonen}} = \frac{k_{E} [M^*]}{k_{Abs.} [M]}
\end{align*}
\subsection{Aufbau des Fluoreszenz-Spektrometers\supercite{skript}}
\begin{figure}[h]

\centering
\begin{tikzpicture}[
  triangle/.style={isosceles triangle,draw=,thick,shape border rotate=90, minimum height=15mm,minimum width=15mm,inner sep=0},
  ]{
\node[triangle]{X};
\node at (0,-0.7) {Licht quelle};
\draw[->,ultra thick] (1,0.25)  -- (2,0.25);
 % \draw (0,0) rectangle (1,0.5) node[pos=.5] (a) {Licht quelle};
\draw[->,ultra thick] (1,0.25)  -- (2,0.25);
\draw (2,0) rectangle (5,0.5) node[pos=.5] (b) {Monochromator};
\draw[->,ultra thick] (5,0.25)  -- (6,0.25);
\draw (6,0) rectangle (8,0.5) node[pos=.5] (c) {Probe};
\draw[->,ultra thick] (8,0.25)  -- (9,0.25);
\draw (9,0) rectangle (11,0.5) node[pos=.5] (d) {Detektor};
\draw[->,ultra thick] (11,0.25)  -- (12,0.25);
\draw (12,0) rectangle (14,0.5) node[pos=.5] (e) {Rechner};

}
\end{tikzpicture}

\end{figure}
\section{Auswertung}
\section{Diskussion}
\printbibliography

\end{document}
