% !TEX encoding   = UTF-8
% !TEX program    = LuaLaTeX
% !TEX spellcheck = de_DE
\documentclass[landscape,a4paper]{article}
\usepackage[usenames,dvipsnames]{color}
\usepackage{siunitx}
\usepackage[german]{babel}
\usepackage{tikz}
\usepackage{pgfplots}
\usepackage{pgfplotstable}
\pgfplotsset{compat=newest}
\usepgfplotslibrary{units}
%\usetikzlibrary{datavisualization} 
\usepackage{graphicx}
\usetikzlibrary{arrows,chains,matrix,shapes}
\usepackage{luacode}
\usepackage{helvet}

%\usepackage[landscape]{geometry}
%\setmainfont{Ubuntu}
%\setmainfont{Linux Libertine}
%\setsansfont{Linux Biolinum}
\usepackage{paratype}
\renewcommand*\familydefault{\sfdefault} %% Only if the base font of the document is to be sans serif
\usepackage[T1]{fontenc}
\usepackage{fancyhdr}
\pagestyle{fancy}
\usepackage{subcaption}
\usepackage{adjustbox}

\usepackage[top=3cm, bottom=3cm, left=2cm, right=2cm]{geometry}

\setlength{\headheight}{30pt} 

\rhead{ \Large \textbf{$\mathsf{Ag^0}$}}
\lhead{\large \textbf{Silber Metal (Z=47) \textit{Sputting}}}
\fancyheadoffset{-1cm}
\pagenumbering{gobble}
\usetikzlibrary{pgfplots.groupplots}

\begin{luacode*}


  fs = dofile("fs.lua")

  function drawAG()

  px,py = fs.readcsv("data/silber/nachkratzen1.txt",0)


	tex.sprint("\\addplot[".."darkblue".."] coordinates{")
		for i=1,#px do
		tex.sprint("("..px[i]..","..py[i]..")")
		end
	tex.sprint("};")
	tex.sprint("\\node[darkblue] (a) at (-26,700) {4d};")
	tex.sprint("\\node[violet] (b) at (30,440) {\\footnotesize 4p};")
	tex.sprint("\\node[violet] (c) at (80,440) {\\footnotesize 4s};")
	
	tex.sprint("\\node[violet] (d) at (328,6000) {$\\mathsf{3d_{\\frac{5}{2}}}$};")
	tex.sprint("\\node[violet] (e) at (328,6000) {$\\mathsf{3d_{\\frac{5}{2}}}$};")
	tex.sprint("\\node[violet] (f) at (420,5050) {$\\mathsf{3d_{\\frac{3}{2}}}$};")
	tex.sprint("\\node[violet](g) at (523,2720) {$\\mathsf{3p_{\\frac{3}{2}}}$};")
	tex.sprint("\\node[violet](h) at (640,2320) {$\\mathsf{3p_{\\frac{1}{2}}}$};")

	tex.sprint("\\node[violet] (j) at (719,1650) {\\footnotesize 3s};")
	tex.sprint("\\node (m) at (990,4073) {MNN-Auger};")

	end

	function drawAGDD()

	px,py = fs.readcsv("data/silber/nachkratzen3d.txt",0)


	tex.sprint("\\addplot[".."darkblue".."] coordinates{")
		for i=1,#px do
		tex.sprint("("..px[i]..","..py[i]..")")
		end
	tex.sprint("};")
	tex.sprint("\\node[violet] (et) at (368,1550) {$\\mathsf{3d_{\\frac{5}{2}}}$};")
	tex.sprint("\\node[violet] (ft) at (374,1150) {$\\mathsf{3d_{\\frac{3}{2}}}$};")


	end




\end{luacode*}


\newcommand\addplotpr{\directlua{drawAG()}}
\newcommand\addplotdd{\directlua{drawAGDD()}}



\definecolor{skyblue1}{RGB}{135, 206, 250}
\definecolor{flame}{RGB}{226, 88, 34}
\definecolor{scarletred1}{RGB}{252, 40, 71}
\definecolor{cyanblau}{RGB}{0, 158, 224}
\definecolor{darkblue}{RGB}{0, 0,139}
\definecolor{charcoal}{rgb}{0.21, 0.27, 0.31}
\definecolor{turquoise}{rgb}{0 0.41 0.41}
\definecolor{rouge}{rgb}{0.79 0.0 0.1}
\definecolor{vert}{rgb}{0.15 0.4 0.1}
\definecolor{mauve}{rgb}{0.6 0.4 0.8}
\definecolor{violet}{rgb}{0.58 0. 0.41}
\definecolor{orange}{rgb}{0.8 0.4 0.2}
\definecolor{bleu}{rgb}{0.39, 0.58, 0.93}
\definecolor{azulen}{RGB}{0,218,255}



\begin{document}

\pgfplotsset{
	standard/.style={
		axis x line*=bottom,
		axis y line=left,
		ylabel= Intensität ,
		xlabel= Bindungsenergie,
		x unit= \si{\electronvolt},
		legend style={at={(0.8,1)},anchor=north},
		y label style={at={(axis description cs:-.1,.5)},anchor=south},
		x tick label style={rotate=90,anchor=east, font=\tiny},
		tick label style={/pgf/number format/assume math mode=true,font=\sffamily},
		yticklabel style={/pgf/number format/fixed,scale=0.6},
		scaled y ticks=false,
		legend cell align=left,
		x dir=reverse
	}
}
%\pgfplotsset{tick style={very thin,gray}}
%
%\begin{figure}
%\begin{tikzpicture}[every node/.style={scale=0.6}]
%  \begin{axis}[standard, enlarge x limits={abs=0.2cm},minor x tick num=1, xlabel=$\tilde{\nu}$]
%    \addplotdd
%    \end{axis}
%  \end{tikzpicture}

%	\caption{Das hochaufgelöste XPS-Spektrum des AG-Drahts nach dem \textit{Sputting}}
%\end{figure}

\begin{figure}[!htpb]
	\centering 
	\resizebox{21cm}{14cm}{%
		\begin{tikzpicture}[every node/.style={transform shape}, every node/.style={scale=0.6}]
			\begin{axis}[
					standard,
					enlarge x limits={abs=0.2cm},
					minor x tick num=1,
					ymax=8000,
					/tikz/line join=bevel]
				\addplotpr
				\end{axis}
			\end{tikzpicture}
		}
	\caption{Das gesamt XPS-Spektrum von $Ag^0$}
	\end{figure}


\begin{figure}[!htpb]
	\centering 
	\resizebox{21cm}{14cm}{%
		\begin{tikzpicture}[every node/.style={transform shape}, every node/.style={scale=0.6}]
			\begin{axis}[
					standard,
					enlarge x limits={abs=0.2cm},
					minor x tick num=1,
					ymax=2000,
					/tikz/line join=bevel]
				\addplotdd
				\end{axis}
			\end{tikzpicture}
		}
	\caption{Das hochaufgelöste XPS-Spektrum von $Ag^0$ um $3d_{3/2}$ $3d_{5/2}$}
	\end{figure}


	\end{document}
