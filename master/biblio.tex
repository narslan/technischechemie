\begin{thebibliography}{100} % 100 is a random guess of the total number of
%references
  \bibitem{adichie}Adichie, \emph{Chimamanda N. The Danger of a Single Story.} TED Talk, gefilmt von TED Global, 2009. 
\bibitem{alsamarai} Al Samarai, Nicola L. „Inspired Topography: Über/Lebensräume, Heim-Suchungen und die
Verortung der Erfahrung in Schwarzen deutschen Kultur- und Wissenstraditionen.“
In \emph{Mythen, Masken Subjekte. Kritische Weißseinsforschung in Deutschland}, herausgegeben von Maureen M.Eggers, Grada Kilomba, Peggy Piesche, Susan Arndt, Münster:Unrast Verlag, 2005.
\bibitem{alcoff} Alcoff, Linda M. und Elizabeth Potter. „Introduktion: When Feminisms Intersect Epistemology. In
\emph{Feminist Epistemologies}, herausgegeben von Linda M. Alcoff und Elizabeth Potter, London: Routledge, 1993.
\bibitem{alcolo} Alcoff, Linda M. „Phenomenology, Post-structuralism, and Feminist Theory on the Concept of
Experience.“ In \emph{Feminist Phenomenology}, herausgegeben von Linda Fisher und Lester Embree, Dordrecht: Kluwer Academic Publishers, 2000.
\bibitem{alcollin} Alcoff, Linda M. „Enrique Dussels Transmodernism.“ \emph{Transmodernity, Journal of Peripheral
Cultural Production of the Luso-Hispanic World}, Vol 1, Iss. 3 (2012).
\bibitem{onami} Bar On, Bat Ami. „Marginality and Epistemic Privilege.“ In
\emph{Feminist Epistemologies},
herausgegeben von Linda M. Alcoff und Elizabeth Potter, London: Routledge, 1993.
\bibitem{bilstein} Bilstein, Johannes, und  Helga Peskoller, Herausgeber\_innen von
\emph{Erfahrung, Erfahrungen}, Wiesbaden: Chronos Verlag, [1968] 2013.
\bibitem{bodin} Bodin, Capucine, James Cohen und Ramón Grosgoguel. „Introduction: From University to
Pluriversity: A Decolonial Approach to the Present Crisis of Western
Universities.“ \emph{Human Architecture: Journal of the Sociology of Self
Knowledge} Vol.10 Issue 1 (2012).
\bibitem{bollnow} Bollnow, Otto F. „Der Erfahrungsbegriff in der Pädagogik.“ In
\emph{Erfahrung, Erfahrungen},
herausgegeben von Johannes Bilstein und Helga Peskoller, Wiesbaden: Chronos Verlag, [1968] 2013.
\bibitem{bristol} Bristol, Laurette S. M. \emph{Plantation Pedagogy. A
Postcolonial and Global Perspektive}. New York:
Peter Lang, 2012.
\bibitem{canning} Canning, Kathleen. „Feminist Discourse after the Linguistic Turn.Historicizing Discourse and
Experience.“ \emph{Signs}, Vol. 19, No. 2 (1994): 368-404
\bibitem{cannkathleen} Canning, Kathleen. „Problematische Dichotomien. Erfahrung zwischen Narrativität und
Materialität.“ In \emph{Erfahrung: Alles nur Diskurs? Zur Verwendung des
  Erfahrungsbegriffes in der Geschlechtergeschichte. Beiträge zur 11.
Schweizerischen HistorikerInnentagung}, herausgegeben von Marguérite Bos, Bettina Vincenz und Tanja Wirz, Zürich: Chronos Verlag, 2004.
\bibitem{catrov} Castro Varela, María do Mar, und Nikita Dhawan. \emph{Postkoloniale Theorie. Eine kritische
Einführung}. Bielefeld: transkript, 2015.
\bibitem{castrova} Castro Varela, María do Mar, und Nikita Dhawan. „Postkolonialer Feminismus und die Kunst der
Selbstkritik.“ In \emph{Spricht die Subalterne deutsch? Migration und
Postkoloniale Kritik},
herausgegeben von Hito Steyerl und Encarnacion Rodriguez, Münster: Unrast Verlag, 2003.
\bibitem{castromar} Castro Varela, María do Mar: „Verlernen und die Strategie des unsichtbaren Ausbesserns. Bildung 
und Postkoloniale Kritik.“ \emph{Bildkunst. Zeitschrift der IG Bildende Kunst},
download unter:
\url{http://www.igbildendekunst.at/bildpunkt/2007/widerstand-macht-wissen/varela}, am 26.11.2014
\bibitem{code} Code, Lorraine. „Experience, Knowledge, and Responsibility.“ In
\emph{Feminist Perspectives} in
Philosophy, herausgegeben von Morwenna Griffith und Margaret Withford, London: The Macmillan Press, 1988. 
\bibitem{darowska} Darowska, Lucyna und Claudia Machhold. Herausgeber\_innen.
\emph{Hochschule als transkultureller
Raum? Kultur, Bildung und Differenz in der Universität}. Bielefeld: transkript, 2011. 
\bibitem{lauretis} De Lauretis, Teresa. „Eccentric Subjects: Feminist Theory
and Historical Concousness.“ \emph{Feminist Studies}, Vol 16, No. 1 (1990): 115-150 
\bibitem{derrida} Derrida, Jacques. \emph{Die unbedingte Universität}. Frankfurt am Main: Suhrkamp, 2001.
\bibitem{dhawan} Dhawan Nikita. „Zwischen Empire und Empower. Dekolonisierung und Demokratisierung“ 
\emph{Femina Politica. Zeitschrift für Feministische Politikwissenschaft}. (2009).
\bibitem{dinkel} Dinkel, Jürgen. „'Dritte Welt.' Geschichte und Semantiken.“Version 1.0 Dokupedia Zeitgeschichte,
     2014. 
\bibitem{duden} Duden, Barbara. „Somatisches Wissen, Erfahrungswissen und 'diskursive' Gewissheiten.
Überlegungen zum Erfahrungsbegriff aus der Sicht einer Körper-Historikerin. In
\emph{Erfahrung: Alles nur Diskurs? Zur Verwendung des Erfahrungsbegriffes in
  der Geschlechtergeschichte. Beiträge zur 11. Schweizerischen
HistorikerInnentagung}, herausgegeben von Marguérite Bos, Bettina Vincenz und Tanja Wirz,  Zürich: Chronos Verlag, 2004.
\bibitem{dussel} Dussel, Enrique. „Transmodernity and Interculturality. An Interpretation from the Philosophy of
Liberation, \emph{Transmodernity: Journal of Peripheral Cultural Production
of the Luso-Hispanic World}, Vol. 1 Nr. 3, (2012).
\bibitem{eggers} Eggers, Maureen M., Grada Kilomba, Peggy Piesche und Susan
Arndt  et al. Herausgeberinnen. \emph{Mythen, Masken Subjekte. Kritische
Weißseinsforschung in Deutschland}. Münster: Unrast Verlag, 2005.
\bibitem{waldraut} Ernst, Waldraut. \emph{Diskurspiratinnen. Wie Feministische
  Erkenntnisprozesse die Wirklichkeit verändern}.Wien: Milena Verlag, 1999.
  \bibitem{fanon} Fanon, Frantz. \emph{Schwarze Haut, Weiße Masken}. Frankfurt am Main: Suhrkamp, 1992. 
  \bibitem{linda} Fisher, Linda. „Feminist Phenomenology.“ In \emph{Feminist   Phenomenology}, herausgegeben von ebd. und Lester Embree, Dordrecht: Kluwer Academic Publishers, 2000.
  \bibitem{linfisher} Fisher, Linda. „Phenomenology and Feminism. Perspectives
  on their Relation. In \emph{Feminist Phenomenology}, herausgegeben von ebd. und Lester Embree,  Dordrecht: Kluwer Academic Publishers, 2000. 
  \bibitem{foucault} Foucault, Michel. \emph{Die Ordnung der Dinge}. Frankfurt am Main: Suhrkamp Verlag, [1966 frz] 1974.
  \bibitem{foucdisp} Foucault, Michel. \emph{Dispositive der Macht. Über
  Sexualität, Wissen und Wahrheit}. Berlin: Merve Verlag, 1978.
  \bibitem{foucmensch} Foucault, Michel. \emph{Der Mensch ist ein
  Erfahrungstier.Gespräch mit Ducio Trombardio}, Frankfurt am Main: Suhrkamp Verlag, [1980] (1996).
  \bibitem{foucaultsex} Foucault, Michel. \emph{Sexualität und Wahrheit. Bd. 2
  Der Gebrauch der Lüste}.  Frankfurt am Main: Suhrkamp Verlag, [frz. Original 1984] 1986.
  \bibitem{kunningham} Gabel-Kunningham, Barbara et al. Vorwort der Übersetzer
  und Übersetzerinnen für \emph{Kritik der Postkolonialen Vernunft. Hin zu
  einer Geschichte der verrinnenden Gegenwart},von Gayatri C. Spivak, Stuttgart: Kohlhammer Verlag, 2013
  \bibitem{gilroy} Gilroy, Paul. \emph{The Black Atlantic. Modernity and Double Conscousness}. London: Verso, [1993] 1999.
  \bibitem{giuliani} Giuliani, Regula. „Körpergeschichte zwischen Modellbildung und haptischer Hexis. Thomas
  Laqueur und Barbara Duden. In \emph{Phänomenologie und Geschlechterdifferenz}, herausgegeben von Silvia Stoller und Helmuth Vetter, Wien: WUV Universitätsverlag, 1997.
  \bibitem{grosfou} Grosfoguel, Ramón. „Decolonizing Western Uni-versalisms:
  Decolonial Pluri-versalisms from AiméCésaire to the Zapatistas.“ \emph{Transmodernity, Journal of Peripheral Cultural Production of the Luso-Hispanic World}, Vol 1, Iss. 3, (2012). 
  \bibitem{grosfugel} Grosfoguel, Ramón. „The Structure of Knowledge in Westernized Universities: Epistemic
Racism/Sexism and the Four Genocides/Epistemicides of the Long 16th Century.“
\emph{Human Architecture: Journal of the Sociology of Self-Knowledge:} Vol. 11: Iss. 1, (2013). 
\bibitem{harding} Harding, Sandra. „Rethinking Standpoint Epistemology. 'What
Is Strong Objectivity' ?“ In \emph{Feminist Epistemologies}, herausgegeben von Linda M. Alcoff und Elizabeth Potter,  London: Routledge, 1993.
\bibitem{collins} Hill Collins, Patricia. \emph{Black Feminist Thought.
Knowledge, Consciousness and the Politics of Empowerment}. New York: Routledge, 2000.
\bibitem{hooksb} hooks, bell. \emph{Talking Back. Thinking feminist, thinking black}. Boston: South End Press, 1989
\bibitem{hookbell} hooks, bell. \emph{Teaching to Transgress. Education as the Practice of Freedom}. New York: Routledge, 1994.
\bibitem{laundry} Laundry, Donna, und Gerald MacLean. Einleitung zu \emph{The Spivak Reader. Selected Works of
Gayatri Chakravorty Spivak.} Herausgegeben von ebd., New York: Routledge, 1996.
\bibitem{levesque} Levesque Lopman, Louise. \emph{Claiming Reality. Phenomenology and
Women's Experience}. Totowa, New Jersey, Rowman and Littlefield Publishers, 1988
\bibitem{lloyd} Lloyd, Genevieve. Das Patriarchat der Vernunft. \emph{„Männlich“ und „weiblich“ in der westlichen
Philosphie}. Bielefeld: Daedalus Verlag,1985. 
\bibitem{lohman} Lohman, Ingrid,  Sinah Mielich, Florian Muhl, Karl-Josef Pazzini, Laura Rieger und Eva Wilhelm. 
\emph{Schöne neue Bidlung? Zur Kritik der Universität der Gegenwart}. Bielefeld: transkript, 2011.
\bibitem{clintock} McClintock, Anne, Aamir Mufti und Ella Shohat,
Herausgerber\_innen. \emph{Dangerous Liaisons.
Gender, Nation and Postcolonial Perspectives}. Minneapolis: University of Minnesota Press, 1997.
\bibitem{mecholal} Mecheril. Paul, Oskar Thomas-Olalde, Claus Melter, Susanne Arens und Elisabeth Romaner,
Herausgerber\_innen. \emph{Migrationsforschung als Kritik? Konturen einer
Forschungsperspektive}. Wiesbaden: Springer Verlag, 2013.
\bibitem{mechpaul} Mecheril, Paul. „Der doppelte Mangel, der das Schwarze Subjekt hervorbringt.“ In \textit{Mythen, Masken Subjekte. Kritische Weißseinsforschung in Deutschland}, herausgegeben von Maureen M.Eggers, Grada Kilomba, Peggy Piesche, Susan Arndt, Münster:Unrast Verlag, 2005.
\bibitem{mechpol} Mecheril, Paul. \emph{Politik der Unreinheit. Ein Essay über die Hybridität}. Wien: Passagen-Verlag, 2003
\bibitem{mechbirte} Mecheril, Paul und  Birte Klingler. „Universität als
transgressive Lebensform. Anmerkungen, die gesellschaftliche Differenz- und
Ungleichverhältnisse berücksichten.“ In \emph{Hochschule als transkultureller
Raum? Kultur, Bildung und Differenz in der Universität}, herausgegeben von L. Darowska et al. Bielefeld: transkript, 2011. 
\bibitem{mignolo} Mignolo, Walter D. \emph{Epistemischer Ungehorsam. Rhetorik
der Moderne, Logik der Kolonialität und Grammatik der Dekolonialität}. Wien: Turia + Kant, 2006.
\bibitem{mohanty} Mohanty, Chandra T. „Feminist Encounters. Locating the
Politics of Experience.“ In \emph{Destabilizing
Theory. Contemporary Feminist Debates}, herausgegeben von Michèle Marett und Anne Phillips, Cambridge: Polity Press, 1992. 
\bibitem{mohchan} Mohanty, Chandra T. „On Race and Voice.Challenges for Liberal
Education in the 1990s.“ \emph{Cultural Critique}, No. 14, (1989-1990): 179-208
\bibitem{narayan} Narayan, Uma, und Sandra  Harding. \emph{Introduction to
Decentering the Center}, herausgegeben von ebd., Indiana University Press: Bloomington, 2000
\bibitem{nederveen} Nederveen Pieterse, Jan und Bhikhu Parek. \emph{The Decolonization of Imagination. Culture,
Knowledge and Power}. London:  Zed Books. 1995.
\bibitem{ruemelin} Nida-Rümelin, Julian. \emph{Der Akademisierungswahn. Zur Krise beruflicher und akademischer Bildung}. Hamburg: Ed. Körber Stiftung, 2014.
\bibitem{nuenning} Nünning, Ansgar. „Wie Erzählungen Kulturen erzeugen. Prämissen, Konzepte und Perspektiven für
eine kulturwissenschaftliche Narratologie.“ In \emph{Kultur-Wissen-Narration.
Perspektiven transdisziplinärer Erzählforschung}, herausgegeben von Andrea Strohmaier,  Bielefeld: transkript, 2013.
\bibitem{peskoller} Peskoller, Helga „Erfahrung/en.“  In \emph{Erfahrung,
Erfahrungen}, herausgegeben von  Johannes Bilstein und Helga Peskoller, Wiesbaden: Chronos Verlag, [1968] 2013.
\bibitem{quix} quix-kollektiv für kritische bildungsarbeit. Gender\_Sexualitäten\_Begehren in der Machtkritischen und Entwicklungspolitischen Bildungsarbeit. Wien:ebd., 2017.
\bibitem{rich} Rich, Adrienne. „Notes Towards a Politics of Location“ Essay, Ort unbekannt, 1984.
\bibitem{riley} Riley, Denise. \emph{'Am I That Name?' Feminism and the Category of 'Women' in History}. London: Palgrave Mcmillan, [1988] 1993.
\bibitem{rodriguez} Rodriguez, Encarnacíon G. „Repräsentationen, Subalternität
und postkoloniale Kritik.“ In \emph{Spricht die Subalterne deutsch? Migration
  und Postkoloniale Kritik, herausgegeben von Hito Steyerl und Encarnacion
Rodriguez}, Münster: Unrast Verlag, 2003.
\bibitem{fegter} Fegter, Susann, und Nadine Rose. „Herstellung von Legitimität. Zum Rekurs auf Erfahrung in der 
Lehre.“ In \emph{Differenz unter Bedigungen von Differenz. Zu Spannungsverhältnissen universitärer Lehre}, 
herausgegeben von Paul Mecheril, Susanne Arens, Susann Fegter, Britta Hoffarth, Birte Klingler, Claudia Machhold, Margarete Menz, Melanie Plößer, Nadine Rose., Wiesbaden: Springer Verlag, 2013. 
\bibitem{saupe} Saupe, Achim und Felix Wiedemann. „Narration und Narratologie. Erzähltheorien in der
Geschichtswissenschaft.“ Version 1.0 \emph{Docupedia-Zeitgeschichte} (2015). 
\bibitem{scott} Scott, Joan. „Experience“ In \emph{Feminists Theorize the Political}, herausgegeben von Judith Butler und Joan Scott, New York: Routledge, 1992.
\bibitem{sium} Sium, Aman, Chandni Desai und Eric Ritskes. „Towards the
'tangible unknown': \emph{Decolonization  and the Indegenous Future.
Decolonization: Indigeneity, Education and Society}, Vol.1, Nr. 1, (2012): I – XIII
\bibitem{smith} Smith, Linda Tuhiwai. \emph{Decolonizing Methodologies.
Research and Indigenous Peoples}. London: Zed Books, 1999.
\bibitem{spivak} Spivak, Gayatri C. \emph{A Critique of Postcolonial Reason. Towards a History of the Vanishing Present.} Cambridge: Harvard University Press, 1999.
\bibitem{steyrl} Steyerl, Hito, und Encarnacion Rodriguez. Einleitung in \emph{Spricht die Subalterne deutsch? Migration und Postkoloniale Kritik}, heraugegeben von ebd. Münster: Unrast Verlag, 2003.
\bibitem{stoller} Stoller, Silvia und Helmuth Vetter, Herausgeber\_innen. \emph{Phänomenologie und Geschlechterdifferenz}. Wien: WUV Universitätsverlag, 1997.
\bibitem{stone} Stone-Mediatore, Shari. „Chandra Mohanty and the Revaluing of
Experience.“ In \emph{Decentering the Center, herausgegeben von Uma Narayan und Sandra Harding}, Bloomington: Indiana University Press, 2000. 
\bibitem{shari} Stone-Mediatore, Shari. \emph{Reading across Borders. Storytelling and Knowledges of Resistance}. New York: Palgrave MacMillian, 2012.
\bibitem{trouillot} Trouillot, Michel-Rolph. \emph{Silencing the Past. Power and the Production of History}, Boston: Beacon Press, 1995.
\bibitem{weedon} Weedon, Chris. \emph{Feminist Practice and Poststructuralist Theory}. New York: Basil Blackwell, 1987.
\bibitem{wimmer} Wimmer, Michael. „Die Agonalität des Demokratischen und die Aporetik der Bildung. Zwölf
Thesen zum Verhältnis zwischen Politik und Pädagogik.“ In \emph{Schöne neue
Bildung? Zur Kritik der Universität der Gegenwart}, herausgegeben von Ingrid Lohman, Sinah Mielich, Florian Muhl, Karl-Josef Pazzini, Laura Rieger un Eva Wilhelm, Bielefeld: transkript, 2011.
\end{thebibliography}
