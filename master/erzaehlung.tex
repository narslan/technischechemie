\section{Erzählung}
\epigraph{
The Black Atlantic developed from my uneven attempts to show these students, that the experience of black people were part of the abstract modernity they found so puzzling and to produce as evidence some of the things that black intellectuals had said – sometimes as defenders of the West, sometimes as its sharpest critics- about their sense of embeddedness in the modern world.
}{Paul Gilroy\footnotemark} \footnotetext{Gilroy,\textit{The Black Atlantic.
Modernity and Double Consciousness},IV.}

Die Unterteilung der Welt in Zentrum und Peripherie wird von postkolonialen
Theoretiker\_innen als eine der maßgebendsten Praktiken der sogenannten
Europäischen Moderne gefasst. Maßgebend in zweierlei Hinsicht. Mit Blick auf
die gewaltvolle Herrschaft der kolonisierenden Länder über die kolonisierten
Länder und mit Blick auf die mit dieser Herrschaftsform einhergehende
dualistische Wissensproduktion, die sich gewiss nicht ohne
Widerstand\footnotemark \footnotetext{Widerstand als allgegenwärtiges Moment
des kolonialen Verhältnisses – bereits widerständige Lesart, da das Phantasma
der Fügsamkeit der Kolonisierten in Ausbeutungsverhältnisse dadurch gebrochen
wird. Vgl.: Troillot, \textit{Silencing The Past. Power and the Production of
History}.} aber
dennoch mit enormer Kraft in das Denken der kolonisierten wie auch
kolonisierenden Gesellschaften und ihren Subjekten einschrieb und
schreibt.\footnotemark \footnotetext{Frantz Fanon, \textit{Schwarze Haut, Weiße
Masken}, Frankfurt: Suhrkamp, 1992 und Smith, \textit{Decolonizing
Methodologies}.} Im letzten Kapitel konnte ich aufzeigen, dass hierbei
insbesondere die Erfindung des 'Menschen' von einer tiefen Spaltung gezeichnet
ist, in der sich vermeintlich Subjekte und Objekte gegenüberstehen. Die Idee
vom Menschen ist damit stets an einen Spiegel gebunden, der einer die eigene
Negation vorhält und erst so die Möglichkeit der Selbstkonstitution eröffnet.
Diese Spaltung, das hat die Auseinandersetzung mit dem Erfahrungsbegriff in der
emanzipatorischen Epistemologie gezeigt, eröffnet den dermaßen verschieden
positionierten Subjekten jedoch auch äußerst verschiedene Spiegelbilder oder
anders, Erfahrungsräume die unterschiedliches Wissen hervorbringen. Subjekte
können sich der Spaltung zwar nicht entziehen, sie können ihre Erfahrungen
jedoch im Verhältnis zu dieser bestimmen, reflektieren und so Wissen erzeugen
das andere Spiegel- und Selbstbilder ermöglicht. Dazu bedarf es nicht nur eines
kritischen Blickes darauf wie sich Dualismen in das eigene Denken einschreiben,
sondern auch, wie jene Dualismen historisch gewachsen sind und so das Denken
mitsamt seinen vermeintlichen Grenzen formen konnten. Dieser Blick in die
Geschichte des Denkens muss, das habe ich im letzten Kapitel aufgezeigt, die
Gewalt der kolonialen Eroberung fokussieren aus der jene Spaltung entspringt.
Das Wissen, das als legitimiertes Wissen aus dieser Gewalt heraus entstanden
ist und die Spaltung des Menschen in Subjekte und Objekte immerwieder von Neuem
vollzieht, schreibt sich dabei auch in die Geschichten ein, die von und über den Menschen erzählt werden.

Im Verlaufe dieses Kapitels möchte ich auf jene Perspektiven Bezug nehmen,
welche diese Praktik der Unterteilung in Zentrum und Peripherie als Erzählung
theoretisieren. Die im vorherigen Kapitel explizierte andro- und eurozentrische
Wissenschaftspraxis, ermöglicht, so die These, eine Erzählung in der sich
Europa nicht nur als Mittelpunkt der Welt, sondern darüber hinaus als
Ursprungsort eines Denkens inszeniert, welches tatsächlich Ausdruck einer
gewaltvollen Begegnungs- und Beziehungsgeschichte war und ist. Meine
Bezugnahme von Repräsentations- und Erzähltheorie zeigt hier auf, dass es sich
bei der Europäischen Moderne nicht um irgendeine Erzählung handelt, sondern in
direkter Abhängigkeit zur hegemonialen Episteme um eine hegemoniale Erzählung.
Dabei wird die Europäische Moderne auf der Ebene ihrer diskursiven Selbst- und
Außenkonstruktion untersucht und der Zusammenhang bzw. die Verschränkung von
Erzählung und Wissensproduktion verdeutlicht. Im Vordergrund steht dabei für
mich nach wie vor die Frage nach der Erfahrung: Welchen Ort erhält Erfahrung?
Wie wird Erfahrung gedacht? Kann Erfahrung für eine dekoloniale Intervention
nutzbar gemacht werden?

Gilroy greift die Bedeutung von Erfahrung auf, wenn er wie im oben stehenden
Zitat zu seinen Schwarzen Student\_innen spricht, und diese auffordert, sich als
Teil der \glqq abstrakten Moderne \grqq \footnotemark \footnotetext{Gilroy,
\textit{The Black Atlantic.}} zu begreifen die ihnen bisweilen \glqq rätselhaft
\grqq \footnotemark \footnotetext{Ebd.}
erscheint. Letzteres verdeutlicht die narrative Macht, die dem kolonialen
Projekt entspringt. Wie sonst kann den Studierenden etwas abstrakt und
rätselhaft erscheinen, das in vielfacher Hinsicht von ihren Erfahrungen und den
Erfahrungen ihrer Vorfahr\_innen konstituiert ist? 

Rätsel haben bei aller Unterschiedlichkeit gemein, dass sie etwas verbergen.
Die Erzählung über die sogenannte Europäische Moderne, das konnte ich in den
letzten Kapiteln herausstellen, verbirgt nicht nur die Gewalt aus der sie
entsprungen ist, sie verbirgt zudem, dass sie nur auf Grund dieser Gewalt
möglich werden konnte. So ist es nicht verwunderlich, dass einer\_m etwas
rätselhaft erscheint, wenn dieses etwas um eine geteilte, gewaltvolle Erfahrung
kreist, die weder beschrieben noch benannt wird. Und es ist noch weniger
verwunderlich, dass diese Ent\_nennung, das Misstrauen jener weckt, die in
dieser Erzählung kategorisch ausgeschlossen werden, wie von Gilroy beschrieben.

Wenn im folgenden die Europäische Moderne, die viel genauer die von Europa
vereinnahmte Moderne genannt werden müsste, als Erzählung theoretisiert wird,
dient dies nicht einer Leugnung der stattgefundenen Gewalt. Eine Erzählung
findet nicht anstatt einer Wirklichkeit statt, sie ist viel eher ein Werkzeug,
das aus der Gewalt entspringt und sie zugleich legitimiert und die darum für
eine Untersuchung interessant ist.\footnotemark \footnotetext{Vgl. Bristol,
\textit{Plantation Pedagogies.}} 

Gilroys Forderung, die tatsächliche Eingebundenheit der Ausgeschlossenen
anzuerkennen, ist Teil eines postkolonialen und feministischen
Theorieprojektes, das es sich zur Aufgabe macht, von und zu dem Außen zu
sprechen, auf das sich Europa in seinem Bestreben, das Zentrum der Welt zu
sein, heimlich bezieht. Dadurch wird die Spaltung, durch die sich die
Europäische Moderne konstituiert, unterwandert und die Europäische Moderne als
Prozess der wechselseitigen Hervorbringung von Zentrum und seinem Außen
theoretisiert. Die Spaltung wurde insbesondere in der feministischen
Epistemologie als Spaltung im Geschlechterverhältnis begriffen.
Dekolonisierung, das Vorhaben die  Europäische Moderne zu Dezentrieren bzw.
Europa zu \glqq provinzialisieren \grqq \footnotemark
\footnotetext{Chakrabarty, \textit{Europa als Provinz.}} problematisiert jedoch nicht nur die Logik der
Zweigeschlechtlichkeit sondern greift darüber hinaus Rassismus als inhärenten
Bestandteil der Europäischen Moderne auf und an. Postkolonialer Feminismus
versucht hier den Dualismus von Körper- Geist, Subjekt-Objekt zu überwinden in
dem es aufzeigt, wie vermeintlich Gegensätzliches zusammen gedacht werden kann
und so Erfahrungen vom Menschsein in das Wissen eingeschrieben werden können,
die im dominierenden Diskurs negiert wurden. Die kontroverse Auseinandersetzung
um Erfahrung in der Wissensproduktion kann hier als ein konkreter Versuch
verstanden werden, emanzipatorische Epistemologien aufzubauen.

Theoretiker\_innen haben in ihrem Versuch, kulturelle Phänomene
nachzuvollziehen, jene Phänomene auf verschiedene Art und Weise
konzeptionalisiert. Was heißt es nun, das soziale Phänomen, der Europäischen
Moderne als Erzählung zu theoretisieren?

In dieser Arbeit, das ist bereits vielfach angeklungen, dominiert eine
diskursive Ebene als theoretischer Zugang. Dies birgt zugleich den Anspruch,
dass das Materielle, Verkörperte etc. auch in dem Diskursiven Ausdruck gewinnen
bzw. durch das Diskursive zugänglich gemacht werden kann. Problematisch ist
hierbei, dass eine Sprache in der Narrativ, Erzählung, Diskurs, Repräsentation
etc. zu den Kerntermini werden es unter Umständen nicht schafft, die Präsenz
der Gewalt, also ihre Gegenwärtigkeit in dem Leben der Menschen, zu
artikulieren.

Die Auseinandersetzung mit Erfahrung kann hier jedoch als Brücke zwischen dem
Materiellen und Diskursiven verstanden werden, da Erfahrungen zum Beispiel im
Anschluss an Duden\footnotemark \footnotetext{Vgl. Duden, \glqq Somatisches
Wissen\grqq. }, als Ausdruck somatischer Erlebnisse verstanden werden
können, die durch ihre Artikulation Einzug in das Diskursive erhalten bzw.
darin intervenieren. Insbesondere Dudens Konzept scheint sich daher für eine
postkoloniale Kritik am Dualismus zu eignen. Dabei muss, wie schon im letzten
Kapitel herausgestellt, stets achtsam damit umgegangen werden, welches Subjekt
imaginiert und damit auch welchen Erfahrungen Raum gegeben wird um die
ausschließende Logik hegemonialen Wissens nicht fortzuschreiben. 

Bevor ich mich, wie eben angekündigt, mit der Bedeutung von Erfahrung für die
Praxis des (Gegen)Erzählens beschäftige, möchte ich mich zunächst noch in
allgemeinerer Form dem Begriff der Erzählung widmen. 

Die Disziplin, die ich hierfür heranziehe ist die Narratologie. Dabei
beschränke ich mich auf Auseinandersetzungen im deutschsprachigen Raum und gehe
nicht auf den, mit dem postmodernen Aufschwung einher gegangenen, und
insbesondere durch Jean-François Lyotard bekannt gewordenen Begriff des Grand
Récit, der Großen Erzählung ein, da es sich hier um ein zwar interessantes,
aber sicherlich den Rahmen dieser Arbeit sprengendes Diskursfeld handelt.

Die Narratologie ist auch ohne die französischen, postmodernen Intellektuellen
ein weites Feld. Deren Aufschwung und Niedergang in Mitten der Disziplinen wird
dabei recht unterschiedlich erzählt. Insbesondere die sogenannten Feinde der
Narratologie könnten gegensätzlicher nicht sein: So wird einmal die Hegemonie
der Naturwissenschaften\footnotemark \footnotetext{Achim Saupe und Felix
Wiedemann, \glqq Narration und Narratologie. Erzähltheorien in der
Geschichtswissenschaft,\grqq Version 1.0, in \textit{Docupedia-Zeitgeschichte}, (2015), 1.} angeführt, ein anderes Mal dem Poststrukturalismus die
Schuld für das sinkende Interesse an der Narratologie in die Schuhe
geschoben\footnotemark \footnotetext{Ansgar Nünning, \glqq Wie Erzählungen Kulturen
erzeugen. Prämissen, Konzepte und Perspektiven für eine kulturwissenschaftliche
Narratologie \grqq in \textit{Kultur-Wissen-Narration. Perspektiven transdisziplinärer
Erzählforschung}. Herausgegeben von Andrea Strohmaier,  (Bielefeld. Transkript,
2013), 16.} Gemeinsam ist jenen Erzählungen jedoch, ihre geteilte Erleichterung
gegenüber dem Come Back der Narratologie in den vielfältigsten Disziplinen. Die
Spanne reicht von literaturwissenschaftlichen Fragestellungen nach der
Erzählstruktur, über psychologische Untersuchungen der narrativen Identität, zu
geschichtswissenschaftlichen Auseinandersetzungen um das Spannungsfeld von
Fiktionalität und Faktizität angesichts der narrativen Aufbereitung von
historischen Ereignissen. Auch wenn es nicht wenig unterhaltsam ist, sich mit
den verschiedenen Niedergangs- und Behauptungsgeschichten der Narratologie zu
befassen, muss an dieser Stelle davon abgesehen werden. Stattdessen möchte ich
mich mit einem Teilbereich der Narratologie befassen, nämlich der
kulturwissenschaftlichen Erzählforschung. Grund für meine Präferenz der
Kulturwissenschaften ist ihre Konzeption der sogenannten
\glqq Wirklichkeitserzählung \grqq \footnotemark \footnotetext{Nünning, 
\glqq Wie Erzählungen Kulturen erzeugen,\grqq  27.}. Anders als andere Zugänge gerät hier das
Wechselspiel, in dem sich das Grand Narrative mit den Partikularen Geschichten
befindet, in den Fokus. Ein zentraler Vertreter der kulturwissenschaftlichen 
Erzählforschung ist Ansgar Nünning. Nünning plädiert für eine Synthese kultur-
und erzählwissenschaftlicher Forschungsansätze.\footnotemark \footnotetext{Ebd. 28.}
Ausgangspunkt dieser Forderung
bilden dabei seine Auffassung, dass sich Kultur u.a. über Praxen des Erzählens
konstituiert bzw. dass Erzählungen immer als kulturelle und damit wandelbare,
an den zeit- und räumlichen Kontext gebundene Phänomene untersucht werden
müssen. Eine kulturwissenschaftliche Narratologie ermöglicht es in diesem Sinne
eine in der Vergangenheit vielfach ahistorisch praktizierte Narratologie für
die Bedeutung des Erzählkontextes sensibel zu machen. Für die
Kulturwissenschaften hingegen ergebe sich durch die Synthese weniger eine
Erweiterung ihrer theoretischen Perspektive, als vielmehr ein Impuls für einen
äußerst relevanten, vielfach jedoch ignorierten Gegenstandsbereich: Erzählungen
werden hier, so Nünnung, nicht auf Prosa reduziert, viel eher interessiert sich
die kulturwissenschaftliche Narratologie für sogenannte
'Wirklichkeitserzählungen' und versteht darunter jene über das literarische
Erzählens hinausgehende Praxen der Bedeutungskonstruktion. Erzählungen werden
hier durch zwei wesentliche Kennzeichen gefasst. Zum einen sind sie stets
Ausdruck \glqq spezifischer Verknüpfungen \grqq \footnotemark
\footnotetext{Saupe und Wiedermann, \glqq Narration und Narratologie,\grqq 2.}, zum anderen werden sie von einer
\glqq temporalen Struktur \grqq \footnotemark \footnotetext{Ebd.}  geformt, sodass sie als \glqq zeitlich strukturierte
Repräsentation von Ereignissequenzen\grqq \footnotemark \footnotetext{Ebd.} begriffen werden können. So könne die
Erzählung auch vom Diskursbegriff abgegrenzt werden, der sich stärker auf die
Ordnungsfunktion von Sprache beziehe und damit synchron angelegt sei, wogegen
sich der Erzählbegriff eher auf linear und temporale Dimensionen
beziehe.\footnotemark \footnotetext{Nünning, \glqq Wie Erzählungen Kulturen
erzeugen,\grqq 28.}
Erzählungen bilden damit eine Form des materialisierten, bzw. performierten
Ausdrucks gesellschaftlicher Ordnungen:
\begin{myenv}
  \textit{
  \glqq Zweifelsohne sind es Erzählungen die kollektiven, nationalen
  Gedächtnissen zugrunde liegen und Politiken der Identität bzw. Differenz
  konstituieren. Kulturen sind immer auch als Erzählgemeinschaften anzusehen,
  die sich gerade im Hinblick auf ihr narratives Reservoir unterscheiden \grqq
  \footnotemark \footnotetext{Wolfang Müller-Funk, \textit{Die Kulturen und ihre
Narrative. Eine Einführung}, (Wien: Springer Verlag, 2008), zitiert  in Nünning, ebd., 29.}
  } \end{myenv}

  Wolfgang Müller-Funk greift hier die politische Dimension kultureller
  Praktiken auf und weist daraufhin, dass jene Ordnungen von
  Auseinandersetzungen um Identität und Differenz geprägt sind, die als Kämpfe
  um Zugehörigkeit zu eben jenen kulturellen Gemeinschaften aufgefasst werden
  können. Die Frage der Macht bahnt sich auf diese Weise ihren Weg in die
  Narratologie und nimmt zugleich Gebrauch von deren formalistischen
  Ausprägungen. Das heißt, dass der Blick auf die Form für die Narratologie
  konstitutiv ist, da er die Möglichkeit bietet, über direkte Erzählbotschaften
  hinaus Aufschlüsse über die ideologische Dimension des Erzählten zu erhalten.
  Frederic Jamesons ideology of the form die davon ausgeht, dass die Art und
  Weise wie erzählt wird mehr über das Motiv der Erzählung aussagt, als eine
  bloße Aufmerksamkeit für den Inhalt es vermag, gilt als einer ihrer
  prominentesten Vertreter. An dieser Stelle wäre eine tiefer gehende
  Auseinandersetzung mit den Methoden der Narratologie bzw. ihrem
  Formverständnis möglich, für meine Arbeit jedoch nicht fruchtbar. Da meine
  Arbeit keine Anwendung dieser Methode beabsichtigt, sondern sich vielmehr der
  Narratologie bedient um den Begriff der Erzählung, wie er im postkolonialen
  Feminismus Verwendung findet, zu schärfen.

  \subsection{Ich-Erzählung als Gegenerzählung}

  Ich knüpfe nun wieder an die feministische Auseinandersetzung um den
  Erfahrungsbegriff in der Erkenntnisproduktion an, konkretisiere die
  Diskussion allerdings auf eine postkoloniale Praxis: die Ich-Erzählung als
  Gegenerzählung. Dazu setze ich mich mit den Möglichkeiten von Repräsentation
  und Dialog auseinander. Kann eine\_r sich selbst erzählen? Was braucht sie\_er
  dafür? Ab wann ist eine Ich-Erzählung eine Gegenerzählung?

  \subsubsection{Repräsentation}

  Eine Auseinandersetzung mit den Möglichkeiten und Grenzen von Repräsentation
  durchzieht diese Arbeit wie ein roter Faden. Bereits im Bildungsbegriff, der
  Wimmer folgend, auf einem Subjektverständnis ruht das niemals in der Lage ist
  sich selbst zu repräsentieren, da es immer auf die Entbehrung eines
  kohärenten Selbst zu Gunsten eines Zwischenraums zwischen Ich und Welt
  zurückgeworfen wird, wird die Problematik der Repräsentation aufgegriffen.
  Diese Problematik wird mit  Mohanty in Anbetracht eines Sprechens über bzw.
  für die 'Dritte Welt' weitergedacht, wenn sie die Herausforderungen
  beschreibt, in ihren Darstellungen nicht zugleich als Vertreterin eben jener
  Dargestellten missverstanden zu werden. Während es also bei Wimmer zunächst
  um die Repräsentation des Selbst geht, beschreibt Mohanty die Schwierigkeit
  andere zu repräsentieren.
  
  Auch die hegemoniale Episteme, die daraufhin von
  mir erörtert wird, beruht letztlich auf einem falschen Versprechen der
  Repräsentation: Sie gibt vor für alle zu sprechen,  in Wirklichkeit
  verkörpert sie jedoch nur die Interessen einer Minderheit.  Die Einordnung
  diesen Wissenssystems als hegemonial, beschreibt darüber hinaus die
  Schwierigkeit sich ihm zu widersetzen und damit aus dem herrschenden
  Repräsentationsregime auszubrechen, da es so stark auf  Zustimmung aufgebaut
  und in der Lage ist auch die Versuche des Widerstands für sich zu
  vereinnahmen.

  Versuche, der hegemonialen Idee des  Menschen eine anderes Fundament
  entgegenzustellen,  kommen hier nicht ohne Widerspruch aus. Emanzipatorische
  Strategien fallen oftmals auf eine von identitären Kategorien durchdrungene
  Sprache zurück – zu Lasten einer Ideen von Erfahrungen als Erfahrung von
  Differenz. Der Repräsentationsanspruch der mit identitären Kategorien, wie
  der Kategorie Frau\_* einhergeht wird hier von jenen Frauen\_* zurückgewiesen,
  die sich in den proklamierten 'weiblichen' Erfahrungen nicht wiedererkennen
  und sich diesem kategorialen Ausschluss widersetzen.

  Die Gegengeschichtsschreibung, oder Gegenerzählung kann nun als Versuch
  beschrieben werden, einen Zwischenraum zu öffnen der dem dissonanten Stimmen
  kein vorgegeben identitären Kategorien überstülpt. Gegenerzählungen  können
  auf diese Weise neue Wege der Repräsentation erkämpfen, die sich dem
  vereinnahmenden Praxen des Verstehens widersetzen und Erfahrung immer als
  Erfahrung von Differenz verstehen. Im Sinne eines strategischen
  Essentialismus, kann sich aus dieser Differenz heraus behauptet werden ohne
  auf die Differenz reduziert zu werden.
  

