\subsection{Hegemoniales Wissen}
\epigraph{\textit{ 
I need to understand how a place on the map is also place in history [...]
}}{Adrienne Rich \footnotemark} \footnotetext{
Adrienne Rich, „Notes toward a Politics of Location,“ unbekannt, 1984.
} 

Wie ein bestimmtes Wissen hegemonial werden und damit andere Wissensformen an
den Rand drängen konnte, ist Gegenstand folgender Auseinandersetzung um die
Grenzen des Denkbaren an der Universität.\\
Die Frage nach den Grenzen des
Denkbaren und danach, wie sie entstanden, verfestigt und gebrochen wurden
erfordert sich die \textbf{Geschichte} des Denkens an der Universität näher anzuschauen,
denn sie bildet den Ort von dem aus und über den ich spreche. Das heißt auch
die bereits angekündigt, konkreter darin zu werden, von welcher Universität in
Zeit und Raum ich ausgehe, aus welchem Standpunkt ich versuche sie zu
beschreiben und zu fragen, inwiefern die Geschichte dieser Universität zum
Gegenstand widerständiger epistemischer Praxen gemacht wird.\\
Kein
deterministisches Geschichtsverständnis, sondern ein Verständnis von Gegenwart,
das sich der Kontingenz ihrer selbst über die Auseinandersetzung mit ihrer
Geschichte nähert, und dabei versucht die Ausschlüsse, die jene Gegenwart
mit\_hervorgebracht haben, nachzuvollziehen, motiviert mich zu diesem Vorhaben. 
\subsubsection{Perspektiven auf hegemoniales Wissen}

Die Perspektiven, aus denen heraus ich mich mit  hegemonialem Wissen
beschäftigen möchte, habe ich bisher als sexismus- und rassismuskritisch
beschrieben. Die epistemische Praxis, in der jene Perspektiven Anwendung
finden,  benenne ich im Anschluss an Mohantys Ausführungen im vorherigen
Kapitel als Dekolonisierung.\\
Meine Auseinandersetzung damit wofür diese
Begrifflichkeiten stehen (können), soll hierbei statt einer einmaligen
Einführung fortlaufend praktiziert werden. Die Anwendung sexismus- und
rassismuskritischer Ansätze bedeutet in diesem Sinne niemals nur die Verwendung
einer Perspektive, sondern immer auch die Befragung ihrer Vorannahmen,
Interessen und Vorgehensweisen, also einem learning on the way. \\
Hierbei
interessiere ich mich insbesondere für die Verbindungslinien zum Postkolonialen
Feminismus, als eine kontextualiserte epistemologische Position, aus der heraus
Dekolonisierung praktiziert wird. Wie Erfahrung gedacht, nutzbar gemacht oder
auch verworfen wird, um eine Idee des 'Menschen' zu entwickeln, bildet hierbei
das Zentrum um das sich meine Diskussion kreist. Je nach theoretischer Schulung
wird sich auf den 'Menschen', das Subjekt oder auch die Identität bezogen.\\

\noindent Zunächst zu meiner Verwendung des Begriffspaars sexismus- und
rassismuskritisch:\\
Jene Perspektiven verhalten sich, anders als das 'und'
vermuten lässt, alles andere als komplementär zueinander. Intersektionalität,
ein politischer und wissenschaftlicher Ansatz zeigt indessen auf, dass die
Differenzverhältnisse die unter diesen Perspektiven analysiert werden,
ineinandergreifen.\\
Intersektionalität beansprucht dabei die Wirkungsweisen von Geschlecht und
'Race' nicht isoliert von anderen Differenzen zu betrachten, und fordert eine
Analyse, in der verschiedene, ineinander wirkende Differenzverhältnisse
berücksichtigt werden. \\
Stoller und Vetter machen hier darauf aufmerksam, dass
jene Differenzverhältnisse nicht als identitäre und damit feststehende
Kategorien missverstanden werden dürfen. Der Vorgriff auf Identität, der ihrer
Ansicht nach die traditionelle Philosophie kennzeichnet, führe unweigerlich
dazu, dass Alterität stets im Anderen, im Fremden gesucht werde und dabei
verkannt wird, dass die Differenz immer schon das Eigene konstituiert und ergo
auch im vermeintlich Unbekannten „immer schon das Bekannte zu finden ist.“
\footnotemark \footnotetext{Silvia Stoller und Helmuth Vetter, „Einleitung,“ in \textit{Phänomenologie und Geschlechterdifferenz}, herausgegeben von ebd. (Wien: WUV Universitätsverlag, 1997), 8. }\\

\noindent Postkolonialer Feminismus kann nun als eine theoretische Position verstanden
werden, die jene Differenzverhältnisse im kolonialen Verhältnis situiert und in
ihrer Verzahnungen mit Wissensproduktion untersucht. Damit rücken
epistemologische Fragen in den Vordergrund, die feministische und postkoloniale
Perspektiven zusammen denken.\\

\noindent Feministische Epistemologie hinterfragt die Möglichkeit eines neutralen Wissens
a priori und geht davon aus, dass der Kontext bzw. der Wille aus dem heraus das
Wissen entsteht, maßgebend ist. Hinter jedem Wissen das produziert wird, steht
also nicht nur ein situierter Blickwinkel sondern auch ein spezifisches
Interesse, das den\_die, der\_die das Wissen erzeugt, leitet.\\
 Ein grundlegendes
Spannungsfeld, das mit dieser radikalen Befragung nicht nur des Erzeugten,
sondern auch des\_der Erzeugenden einhergeht, fragt nun nach dem Verhältnis von
Subjektivität zu Objektivität bzw. Relativität zu Universalität und den
Konsequenzen, die dies für die Relevanz, ja die Sprengkraft eines Wissens mit
sich bringt.\footnotemark \footnotetext{Linda M. Alcoff und Elizabeth Potter:
  „Introduction. When Feminisms Intersect Epistemology,“ in
\textit{Feminist Epistemologies}. Herausgegeben von ebd. (London: Routledge, 1993), 1.} \\
Die Politisierung von Wissensproduktionen als machtvolle Gefüge,
die immer auch von vergeschlechtlichten (sexed) Subjekten aus einem
spezifischen gesellschaftlichen Staus heraus gemacht werden, greift damit die
grundlegenden Strukturen der männlich dominierten Wissenschaft an, indem sie
das benennt, was bis dahin als unsichtbare Größe die Neutralität des Wissens
beanspruchen konnte: Das Zentrum wird in Frage gestellt.\footnotemark
\footnotetext{Alcoff und Potter, „Introduction,“ 3.}\\

\noindent Postkolonialer Feminismus kann nun gewissermaßen als Korrektiv einer
feministischen Epistemologie gelesen werden, die die „subjekttheoretische
Überlegungen [...] lediglich als dekontextualisierte erkenntnistheoretische
Fragestellungen[...]“ \footnotemark \footnotetext{Hito Steyerl und
Encarnacion Rodriguez, „Einleitung,“ in \textit{Spricht die
Subalterne deutsch?  Migration und Postkoloniale Kritik}, herausgegeben von
ebd. (Münster:Unrast Verlag, 2003), 9.} betrachtet. \\
Postkolonialer Feminismus fordert dazu auf die kolonialen
Diskurse in den Blick zu nehmen, die u.a. die Kategorie Geschlecht bestimmen
und „analysiert die Gewalt des Westens“ \footnotemark
\footnotetext{Castro Varela, und Dhawan „Postkolonialer Feminismus und die
  Kunst der Selbstkritik“ in \textit{Spricht die Subalterne deutsch? Migration und
  Postkoloniale Kritik}, herausgegeben von Hito Steyerl und Encarnacion Rodriguez
(Münster: Unrast Verlag, 2003), 271.} in der Herstellung der 'Anderen
Frauen\_*' über die sich die unmarkiert '\textit{w}eiße Frau\_*' konstituiert.\\
Castro Varela und Dahwan stellen Postkolonialen Feminismus hier als
„widerständigen, antihegemonialen Gegendiskurs dar, der sich insbesondere gegen
epistemologische Ausschlussverfahren richtet.“ \footnotemark
\footnotetext{Castro Varela und Dhawan, „Postkolonialer Feminismus“, ebd.}\\

\noindent Meine Präferenz für die Position des Postkolonialen Feminismus liegt nun darin
begründet, dass sie, wie es der Anspruch auf Reflexivität vorsieht, sich nicht
außerhalb dessen versteht, was sie zu untersuchen bestrebt. Postkolonialer
Feminismus ist mit dem Gegenstand, nämlich kolonialer und patriarchaler
Strukturen der Europäischen Moderne verwoben und es ist eben jene
Auseinandersetzung mit der eigenen Involvierung, die diese Position für mich,
für ein an Reflexivität orientiertes Vorhaben, interessant macht. \\
Die
Thematisierung von der sogenannten unvermeidbaren Komplizenschaft mit jenem,
das es zu überwinden gilt, macht in Spivaks Sinne Mut, sich nicht
außerhalb, sondern innerhalb der Struktur zu begreifen, die eine kritisiert,
und mit denen sie doch „aufs engste vertraut ist.“ \footnotemark 
\footnotetext{Spivak, zitiert in Castro Varela und Dhawan,
\textit{Postkolonialismus. Eine Kritische Einführung}, 202.}\\

\noindent Wie eingangs formuliert, gehe ich jedoch nicht davon aus, Lösungen zu finden.
Mich interessiert, welche Möglichkeiten Postkolonialer Feminismus für eine
Suche nach dem Nicht\_denkbaren an der Universität bietet: Welchen Beitrag
leisten Konzeptionen von Erfahrung(swisssen) für eine Kritik an hegemonialen
Wissensbeständen? Ist die Überschreitung des Mensch(lichen) möglich?

\subsubsection{Entstehung und Durchsetzung hegemonialen Wissens}

Nachdem ich nun in Ansätzen die Perspektive aus der ich auf hegemoniales Wissen
blicke vorgestellt habe, beschäftige ich mich nun aus jener Perspektive mit der
Konkretisierung des Ortes über den ich spreche und der bisher mit 'der
Universität' zeit- und räumlich unbestimmt blieb. Dazu beziehe ich mich mich
auf wissenschaftliche Beiträge einer Konferenz, die vor nicht allzulanger Zeit
in Paris stattfand und den Titel: „Quelles universités et quels
universalismes demain en Europe? Un dialogue aves les Amériques“ \footnotemark \footnotetext{Boidin et al., „From University to
Plurisversity “ 2012.} trug. Während der Titel auf das zukünftige
verweist und nach den Universitäten und Universalismen von Morgen in Europa
fragt, bestand der Ausgangspunkt dieser Konferenz in der Dekolonisierung der
bisherigen 'Westernized University' mitsamt ihren eurozentrischen
Wissensstrukturen.\\ 

\noindent Die Konferenz proklamierte dabei das Ziel einer dekolonialen
Intervention in akademische Wissensproduktion und verwestlichte
Universitätsstrukturen. Die Organisator\_innen gingen, ähnlich wie die in dieser
Arbeit eingangs vorgestellten Perspektiven, davon aus, dass eine
kapitalismuskritische Analyse nicht ausreicht, um die Ursachen der Krise der
Universität zu bestimmen. Kritik an neoliberalen Vereinnahmungen wie sie
beispielsweise in Zusammenhang mit Bologna diskutiert werden\footnotemark
\footnotetext{Vgl. Lohman et al. Herausgerber\_innen, \textit{Schöne Neue
Bildung.}} greifen ihrer
Ansicht nach nicht tief genug, da sie die Bedeutung rassistischer und
sexistischer Praxen der Dehumanisierung/Entmenschlichung als konstitutiven
Bestandteil in der Geschichte der Universität verkennen.\\

\noindent Meine Frage von
welcher Universität in Zeit und Raum ich ausgehe, wird durch ihre
Konferenzbeitrage dabei um zwei Dimensionen konkretisiert:\\
 Zum einen fassen die
Autor\_innen Uni-versität in Abgrenzung zur Pluri-versität und beziehen sich
damit auf den auf Universalität ausgerichteten Wissensanspruch der Eurpäischen
Moderne als Entstehungskontext von Uni-versität. Zum anderen beschreiben sie
die Bedingungen unter denen Uni-versität entstanden ist: Europas Rolle in
Epistemiziden und ihre Mündung in solipsistischen und dualistischen
Denkstrukturen. \\
Wie bereits im Zusammenhang mit der Konturierung des
Postkolonialen Feminismus erwähnt, wird Wissensproduktion auch hier in einem
durch koloniale Herrschaft strukturierten Raum verortet. Diesen zu explizieren,
ist nun das Anliegen von Ramón Grosfoguel, einem der Vortragenden, und wird im
folgenden erörtert.\footnotemark \footnotetext{ Ramón Grosfoguel, „The
Structure of Knowledge in Westernized Universities: Epistemic Racism/Sexism and
the Four Genocides/Epistemicides of the Long 16th Century, “ in \textit{Human
Architecture: Journal of the Sociology of Self-Knowledge}: Vol. 11: Iss. 1,
Article 8., (2013): VIII.} Dabei steht zunächst der Zusammenhang, der zwischen
den sogenannten (kolonialen) Epistemiziden und einer solipsistischen und
dualistischen Denkstruktur hergestellt wird, im Vordergrund.\\ 

Grosfoguel bezieht sich in seinen Ausführungen zum Epistemizid der Europäischen
Moderne auf vier Genozide bzw. Epistemizide die allesamt im 17. Jahrhundert
verübt wurden. Diese Epistemizide richteten sich, so Grosfoguel, gegen die
muslimische und jüdische Bevölkerung der spanisch\_iberischen Halbinsel
Al-Andalus, gegen die indigene Bevölkerung der Amerikas und Asiens, gegen die
Bevölkerung Westafrikas, die versklavt und verschleppt wurde und gegen
europäischen Frauen\_*, die als Hexen verfolgt wurden.\\
Grosfoguel betont in
seiner Darstellung die Notwendigkeit, die Zusammenhänge dieser, meist als
voneinander getrennt untersuchten Phänomene, zu markieren und die rassistische
Logik darin als konstitutiv für die Etablierung einer westeuropäischen
Vorherrschaft und ihres Selbstverständnisses zu begreifen: 
\begin{myenv}
  „\textit{The attempt here is to see them [the epistemicides] as interlinked,
 inter-related to each other and as constitutive of the modern/colonial world’s
 epistemic structures. These four genocides were at the same time forms of
 epistemicide that are constitutive of Western men epistemic privilege. To
 sustain this argument we need to not only go over the history but also explain
 how and when racism emerged.“\footnotemark}\footnotetext{Grosfoguel „The
 Structure of Knowledge “, 77.}
\end{myenv}
 Seine historische Darstellung ist entsprechend durch das Anliegen geprägt, die
 Denkstrukturen, die o.g. Epistemizide ermöglichten und legitimierten,
 aufzudecken.\footnotemark \footnotetext{Vorab möchte ich dabei bereits
 einschränkend darauf hinweisen, dass Grosfoguel keine differenzierte
 historische Rekonstruktion der angesprochenen Genozide auflistet. Statt einer
 umfangreichen Deskription kennzeichnet sich seine Auseinandersetzung eher
 durch eine punktuelle Analyse in der insbesondere Verbindungslinien innerhalb
 der Genozide hervorgehoben werden. Seine Analyse ist jedoch insbesondere darum
 interessant, weil er die koloniale Eroberungspolitik mit der rassistischen und
 sexistischen Gewaltausübung auf dem europäischen Festland in Verbindung bringt
 bzw. letztere als konstitutiv für die koloniale Gewaltherrschaft expliziert.}\\

\noindent Grosfoguel skizziert dazu zunächst die historischen Entwicklungen im heutigen
Spanien um Fünfzehnhundert. Die Eroberung der iberischen Halbinsel Al-Andaluz
durch die spanische Monarachie, die mit einer gewaltvollen Unterwerfung der
dort lebenden muslimischen Bevölkerung einherging, verfolgte, so Grosfoguel,
die Etablierung einer 'spanisch-christlichen' Vorherrschaft über die dort
lebenden Menschen. Dabei hätte die Idee der Einheit von Staat und Bevölkerung,
im Sinne einer direkten Korrespondenz von subjektiver und kollektiver
Identität, eine zentrale Rolle gespielt.\footnotemark
\footnotetext{Grosfoguel, „The Structure of Knowledge,“ 79.} Die mit
der gewaltsamen Eroberung der iberischen Halbinsel einsetzende
Zwangsassimilierung der muslimischen Bevölkerung, käme einem kulturellen
Genozid gleich, da hier eine kulturelle Selbstbestimmung, sowie
intergenerationale Weitergabe kultureller Vorstellungen und Praktiken massiv
unterbunden wurde.\footnotemark \footnotetext{Ebd.}\\

\noindent Theoretisch fasst Grosfoguel diesen Epistemizid in Abgrenzung zu Rassismus als
religiöse Diskriminierung, da im Unterschied zu Rassismus die Menschlichkeit
der unterdrückten Bevölkerung nicht in Frage gestellt worden sei.\footnotemark
\footnotetext{Ebd.} An dieser Stelle verweist er auf den direkten Zusammenhang, der zwischen der Eroberung
der iberischen Halbinsel und dem kolonialen Projekt bestand: Die sogenannnte
'Indian Enterprise' die Christopher Columbus dem König und der Königin des
castilianischen Königreiches vorstellte, sei nur unter der Voraussetzung einer
vollständigen Eroberung der iberischen Halbinsel, bewilligt worden.\\

\noindent Die direkte Abfolge der Eroberung Al- Andalus mit der Eroberung der Amerikas,
ist dabei laut Grosfoguel bisher nur wenig erforscht.\footnotemark
\footnotetext{Ebd.} Im Weiteren zeigt er
auf, wie sowohl die Eroberung von Al-Andalus, als auch die Eroberung der
Amerikas durch einen epistemischen Genozid getragen wurden, der die
vollständige Zerstörung der Wissensarchive wie beispielsweise Bibliotheken und
Schriftsysteme zum Ziel hatte.\footnotemark
\footnotetext{Ebd., 80}  Der Epistemizid wurde, so Grosfoguel dabei
durch die evangelikale Missionierung sowohl in Al-Andalus als auch in der
Amerikas vorangetrieben, die die Vernichtung von insbesondere spirituellen
Formen des Wissens vorantrieb. Mit der Kolonialisierung der Amerikas habe
hierbei erstmals auch eine rasisstische Logik eingesetzt: „But with the
colonization of the Americas, these old medieval discriminatory religious
discourses mutated rapidly, transforming into modern racial domination.“
\footnotemark \footnotetext{Ebd.,82}\\

\noindent Religiosität kam in dieser rassistischen Logik, so argumentiert Grosfoguel
weiter, eine zentrale Bedeutung zu. Denn die mit der Kolonialisierung
einhergehende rassistische Unterteilung in Menschen, denen Menschlichkeit zu-
bzw. abgesprochen wurde, sei unmittelbar an die Praktizierung von Religion
gebunden worden. Diejenigen mit der falschen Religion (muslimische und jüdische
Bevölkerung der iberischen Halbinsel) konnten missioniert werden, denjenigen
ohne Religion (indigene Bevölkerung der Amerikas) wurde das Menschsein an sich
abgesprochen.\footnotemark \footnotetext{Ebd.} 
Diese Unterteilung hätte jedoch niemals absoluten Charakter
gehabt sondern sei immer auch von Stimmen innerhalb der christlichen Kirche
infrage gestellt worden.\\
Diese 'kritischen' Stimmen argumentierten, dass die Indigenen zwar keine
Religion, aber dennoch eine Seele hätten und aus ihrem 'Barbarismus' mittels
einer Missionierung befreit werden könnten.\footnotemark \footnotetext{Ebd.}
Eine Versklavung wäre damit eine
Sünde und würde von Gott bestraft werden. Der Disput innerhalb der christlichen
Kirche, ist für Grosfoguel das erste schriftliche Anzeichen nicht nur von
Rassismus, sondern damit unmittelbar verknüpft, von der Erfindung der
Identität: „This debate was the first racist debate in world history and
'Indian' as an identity was the first modern identity.“\footnotemark
\footnotetext{Grosfoguel schreibt hierzu genauer: „[...] at the time, the
debate about having a soul or not was already a racist debate in the sense used
by scientific racism in the 19th century. The theological debate of the 16th
century about having a soul or not had the same connotation of the 19th century
scientificist debates about having the human biological constitution or not.
Both were debates about the humanity or animality of the others articulated by
the institutional racist discourse of states such as the Castilian Christian
monarchy in the 16th century or Western European imperial nation-states in the
19th century. These institutional racist logics of 'not having a soul' in the
16th century or 'not having the human biology' in the 19th century became the
organizing principle of the international division of labor and capitalist
accumulation at a worldscale. “ ebd.} \\
Grosfoguel skizziert an dieser Stelle
die Folgen des Disputes zwischen Bartolomé und Sepulveda für die
Eroberungspolitik und Versklavung der indigenen Bevölkerung der Amerikas und
der Versklavung und Verschleppung der afrikanischen Bevölkerung in die
Amerikas, sowie den Effekt, den die rassistischen Epistemizide auf die Politik
gegenüber den ursprünglich sich muslimisch-jüdisch identifizierenden und später
christlich konvertierten Menschen z.B. aus der iberischen Halbinsel
hatte.\footnotemark \footnotetext{Ebd., 85.} \\

\noindent Die von Spivak und Foucault
problematisierte Kategorie 'Mensch' wird von Grosfoguel hier in ihrem
kolonialen Enstehungszusammenhang expliziert. Er macht deutlich, dass die
Entscheidungsmacht darin, wem das Menschsein zu- oder abgesprochen wurde, bei
den kolonialen Herrschern bzw. den von ihnen beauftragten christlichen
'Gelehrten' lag. \\
Überraschend ist für mich an dieser Stelle, dass seiner
Ansicht nach die Erfindung von Identität, also die Fixierung des Menschlichen
auf konkrete Wesensmerkmale in der Fremdzuschreibung fungierte. Indem er die
„Indian Identity “ als erste moderne Form der Identität beschreibt, zeigt er
auf, dass diejenigen, die sich selbst unmissverständlich als Menschen
begriffen, die Idee der Identität jenen aufzwangen, die sich aus ihrer Sicht
das Menschsein erst noch, durch die Annahme der (richtigen) Religion, verdienen
mussten.\\

\noindent Die Auslöschung von Wissen, das hemonial gemachtes Wissen verunsichern konnte,
wurde ebenfalls durch 'Hexenverbrennung'  versucht. Die Verbrennung von
Frauen\_*, die durch die Inquisition in Europa als Hexen verfolgt wurden wird
von Grosfoguel in Anlehnung an die marxistisch-feministische Theoretikerin
Silvia Federici, mit der frühkapitalistischen Expansion in Zusammenhang
gebracht. Federici argumentiert, dass die Verbrennung von Frauen\_* als Hexen ebenso wie die Versklavung afrikanischer Menschen Effekte kapitalistischer Expansionslogik waren:  
\begin{myenv}
  \textit{„the African enslavement in the Americas [and] the witch hunt of
    Women in Europe [were] two sides of the same coin: capital accumulation at a
    world-scale in need of incorporating labor to the capitalist accumulation
  process“\footnotemark \footnotetext{Federici in Grosfoguel, ebd., 86.} }
\end{myenv}
Der Epistemizid richtete sich gegen tausende Frauen\_*, deren „Autonomie,
Führungskompetenzen und Wissen“ \footnotemark \footnotetext{Ebd.} 
eine Gefahr für die transnationale Etablierung
eines kapitalistischen Klassensystems darstellte.\\
Im Gegensatz zu dem Epistemizid an Indigenen der Amerikas und der muslimischen
und jüdischen Bevölkerung der iberischen Halbinsel, konnte die Inquisition
keine Bücher verbrennen, da die Weitergabe von Wissen in der Regel durch
mündliche Überlieferung stattfand. „The 'books' were the women’s bodies and,
thus, similar to the Andalusian and Indigenous 'books' their bodies were burned
alive“\footnotemark \footnotetext{Grosfoguel, ebd.}\\

\noindent Zuletzt diskutiert Grosfoguel die Auswirkungen der Genozide auf die
Monopolisierung des Wissens auf den europäischen Mann\_* nördlich der Pyrenäen,
infolgedessen die 'Westernized University' bis heute ihren universellen
Anspruch verteidigt.\footnotemark \footnotetext{Ebd., 89.}\\

\noindent Wissen, das wird mit Grosfoguel deutlich, das bis heute seine Legitimität
bewahren konnte, hat eine hegemoniale Stellung durch die gewaltvolle Zerstörung
konkurrierender Wissenssysteme erhalten können und entwickelte sich analog zu
der imperialen und kolonialen Etablierung eines transnationalen Kapitalismus,
mitsamt seiner rassistischen und patriarchalen Strukturlogik. 
Mit Grosfoguel wird die Zeit und der Raum der Universität von der ich ausgehe um die
gewaltsame Geschichte europäischer Herrschaft und der der damit einhergehenden
Monopolisierung von Wissen konkreter.\\

\noindent Das Wissen, das an der Universität als legitimiertes Wissen entstehen konnte,
ist in Zeiten und an Orten entstanden, an denen nicht nur Menschen unterworfen
und ermordet wurden, die dem \textit{w}eißen europäischen Mann\_* und dessen Bild vom
Menschen nicht entsprachen, sondern mit ihnen ihre Archive und damit ihre
Bilder und Vorstellungen vom Mensch-sein. Mit den Worten von Encarnacíon
Gutiérrez Rodrigues: „Die Expansion Europas verfolgte nicht nur die Ausbeutung
und Aneignung von Arbeit, Ressourcen und Land, sondern auch eine politische und
kulturelle Unterwerfung der kolonisierten Menschen.“\footnotemark
\footnotetext{Encarnacíon G. Rodriguez „Repräsentationen, Subalternität und
  postkoloniale Kritik,“ in \textit{Spricht die Subalterne deutsch? Migration und
  Postkoloniale Kritik}, herausgegeben von Hito Steyerl und Encarnacion Rodriguez,
(Münster: Unrast, 2003), 19.}
\subsubsection{Methodologie hegemonialen Wissens}
Mit Grosfoguel habe ich die geschichtlichen Möglichkeitsbedingungen euro- und androzentrischen Wissens nachvollzogen. Denn die eben beschriebenen geopolitischen, ökonomischen, historischen und kulturellen Prozesse haben nicht nur andere Wissenssysteme zerstört, sondern durch diese Zerstörung auch eine spezifische Erfahrung ermöglicht, die zur Grundlage für Wissensproduktion wurde. Rodriguez formuliert dies folgendermaßen:
\begin{myenv}
  \textit{
  „
[....] neben der territorialen Annektierung, der Ausbeutung von Ressourcen und des Genozids an der indigenen Bevölkerung ging Kolonialismus mit einer 'Neu-Schreibung' der Kolonien einher, die auf der Grundlage des Erfahrungshintergrundes und der Wissenstradition der Kolonisatoren stattfand“ }\footnotemark\footnotetext{ Rodriguez, „Repräsentation, Subalternität und postkoloniale Kritik,“ 21.}
\end{myenv}
Wie aus dieser Erfahrung der Kolonisatoren eine Wissenschaft mitsamt einer
Methodologie gewachsen ist, möchte ich im folgenden mit Enrique Dussel
erläutern. \\
Seine Ansicht, nach der erst die koloniale Expansion und Herrschaft
das europäischen Selbstverständnis mit all seiner Allmacht- und
Allwissensfantasie ermöglichte, wird auch von weiteren Theoretiker\_innen
geteilt. Todorov und Spivak sowie weitaus früher die Négritude Bewegung haben
argumentiert, dass das europäische (Wissenschafts)verständnis auf Kolonisierung
aufbaut: Erst durch die „Schaffung […] der 'Neuen Welt' als
Erkenntnisobjekt wurde die 'Kolonialmacht' im Namen des autonomen Subjekts als
regierendes und wissendes Subjekt geschaffen.“\\
Die koloniale Expansion
hat also, so lässt sich mit Rodriguez vermuten, ein kollektives Subjekt
geschaffen das sich durch die Erfahrung der Kolonisierung konstituiert und
dabei all jene mit einschließt, die sich der Kolonialmacht von nah und fern
zugehörig fühlen.  \\
Worin der Zusammenhang zwischen kolonialer Expansion und dem
europäischen Wissenschafts- und Subjektverständnis dabei genau besteht, möchte
ich nun mit Dussel aufzeigen. Hierzu beziehe ich mich auch auf die Arbeiten von
Alcoff, die sich intensiv mit Dussels Positionen auseinandergesetzt hat.\\

Ein Subjekt, so die These von Dussel, das sich selbst als Zentrum der Welt
versteht, kann nur vor dem Hintergrund einer Eroberungsgeschichte dieser Welt
entstanden sein. Nur wer sich als Eroberer\_* dieser Welt versteht, kann den
Anspruch erheben, durch seinen Standort und seine Perspektive dermaßen über sie
verfügen zu können.\\
Erst die Erfahrung der Eroberung der Welt ermöglicht also
ein Denken, das sich im Zentrum dieser Welt verortet. \\
Die von Grosfoguel
beschriebenen Epistemizide haben in Westeuropa ein kollektives Subjekt
hervorgebracht, das von Dussel als „masterful ego“ \footnotemark
\footnotetext{Alcoff, „Enrique Dussels Transmodernism,“ in
  \textit{Transmodernity, Journal of Peripheral Cultural Production of the
Luso-Hispanic World}, Vol 1, Iss. 3, (2012):62.} bezeichnet wird. Dieser
„masterful ego“ zeichnet sich durch eine Verweigerung gegenüber einer
kritischen Befragung der Vorannahmen und Kontextbedingungen, die sein Denken
bestimmen, aus.\\
Dass über die konkrete Vernichtung von Bevölkerungsgruppen und
ihren Wissensbeständen hinaus, auch die Erinnerung an diese Gewaltausübung
verweigert wird, ermöglicht die Etablierung einer solipsistischen und
dualistischen Wissenschaft, in der 'die Anderen' im 'Eigenen' vereinnahmt
werden:
\begin{myenv} 
  \textit{„Such an epistemic solipsism is affected through
    subsuming  'the Other under the Same', or refusing to entertain the
    possibility that there is a plurality of reasonable founding premises and
    conceptual categories. When my particular standards of judgment […] becomes
    the universal, I can judge the Other under a cloak of neutral anonymity
    with no need for hermeneutic humility.“ \footnotemark \footnotetext{Alcoff, „Enrique Dussels
    Transmodernism,“ 63.} }
\end{myenv}
Vereinnahmung wird hier von Alcoff in Anlehung an Dussel als epistemische Strategie beschrieben,
in der die Möglichkeit verschiedener Prämissen, also Grundannahmen verweigert
wird. Grund- oder Vorannahmen bilden in der Philosophie den Anfangspunkt einer
Argumentation und bestimmen damit maßgeblich darüber, von wo aus und wie über
etwas nachgedacht wird. Wenn die eigenen Vorannahmen für universell gültig
erklärt werden, negieren sie die Möglichkeit anderer Ausgangspunkte und umgehen
damit auch den Anspruch, andere Perspektiven zuzulassen. Interessant ist hier
der Begriff der 'Hermeneutischen Bescheidenheit', der von Alcoff
eingebracht wird. \\
Bescheidenheit, ähnlich wie Zurückhaltung tritt stets dann
ein, wenn das eigene angesichts des großen Ganzen als gering, vielleicht schon
fast unwichtig, eingeschätzt wird. Bescheidenheit geht also die Einsicht
voraus, dass das, was eine hat oder denkt oder weiß, angesichts dessen, was es
darüber hinaus gibt oder geben könnte, gedacht oder gewusst wird, unwesentlich,
ja vielleicht auch belanglos sein könnte. Solipsismus und Dualismus, so
verstehe ich Alcoff, zeichnen sich durch eine Weigerung aus, Bescheidenheit
zu üben.\\

\noindent Im Folgenden möchte ich die Entwicklung diesen, dualistisch und solipsistischen
Erkenntnismodels angefangen mit Descartes als dessen Begründer, über seine
Fortführung durch Kant, Hegel und Marx, nachvollziehen und beziehe mich dabei
auf die Arbeiten von Grosfoguel.\\

Das von René Descartes begründete Subjekt-
und Wissenschaftsverständnis unterliegt, so Grosfoguel, der Grundannahme, dass
Geist und Körper voneinander getrennt seien. Dies ermöglicht eine Logik, nach
der der Geist nicht an einen Körper und damit auch nicht an einen spezifischen
Standort gebunden ist.\footnotemark \footnotetext{Grosfoguel,  „The Structure of Knowledge,“  74.}
Das Subjekt der Erkenntnis  „[...]which has no
sexuality, gender, ethnicity, race, class, spirituality, language [...]“
\footnotemark \footnotetext{Grosfoguel  „Decolonizing Western
Uni-versalisms: Decolonial Pluri-versalisms from Aimé Césaire to the
Zapatistas, “  in \textit{Transmodernity, Journal of Peripheral Cultural Production
of the Luso-Hispanic World}, Vol 1, Iss. 3, (2012):89.} tritt an die Stelle
Gottes, und übernimmt dessen Position, von der aus es alles sehen kann.
Grosfoguel nennt dies die Ablösung der  „Theo-Politics“ \footnotemark
\footnotetext{Grosfoguel,  „Decolonizing Western Uni-versalisms,“  88.} durch die
„Ego-Politics.“\footnotemark \footnotetext {Ebd.}\\
Das solipsistische Verständnis von Wissensproduktion besteht nun in der
Annahme, dass Erkenntnis allein durch einen Monolog der\_des Theoretiker\_in mit sich
selbst- und damit außerhalb jeglicher sozialer Beziehungen entsteht.\footnotemark \footnotetext{Ebd., 76.}\\

\noindent Der Solipsismus ist dabei unmittelbar an einen Universalismus gebunden, der von
Grosfoguel in zwei Ebenen der Universalität unterteilt wird: Zum einen
beansprucht Descartes eine uneingeschränkte Gültigkeit des hergestellten
Wissens, und damit Wissen für die Ewigkeit und alle Teile der Welt. Zum anderen
zeigt sich die Universalität in dem vom Körper losgelösten Subjektverständnis,
das ungebunden seiner geopolitischen Situiertheit, Wissen über die Welt
erzeugt. Während der bezüglich Zeit und Raum  uneingeschränkten Gültigkeit des
Wissens, in den folgenden Jahrhunderten auch innerhalb der hegemonialen
philosophischen Tradition widersprochen wurde, setzt sich, so Grosfoguel, das
entkörperte Subjektverständnis bis heute fort.\\

Bei Immanuel Kant sind, anders als bei Descartes, Zeit und Raum nicht außerhalb
des Subjektes, nicht von ihm losgelöst sondern, jedes Subjekt ist durch diese
Kategorien konstruiert, sie sind a priori Kategorien von Wissen, das heißt, sie
gehen Wissen voraus.  Dadurch, dass diese Kategorien alles Wissen bestimmen,
also universell sind, so Kant, kann eine intersubjektive Nachvollziehbarkeit,
ein Verständnis zwischen verschiedenen Subjekten erfolgen, da sie sich auf
dieselben Kategorien beziehen, bzw. durch sie konstituiert sind.\footnotemark
\footnotetext{Ebd., 90.} Das intersubjektive Wissen kann dann Wahrheit
beanspruchen und ist universell, es kann jedoch nie eine vollständige Wahrheit
über die Dinge beansprucht werden, da die tatsächliche Wahrheit immer in den
Dingen verborgen bleibt. \\
Dabei wird bei Kant der Eurozentrismus, der bei Descartes nur implizit
angenommen wird, explizit, da er eine klare Definition dessen liefert, wer das
Subjekt der transzendentalen Vernunft sein kann. Grosfoguel leitet dieses
Subjektverständnis aus den anthropologischen Arbeiten Kants ab, und stellt
fest, dass es \textit{w}eiße europäische Männer\_* sind. Alle auf die diese Kategorien
nicht zutreffen, sind für Kant von der Möglichkeit transzendentale Vernunft zu
erlangen ausgeschlossen.\footnotemark \footnotetext{Ebd.}
 Intersubjekive Nachvollziehbarkeit besteht also nur
unter \textit{w}eißen europäischen Männern\_*.\\

\noindent Die Trennung von Körper und Geist, wie von Descartes beschrieben, bleibt bei
Kant bestehen, das Gewand des Solipsismus ändert sich jedoch und wird von der
intersubjektiven Nachvollziehbarkeit abgelöst. Während Descartes abstrakter
Universalismus von Kant hinterfragt wird, also die Möglichkeit, dass Wissen
völlig entkörpert und ungebunden von Zeit und Raum gilt, besteht in Bezug auf
das Subjekt die Ansicht fort, dass der \textit{w}eiße europäische Mann\_* die gesamte
Menschheit repräsentiere. \\
Kosmopolitismus, die intersubjektive
Nachvollziehbarkeit von Wissen, ist in Wirklichkeit, so behauptet Grosfoguel,
ein europäischer Provinzialismus.\\

Mit Georg Friedrich Hegel findet nun eine Neubegründung der sogenannten
westlichen Philosophie statt. Zum einen plädiert er für eine Situierung des
Subjektes in Raum und Zeit.\footnotemark \footnotetext {Ebd., 91.}
 Zum anderen distanziert er sich vom Dualismus
Subjekt- Umwelt, und konstatiert eine Gleichsetzung von Subjekt und Objekt. Der
Transzendentalismus bei Kant wird infrage gestellt, da Hegel anders als Kant
nicht von Kategorien a priori ausgeht, sondern sie einer historischen Analyse
unterzieht. Dabei untersucht er ihre Entstehung und negiert nicht wie Kant die
Möglichkeit, die Dinge vollkommen zu erfassen, sondern geht von einem
dialektischen Prozess aus, durch den eine umfassende Wahrheit entspringt.\footnotemark \footnotetext {Ebd.}\\ 
Die Dialektik bei Hegel begreift Wahrheit als das Ganze und fragt nach dem
Prozess der dialektischen Bewegung von Gedanken, die sich stets vom Abstrakten
zum Konkreten hin bewegen. Die Entwicklung von Kategorien verläuft, so Hegels
Theorie, parallel zur Weltgeschichte, die in der letzten Konsequenz ein
Ergebnis von Kategorien sind.\footnotemark \footnotetext {Ebd.}\\
Abstrakte Kategorien werden auf ihrer Widersprüchlichkeit hin befragt und in
diesem Prozess entstehen weitere Kategorien, die das konkretisieren, was in der
Auseinandersetzung mit den abstrakten Kategorien entstanden ist. Durch diese
Bewegung vom Abstrakten zum Konkreten entsteht, so Hegel, absolutes Wissen, das
über Zeit und Raum hinaus gültig ist.\\
\textit{Abstract universals} birgen noch keine
Erkenntnis, sie sind leere Hüllen die erst noch durch eine Befragung und
Auseinandersetzung mit konkreteren Kategorien gefüllt werden. Erst diese
konkreteren Kategorien, die \textit{simple categories}, sind der Ausgang von
Erkenntnis.\footnotemark \footnotetext {Ebd.}\\
Die \textit{concrete Universals} sind Kategorien, die bereits durch eine
Auseinandersetzung mit ihren inneren Widersprüchlichkeiten, entstanden
sind.\footnotemark \footnotetext {Ebd.} Dabei werden alle Widersprüche aufgelöst und alle Differenzen in das eine
Ganze vereinnahmt. So entsteht das absolute Wissen, das Wissen allen
Wissens.\footnotemark \footnotetext {Ebd.} Das ist zugleich das Ende der Geschichte, da dieses absolute Wissen
durch nichts mehr erweitert werden kann.\\
 Grosfoguel bezeichnet dieses
Verständnis vom Ende der Geschichte und das absolute Wissen als Widerspruch zu
dem eigentlichen Anliegen Hegels, Wissen zu kontextualisieren. Also die
Trennung von Wissen und Kontext aufzubrechen, indem die Kategorien, die unser
Wissen bestimmen, historisiert werden. Das absolute Wissen hingegen kann als
eine neue Form des cartesianischen Universalismus bezeichnet werden, da diese
absolute Wissen für immer und überall Gültigkeit verspricht.\\


\noindent Der Unterschied zwischen Descartes und Hegel besteht nun darin, dass bei
Descartes die unendliche Universalität dem Wissen vorausgeht, wohingegen bei
Hegel das unendliche Wissen erst anhand einer Historisierung der Kategorien
erarbeitet wird. Dafür müssen wiederum die Kategorien der gesamtem Menschheit
analysiert werden.  Mit 'Menschheit' sind aber nicht alle Menschen gemeint. Hegel
begreift den Osten als Ursprung des Geistes, der in diesem Geiste stagniert
ist, und den Westen als die Weiterentwicklung diesen Geistes. Der Fortschritt,
das absolute Wissen kann entsprechend auch nur von \textit{w}eißen, christlichen,
europäischen Männern\_* beansprucht werden.\\
Damit werden auch die vielfältigen pluralen Kategorien letztlich auf die
Perspektiven dieser Männer\_*, deren Wissen so absolut ist, dass nichts mehr im
Außen bleibt, reduziert.  
\begin{myenv} 
  \textit{„As a result, the Cartesian and Kantian
  epistemological racism of abstract epistemic universalism (Type II), in which
the universal is defined on the basis of the particular (Western man), remains
intact in Hegel.“\footnotemark \footnotetext{Ebd., 92.} }
\end{myenv}
Andere philosophische Gemeinschaften und Wissensproduktionen werden abgewertet oder überhaupt nicht in Betracht gezogen.\\

Karl Marx wendet die Vorstellung Hegels, nach der die die Deduktion abstrakter
Kategorien ein Verständnis des Konkreten ermöglicht um, und priorisiert das
Konkrete, in seinem Fall die Produktionsbedingungen als Determinanten aus der
die abstrakten Kategorien entspringen.\footnotemark \footnotetext{Ebd.} Die materiellen Bedingungen einer
Epoche bedingen damit die abstrakten Vorstellungen. Damit ist für Marx das
Proletariat die Quelle der Kritik der Verhältnisse. Marx wie Hegel
kontextualisieren damit Wissensproduktion, Marx verortet sie nur anders als
Hegel nicht im Denken, sondern in den materiellen Produktionsbedingungen. Die
Bedingungen bestimmen in seiner Theorie das Bewusstsein, weshalb in diesem
Zusammenhang davon gesprochen wird, dass er Hegel vom Kopf auf die Füße stellt.\\
In Bezug auf die Subjektposition verortet Marx, anders als seine Vorgänger, das
Subjekt nicht im luftleeren Raum, sondern bezieht Klassenverhältnisse mit ein.
Dafür nimmt er die europäische Arbeiterklasse zum Ausgang um eine Theorie eines
politischen Systems zu entwickeln, das für die gesamte Menschheit Gültigkeit
verspricht: Kommunismus. Hier problematisiert er jedoch nicht, dass das Subjekt
von dem er ausgeht zugleich  „European, masculine, heterosexual, white,
Judeo-Christian, etc.“\footnotemark \footnotetext{Ebd., 93.} ist.\\

\noindent Damit führt Marx den epistemischen Rassismus fort. Das Subjekt das bisher
vollkommen unmarkiert Wissen produzierte, wird bei Marx durch den Proletarier
konkret. Die eurozentristische Logik, durch die die Existenz von
Subjektposition außerhalb Europas, sowie die Relevanz ihrer Wissensbestände
vollkommen negiert wird, bleibt aber auch bei Marx bestehen.\footnotemark \footnotetext{Ebd., 92.}  Dies wird
insbesondere an seiner Theorie der ökonomischen Entwicklung deutlich, in der er
davon ausgeht, dass nur die Kolonisierung durch die imperialen europäischen
Mächte die restliche Welt aus ihrer vermeintlichen Stagnation befreien und so
Produktionsverhältnisse erzeugen kann, die für die kommunistische
Weltrevolution die Voraussetzung bilden.\\

\noindent Zusammenfassend kann das Subjektverständnis, das dem cartesianischen Denksystem
entspringt und wie hier dargelegt mit Kant, Hegel und Marx fortgeschrieben
wurde, als Gründungsmythos der Europäischen Moderne interpretiert werden.\footnotemark \footnotetext{Mignolo in Grosfoguel, ebd., 88.} 
Dieser Mythos erzählt die Geschichte Europas im Licht der Unabhängigkeit und
Ungebundenheit ohne jegliche Beziehung zu einem Außen.\\
 In dieser Erzählung hat
Europa nicht nur Barbarei und Feudalismus besiegt eine Wissenschaft begründet
und demokratische Werte wieder entdeckt, sondern all dies vollkommen losgelöst
vom Rest der Welt bewerkstelligt.\footnotemark \footnotetext{Grosfoguel, ebd.}\\
 Das vermeintlich völlig isolierte Subjekt
auf dem jene solipsistische und dualistische Methodologie aufbaut, ist jedoch,
dass konnte ich mit Grosfoguel zeigen auf fundamentale Weise auf das Objekt
angewiesen, welches es negiert.\\
 Meine Auseinandersetzung mit Dualismus und
Solipsismus als konstitutive Bestandteile der cartesianischen Philosophie
verdeutlichen, dass die Moderne tatsächlich keine europäische, sondern eine
globale Moderne ist, die jedoch eine Methodologie entwickelt hat, welche es ihr
ermöglicht von der eigenen Eingebundenenheit abzusehen und die Erkenntnisse
zunächst für sich zu beanspruchen, um sie dann für universell gültig zu
erklären. Diese Negation führt zu einem gewaltvollen Ausschluss nicht nur von
Subjektivitäten und ihrem Handeln, sondern auch dessen Bedeutung für die
Erlangung von Erkenntis.\\

\noindent Mit Grosfoguel und Dussel habe ich aufzeigen können, dass erst die koloniale
Eroberung die Entwicklung des solipsistischen und dualistischen Wissens- und
Methodenverständnisses ermöglichte. \\
Die euro- und androzentrische
Wissensproduktion fußt auf einer gespaltenen Erfahrung, die an den Ausschluss,
die Negation, letztlich an die Vernichtung der Peripherie gebunden ist.\\
Descartes, Kant, Hegel und Marx negieren nicht nur die eigene Identität als
\textit{w}eiße, (christliche)\footnotemark \footnotetext{Zumindest bei Marx ist es
problematisch von christlich auszugehen, da sein jüdischer Vater auf Grund von
Antisemitismus zum Protestantismus konvertierte und Marx selbst Atheist und
Kritiker von Religion/Religiosität im Allgemeinen war. } Männer\_*, sie
negieren die Erfahrung von Eroberung, Verschleppung, Zerstörung, von all der
Gewalt, die mit der kolonialen Herrschaft einherging und von der sie wussten
und von der sie profitierten, wie auch sie von ihnen und ihrem selbstbezogenen
und ignoranten Subjektverständnis profitierte. \\
Universalismus der Aufklärung
baut somit auf einem rassifizierten und vergeschlechtlichten Subjektverständnis
auf - das \textit{w}eiß und männlich ist - bei einer gleichzeitigen Behauptung für alle
zu sprechen.
