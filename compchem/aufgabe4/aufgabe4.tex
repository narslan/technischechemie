\documentclass[12pt]{article}
\usepackage{amsmath,mathtools}
\usepackage[usenames,dvipsnames]{xcolor}
\usepackage[german]{babel}
\usepackage[utf8]{inputenc}
\usepackage[T1]{fontenc}
\usepackage{textcomp}
%\usepackage{libertine}
%\usepackage{helvet}

\usepackage{microtype}
% Minion and Myriad fonts
\usepackage[minionint,mathlf]{MinionPro}
\renewcommand{\sfdefault}{Myriad-LF}
\usepackage{siunitx}
\usepackage{fancyhdr}
\usepackage{sectsty}
\usepackage{setspace}
\usepackage{booktabs} % To thicken table lines
\usepackage[version=4]{mhchem}
\usepackage[draft]{graphicx}
\usepackage[labelfont=bf]{caption}
\usepackage{subcaption}
\usepackage{chemstyle}
\usepackage{tabularx}
\usepackage{tikzorbital}
\usepackage{chemfig}

%\usepackage[compatibility=4.7,language=german]{chemmacros}
\renewcommand{\familydefault}{\sfdefault}
\sisetup{detect-all}
\usepackage{chngcntr}
\counterwithin{table}{section}
\counterwithin{figure}{section}

\usepackage{titlesec}
\renewcommand*\printatom[1]{\ensuremath{\mathsf{#1}}}
\titleformat*{\section}{\large\bfseries}
\titleformat*{\subsection}{\normalsize\bfseries}

\usepackage{geometry}
 \geometry{
 a4paper,
 left=20mm,
 top=30mm,
 right=20mm
 }

\begin{document}

\begin{onehalfspace}

\section{Potentialkurve von \ce{N_2}}
Ziel dieser Aufgabe ist es, die Potentialkurve des Stickstoffmoleküls zu berechnen und anschließend die Orbitale
 visualisiert und ein MO-Diagramm erstellt werden.
\subsection{Berechnung der Potenzialkurve auf Hartree-Fock Niveau}
\textbf{Auswertung 1 und 2}
\begin{table}[!htpb]
\centering
\caption{}
\begin{tabularx}{\textwidth}{lllclll}
\toprule
Molekül &
Methode &
Basissatz &
\parbox[t]{2cm}{Gitter\\ \scriptsize{(Min, Max, Incr)}} &
Symmetrie &
GGW-Abstand &
GGW-Energie $E _h$ \\
\midrule
\ce{N _2} & RHF & 6-311G(d,p) & 0.6, 4.6, 0.2 \si{\angstrom}& $D _{\infty h}$ & 1 \si{\angstrom} & -108.95140449 \\
\bottomrule
  \label{table:morse}
\end{tabularx}
\end{table}
\begin{figure}[!htpb]
\centering
  \fbox{\includegraphics[width=\textwidth]{data/morse.png}}
  \caption{Potenzialkurve für Stickstoff auf RHF/6-311G(d,p)-Niveau}
  \label{figure:morse}
\end{figure}
Das Morse Potential beschreibt den Verlauf des elektronischen Potentials eines zweiatomigen Moleküls
in Abhängigkeit vom Kern­bindungsabstand {\displaystyle R} R durch eine exponentielle Näherung:
\begin{equation}
 E(r) = D_e (1-e^\alpha{r-r_e})^2
\end{equation}
 Darin ist $r$ der Kernabstand der beiden betrachteten Atome, $r_e$ der Kernabstand bei der
geringsten potentiellen Energie $D_e$. $D_e$ ist die minimale Energie, auch spektroskopische
Dissoziationsenergie genannt, $\alpha$ ist eine spezifische Stoffkonstante, abhängig von den
betrachteten Atomen.
Es ist an der ~\ref{figure:morse} erkennen, dass der Verlauf des berechneten Potentials
 den exponentiellen Morse-Verlauf in guter Näherung wiedergibt.

\noindent
\textbf{Auswertung 3}\\
Die  Bindungsspaltung war eine heterolytische Bindungsspaltung. wird
der Input der Rechnung betrachtet, so weist die Vorgabe „charge: 0, Spin:
Singlet“ darauf hin.

\textbf{Auswertung 4 }\\

\begin{table}[!htpb]
\begin{tabular}{c|ccc}
 \large Homolytisch & &\multicolumn{2}{c}{\large Heterolytisch}\\
 & &\\
 \ce{\Lewis{0.2.4:6.,N}} & & \ce{N+} & \ce{N-}\\
  & &\\
\begin{tikzpicture}
%\draw [->,ultra thick] (-1,-2) --(-1,4) node[above] { Energie};
\drawLevel[elec = updown,pos = {(0,0)},    width = 1]{d1};
\drawLevel[elec = updown,pos = {(0,1.3)},  width = 1]{};
\drawLevel[elec = up,pos = {(0,2.6)},  width = 1]{};
\drawLevel[elec = up,pos = {(1.3,2.6)},  width = 1]{};
\drawLevel[elec = up,pos = {(2.6,2.6)},  width = 1]{};
\node[right] at (right d1) { Quartett} ;
\end{tikzpicture}
& &
\begin{tikzpicture}

%\draw [->,ultra thick] (-1,-2) --(-1,4) node[above] { Energie};
\drawLevel[elec = updown,pos = {(0,0)},    width = 1]{d1};
\drawLevel[elec = updown,pos = {(0,1.3)},  width = 1]{};
\drawLevel[elec = up,pos = {(0,2.6)},  width = 1]{};
\drawLevel[elec = up,pos = {(1.3,2.6)},  width = 1]{};
\drawLevel[pos = {(2.6,2.6)},  width = 1]{};
\node[right] at (right d1) { Triplett} ;
\end{tikzpicture}
&
\begin{tikzpicture}
%\draw [->,ultra thick] (-1,-2) --(-1,4) node[above] { Energie};
\drawLevel[elec = updown,pos = {(0,0)},    width = 1]{d1};
\drawLevel[elec = updown,pos = {(0,1.3)},  width = 1]{};
\drawLevel[elec = updown,pos = {(0,2.6)},  width = 1]{};
\drawLevel[elec = up,pos = {(1.3,2.6)},  width = 1]{};
\drawLevel[elec = up,pos = {(2.6,2.6)},  width = 1]{};
\node[right] at (right d1) { Triplett} ;
\end{tikzpicture}\\
&&&\\
&&&\\
\begin{tikzpicture}
%\draw [->,ultra thick] (-1,-2) --(-1,4) node[above] { Energie};
\drawLevel[elec = updown,pos = {(0,0)},    width = 1]{d1};
\drawLevel[elec = updown,pos = {(0,1.3)},  width = 1]{};
\drawLevel[elec = updown,pos = {(0,2.6)},  width = 1]{};
\drawLevel[elec = up,pos = {(1.3,2.6)},  width = 1]{};
\drawLevel[pos = {(2.6,2.6)},  width = 1]{};
\node[right] at (right d1) { Dublett} ;
\end{tikzpicture}
&&
\begin{tikzpicture}
%\draw [->,ultra thick] (-1,-2) --(-1,4) node[above] { Energie};
\drawLevel[elec = updown,pos = {(0,0)},    width = 1]{d1};
\drawLevel[elec = updown,pos = {(0,1.3)},  width = 1]{};
\drawLevel[elec = updown,pos = {(0,2.6)},  width = 1]{};
\drawLevel[pos = {(1.3,2.6)},  width = 1]{};
\drawLevel[pos = {(2.6,2.6)},  width = 1]{};
\node[right] at (right d1) { Singlet} ;
\end{tikzpicture}
&
\begin{tikzpicture}
%\draw [->,ultra thick] (-1,-2) --(-1,4) node[above] { Energie};
\drawLevel[elec = updown,pos = {(0,0)},    width = 1]{d1};
\drawLevel[elec = updown,pos = {(0,1.3)},  width = 1]{};
\drawLevel[elec = updown,pos = {(0,2.6)},  width = 1]{};
\drawLevel[elec = updown,pos = {(1.3,2.6)},  width = 1]{};
\drawLevel[pos = {(2.6,2.6)},  width = 1]{};
\node[right] at (right d1) { Singlet} ;
\end{tikzpicture}\\

\end{tabular}

\caption{Die Elektronenbesetzungschemen der Stickstoff-Fragmenten von möglichen Dissoziationsarten \cite{wiberg98}}
\label{table:besetzung}
\end{table}

Die anderen Möglichkeiten zur Dissoziation und der Multiplizitäten sind oben in
der ~\ref{table:besetzung} dargestellt.
Bei einer heterolytischen Dissoziation erhält man ein
Stickstoffkation und ein Stickstoffanion, die Singlet- und Dublett-Zustand aufweisen.
Möglich ist eine homolytische Bindungsspaltung,
bei der man zwei Stickstoffradikale erhält. Die dabei möglichen Multiplizitäten sind
in der ~\ref{table:besetzung} zu sehen; es ist der Quartett- und der Dublett-Zustand
möglich.\\
\textbf{Auswertung 5}
Nach der Hundschen Regel die Besetzung entarteter Orbitale so erfolgt, dass die größtmögliche Zahl
 ungepaarter Elektronen erreicht wird (maximale Spinmultiplizität).
 Solche Zustände sind energetisch stabiler als Zustände, bei denen die entarteten Orbitale nicht so besetzt werden,
dass die Anzahl der ungepaarten Elektronen maximal wird. Demnach lautet die energetische Reihenfolge der
Dissoziationskanäle wie folgt (nach aufsteigender Energie):\\
 Quartett < Dublett < Triplett < Singlet.\\
Daraus ergibt es sich dass es eine homolytische Dissoziation
  energetisch günstiger als eine heterolytische Dissoziation.

\subsection{Geometrieoptimierung des Stickstoffmoleküls}
Mit der gleichen Methode (RHF) und dem gleichen Basissatz 6-311G(d,p)
wurde die Geometrie des Stickstoffmoleküls optimiert.
\textbf{Auswertung 1 und 2}
\begin{table}[!htpb]
\caption{}
\begin{tabular}{lllll}
\toprule
Molekül &
Methode &
Basissatz &
Bindungslänge \si{\angstrom} &
Gesamtenergie \si{\hartree}\\
\midrule
\ce{N _2} & RHF & 6-311G(d,p) & 1.07027 \si{\angstrom} & -108.98655 \\
\bottomrule
\end{tabular}
\end{table}

Im Vergleich zur Gesamtenergie aus dem ersten Aufgabenteil ist die Energie der optimierten Geometrie geringfügig niedriger.\\

\textbf{Auswertung 3}\\
\begin{table}[!htpb]
\centering
\caption{Der experimentelle Gasphasenabstand im \ce{N_2} Molekül }
\begin{tabular}{ccc}
\toprule
Abstand Teil 1 & Abstand Teil 2 (opt. Geometrie)  & experimenteller Abstand ~\cite{holleman} \\
1.0 \si{\angstrom} & 1.07027 \si{\angstrom} & 1.0976 \si{\angstrom} \\
\midrule
\bottomrule
\end{tabular}
\end{table}\\
\noindent
\textbf{Auswertung 4}\\


Beim Vergleich der berechneten Orbitale mit denen aus der Literatur ~\cite{ritaatom} fällt auf,
 dass die Orbitale von Stickstoffmoleküls in richtiger Reihenfolge aufgeführt sind. Es liegt daran

\begin{figure}[!hptb]

        \fbox{\includegraphics[width=\textwidth,height=8cm,keepaspectratio]{data/mohf.png}}

\caption{MO-Diagramm für ein Stickstoff-Molekül auf RHF/6-311G(d,p) und aus der Literatur ~\cite{ritaatom}}

\end{figure}





\textbf{Auswertung 5 und 6} \\
Im Folgenden werden die Orbitale als
Abbildungen dargestellt und mit Hauptquantenzahl, Drehimpuls und Symmetrie
gekennzeichnet.
\begin{figure}[!hptb]
    \centering
    \begin{subfigure}[b]{0.4\textwidth}
        \fbox{\includegraphics[width=0.4\textwidth]{data/orbitale/1.png}}
      \subcaption*{$1 \sigma _g$  }
    \end{subfigure}
    \begin{subfigure}[b]{0.4\textwidth}
        \fbox{\includegraphics[width=0.4\textwidth]{data/orbitale/2.png}}
         \subcaption*{$1 \sigma _u$ }
    \end{subfigure}

\end{figure}

\begin{figure}[!hptb]
    \centering
    \begin{subfigure}[b]{0.4\textwidth}
        \fbox{\includegraphics[width=0.4\textwidth]{data/orbitale/3.png}}
        \subcaption*{$2 \sigma _g$ }
    \end{subfigure}
    ~ %add desired spacing between images, e. g. ~, \quad, \qquad, \hfill etc.
      %(or a blank line to force the subfigure onto a new line)
    \begin{subfigure}[b]{0.4\textwidth}
       \fbox{ \includegraphics[width=0.4\textwidth]{data/orbitale/4.png}}
     \subcaption*{$2 \sigma _u$}
    \end{subfigure}

\end{figure}

\begin{figure}[!hptb]
    \centering
    \begin{subfigure}[b]{0.4\textwidth}
        \fbox{\includegraphics[width=0.4\textwidth]{data/orbitale/5.png}}
        \subcaption*{$1 \pi _u^x$  }
    \end{subfigure}
    ~ %add desired spacing between images, e. g. ~, \quad, \qquad, \hfill etc.
      %(or a blank line to force the subfigure onto a new line)
    \begin{subfigure}[b]{0.4\textwidth}
        \fbox{\includegraphics[width=0.4\textwidth]{data/orbitale/6.png}}
        \subcaption*{$1 \pi _u^y$}
    \end{subfigure}
\label{figure:orbitalen2}
\end{figure}
\begin{figure}[!hptb]
    \centering
    \begin{subfigure}[b]{0.4\textwidth}
        \fbox{\includegraphics[width=0.4\textwidth]{data/orbitale/7.png}}
                \subcaption*{$3 \sigma _g$}

    \end{subfigure}
    ~ %add desired spacing between images, e. g. ~, \quad, \qquad, \hfill etc.
      %(or a blank line to force the subfigure onto a new line)
    \begin{subfigure}[b]{0.4\textwidth}
        \fbox{\includegraphics[width=0.4\textwidth]{data/orbitale/8.png}}
               \subcaption*{$1 \pi _g$}

    \end{subfigure}
\end{figure}

\begin{figure}[!hptb]
    \centering
    \begin{subfigure}[b]{0.4\textwidth}
        \fbox{\includegraphics[width=0.4\textwidth]{data/orbitale/9.png}}
               \subcaption*{$1 \pi _g$}
    \end{subfigure}
    ~ %add desired spacing between images, e. g. ~, \quad, \qquad, \hfill etc.
      %(or a blank line to force the subfigure onto a new line)
    \begin{subfigure}[b]{0.4\textwidth}
        \fbox{\includegraphics[width=0.4\textwidth]{data/orbitale/10.png}}
          \subcaption*{ $3 \sigma _u$}
    \end{subfigure}

\end{figure}

\end{onehalfspace}
\end{document}