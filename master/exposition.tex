\section{Exposition}
\epigraph{
  \textit{What, after all, is opposed by the movements of slaves and their descendents?
Slavery? Capitalism? Coerced industrialisation? Racial terror?
Or the ethnocentrism and european solipsism that these processes help to reproduce?
}}{Paul Gilroy\footnotemark}\footnotetext{Paul Gilroy, \textit{The Black
Atlantic. Modernity and Double Conscousness} (London: Verso, [1993] 1999), 30.} 

\epigraph{\textit{
    \textit{Wenn die Erkenntnis ein imperiales Instrument der Kolonialisierung
      ist, dann ist Dekolonisierung der Erkenntnis eine der dringlichsten
    Aufgaben.}
  }}{Aníbal Quijano\footnotemark}\footnotetext{Aníbal Quijano,
\textit{Colonialidad y modernidad/racionalidad}, 1989, zitiert in: Walter D.
Mignolo, \textit{Epistemischer Ungehorsam. Rhetorik der Moderne, Logik der
Kolonialität und Grammatik der Dekolonialität} (Wien: Turia + Kant, 2006), 48.}

Auf den ersten Blick scheint Aníbal Quijano auf Paul Gilroys Frage eine
unmittelbare Antwort zu liefern. Gilroy fragt, wogegen sich antikoloniale
Widerstände richten und beginnt mit einer Aufzählung:\\
Ist es Sklaverei,
Kapitalismus, erzwungene Industrialisierung, rassistischer Terror? Oder der
Ethnozentrismus und europäische Solipsismus, mit Hilfe dessen sich jene
Zustände immer wieder von Neuem zementieren? Quijano schließt hieran an, so
lässt sich meine Collage weiterdenken, indem er die von Gilroy benannte
ethnozentrische Perspektive und solipsistische Methode der Erkenntnisproduktion
als imperiales Instrument der Kolonialisierung beschreibt. Dekolonisierung
müsse sich entsprechend gegen dieses Instrument und damit zuallererst gegen
Ethnozentrismus und europäischen Solipsismus wenden.\\
Auf den zweiten Blick wird deutlich, dass Gilroys Frage eine Beschreibung der
Verhältnisse bestrebt; Quijanos Antwort hingegen liest sich eher wie eine
Forderung. Gilroy fragt, was passiert, Quijano antwortet, was passieren sollte.
Der Unterschied mag trivial erscheinen, für mich repräsentiert er ein
grundlegendes Dilemma der vorliegenden Arbeit:\\
Die postkolonialen Theoretiker\_innen, mit denen ich mich in dieser Arbeit
auseinandersetze, sprechen zu unterschiedlichen Zeiten, von unterschiedlichen
Orten und aus unterschiedlichen Beweggründen. Mein Versuch sie
zusammenzuführen, kann nur gelingen, wenn der Anspruch nach einer kohärenten
Erzählung der Einsicht weicht, dass es gerade die Kohärenz ist, die sich so
machtvoll, so beschneidend auf die vielfältigen Stimmen und Dissonanzen legt,
die sich gegenüber den unterschiedlichen Kolonialismen zur Wehr setzten und
setzen.\\

\noindent\textbf{\large Erzählung}\\
Chimamanda Ngozi Adichie beschäftigt sich mit dieser Macht, die von kohärenten
Erzählungen ausgeht. Sie bezeichnet sie als Single Story und zeigt auf, welchen
Einfluss sie auf die Wirklichkeit, also auf die Handlungs- und Denkräume von
Menschen haben und wie sich ihnen widersetzt werden kann. In ihrem Vortrag mit
dem Titel „The danger of a single story“\footnotemark\footnotetext{Chimamanda Ngozi Adichie, The danger of a single story, Ted Talk, 2009.} 
 zeigt sie anhand von
autobiographischen Beispielen, wie sich Erzählungen in die Selbst- und
Weltverständnisse ihrer Zuhörer\_innen bzw. Leser\_innen einschreiben. Dabei ist
ihrer Ansicht der Kontext, in dem Erzählungen wirkmächtig werden, entscheidend:

\begin{myenv}
  \textit{„How they are told, who tells them, when they're told, how many
stories are told, are really dependent on power. Power is the ability not just
to tell the story of another person, but to make it the definitive story of
that person.” \footnotemark\footnotetext{Ebd.} } 
\end{myenv}
Adichie beschreibt hier die Macht der Eindeutigkeit. Eindeutigkeit darüber, wer
eine\_r ist und sein kann lässt keinen Raum für eine selbstbestimmte Suche und
keine Möglichkeit für einen widersprechenden Ausdruck dessen, was eine\_r ist oder
sein möchte. Eindeutigkeit ist immer eine Beschneidung aller anderen möglichen
Bedeutungen.\\
Macht, so lese ich Adichie, beschreibt im wesentlichen eine
Position, die Eindeutigkeiten festlegt, hier, indem sie die einzig mögliche
Geschichte über jemanden erzählt, also bestimmt wer sie\_er ist.\\

\noindent Erzählungen werden hier mit Adichie zu machtvollen Praxen der
Subjektkonstruktion, wobei das Ausmaß wie tief sich diese Erzählung in diese
die Selbst- und Weltverständnisse der Subjekte einschreibt, davon abhängt, wer,
was über wen erzählt.\\
Erzählungen sind also Praxen, über die Macht ausgeübt wird. Erzählungen sind
jedoch nicht allmächtig, sie können eher als gewaltsame Versuche verstanden
werden, Wirklichkeit zu ordnen. Ein Vorhaben, das jedoch nie ganz gelingen
kann, wie Troillot anhand der Sklavenrevolution in Haiti zeigt.\\
Denn Widerstände gegen koloniale Herrschaft hat es immer gegeben, auch wenn diese
Widerstände in  herrschenden Erzählungen undenkbar waren. Dies zeigt nicht nur,
wie störrisch sich Erzählungen angesichts einer sie widerlegenden Materialität
darstellen, es zeigt auch, dass jene Erzählungen in keinem deterministischen
Verhältnis zu dieser Materialität stehen: Es ist machbar, was nicht erzählbar
ist, und auch unerzählbar bleibt, nachdem es gemacht wurde.\\

Adichie gibt in ihrem Vortrag nun Hinweise darauf, wie sich manche
Erzählungen behaupten können, obwohl die materielle Realität ihnen diamentral
entgegensteht. Sie weist damit auf den für diese Arbeit zentralen Begriff der
Hegemonie hin.\\
Hegemoniale Verhältnisse sind
Herrschaftsverhältnisse.\footnotemark\footnotetext{Brian Longhurst et al,
\textit{Introducing Cultural Studies} (Essex: Longman, 2008), 73.}\\
Herrschaft
funktioniert, so die Annahme, nicht nur durch direkte Gewaltausübung, sondern
zu einem nicht unbedeutenden Teil darüber, dass eine Erzählung diese Herrschaft
alternativlos erscheinen lässt. Die Erzählung baut einen Konsens der
Beherrschten auf, die sich somit der Herrschaft fügen. Diese Erzählung wird von
Antonio Gramci, einem der Begründer der Hegemonietheorie, auch in
Institutionen, wie der Universität verortet.\\
Hegemonie ist jedoch niemals statisch, sondern muss sich immer wieder auch
gegen antihegemoniale Praktiken behaupten. Widerstand, oder antihegemoniale
Praktiken müssen sich also neben den materiellen Widerstandshandlungen auch
gegen die hegemoniale Erzählung wenden, da wie von Trouillot gezeigt,
Erzählungen sich nicht per se den Widerstandshandlungen anpassen. Erst die
Erfahrungen des Widerstandes und ihr Einzug in Erzählungen über den Widerstand
werden damit zu echten Gefahren für die herrschenden Erzählungen und damit auch
die herrschenden Verhältnisse.\\


\noindent In dieser Arbeit suche ich nach jenen Erzählungen aus (Widerstands)erfahrungen
und frage nach den Strategien, die sie anwenden, um sich gegenüber der
hegemonialen Erzählung zu behaupten. Statt einer kategorialen Gegenüberstellung
beispielsweise feministischer und postkolonialer Ansätze, interessiere ich mich
für die verschiedenen divergierenden Möglichkeitsräume, die mit
unterschiedlichen Perspektiven von unterschiedlichen Standorten einhergehen.
Ich möchte im Sinne Walter D. Mignolos begreifen, „[...] was diese Projekte wem
anzubieten haben. [und fragen:] Wer braucht sie? Wer profitiert von ihnen? Wer
sind die möglichen Agent\_innen von Emanzipations- oder Befreiungsprojekten?
Welche Subjektivität wird durch diese Projekte
aktiviert?“\footnotemark\footnotetext{Mignolo, \textit{Epistemischer
Ungehorsam}, 63.}\\

\noindent\textbf{\large Universität}\\
Die Universität bildet den Rahmen für mein Nachdenken über die Möglichkeiten
widerständigen Wissens. Denn die Universität stellt einen Ort dar, der
untrennbar mit der Entstehung, Behauptung, Kritik und Unterwanderung des
kolonialen Wissenssystems verbunden ist.\footnotemark\footnotetext{Ramón
Grosfoguel, „The Structure of Knowledge in Westernized Universities: Epistemic
Racism/Sexism and the Four Genocides/Epistemicides of the Long 16th Century,“
in \textit{Human Architecture: Journal of the Sociology of Self-Knowledge}: Vol. 11: Iss. 1, Article 8., (2013).}
Das macht Universität zu einem ambivalenten Ort, da hier die Kerben des
Kolonialismus neben den Versuchen (weiter)bestehen, ein irgendwie würdevolleres
Bild vom Menschen zu zeichnen.\\
Mich interessiert hierbei, wie der Zustand des „knowing about
something“\footnotemark \footnotetext{Spivak, „Teaching for the Times,“ in
\textit{The Decolonization of Imagination}, herausgegeben von Jan N. Pietersen und Bhikhu Parakeh (London: Zed Books, 1995), 181.} im
Sinne eines Wissens um die gewaltvolle Entstehungs- und Nutzungsgeschichte
weißen Wissens in eine Praxis des „learning to do something“\footnotemark
\footnotetext{Spivak, „Teaching for the Times,“ ebd.}, also in einen Umgang mit bzw. lernen aus der Kritik, an
ihm übersetzt werden kann. Ein Schritt, der Gayatri Chakravorty Spivak folgend
unabdingbar ist, „as we move from the space of opposition to the menaced space
of the emergent dominant“\footnotemark\footnotetext{Ebd.}, und der für mich damit jene Herausforderung
benennt, dekoloniale Kritik auf Praxen der Bildung an der Universität
anzuwenden, umzusetzen, ja zu übersetzen.\\
Doch von welchem 'wir' geht Spivak
aus? \textit{W}er dekolonisiert, und \textit{w}er erzählt \textit{w}ie über Strategien der
Dekolonisierung?\\

\noindent\textbf{\large Dekolonisierung}\\
Dekolonial, wird im Glossar von Quix\footnotemark\footnotetext{quix-kollektiv für kritische
bildungsarbeit, „Gender\_Sexualitäten\_Begehren in der Machtkritischen und
Entwicklungspolitischen Bildungsarbeit,“ (Wien: ebd., 2017).} als
\begin{myenv} \textit{„eine Haltung oder eine Vorstellung von der Welt
  [beschrieben], die versucht, Geschichte nicht von Europa aus zu denken und zu
schreiben, und die jene Menschen und Weltgegenden, die seit der europäischen
kolonialen Expansion im 15. Jahrhundert auf verschiedene Weise unterdrückt
worden sind/werden, als Subjekte zu begreifen. Dekolonial bezieht sich nicht
nur auf eine praktische politische Ent-kolonisierung von Nationalstaaten,
sondern auf ein Dekonstruieren, Verlernen und Erneuern von Denkmustern und
Strukturen.“\footnotemark\footnotetext{Quix-kollektiv, „Gender\_Sexualitäten\_Begehren,“ 91.}} \end{myenv}
Eine dekoloniale Haltung stellt Subjekte von Bildungsprozessen, so verstehe ich
das Quix Kollektiv, vor mehrere Aufgaben: Sie beansprucht von ihnen einen
Perspektivwechsel (Geschichte nicht von Europa aus denken), erwartet
Vorstellungskraft (Menschen als Subjekte begreifen) und erfordert Überwindung
des Bekannten und Vertrauten (bisher Gewusstes verlernen und neues Wissen
erlernen).\\
Hier werden, so scheint es, jene angesprochen, die die Weltgeschichte bisher
aus europäisch-kolonialer Sicht erfahren haben. Für sie bedarf es einer
Anstrengung, in Menschen das Menschliche zu erkennen, die im Kolonialismus wie
Dinge betrachtet und behandelt wurden. Dekoloniale Bildung bedeutet für sie
darum auch, Quix folgend, den sicheren Boden aus Selbstverständlichkeiten und
Gewissheiten über das Selbst und die zu Anderen gemachten in der Welt zu
verlassen.\\
Dekolonisierung beschränkt sich jedoch nicht auf (Ver)Lernprozesse \textit{w}eiß
positionierter Europäer\_innen. Die Möglichkeit aus einer \textit{w}eißen Position heraus
hegemoniale Wissensordnungen zu überschreiten, wird von einigen postkolonialen
Theoretiker\_innen hingegen infrage gestellt:

\begin{myenv}
  \textit{
  „Indigenous knowledges are the starting point for resurgence and
  decolonization, are the medium through which we engage in the present, and
  are the possibility of an Indigenous future. Without this power base,
  decolonization becomes a domesticated industry of ideas. Decolonization is
  not always about the co-existence of knowledges, nor knowledge synthesis,
  which inevitably centers colonial logic. Whiteness does not ‘play well with
  others’ but, rather, fragments and marginalizes - so it must be asked:
  Co-existence at what cost and for whose benefit? Decolonization necessarily
  unsettles. In the face of the beast of colonialism, thirsty for the blood of
  Indigeneity and drunk on conquest, assimilation is submission and
  decolonization calls on those who will 'beat the beast into submission and
  teach it to behave'.”\footnotemark\footnotetext{Sium, Aman, Chandni Desai und
  Eric Ritskes, „Towards the 'tangible unknown': Decolonization and the
  Indegenous Future,“ \textit{Decolonization: Indigeneity, Education and
  Society}, Vol.1,
Nr. 1, (2012): I – XIII.} }
\end{myenv}
Aman Sium und Kolleg\_innen beschreiben das Verhältnis, \textit{w}eißen und Indigenen
Wissens als Kampf, der nicht ohne Verluste geführt werden kann. Hegemoniale
Diskurse spalten und verdrängen Indigenes Wissen und führen damit koloniale
Logiken fort, so ihre Argumentation.\\
Dekolonisierung kann also nicht in der
Zusammenführung unterschiedlicher Wissensbestände bestehen, sondern muss laut
der Autor\_innen die Unterwerfung hegemonialen Wissens zum Anliegen erklären.
Indigenes Wissen ist demnach Ausgangspunkt, Mittel und Ziel von
Dekolonisierung. Nicht nur der Ton wird schärfer in dieser Ausführung.\\
Im
Unterschied zur Definition des Quix Kollektivs werden bei Aman et al.
nicht-\textit{w}eiße (Indigene) Subjekte adressiert, sie werden als Akteur\_innen der
Dekolonisierung angesprochen und dazu aufgerufen, sich nicht domestizieren zu
lassen, sondern selbst zu domestizieren.\\

Die Ansätze von Quix und Sium et al. unterscheiden sich in der Frage, wer den
Subjektstatus im dekolonialen Projekt beanspruchen und damit wer wen mit
welchen Mitteln befreien kann. Diese Perspektiven als einander konkurrierend
und damit als unvereinbar zu lesen, funktioniert jedoch nur, solange die Frage,
was mit \textit{w}eißem vs. Indigenem Wissen gemeint ist, nicht gestellt wird.\\
Geht das Ergebnis jahrhundertelanger Gewaltherrschaft in diesem Gegensatzpaar
auf?\\
Es mag wie eine rhetorische Frage klingen, doch die Antwort ist komplex. Sie
muss aufzeigen, „wie Differenz zu denken ist, ohne sie in Identität zu
vereinnahmen“\footnotemark\footnotetext{Silvia Stoller und Helmuth Vetter,
„Einleitung,“ in \textit{Phänomenologie und Geschlechterdifferenz}, herausgegeben von ebd. (Wien: WUV Universitätsverlag, 1997), 8. } und damit Wege finden, wie historisch verankerte, stetig
wachsende und widerstandene Machtverhältnisse und ihre Effekte anerkannt werden
können, ohne ihre Logik fortzuschreiben.\\
Diese Arbeit befindet sich mitten in
diesem Spannungsfeld, denn radikale Infragestellungen europäischer
Erkenntnispolitik, auf die ich mich beziehe, bestreben weder deren
Komplementierung, noch eine vollständig unabhängige Neuschaffung eines
alternativen Denk- und Wissenssystems.\\
Epistemische
Widerstandsstrategien\footnotemark\footnotetext{ Walter D. Mignolo,
  \textit{Epistemischer Ungehorsam. Rhetorik der Moderne, Logik der
  Kolonialität und Grammatik der Dekolonialität} (Wien: Turia + Kant, 2006),
  sowie Enrique Dussel, „Transmodernity and Interculturality. An Interpretation
  from the Philosophy of Liberation,“ \textit{Transmodernity: Journal of Peripheral
  Cultural Production of the Luso-Hispanic World}, Vol. 1 Nr. 3, (2012), sowie
  Gayatri Chakravorty Spivak, „Righting Wrongs,“ zitiert in : Maria do Mar
  Castro Varela und Nikita Dhawan, Hg., \textit{Postkoloniale Theorie. Eine kritische
Einführung}, (Bielefeld: transkript Verlag, 2015).} zeigen im Gegensatz dazu
  auf, wie die eigene Involviertheit in und gleichsame Distanzierung von
  europäischen Wissensregimen genutzt werden kann, um radikale Neuordnungen und
  damit einhergehende Perspektivwechsel zu ermöglichen.\\
  Diese Strategien bezweifeln, dass es ein Außen geben könnte, von dem aus die
  Geschichte betrachtet, oder dass es einen Nullpunkt gibt, von dem aus noch
  einmal von Vorne begonnen werden kann.\\
  Das bedeutet jedoch nicht, dass alle gleichermaßen involviert sind. Die
  Unterscheidung in weißes versus Indigenes Wissen oder hegemoniales versus
  widerständiges Wissen ist keinesfalls obsolet. Sie ist nicht nur Ausdruck der
  verschiedenen Ausgangspositionen, von denen aus Subjekte kolonialer Gewalt
  begegnet sind, sondern auch der unterschiedlichen Erfahrungen, die mit
  Kolonialismus einhergegangen sind und einhergehen. Postkoloniale Positionen
  berücksichtigen dieses Machtverhältnis in ihren Versuchen, dekoloniale
  Strategien zu entwickeln.\\

\noindent\textbf{\large Gegenerzählung}\\
Postkolonialismus kann nun als eine Perspektive verstanden werden, aus der
heraus antihegemoniale Praktiken im Sinne einer Gegenerzählung geübt werden:
\begin{myenv}
  \textit{„Against the meta-narrative of European colonialism stands
  postcolonialism as a set of perspectives through which the contemporary world
is re-interrogated, re-interpreted and re-positioned discursivly through
practices and policies of and for social justice.“\footnotemark
\footnotetext{Bristol, \textit{Plantation Pedagogies}, 27.}}
\end{myenv}
Hier wird das Zusammenwirken von diskursiven Praxen der Infragestellung,
Reinterpretation und Repositionierung der Meta-Erzählung des europäischen
Kolonialismus in eine direkte Verbindung mit konkreten und damit materiellen
Veränderungen durch und für soziale Gerechtigkeit gebracht. Kolonialismus wird
somit nicht auf die diskursive Praxis der Meta-Erzählung der Europäischen
Moderne reduziert. Die Meta-Erzählung stellt indessen \textit{eine} Form der
Legitimierung und Ausübung kolonialer Macht dar, die im Postkolonialismus zum
Gegenstand von Interventionen gemacht wird.\footnotemark\footnotetext{Laurette
S. M. Bristol, \textit{Plantation Pedagogies. A Postcolonial and Global
Perspective} (New York:Lang, 2012).}\\
Die Praxis der Gegenerzählung greift das
koloniale Projekt damit auf der Ebene des Diskursiven an und muss in seiner
Verschränkung mit Praxen von Gesetzen, und der Umverteilung materieller
Ressourcen verstanden werden, die ebenso in der postkolonialen Perspektive
aufgehen, beziehungsweise sich in ihr subsumieren lassen.\\

Laurette Bristol beschreibt Kolonialismus als „story of European legitimacy,
control, enlightenment and civility [that] used research to claim and maintain
its hold over the psyche of the colonised.“\footnotemark \footnotetext{Bristol,
\textit{Plantation Pedagogies}, 21.}. Sie nimmt damit in Anschluss an
Frantz Fanon die Effekte in den Blick, die das koloniale Projekt auf Körper und
Psyche der Menschen hatte und hat. So tritt neben die militärische Herrschaft,
die mit Waffengewalt das Leben und den Widerstand der Bevölkerung brutal zu
bezwingen suchte, auch die epistemische Herrschaft, die mittels einer
spezifischen Geschichtsschreibung, Subjektivitäten
hervorbrachte.\footnotemark\footnotetext{Bristol, \textit{Plantation
  Pedagogies}, Smith,
Linda Tuhiwai, \textit{Decolonizing Methodologies. Research and Indigenous Peoples} (London: Zed Books, 1999).}\\

\noindent Postkoloniale Theorie könnte als eine Position beschrieben werden, die  auf
unterschiedliche Weise  koloniale Metanarrative mitsamt ihrer
Subjektkonzeptionen in Frage stellt. Dabei gilt für Bristol die Praxis des
Erzählens, und damit das Narrativ als Referenzrahmen, indem sich sowohl das
koloniale als auch das postkoloniale Moment greifen lässt und das unter anderem
Universität als Ort der Wissensproduktion in den Blick nimmt:
\begin{myenv} \textit{  „If research is constructed as a dominant story told
    about a particular moment in which a particular group responded to social,
    economic, political and historical forces, then postcolonial research
    offers the possibility of a new story, a different story and a contested
story.“\footnotemark \footnotetext{Bristol, ebd., 21.} } \end{myenv}
Das Erzählen einer Geschichte wird hier sowohl als hegemoniale als auch als
subversive Praxis dargestellt. Es ist also ein höchst ambivalentes Instrument,
mit dem Macht stabilisiert und destabilisiert werden kann. Der kolonialen
Erzählung (die ihre Kontextgebundenheit negiert), stehen eine Vielzahl
widersprechender, dekolonialer Geschichten gegenüber.\\

\noindent Meta- und Mikronarrative können jedoch nicht als getrennte, diskursive Räume
betrachtet werden. Mit Blick auf die universitäre Wissensproduktion, lässt sich
viel eher vermuten, dass sich die großen Erzählungen in die kleinen Geschichten
einschreiben, sie in Formen zwängen und damit nicht nur Inhalte bestimmen,
sondern auch dafür sorgen, dass die große Erzählung immer wieder von Neuem
bestätigt wird.\\

\noindent\textbf{\large Erfahrung}\\
Hierbei steht für mich das Verhältnis, von Erfahrung zu Wissen oder viel eher
die wechselseitige Angewiesenheit dieser Konzepte im Vordergrund. Mich
interessiert, ob die theoretische Auseinandersetzung mit Erfahrung das
Potential hat, die machtvollen Prozesse der Differenzbildung und -verschiebung
so zu befragen, dass Verknüpfungen, die die Neuschreibung der hegemonialen
Erzählung der Moderne ermöglichen, geübt werden können. Denn Erfahrung bzw.
Wissen aus Erfahrung beschreibt, so meine Vermutung, einen äußert interessanten
Zwischenraum, der einander vermeintlich ausschließende Konzepte in neue
Beziehungen versetzt: \\
Singularität und Kollektivität, Vergangenheit und
Gegenwart, Erleben und Interpretation, Subjekt und Diskurs, Körper und Geist
werden im Nachdenken über Erfahrung nicht mehr als gegensätzlich, sondern
vielmehr in ihrer Verschränkung erkennbar.\\

\noindent Der Fokus auf bestimmte, einheitlich gedachte (Differenz-)Erfahrungen kann
jedoch auch eine ganz andere Wirkung entfalten. Im Postkolonialen Feminismus
wurde und wird beispielsweise problematisiert, dass eine homogen imaginierte
genuin 'weibliche' Erfahrung als Ausgangspunkt eines universalen Feminismus
Differenzen zwischen Frauen\_* negiert und somit normative Identitätspolitiken
antreibt.
\begin{myenv}
  \textit{
  „If the assumption of the sameness of experience is what ties woman
  (individual) to women (group), regardeless of class, race, nation and
  sexualities, the notion of experience is anchored firmly in the notion of the
  individual self, a determined and specifiable constituent of European
  modernity.“\footnotemark \footnotetext{Chandra Talpade Mohanty, „Feminist
    Encounters. Locating the Politics of Expereince“ in \textit{Destabilizing
  Theory. Contemporary Feminist Debates}, herausgegeben von Michèle Barret und Anne Phillips, (Stanford: Stanford Univerity Press, 1992), 82.}
  }
\end{myenv}
Erfahrung muss also stets als Erfahrung von Differenz, von Uneindeutigkeit
verstanden werden. Nur so kann einem singulären, ahistorischen
Erfahrungsbegriff vorgebeugt werden, der Erfahrung als individuell versteht und
damit auf eine Kategorie reduziert, die wie Chandra Talpade Mohanty hier
schreibt, untrennbar mit der Konstitution der europäischen Moderne verbunden
ist.\footnotemark\footnotetext{Mohanty, „Feminist Encounters. Locating the Politics of Expereince“, ebd.}\\

Mohanty zeigt hier, ebenso wie eingangs Lauré Al Samarays Begriff der
kolonialen Erfahrung, wie eng Erfahrungen mit Subjektpositionen verknüpft sind,
und dass koloniale Erfahrungen nicht ausgeblendet werden dürfen, wenn es darum
geht, Bedingungen für einen Dialog zu schaffen, in dem einander widerstreitende
Erfahrung artikulierbar werden.\\

Für bell hooks ist ein wesentliches Kennzeichen
von Dialog, die Einsicht, dass sich zwei Subjekte
gegenüberstehen.\footnotemark\footnotetext{bell hooks, \textit{Talking back.
Thinking feminist, thinking black}, (Bosten: South End Press, 1989), 131.} Es gilt
also nach möglichen Formen des miteinander Umgehens zu suchen, die es
ermöglichen, einander als Subjekte zu erkennen. Subjekte, die unvorhersehbar,
unvollständig unbeschreibbar sind und bleiben werden.\\

\noindent Dabei versteht sich diese Arbeit nicht als Prüfung, inwiefern die Praxis einer
erfahrungsorientierten Wissensproduktion für eine dekoloniale Bildung sinnvoll
ist. Mir geht es im Anschluss an Lorraine Code darum, herauszufinden, wie
Erfahrung erkenntnistheoretisch genutzt werden kann, ohne dabei die Strukturen,
denen die Erfahrungen zu Grunde liegen, außer Acht zu lassen, und damit eine
Theorie der Erkenntnis und damit auch (Subjekt-)Bildung zu entwickeln, die der
Bedeutung der Erfahrung in ihrer komplexen Verschränkung mit Diskursen und
ihrem historischen Gewordensein einen angemessenen Ort
bietet.\footnotemark\footnotetext{Lorraine Code, „Experience, Knowledge, and
Responsibility,“ in \textit{Feminist Perspectives in Philosophy}, herausgegeben von Morwenna Griffith und Margaret Withford ( London: The Macmillan Press, 1988), 157.}\\

\noindent Eine reflexive Praxis des Dialogs stellt nun einen wesentlichen Bezugspunkt
dar, anhand dessen ich die Möglichkeiten eines Austausches unterschiedlicher
Wissensbestände und -formen diskutieren möchte.\\
Interessant ist für mich
hierbei, wie die großen und teils abstrakten epistemologischen Forderungen, die
postkoloniale Theoretiker\_innen stellen, wenn sie z.B. dazu aufrufen, Europa zu
provinzialisieren,\footnotemark\footnotetext{Dipeh Chakrabarty, \textit{Europa als
Provinz. Perspektiven postkolonialer Geschichtsschreibung} (Frankfurt am
Main:Campus Verlag, 2010).} auf universitäre Bildungskontexte angewandt werden
können.\\

\noindent\textbf{\large Zusammenfassung}\\
Zusammenfassend ist festzuhalten, dass die
Universität einen Fixpunkt am Horizont dieser Arbeit bildet. Dabei gehe ich
davon aus, dass (universitäre) Bildungsprozesse immer in Wissensproduktionen
eingebettet sind, und damit Praxen hervorbringen, in denen das Wissen nicht nur
her- und infrage- gestellt wird, sondern sich in die spezifischen Lebens- und
Selbstverständnisse der Akteur\_innen an der Universität einschreibt bzw. von
hier aus neu gedacht wird.\\
Die für diese Arbeit ausgewählten Positionierungen in und zur europäischen
Moderne und die daraus folgenden Utopien einer Wissenschaft, die Befreiung
ermöglichen soll, ohne dabei von einem universalen Subjekt auszugehen, das sich
auf eine vorgeschriebene und vorhersehbare Art und Weise zu dieser Befreiung
verhält, bilden somit das komplexe Gefüge, aus dem heraus ich Fragen über
mögliche Strategien der dekolonialen Bildung an der Universität stellen möchte.\\
Dabei sprechen sowohl die Perspektiven, als auch ich selbst in dieser Arbeit
zugleich aus dem Raum heraus und in den Raum hinein der Gegenstand meiner
Auseinandersetzung ist. \\
Die Universität wird damit zu einem Sinnbild für das
Spannungsfeld, in dem sich postkoloniale Erkenntnistheorien bewegen, wenn sie
fragen, ob das Haus des Herrn mit seinen eigenen Waffen zerstört werden
kann.\footnotemark \footnotetext{Vgl. Audre Lorde, „The Master's Tools Will
Never Dismantle the Master's House“ zitiert in Nikita Dhawan „Zwischen Empire
und Empower: Dekolonisierung und Demokratisierung,“ \textit{Femina Politica} (2009), 6.} Das erfordert, die Kritik, die aus der Universität und damit durch sie hindurch formuliert wird, auf sie selbst zurück zuwerfen und die Instrumente in den Blick zu nehmen, die für dieses Vorhaben Nützlichkeit versprechen.
\newpage
\section{Vorgehen}

Wie Titel und Exposition dieser Arbeit bereits versprechen, setze ich mich mit
den Möglichkeitsräumen dekolonialer Bildung an der Universität auseinander.
Praxen der Gegenerzählens, die sich kritisch auf die Kategorie der Erfahrung
beziehen werden von mir hier als mögliche Strategien dekolonialer Bildung
untersucht.\\
Doch was meine ich mit dekolonialer Bildung und von welcher Universität gehe
ich aus? Gegen was wird gegenerzählt und was hat das mit Erfahrung, bzw. einer
kritischen Auseinandersetzung mit Erfahrung zu tun?\\

\noindent In meinem Lese- und Schreibprozess habe ich eben gestellte Fragen nicht in
analytischer Manier Schritt für Schritt beantwortet. Die Fragen, mit denen ich
den\_die Leser\_innen in den nun folgenden Text entlasse sind auf dem Weg
entstanden. Sie haben Antworten und neue Frage mit sich gebracht. Der Text, der
nun dabei entstanden ist, ist mögliches Abbild dieser Suche, neben vielen
anderen Abbildern, die an anderen Orten für andere Zeiten verweilen.\\
Wenn ich im folgenden mein Vorgehen beschreibe, beziehe ich mich auf dieses,
hier vorliegende  Abbild. Das führt auch dazu, dass die Erkenntnisse, dich im
im Lese- und Schreibprozess gewonnen habe in der Beschreibung meines Vorgehens
zu Ausdruck kommen. Auch wenn ich versuchen will, nicht zu viel zu verraten
komme ich nicht umhin, Zwischenergebnisse zu markieren um mein darauffolgenden
Schritte nachvollziehbar zu machen.\\

\noindent\textit{Von welchen Subjekten geht die Universität aus?}\\
Die Frage, von
welcher Universität ich ausgehe wird im ersten Kapitel von der Frage:„Von
welchen Subjekten geht die Universität aus?“ abgelöst.\\
 Meine Analyse zeigt,
dass sich in den Überlegungen über Ausgangslagen und Ziele von
Bildungsprozessen an der Universität durchaus sehr unterschiedliche
Subjektverständnisse wiederfinden. Der Unterschied, so meine Vermutung
entspringt hier weniger divergierenden theoretischen Positionen sondern viel
mehr einander widerstreitenden Vorstellungen davon, von welchen Erfahrungen
Subjekte konstituiert sind und damit auch mit welchen Selbstverständnissen
Subjekte in Bildungsprozessen agieren können. Mein Interesse für die Bedeutung
von Erfahrung in Bildungsprozessen führt mich zu der Frage: Von welchen
(Vor)Erfahrungen der Subjekte, die in Bildungsprozessen Gegenstand der
Auseinandersetzung werden, wird hier ausgegangen? Inwiefern sind diese
(Vor)Erfahrungen in dem Wissen und damit Denken, das an der Universität
'herrscht' (nicht) enthalten?\\

\noindent\textit{Aus welchen Erfahrungen geht das Wissen und Denken an der Universität
(nicht) hervor?}\\ 
Hierfür setzte ich mich mit hegemonietheoretischen Ansätzen
auseinander, die sich mit der Methodologie, also der Frage wie Wissen
hergestellt wird, machtkritisch auseinandersetzen. Dabei zeige ich anhand einer
dekolonialen Lesart einschlägiger europäischer Denkansätze auf, wie die
Etablierung euro- und androzentrischer Perspektiven mit den vorherrschenden
(welt)gesellschaftlichen Verhältnissen verknüpft waren und sind. Ich setze mich
also mit der Frage auseinander, welche Erfahrungen sich in das hegemonial
gesetzte Wissen einschreiben konnten, bzw. wie eine hegemoniale Setzung dieses
Wissens überhaupt möglich werden konnte. Eine dekoloniale Analyse fokussiert
hier die Europäische Moderne im kolonialen Verhältnis und untersucht die
Bedeutung, die die koloniale Erfahrung für die Entwicklung von Wissen und
Legitimationspraktiken von Erkenntnissen hatte und hat.\\[0.75em]
\textit{Welchen Ort erhält Erfahrung in universitären Wissensproduktionen und
was kann eine emanzipatorische, dekoloniale Bildung daraus lernen? }\\
Wie wird diese Erfahrung und ihre Bedeutung für Wissensproduktion nun in diesem
Wissen reflektiert bzw. verhandelt?\\ Hier unterscheide ich zwischen
hegemonialen und widerständigen epistemischen Strategien und analysiere ihre
Auseinandersetzungen bzw. Reflexionen über die Bedeutung von Erfahrung in
Erkenntnisproduktion. Ob sich emanzipatorische Praxen hier aus einer Abgrenzung
zu hegemonialen Strategien entwickeln können oder sich an etwas ganz anderem
orientieren müssen ist Gegenstand der anschließenden Diskussion um
Möglichkeiten und Grenzen widerständigen Wissens. Hier treffen durchaus sehr
unterschiedliche Vorstellungen davon, ob und wenn ja wie Erfahrung
erkenntnistheoretisch genutzt werden kann, aufeinander.\\

\noindent An dieser Stelle habe ich mich so mag der\_die Leser\_in vermuten, von dem
Bildungs- und evt. auch Subjektbegriff mit dem ich die Arbeit einleitete weit
entfernt. Doch was bring eine Beschäftigung mit Subjekten wenn nicht die
Geschichte dieser Subjekte thematisiert wird? Und wie kann über Bildung
nachgedacht werden, ohne dafür die Verhältnisse in den Blick zu nehmen, in
denen die Subjekte eben jener Bildung sich befinden?\\
Der Erfahrungsbegriff, das konnte der Umweg über Erkenntnistheorie zeigen,
ermöglicht eine Auseinandersetzung mit und über Subjekte, die historische und
gegenwärtige Verhältnisse und ihre Bedeutung für Subjekte berücksichtigt. Eine
erkenntnistheoretische Reflexion um die Frage: „Was ist der Mensch?“, die sich
mit den Ausschlüssen befasst, die mit dieser Frage und ihren möglichen
Antworten einhergehen, wird somit Bedingung für eine dekoloniale Bildung, die
sich mit ihrer Geschichte und damit auch der Geschichte ihrer imaginierten
Subjekte befasst.\\

\noindent\textit{Gegenerzählstrategien als Formen erfahrungsorientierte Intervention?}\\
Die Auseinandersetzung mit Erfahrung und ihrem Bedeutung für Erkenntnisse über
Selbst- und Weltverhältnisse führt mich im letzten Kapitel zu den Möglichkeiten
der Anwendung von erfahrungsorientiertem Erzählen als Strategie der
Gegenerzählung. Der Begriff der kolonialen Erzählung, auf den sich jene
Strategie bezieht, kann hier als Analyseperspektive beschreiben werden, in der
Prozesse der Selbst- und Außenkonstruktion der Europäische Moderne in ihrer
diskursiven Dimension in den Blick geraten. Sowohl die erfahrungsorientierte
Ich - Erzählung als auch die erfahrungsorientierte Global-Erzählung werden hier
bezüglich ihrer Interventionsmöglichkeiten in das bestehende hegemoniale
Narrativ der Europäischen Moderne von mir untersucht. Dabei steht für mich die
Verknüpfung mit den bereist geführten Auseinandersetzen um die Bedeutung von
Erfahrung in Wissensproduktion im Vordergrund.\\

\noindent Zusammenfassend kann die Struktur meiner Arbeit in vier, aufeinanderfolgende
Schritten, beschrieben werden. So stelle ich im ersten Teil die Frage, wie (ein
bestimmtes, hegemonial gesetztes) Wissen beschaffen ist, und übe im zweiten
Teil Kritik an den Auslassungen und Dethematisierungen von Erfahrung als
konstitutivem Bestandteil dieses Wissens. Der dritte Teil besteht in einer Diskussion verschiedener (feministischer)
\textit{Forderungen} die mit der Kritik an diesem hegemonialen Wissensanspruch
einhergehen und im letzten Teil der Arbeit versuche ich einer
\textit{Anwendung} eben
jener Forderungen für eine dekoloniale Intervention im Form einer
Gegenerzählung.

