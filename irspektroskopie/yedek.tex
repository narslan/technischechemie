% !TEX encoding   = UTF-8
% !TEX program    = LuaLaTeX
% !TEX spellcheck = de_DE
\documentclass[12pt]{article}
\usepackage{amsmath,mathtools}
\usepackage{amssymb}  

\usepackage[usenames,dvipsnames]{xcolor}
%\usepackage[bitstream-charter]{mathdesign}
\usepackage{microtype}
\usepackage{fontspec}
\usepackage{graphicx}
\usepackage{siunitx}
\usepackage[german]{babel}
\usepackage{comment}
\usepackage{nicefrac}
\usepackage{booktabs}
\usepackage{float}
\usepackage{tikz}
\usetikzlibrary{arrows,chains,matrix,shapes}
\usepackage{pgfplots}
\pgfplotsset{compat=newest}
\usepackage[backend=biber,sorting=none,autocite = superscript,natbib=true]{biblatex} \addbibresource{books.bib}
\usepackage{caption}
\usepackage{subcaption}
\usepackage{luacode}
\usepackage{wrapfig}

\floatstyle{plaintop}
\restylefloat{table}
%\usepackage[sfdefault]{AlegreyaSans} %% Option 'black' gives heavier bold face
%% The 'sfdefault' option to make the base font sans serif
%\renewcommand*\oldstylenums[1]{{\AlegreyaSansOsF #1}}

\defaultfontfeatures{Ligatures=TeX,Numbers=OldStyle}% ,Scale=MatchLowercase} bug in current Biolinum
\setmainfont{Linux Libertine O}
\setsansfont{Linux Biolinum O}

\usepackage{fancyhdr}
\fancyhf{}
\rhead{13.12.2016}
\lhead{Nevroz Arslan, Justin König Gruppe 6}
\setlength{\headheight}{15pt}
\rfoot{\thepage }
\lfoot{Versuch 1: Infrarotspektroskopie}
\pagestyle{fancy}



\usepackage{url} \usepackage{csquotes}
\sisetup{inter-unit-product=\ensuremath{{}\cdot{}}}
\renewcommand{\arraystretch}{1.5}
\definecolor{skyblue1}{RGB}{135, 206, 250}
\definecolor{flame}{RGB}{226, 88, 34}
\definecolor{scarletred1}{RGB}{252, 40, 71}
\definecolor{cyanblau}{RGB}{0, 158, 224}
\definecolor{charcoal}{rgb}{0.21, 0.27, 0.31}

\begin{document}

\section{Ziel des Versuches}
Der Versuch befasst sich mit der Infrarotspektroskopie und den physikalischen Ursachen der Schwingungen der Materie.
Dazu sollen Spektren von Kohlenmonoxid aufgenommen werden.
Die Bindungslänge und Dissoziationsenergie von Kohlenmonoxid sollen mit Hilfe der Spektren bestimmt werden. 
\section {Theorie\supercite{fadini}}
\subsection{Molekülschwingungen}
Die Wechselwirkung zwischen elektromagnetischer Strahlung und Molekülen
bildet die Grundlage verschiedener Methoden der Strukturbestimmung. Zu den wichtigsten
gehören diejenigen, die auf die Absorption der Strahlung (Infrarot Spektroskopie),
ihre Streuung (Raman Spektroskopie) oder Beugung (röntgenographische Methoden)
zurückzuführen sind.
Die Schwingungen treten im allgemein im mittleren IR-Bereich,
also bei Wellenzahlen 4000 \si{\per\centi\meter} und 200 \si{\per\centi\meter}.

In einem makroskopischen Modell stellt man ein zweiatomiges Molekül
durch zwei Massen m$_1$ und $_2$ dar, die durch eine elastische Feder verbunden sind.
Wenn der Gleichgewichtsabstand r$_e$ zwischen den beiden Massen um den Betrag $\delta r$
verzerrt und das System darauf losgelassen wird, so führt es eine Schwingungsbewegung
um die Gleichgewichtslage aus, die durch die rücktreibende Kraft $f$ verursacht wird. $f$
ist in erster Näherung zur Auslenkung proportional, jedoch entgegengerichtet:
\begin{equation}
f = - k \cdot \Delta r = -k \cdot (r_{max} -r_e)
\end{equation}
Der Proportionalitätsfaktor $k$, dem im Modell die Federkonstante entspricht,
wird bei der Beschreibung von Molekülschwingungen als Kraftkonstante bezeichnet.
Anschaulich ist $k$ ein Maß für die Bindungsstärke, deren Berechnung eine der wichtigsten
theoretischen Anwendungen der Schwingungsspektroskopie darstellt.
\begin{figure}
\centering
\resizebox {.6\linewidth} {!} {
\begin{tikzpicture}

\node at (1,8) {\footnotesize Gleichgewichtslage};
\node at (1,5) {\footnotesize größte Auslenkung};
\node at (1,2) {\footnotesize kleinste Auslenkung};
\node at (5,-0.25) {\footnotesize Schwerpunkt};


\node[circle,fill=gray,draw=black,inner sep=6mm] at (4.1,2.0) (p1) {};
\node[circle,fill=gray,draw=black,inner sep=3mm] at (7,2.0) (p2) {};
\draw[very thick] (4.1,2.0) -- (7,2.0);



\node[circle,fill=gray,draw=black,inner sep=6mm] at (3.5,4.5) (p3) {};
\node[circle,fill=gray,draw=black,inner sep=3mm] at (8,4.5) (p4) {};
\draw[very thick] (3.5,4.5) -- (8,4.5) ;
\draw[<->,thick] (3.5,3.5) -- (8,3.5) node[midway,below] {$r_{max}$};


\draw[ thick]  (0,6) -- (10, 6) ;
\node[circle,fill=gray,draw=black,inner sep=6mm] at (3.5,7.5) (p5) {};
\node[circle,fill=gray,draw=black,inner sep=3mm] at (7.3,7.5) (p6) {};
\node (v1) at (3.5,7.3) {\scriptsize $m_1$};
\node (v2) at (7.3,7.3) {\scriptsize $m_2$};
\draw[<-> ,thick]  (3.5,6.5) -- (7.3,6.5)  node[midway,below] {$r_e$} ;


\draw[very thick] (3.5,7.5) -- (7.3,7.5);

\draw[ thick, dashed]  (5,0) -- (5,10) ;

\end{tikzpicture}
}
\caption{Das Modell eines schwingenden zweiatomigen Moleküls} \label{fig:M1}
\end{figure}

\subsubsection{Der harmonische Ansatz}
Die potentielle Energie als Funktion des Kernabstandes $r$ ergibt sich durch Integration zu
\begin{equation}
  v(r) = \frac{1}{2} \cdot (k \cdot (\Delta r)^2) = \frac{1}{2}(k \cdot (r -r_e)^2) 
\end{equation}
und hat damit einen parabelförmigen Verlauf. Das Potential $V$ und die dadurch beschriebene Schwingungsbewegung werden,
da sie auf dem linearen Kraftgesetz beruhen, als \textit{harmonisch} bezeichnet.
Aus der quantentheoretischen Beschreibung des \glqq harmonischen Oszillators \grqq erhält man seine Energie-Eigenwerte $E _{\nu}$:

\begin{equation}
      E _{\nu}= (\nu + \frac{1}{2}) \hslash \omega \, \, \text{mit } \omega=\sqrt{\frac{k}{m}} \,\, \text{und} \,\, \nu = 0, 1, 2 ..
\end{equation}

Dabei ist $m$ die Teilchenmasse und  $\omega$  die die Frequenz der ausgeführten Schwingungsbewegung,
und die Schwingungsquantenzahl $\nu$ kennzeichnet Schwingungsniveaus,
so dass $\nu = 0$ dem Grundzustand mit der Energie $E _0 = \frac{1}{2}  \hslash \omega  $ entspricht;
$\nu = 1$ bezeichnet den ersten angeregten Zustand ( $E _1 = \frac{3}{2}  \hslash \omega $) usw.
Der Abstand $\Delta E$ zwischen zwei benachbarten Schwingungsniveaus ist immer gleich ( $ \Delta E =  \hslash \omega  $ )
\begin{figure}[!ht]
  \centering
  \resizebox {.7\linewidth} {!} {
    \begin{tikzpicture}
      \def\xmin{0}
      \def\xmax{10}
      \def\ymin{0}
      \def\ymax{6}

      (\xmax,\ymax);
      \draw[->] (\xmin,\ymin) -- (\xmax,\ymin) node[right] {$r$};
      \draw[->] (\xmin,\ymin) -- (\xmin,\ymax) node[above] {$V(r)$};

      \node (v1) at (0.6,1.5) {\scriptsize $V = 0$};
      \node (v2) at (0.6,2.25) {\scriptsize $ V = 1$};
      \node (v3) at (0.6,3) {\scriptsize $ V = 2 $};
      \node (v4) at (0.6,3.75) {\scriptsize $ V = 3$};

      \node (e1) at (7.8,1.5) {\scriptsize $E_0 = \frac{1}{2} \cdot h \cdot v{'}$};
      \node (e2) at (7.8,2.25) {\scriptsize $E_1 = \frac{3}{2} \cdot h \cdot v{'}$};
      \node (e3) at (7.8,3) {\scriptsize $E_2 = \frac{5}{2} \cdot h \cdot v{'} $};
      \node (e4) at (7.8,3.75) {\scriptsize $ E_3 = \frac{7}{2} \cdot h \cdot v{'}$};

      \node (re) at (4, -0.25) {\scriptsize $ r _e$};

      \draw[] (3.12,1.5) -- (4.87,1.5);
      \draw[] (2.62,2.25) -- (5.37,2.25);
      \draw[] (2.25,3) -- (5.75,3);
      \draw[] (2,3.75) -- (6,3.75);

      \draw[<->, thick] (4,1.5) -- (4,2.25) ;
      \draw[<->, thick] (4,2.25) -- (4,3) ;
      \draw[<->, thick] (4,3) -- (4,3.75) ;

      \draw[ thick, dashed]  (1,1.5) -- (3.125,1.5) ;
      \draw[ thick, dashed]  (1,2.25) -- (2.5,2.25) ;
      \draw[ thick, dashed]  (1,3) -- (2.25,3) ;
      \draw[ thick, dashed]  (1,3.75) -- (1.75,3.75) ;

      \draw[ thick, dashed]  (5,1.5) -- (7,1.5) ;
      \draw[ thick, dashed]  (5.5,2.25) -- (7,2.25) ;
      \draw[ thick, dashed]  (5.75,3) -- (7,3) ;
      \draw[ thick, dashed]  (6.1,3.75) -- (7,3.75) ;

      \draw (1.25,6) parabola[parabola height=-5cm,thick] (6.75,6);
    \end{tikzpicture}
  }
  \caption{Potenialverlauf für den harmonischen Oszillator} \label{fig:M2}
\end{figure}


\subsubsection{Der anharmonische Ansatz}
Einige Phänomene wie die Dissoziation des Moleküls bei Zufuhr
hinreichend hoher Energie oder das Auftreten von Kombinations- und Oberschwingungen lassen sich durch den
harmonischen Ansatz nicht erklären. Eine realistischere Betrachtung ist mit einem erweiterten Modell, dem des
\textit{anharmonischen Oszilattors}, möglich, dessen potentielle Energie annähernd durch das sog. Morse-Potential
beschrieben wird:
\begin{equation}
  V(r) = D \cdot [ 1- e^{-a(r-r_e)} ]^2
\end{equation}
Dies entspricht einer asymmetrischen Potentialkurve, deren Krümmung durch die Konstante $a$ charakterisiert wird. Der Faktor $D$
bedeutet die Summe aus Nullpunktsenergie $E _0$ und Dissoziationsenergie $E _D$.
\begin{figure}[!ht]
  \centering
  \resizebox {.7\linewidth} {!} {
    \begin{tikzpicture}
      \def\xmin{0}
      \def\xmax{10}
      \def\ymin{0}
      \def\ymax{6}

      (\xmax,\ymax);
      \draw[->] (\xmin,\ymin) -- (\xmax,\ymin) node[right] {$r$};
      \draw[->] (\xmin,\ymin) -- (\xmin,\ymax) node[above] {$V(r)$};

      \draw[] (1.25,6) parabola[parabola height=-4cm,very thick] bend (4.5,2) (6,3);
      \draw[] (6,3) .. controls (7,4.5) .. (10,4.5);

      \node (v0) at (0.6,1) {\scriptsize $V = 0$};
      \node (v1) at (0.6,2.25) {\scriptsize $ V = 1$};
      \node (v2) at (0.6,3) {\scriptsize $ V = 2 $};
      \node (v3) at (0.6,3.6) {\scriptsize $ V = 3$};

      \draw[ thick, dashed]  (1,1) -- (3.125,1) ;
      \draw[ thick, dashed]  (1,2.25) -- (2.5,2.25) ;
      \draw[ thick, dashed]  (1,3) -- (2.25,3) ;
      \draw[ thick, dashed]  (1,3.6) -- (1.75,3.6) ;

      \draw[ thick, dashed]  (4.75,1) -- (9,1) ;
      \draw[ thick, dashed]  (3.75,0.5) -- (10,0.5) ;

      \draw[thick, dashed]  (3.75,0) -- (3.75,0.5) ;
      \node (e0) at (8, 0.75) {\scriptsize $ E _0$};
      \draw[->,thick]  (8,0) -- (8,0.5) ;

      \draw[<-,thick]  (8,1) -- (8,2.25) ;
      \node (ed) at (8, 2.5) {\scriptsize $ E _D$};
      \draw[->,thick]  (8,2.75) -- (8,4.5) ;


      \draw[<-,thick]  (8.5,0.5) -- (8.5,2.3) ;
      \node (d) at (8.5, 2.5) {\scriptsize $ D$};
      \draw[->,thick]  (8.5,2.75) -- (8.5,4.5) ;


      \node (re) at (3.75, -0.25) {\scriptsize $ r _e$};

      \draw[] (2.9,1) -- (4.625,1);
      \draw[] (2.4,2.25) -- (5.45,2.25);
      \draw[] (2,3) -- (6,3);
      \draw[] (1.85,3.6) -- (6.4,3.6);

      \draw[->, thick] (3,1) -- (3,2.25) ;
      \draw[->, thick] (3.5,1) -- (3.5,3) ;
      \draw[->, thick] (4,1) -- (4,3.6) ;


    \end{tikzpicture}
  }
  \caption{Potenialverlauf für den anharmonischen Oszillator} \label{fig:M3}
\end{figure}

\subsection{IR-Spektroskopie}
Die Anregung einer \textit{Grundschwingung} kann dadurch beschrieben werden,
dass das Molekül unter Absorption eines Lichtquants vom 
Schwingungsgrundzustand in den nächsthöheren übergeht.
Dieser Vorgang ist nur dann möglich, 
wenn die damit verbundene Energieänderung 
gleich der Energie der einfallenden Lichtquanten ist:
\begin{equation}
  E _1 - E_0 = h. \nu ^{'}_{vib} = E _{LQ} = h \cdot \nu ^{'}_{LQ}
\end{equation}
Grundbedingung ist also, dass $\nu ^{'}_{vib} = \nu ^{'}_{LQ} $ ist ($LQ =$ Lichtquant). \par
Zur Aufnahme des Infrarot-Spektrums wird der Probe daher \textit{polychromatischer Strahlung} ausgesetzt,
deren Energie im mittleren IR-Bereich (3 \si{\micro\meter} -
50 \si{\micro\meter}) liegt; durch sukzessiven
Intensitätsvergleich mit einem die Probe nicht durchlaufenden Referenzstrahl können dann die Frequenzwerte
der absorbierten Strahlung festgestellt und somit die Schwingungsfrequenzen $\nu ^{'}_{vib} $ als Absolutwerte
ermittelt werden. \par
Als Folge der Anharmonizität sind neben dem Übergang eines Moleküls zum nächsthöheren Schwingungsniveau
(entsprechend der Auswahlregel $ \Delta v = +1$) auch solche mit $ \Delta v = +2, +3$ usw. erlaubt;
ihre Wahrscheinlichkeit und damit die Intensität der betreffenden Absorptionsbande nehmen
jedoch mit zunehmender Größe des Quantensprungs ab. Der Übergang $v =0 \rightarrow v =1$ entspricht der Grundschwingung,
$ v =0 \rightarrow v=2 $ der ersten Oberschwingung, die bei einer etwas kleineren als der doppelten Grundfrequenz
zu einer wesentlich schwächeren Bande führt, usw.
\subsection{Rotationsspektrum zweiatomiger Moleküle~\supercite{rovib}}
Das Modell zur Beschreibung von Rotationsübergängen 
ist das starre zweiatomige Molekül, auch als starrer Rotator bezeichnet. 
Beim starren Rotator sind zwei Atome mit den Massen $m_1$ und $m_2$ 
durch eine starre Bindung mit der Länge $r_0$ verbunden. 
Die Energie der Rotation ist quantisiert und man erhält nach Lösen 
der entsprechenden Schrödinger-Gleichung folgende Energieeigenwerte der Rotation:
\begin{equation}
\label{eq:erot}
E_{rot} = \hslash c B J (J+1) \,\, \text{ mit } J = 0, 1,2 
\end{equation}

\begin{equation}
\label{eq:bconst}
B = \frac{\hslash}{8 \pi ^2 c I} \,\,  \text{(B in } cm^{-1})  
\end{equation}
wobei $I$ das Trägheitsmoment des Moleküls, 
wie folgt zu berechnen ist.
\begin{align}
\label{eq:trag}
I &=\frac{m_1 m_2 }{m_1 + m_2} r_0^2\\
&=\mu r_0^2
\end{align}
Der Ausdruck $\mu$ ist die reduzierte Masse.
$B$ wird als Rotationskonstante bezeichnet; aus ihr kann die Bindungslänge des Moleküls
$r_0$ bestimmt werden. Die Rotationsenergieniveaus sind nicht äquidistant, sie
wachsen quadratisch mit der Rotationsquantenzahl J an.

\subsection{ Fourier-Transformations-Infrarotspektroskopie (FT-IR Spektroskopie)~\supercite{harris}}

Für das Infrarotgebiet ist die Fourier-Transform-Spektroskopie die wichtigste und verbreitetste Methode zur Sofortaufnahme eines kompletten
Spektrums. Das grundliegende Prinzip der FT-IR ist es,dass das Spektrum in seine Wellenlängenbestandteile zerlegt. In Abbildung ~\ref{fig:inter} wird ein Michelson-Interferometer dargestellt.  Die Strahlung der Quelle auf der linken Seite trifft auf einen Strahlteiler, der einen Teil des Lichtes durchlässt und den anderen Teil reflektiert. Wenn das Licht den Strahlteiler am Punkt \textbf{O} trifft, wird ein Teil davon auf einen stationären Spiegel
im Abstand \textbf{OS} reflektiert und der andere Teil auf einen beweglichen Spiegel im Abstand \textbf{OM} durchgelassen. Die von den Spiegeln reflektierten Strahlen laufen zurück zum Strahlteiler, wo jeweils die Hälfte jedes Strahles reflektiert und durchgelassen wird. Ein vereinigter Strahl geht weiter in Richtung zum Detektor, ein zweiter kehrt zur Lichtquelle zurück.
Im Allgemeinen sind die Strecken \textbf{OM} und \textbf{OS} nicht gleich, so dass die beiden den
Detektor erreichenden Strahlen gegeneinander phasenverschoben sind. Wenn die zwei Wellen,sich zueinander in Phase befinden, findet konstruktive Interferenz statt und es resultiert eine Welle mit doppelter Amplitude.
Sind die Wellen um eine Halbwelle $180^°$ gegeneinander verschoben,
findet destruktive Interferenz und Auslöschung der Wellen statt. Für jeden Phasenunterschied zwischen diesen beiden Extremen erfolgt eine teilweise Auslöschung.
\begin{figure}[!htbp]
  \begin{tikzpicture}[decoration=snake,
      spiegel/.style={rectangle,gray,fill=black!20,thin,minimum width=3cm,minimum height=0.1cm},
      teiler/.style={rectangle,black,dashed,rotate=45,minimum width=3.2cm,minimum height=0.07cm},
      probe/.style={rectangle,black,fill=black!20,dashed,rotate=90,minimum width=1.5cm,minimum height=0.7cm},
      ]{
     % \draw (0,0) rectangle (1,1) node[above] (b) {\footnotesize Strahlungsquelle};
      \path ( 5,4) node [spiegel,draw] (sspiegel) {}
        ( 5,0) node [teiler,draw] (steiler) {}
        ( 4.8,-4) node [probe,draw] (probe) {}
        ( 5,-7) node [circle,draw] (detektor) {\footnotesize Detektor}
        ( 8,0) node [spiegel,draw,rotate=90] (bild) {}
        ( 10,0) node [spiegel,draw,rotate=90,dashed] (spiegelbild) {}
        (-1,0) node [circle,draw,inner sep=0.3cm] (source) {};
      }
      \draw [->,decorate,decoration={snake,amplitude=1mm}] (steiler) -- (probe);
      \draw [->,decorate,decoration={snake,amplitude=2mm},
        transform canvas={xshift=-3mm,yshift=1mm}] (steiler) -- (probe);
      \draw [<->,skyblue1] (source) -- (steiler);
      \draw [<->,skyblue1] (sspiegel) -- (steiler);
      \draw [->,skyblue1] (steiler) -- (detektor);
      \draw [->,skyblue1,transform canvas={xshift=-3mm}] (steiler) -- (detektor);
      \draw [<->,skyblue1] (steiler) -- (bild);
      \draw [<->] (bild) -- (spiegelbild) node [midway,yshift=3mm] {\tiny Abstand $= \frac{\lambda}{4} $};
      \node [above] at (source.north) {Strahlungsquelle};
      \node [above] at (9,2) {\footnotesize{Beweglicher Spiegel \textbf{M}}};
      \node [above] at (sspiegel) {\footnotesize{stationärer Spiegel \textbf{S}}};
      \node [above,rotate=90] at (probe.north) {\footnotesize{Probe}};
      \node [below,yshift=-15mm] at (bild.south) {\footnotesize{ $ \delta=0$}};
      \node [below,yshift=-15mm] at (spiegelbild.south) {\footnotesize{ $ \delta= \frac{\lambda}{2} $}};
      \node [above] (texta) at (1.6,-4) {\scriptsize{die Lichtwellen von den Spiegeln M und S }};
        \node [above] at (1.5,-4.3) {\scriptsize{befinden sich zueinander gewöhnlich}}; 
        \node [above] at (1,-4.6) {\scriptsize{nicht in der Phase}}; 

        \node [above] (textb) at (3,1) {\footnotesize{Strahlungsteiler}};
      \draw [->] (texta) -- (4.5,-1.8);
      \draw [->] (textb) -- (steiler);
          \node [above] (textb) at (4.8,0) {\footnotesize{\textbf{O}}};

    \end{tikzpicture}
    \caption{Schematische Darstellung des Michelson-Interferometers}
    \label{fig:inter}
  \end{figure}
\subsection{Das Verfahren der Kombinationsdifferenz~\supercite{atkins}}
Um die Werte beider Rotationskonstante $B_0$ und $B_1$ zu berechnen, wird das Verfahren der Kombinationsdifferenz verwendet. Diese
Methode dient dazu, Informationen über einen bestimmten Zustand zu erhalten. Dazu werden zuerst die Ausdrücke für die Differenz der Wellenzahlen der Übergängen zu einem gemeinsamen Zustand gefunden.
Wie aus Abbildung ~\ref{fig:kombi} zu erkennen ist; besitzen die Übergänge  $\tilde{\nu} _R (J-1)$ und $\tilde{\nu} _P (J+1)$ einen gemeinsamen oberen Zustand. Sie sollten daher die gleiche Abhängigkeit von der Rotationskonstante $B_0$ zeigen:
\begin{equation}
\tilde{\nu} _R (J-1) - \tilde{\nu} _P (J+1) = 4 B_0 (J+\frac{1}{2})
\end{equation}

Aus neiner Auftragung der Wellenzahldifferenzen gegen $J + \frac{1}{2}$ wird eine Gerade mit der Steigung $4B_0$ erhalten, sodass auf diesem Weg die Rotationskonstante des Grundzustands bestimmt wird.

Die Übergänge  $\tilde{\nu} _R (J)$ und $\tilde{\nu} _P (J)$ haben einen gemeinsamen unteren Zustand. Die Differenz liefert $B_1$ bezüglich des oberen Zustandes.

\begin{equation}
\tilde{\nu} _R (J) - \tilde{\nu} _P (J) = 4 B_1 (J+\frac{1}{2})
\end{equation}


\begin{figure}[!htbp]

   \begin{tikzpicture}[]{
        \draw (0,1) -- (4,1) node[left=-12mm] {$J-1$};
        \draw (0,2) -- (4,2) node[left=-5mm] {$J$};
        \draw (0,3) -- (4,3) node[left=-12mm] {$J+1$};
     \draw (0,-2) -- (4,-2) node[left=-12mm] {$J+1$};
        \draw (0,-3) -- (4,-3) node[left=-5mm] {$J$};
         \draw (0,-4) -- (4,-4) node[left=-12mm] {$J-1$};
 
         \draw[color=violet] (0.5,-4) -- (0.5,2) node[pos=0.6,above=2pt,rotate=90] {$\tilde{\nu} _R (J-1)$};
         \draw[color=violet] (1.5,-2) -- (1.5,2) node[pos=0.44,above=2pt,rotate=90] {$\tilde{\nu} _P (J+1)$};
         \draw[color=violet] (2.5,-3) -- (2.5,3) node[pos=0.45,above=2pt,rotate=90] {$\tilde{\nu} _R (J)$};
         \draw[color=violet] (3.8,1) -- (3.8,-3) node[pos=0.4,above=2pt,rotate=90] {$\tilde{\nu} _P (J)$};
         \node [shading = axis,rectangle, left color=charcoal, right color=charcoal!30!white, anchor=north, minimum height=2cm] (box) at (-1,-2){$B_0$};
         \node [shading = axis,rectangle, left color=charcoal, right color=charcoal!30!white, anchor=north, minimum height=2cm] (box) at (-1,3){$B_1$};
 
       }
    \end{tikzpicture}
      \caption{Das Verfahren der Kombinatonsdifferenzen}
      \label{fig:kombi}

\end{figure}
\subsection{Birge-Sponer-Extrapolation~\supercite{atkins}}
Um die Dissoziationsenergie $hc \tilde{D} _0$ des Moleküls zu bestimmen, wird ein unter dem namen Birge-Sponer-Extrapolation bekanntes graphisches Verfahren verwendet. Es beruht auf der Tatsache, dass die Summe aller Abstände der Energieniveaus $\Delta \tilde{G}_{1/2}$ vom Grundzustand bis zur Dissoziationsgrenze die Dissoziationsenergie ergibt.
\begin{equation}
    \tilde{D} _0 = \Delta \tilde{G}_{1/2} + \Delta \tilde{G}_{3/2} + ... = \sum\limits_{\nu} \tilde{G} _{\nu +1/2}
\end{equation}
Wenn $\tilde{G} _{\nu +1/2}$ gegen $\nu + \frac{1}{2}$ aufgetragen wird, lässt es  die Summe (Dissoziationsenergie) sich aus der Fläche unter der Gerade berechnen.

\section{Auswertung}

\subsection{Untersuchung der Spektren}

Bei der Aufnahme der Spektren wurden die Auflösung sowie die Scanzahl variiert. Anhand der aufgenommenen Spektren wird ersichtlich, das mit höherer Scanzahl und Auflösung die Spektren deutlicher sind als mit geringerer Auflösung und Scanzahl. 
Wird die Anzahl der Scans bei einer Messung erhöht, so wird das Rauschen geringer und die Peaks können deutlicher erkannt werden. Bei niedriger Auflösung verschwimmen die Peaks wobei man bei einer höher gewählten Auflösung die Peaks besser unterscheiden kann.

Im Hintergrundsspektrum lassen sich Peaks bei einer Wellenzahl von circa 3600 \si{\per\centi\meter} und 1595 \si{\per\centi\meter}  
erkennen, welche für Wasser sprechen. Die sind tatsächlich drei normalen Modi des Wasser-Moleküls \supercite{atkins}. 
\begin{table}[htpb]
  \centering
  \caption{Schwingungsfrequenzen von $H_2O$ aus der Literatur~\supercite{atkins}}
  \label{tab:label}
  \begin{tabular}{llc}
    $v_1$ & 3652 \si{\per\centi\meter} & Streck\\  
    $v_2$ & 1595 \si{\per\centi\meter} & Deformations \\  
    $v_3$ & 3756 \si{\per\centi\meter} & Streck\\  
  \end{tabular}
\end{table}
Bei den Wellenzahlen von circa 2250-2400 \si{\per\centi\meter} erkennt man die Peaks von $CO_2$.

Am besten sollte bei hoher Auflösung und einer hohen Scanzahl messen, da die Spektren so am deutlichsten lesbar sind. 
Dies lässt sich durch einen Vergleich von Spektren am \textit{Anhang I} und am \textit{Anhang II} mit dem am \textit{Anhang III} verdeutlichen.

\subsection{Bestimmung der Bindungslänge des CO-Moleküls}
In dem Versuch sollte die Bindungslänge des CO-Moleküls mit Hilfe der Kombinationsdifferenz berechnet werden.
Die Werte, die für die Kombinationsdifferenz benötigt werden sind im R- bzw. P-Zweig enthalten. 
Hierzu werden zunächst die Rotationskonstanten $B_0$ und $B_1$ benötigt.
Durch Auftragung von $\Delta \tilde {\nu}$ gegen $J$ wird eine Gerade erhalten, 
mit deren Steigung die Rotationskonstanten berechnet werden können. 
Für den Grundzustand ergeben sich somit folgende Werte:

\begin{table}[htpb]
  \centering
  \caption{Werte für den Grundzustand}
  \label{tab:grund}
  \begin{tabular}{cccc}
    R-Zweig [$cm^{-1}$] & P-Zweig [$cm^{-1}$] & $\Delta \tilde {\nu}$ [$cm^{-1}$] & J+0.5\\
    2146.27 & 2135.39 & 10.88 & 1.5 \\
2150.72 & 2131.44 &  19.28 & 2.5\\
2154.58 & 2127.58 &  27.00 & 3.5\\
2157.95 & 2123.72 &  34.23 & 4.5\\
2161.81 & 2119.39 &  42.42 & 5.5\\
2165.67 & 2115.53 &  50.14 & 6.5\\
2169.04 & 2111.19 &  57.85 & 7.5\\
  
  
  \end{tabular}

\end{table}


\begin{figure}[H]
\centering
\includegraphics[width=12cm]{grundzustand}
\end{figure} 

Mit der Steigung der Regressionsgeraden m lässt sich nun die Rotationskonstante mit folgender Gleichung berechnen.

\begin{equation}
    m = B_0 \cdot 4
\end{equation}
 Hierraus folgt 
 \begin{equation}
     B_0 = \frac{m}{4} = \frac{7.79 cm^{-1}}{4} = 1.95 cm ^{-1}
 \end{equation}
  Die Rotationskonstante und der Abstand der Kerne stehen über folgende Gleichung im Zusammenhang:
  
  \begin{equation}
      B =\frac{\si{h}}{8\pi^2c\mu r^2}
  \end{equation}
  
  Nach der Berechnung und einsetzen der reduzierten Masse und der Rotationskonstante lässt sich der Abstand r berechnen. Die Bindungslänge im Grundzustand beträgt somit:
  
  \begin{equation}
      r = \sqrt {\frac{6.626\cdot 10^{-34} J\cdot s}{8\pi^2\cdot 2.99 \cdot 10^8 ms^{-1} \cdot  1.14\cdot 10^{-26}kg\cdot 1.95\cdot 10^2m^{-1}}}
  \end{equation}
  Somit beträgt r = 1.132 \cdot 10$^{-10}$ m = 113.2 pm
  Der Fehler für r wird wie folgt berechnet:
  
  Der Fehler der Steigung wird nach der Methode der kleinsten Quadraten bestimmt:
  
  \begin{equation}
      \Delta m = \sqrt{\frac{s^2_{x,y}}{\sum (x-\bar{x})^2}}
  \end{equation}
  
  Der Fehler $\Delta m$ beträgt somit 0.014 cm $^{-1}$
  
  Hier raus folgt $\Delta B_0$ 
  
  \begin{equation}
      \Delta B_0 = \sqrt{\frac{dB_0}{dm}\cdot \Delta m^2} = \sqrt{\frac{1}{4}\cdot (0.014 cm^{-1})^2} = 0.007 cm^{-1}
  \end{equation}
  
  Somit ergibt sich für den Fehler der Bindungslänge:
  \begin{equation}
      \Delta r = \sqrt{\frac{dr}{dB_0} \cdot (\Delta B_0)^2} =  \sqrt{-\frac{h}{8\pi^2c\mu B_0^2}\cdot (0.007 cm^{-1})^2} = 5.72 \cdot 10^{-14} m 
  \end{equation}
  
  Mit Einbezuge des Fehlers beträgt der Abstand somit 113.2 \pm 0.057 pm
  
  Für den angeregten Zustand ergeben sich folgende Werte.
  
  \begin{table}[htpb]
  \centering
  \caption{Werte für den angeregten Zustand}
  \begin{tabular}{cccc}
    R-Zweig [$cm^{-1}$] & P-Zweig [$cm^{-1}$] & $\Delta \tilde {\nu}$ [$cm^{-1}$] & J+0.5\\
2150.72 & 2139.15 &  11.57 & 1.5\\
2154.58 & 2135.29 &  19.29 & 2.5\\
2157.95 & 2131.44 &  26.51 & 3.5\\
2161.81 & 2127.58 &  34.23 & 4.5\\
2165.67 & 2123.72 &  41.95 & 5.5\\
2169.04 & 2119.39 &  49.65 & 6.5\\
2172.42 & 2115.53 &  56.89 & 7.5\\
  
  \end{tabular}

\end{table}
 
 \begin{figure}[H]
\centering
\includegraphics[width=12cm]{angeregt}

\end{figure} 

 
 Nach Gleichung (19) ergibt sich für B$_1$ im angeregten Zustand  B$_1$ = 1.90 cm$^{-1}$
 
 Nach Gleichung (13) ergibt sich für die Bindungslänge im angeregten Zustand somit 113.8 pm  Der Fehler wird Analog zur Berechnung für den Grundzustand durchgeführt. 
 Hierraus folgt:
$\Delta$m = 0.021 cm$^-1$ daraus folgt, das $\Delta$B$_1$ = 0.011 cm$^{-1}$. Somit beträgt der Fehler für die Bindungslänge nach Gleichung (16) 9.08\cdot 10$^{-14}$ m.

Mit Angabe des Fehlers beträgt der berechnete Abstand somit 113.8 \pm 0.091 pm:


\subsection{Bestimmung der Dissoziationsenergie des CO-Moleküls}

Die Dissoziationsenergie kann mit Hilfe der Birge-Spooner-Extrapolation berechnet werden. Hierfür werden zunächst die Werte des Q-Zweiges aus den Mittelwerten der ersten Übergänge des P- und R-Zweiges ermittelt.
Die verwendeten Werte für die Birge-Spooner-Extrapolation betragen somit:

\begin{equation}
    \frac{2146.27cm^{-1}+2139.15cm^{-1}}{2}= 2142.71 cm^{-1}
\end{equation}

\begin{equation}
    \frac{4256.13cm^{-1}+4263.36cm^{-1}}{2}-2142.71cm^{-1}= 2117.04cm^{-1}
\end{equation}

Die Dissoziationsenergie kann aus der Fläche zwischen dem Graphen und den Achsen ermittelt werden mit:

 \begin{figure}[H]
\centering
\includegraphics[width=12cm]{birg}
\end{figure} 

\begin{equation}
    y= -25.67x+2155.5
\end{equation}

Nach Nullsetzen der Funktion erfolgt x = 83.97

Der Flächeninhalt des Dreiecks wird wie folgt berechnet:

\begin{equation}
    A = \frac{2155.5cm^{-1} \cdot 83.97}{2} = 90498.668cm^{-1} = 9049866.8 m^{-1}
\end{equation}

Der Flächeninhalt steht wie in folgender Gleichung mit der Dissoziationsenergie in Beziehung:

\begin{equation}
    A = \frac{D_0}{h\cdot c}
\end{equation}

Nach umformen:
\begin{equation}
    D_0 = A \cdot h \cdot c \cdot N_A
\end{equation}

Somit beträgt die Dissoziationsenergie 1079.71 kjmol$^{-1}$

\section{Diskussion}

Die Literaturwerte für die Bindungslänge und die Dissoziationsenergie des Kohlenmonoxid betragen bei 298K 112.83pm \supercite{atkins} und 1076kjmol $^{-1}$ ~\supercite{atkins}

Der berechnete Wert der Bindungslänge weicht nur um 0.3 \% ab. Die berechnete Bindungslänge war größer als die Literaturangabe, was an Temperaturunterschieden liegen kann. Die berechnete Dissoziationsenergie ist geringfügig größer als der Literaturwert (0.3\%]  Dies kann an der Verwendung der Birge-Spooner.Exrapolation liegen, da die Kurve nicht linear verläuft.

\printbibliography



\end{document}

