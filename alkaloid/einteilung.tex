\begin{frame}[t]{Alkaloiden - Klassifizierung}
\begin{quote}
  Die \enquote{echten} Alkaloide werden nach den in ihnen enthaltenen
  heterocyclischen Ringsystemen eingeteilt.
\end{quote}
  \begin{table}[htpb]
    \tiny
    %\caption{Wichtige Gruppen der Alkaloide mit Aminosäurenvorstufen}
    \begin{tabular}{llll}
      \hline
      Strukturgruppe & Alkaloid Gruppe & Vorstufe & Beispiele \\
      \hline
      \chemfig[][scale=0.5]{
        =^[:270]% 2
        -[:330]% 3
        =^[:30]% 4
        -[:90]% 5
        (
        =^[:150]% 6
        -[:210]% -> 1
        )
        -[:30]% 7
        =_[:330]% 8
        -[:270]% 9
        =_[:210]N% 10
        (
        -[:150]% -> 4
        )
      }& Chinolin & Tryptophan & Chinin \\
      \multirow{3}{*}{\chemfig[][scale=0.5]{
        -[:300.8]N% 2
        -[:279.7]% 3
        -[:233.9,1.069]% 4
        -[:31.3,0.88]% 5
        -[:65.7,0.943]% 6
        (
        -[:141.9,0.864]\phantom{N}% -> 2
        )
        -[:1,1.126]% 7
        -[:279.6,1.005]% 8
        (
        -[:30,,,1]OH% 10
        )
        -[:140.3,0.813]% 9
        (
        -[:180,1.178]% -> 3
        )
    } } &  &  & Cocain \\
        & Tropan & Ornithin& Scopolamin\\
        && & Hyoscyamin\\
        &&&\\
        &&&\\
   \multirow{4}{*}{\chemfig[][scale=0.5]{
        R=[:0]% 6
       -[:300]% 5
       =^[:0]% 4
                 (
           -[:300]% 34
                     (
            -[:0,,,1]OH% 36
                     )
           =[:240]O% 35
                 )
    -[:60,,,1]NH% 3
    -[:120,,1]% 2
                 (
           -[:180]% 1
           -[:240]% -> 6
                 )
        <[:60]% 37
                 (
           =[:120]O% 38
                 )
    -[:0,,,1]OH% 39
    } } & Betalaine & Tyrosin & Betanidin \\
        && & Indicaxanthin\\
        &&&\\
        &&&\\
        &&&\\
        &&&\\
\multirow{4}{*}{\chemfig[][scale=0.5]{
    =^[:270]% 2
     -[:330]% 3
     =^[:30]% 4
      -[:90]% 5
               (
        =^[:150]% 6
         -[:210]% -> 1
               )
      -[:18]% 7
    =_[:306]% 8
     -[:234]\chembelow{N}{H}% 9
               (
         -[:162]% -> 4
               )
    } } & Indol & Tryptophan & Harmin \\
        && & Ergolin\\
        && & Strychinin\\
        && & Reserpin
    \end{tabular}
  \end{table}

\end{frame}
\begin{frame}[t]{Alkaloide - Klassifizierung}
   \begin{table}[htpb]
    \tiny
  %  \caption{Wichtige Gruppen der Alkaloide - II}
    \begin{tabular}{llll}
      \hline
      Strukturgruppe & Alkaloid Gruppe & Vorstufe & Beispiele \\
      \hline
      \multirow{3}{*}{\chemfig[][scale=0.5]{
    =^[:270]% 2
     -[:330]% 3
     =^[:30]% 4
     -[:330]% 5
     =^[:30]N% 6
      -[:90]% 7
    =^[:150]% 8
     -[:210]% 9
               (
         -[:270]% -> 4
               )
    =^[:150]% 10
               (
         -[:210]% -> 1
               ) 
    } } & Isochinolin & Tyrosin & Morphin \\
        && & Codein\\
        && & Papaverin\\
   \chemfig[][scale=0.5]{
       =^[:180]% 2
     -[:240]% 3
    =^[:300]N% 4
           -% 5
     =^[:60]% 6
               (
         -[:120]% -> 1
               )
    }  & Pyridin & Ornithin & Nicotin \\
\multirow{4}{*}{\chemfig[][scale=0.5]{
     =_[:330]% 2
     -[:270]% 3
               (
         -[:342]N% 7
         =^[:54]% 8
         -[:126]\chemabove{N}{H}% 9
         -[:198]% -> 2
               )
    =_[:210]N% 4
     -[:150]% 5
     =_[:90]N% 6
               (
          -[:30]% -> 1
               )
    } } & Purin & Glycin & Adenin \\
        && & Guanin\\
        && & Theobromin\\
        && & Coffein
    \end{tabular}
  \end{table}

 
\end{frame}
  \begin{frame}[t]{Alkaloide - weitere Einteilung}
    \begin{quote}
      Angehörige einer dieser Alkaloidgruppen können weiter nach am genannten
      Ringsystem ankondensierten zusätzlichen Ringen bestimmten Typen zuordnen.  
    \end{quote}
  \begin{table}[htpb]
    \tiny
  %  \caption{Wichtige Gruppen der Alkaloide - II}
    \begin{tabular}{llll}
      \hline
      Gerüßt & Alkaloid & Typ & Zusatzfunktionen \\
      \hline
      \chemfig[][scale=0.5]{
          -[:270]% 2
       >[:330]% 3
        -[:30]% 4
                 (
        -[:90,,,1]NH% 15
        -[:150,,1]% 16
           -[:210]% -> 1
                 )
      <:[:330]% 5
       -[:270]% 6
    =_[:308.5]% 7
       -[:240]\chembelow{N}{H}% 8
     -[:171.5]% 9
      =_[:210]% 10
       -[:150]% 11
       =_[:90]% 12
        -[:30]% 13
                 (
            -[:90]% -> 3
                 )
      =_[:330]% 14
                 (
            -[:30]% -> 6
                 )
                 (
           -[:270]% -> 9
                 ) 
      }  & Ergolin & Ergolin & Isopren-Einheit, Methyl \\
   \chemfig[][scale=0.5]{
              -[:30]% 2
       >:[:330]% 3
        -[:270]% 4
        >[:330]% 5
         -[:30]% 6
        >:[:90]% 7
        -[:150]% 8
                  (
            -[:210]% -> 3
                  )
      <:[:98.6]N% 9
                  (
    -[:230.2,2.036]% 10
    -[:315.5,2.036]% -> 5
                  )
       -[:47.1]% 11
      -[:355.7]% 12
      -[:304.3]% 13
     =^[:252.9]% 14
                  (
          -[:201.4]% -> 7
                  )
      -[:324.9]\chembelow{N}{H}% 15
       -[:36.9]% 16
     =^[:108.9]% 17
                  (
          -[:180.9]% -> 13
                  )
       -[:48.9]% 18
     =_[:348.9]% 19
                  (
           -[:48.9]O% 22
          -[:348.9]% 23
                  )
      -[:288.9]% 20
     =_[:228.9]% 21
                  (
          -[:168.9]% -> 16
                  )    
      }  & Ibogain & Iboga & Monoterpen-Einheit, Alkoxy \\
       \chemfig[][scale=0.5]{
                  O% 12
           =[:210]% 11
           -[:150]% 10
           -[:210]% 9
         <[:135.7]O% 8
         -[:187.8]% 7
           -[:240]% 6
        =^[:292.2]% 5
    -[:280.1,1.91]% 4
      -[:333,1.91]N% 3
           -[:354]% 2
            -[:66]% 1
           >[:138]% 19
           -[:210]% 18
                     (
               -[:282]\phantom{N}% -> 3
                     )
           <[:150]% 17
            >[:90]% 16
                     (
             -[:164.3]% -> 5
                     )
            -[:30]% 15
                     (
                -[:90]% -> 9
                     )
           <[:330]% 14
                     (
               -[:270]% -> 19
                     )
           <:[:30]N% 13
                     (
                -[:90]% -> 11
                     )
         -[:308.5]% 25
                =_% 24
           -[:300]% 23
          =_[:240]% 22
           -[:180]% 21
          =_[:120]% 20
                     (
             -[:171.5]% -> 19
                     )
                     (
                -[:60]% -> 25
                     )
       }  & Strychnin & Strychnin & Secoiridoid monoterpene \\
   
    \end{tabular}

  \end{table}
 


  \end{frame}



