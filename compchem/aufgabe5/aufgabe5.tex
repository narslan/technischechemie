\documentclass[12pt]{article}
\usepackage{amsmath,mathtools}
\usepackage[usenames,dvipsnames]{xcolor}
\usepackage[german]{babel}
\usepackage[utf8]{inputenc}
\usepackage[T1]{fontenc}
\usepackage{textcomp}
%\usepackage{libertine}
%\usepackage{helvet}

\usepackage{microtype}
% Minion and Myriad fonts
%\usepackage[minionint,mathlf]{MinionPro}
%\renewcommand{\sfdefault}{Myriad-LF}
\usepackage{siunitx}
\usepackage{fancyhdr}
\usepackage{sectsty}
\usepackage{setspace}
\usepackage{booktabs} % To thicken table lines
\usepackage[version=4]{mhchem}
\usepackage[draft]{graphicx}
\usepackage[labelfont=bf]{caption}
\usepackage{subcaption}
\usepackage{chemstyle}

%\usepackage[compatibility=4.7,language=german]{chemmacros}
\renewcommand{\familydefault}{\sfdefault}
\sisetup{detect-all}
\usepackage{chngcntr}
\counterwithin{table}{section}
\counterwithin{figure}{section}

\usepackage{titlesec}

\titleformat*{\section}{\large\bfseries}
\titleformat*{\subsection}{\normalsize\bfseries}

\usepackage{geometry}
 \geometry{
 a4paper,
 left=20mm,
 top=30mm,
 right=20mm
 }

\begin{document}

\begin{onehalfspace}

\raggedright
\section{Aufgabe 5 S\textsubscript{N}2-Substitution des Br- durch Cl- an CH\textsubscript{3}Br}

Ziel dieser Aufgabe ist die Bestimmung des Übergangszustandes bei der
Substitution des Bromidions durch ein Chloridion an CH\textsubscript{3}Br,
sowie die Auswahl der Startgeometrie (die Geometrie am höchsten höchsten Punkt
des Energieprofiles, bei der schrittweisen annäherung des Chloridions an an
das Methylbromid) und die Berechnung der Aktivierungsbarriere und der Ausbeute
dieser Reaktion.


\subsection{Bestimmung des Reaktionspfades und Auswahl einer geeigneten Startgeometrie}



\textbf{Auswertung 1}:

\begin{figure}[!htbp]
\centering
  \includegraphics[width=0.8\textwidth]{data/a5_teil1_scan.png}%
  \caption{Optimierte Geometrie des Übergangszustandes}
\end{figure}
\noindent

 Die Gesamtenergie für die Geometrie beträgt -3071.56972781 \si{\hartree}.
 Der Basissatz ist aug-cc-pVDZ. Die Punktgruppe beträgt C\textsubscript{1}.

\textbf{Auswertung 2}\\

Nach der Optimierung kommt es zur änderung der Punktgruppe von
C\textsubscript{1} zu C\textsubscript{3V}. Die Gesamtenergie beträgt nun
-3071.56971543 \si{\hartree}. Die Bildungsenthalpie beträgt -3071.560004
\si{\hartree} Die optimierte Bindungslänge der C-Cl bindung beträgt 2.45
\si{\angstrom}

Der Zustand besitzt die imaginäre Schwingungsfrequenz -390.12
cm\textsuperscript{-1} Die Bewegung bei dieser Frequenz entspricht der
Walden'sche Umkehr.

\begin{figure}[!htbp]
\centering
  \includegraphics[width=\textwidth]{data/A5_opt_darstellung.png}%
  \caption{Optimierte Geometrie des Übergangszustandes}
\end{figure}
\noindent

\subsection{Optimierung und Frequenzrechnung des Übergangszustandes}

\textbf{Auswertung 1}\\
\begin{table}[!htpb]
\centering
\begin{tabular}{ccc}
\toprule
Molekül & E/\textit{Hartree} & Bildungsenthalpie $\Delta G$  \textit{Hartree}\\
\ce{CH_3Br}  & -2612.00556924 & -2611.99026 \\
\ce{CH_3Cl}  & -499.12227340 & -499.105102\\
\ce{Br^-}  & -2572.46248971 & -2572.478665 \\
\ce{Cl^-}  & -459.56364460 & -459.578667 \\
\midrule
\bottomrule
\end{tabular}
\caption{Molekülbindungen und Energien}
\end{table}
Mit Hilfe der ermittelten Werte kann nun die freie Enthalpie wie folgt berechnet werden:
\begin{equation}
\Delta _R G = \sum\limits_{Produkte} \Delta _B G - \sum\limits_{Edukte} \Delta _B G
\end{equation}
Nach einsetzen der Werte erhält man $ \Delta _R G = -0.014834$ \si{\hartree}. Nach Umrechnung des Ergebnisses in  \si{\kilo\joule\per\mol},erhält man $ \Delta _R G = -38.95$ \si{\kilo\joule\per\mol}.
Die freie Aktivierungsenthalpie wird definiert als Energiedifferenz der Edukte und der raktiven Zwischenstufe. Und wird wie folgt berechnet:
\begin{equation}
\Delta _R G^{\neq} = \sum\limits_{Edukte} - \Delta _B G _U
\end{equation}
$\Delta _B G _U$ stellt hier die Energie der reaktiven Zwischenstufe dar und  beträgt $\Delta _B G _U = -3071.560004$ \si{\hartree} (Wert aus Aufgabe 5 Teil 1). $\Delta _R G^{\neq}$ steht für die freie Aktivierungsenthalpie. Nach der Berechnung und Umrechnung in SI-Einheiten ergibt sich für die freie Aktivierungsenthalpie: $\Delta _R G^{\neq} = 24.178$ \si{\kilo\joule\per\mol}.

\noindent
 \textbf{Auswertung 2:} Bei Betrachtung des Diagrammes ~\ref{figure:pfad} im
Skript auf S.51 mit der auftragung der Reaktionsenthalpie gegen die Ausbeute
für Reaktionen bei 25 °C bzw. 100 °C, lässst sich für den Berechneten Wert
eine Ausbeute von annähernd 100\% vorraus sagen. Aus dem niedrigen Wert für
die freie Aktivierungsenthalpie lässt sich schließen, dass die Reaktion sehr
schnell ablaufen wird, da der Graph der halbwertszeiten für die Reaktion 2.
Ordnung mit  Aktivierungsenthalpien unter $\Delta _R G^{\neq} = 90$
\si{\kilo\joule\per\mol} stark gegen Null geht. Zur Betrachtung eines Fehlers
von $\pm 10$ \si{\kilo\joule\per\mol} lässt sich sagen das sich dieser nur
gerfingfügig auf eine Vorhersage auswirkt, da es auch bei einem so hohen
Fehler nur zu unsignifikanten änderung der Reaktionszeit und der Ausbeute
kommen würde.
\subsection{Bestimmung des Reaktionspfades}
\textbf{Auswertung 1}
 \begin{figure}[!htpb]
 \centering
 \includegraphics[width=\textwidth,height=\textheight,keepaspectratio]{data/potentialkurve.png}%
\captionsetup{justification=raggedright}
 \caption{Reaktionspfad: Auftragung der Energie der optimierten Geometrien gegen den C-Cl-Bindungsabstand}
 \label{figure:pfad}
\end{figure}
\subsection{Bestimmung des Reaktionspfades}


Die Bewegung, welche bei der S\textsubscript{N}2 Reaktion abläuft, ist die
eines Rückseitenangriffs des Chloridions an das Methylbromid. Bei dieser
Reaktion kommt es zunächst zu einem Rückseitenangriff des Chloridions,
daraufhin zur Bildung eines Intermediates und letztlich zum Bindungsbruch zum
Bromidion. Diese Bewegung ist aperiodisch.

Die Vorstellung des \glqq umklappenden Regenschirms \grqq ist unpassend, da
die H-Atome bei der Übergangsbewegung starr auf ihrem Platz bleiben, somit
nicht umklappen, sondern das C-Atom die Fläche der H-Atome durchwandert
(Inversion am C-Zentrum).
 \end{onehalfspace}
 \end{document}
