\documentclass[12pt]{article}
\usepackage{amsmath,mathtools}
\usepackage[usenames,dvipsnames,table]{xcolor}



\usepackage[german]{babel}
%\usepackage[utf8]{inputenc}




\usepackage{textcomp}
%\usepackage{libertine}
%\usepackage{helvet}
\usepackage{lscape}



\usepackage{microtype}
% Minion and Myriad fonts
\usepackage[minionint,mathlf]{MinionPro}
\renewcommand{\sfdefault}{Myriad-LF}
\renewcommand*\familydefault{\sfdefault} %% Only if the base font of the document is to be sans serif
\usepackage[T1]{fontenc}
\usepackage{siunitx}
\usepackage{fancyhdr}
\usepackage{sectsty}
\usepackage{setspace}
\usepackage{booktabs} % To thicken table lines
\usepackage[version=4]{mhchem}
\usepackage[draft]{graphicx}
\usepackage[labelfont=bf]{caption}
\usepackage{subcaption}
\usepackage{chemstyle}
\usepackage{tabularx}
\usepackage[framemethod=tikz]{mdframed}
\usepackage{pgfplots}
\usepackage{pgfplotstable}
\usepackage{csquotes}
\pgfplotsset{compat=newest}
\usepgfplotslibrary{units}
\mdfdefinestyle{mystyle}{%
  innerleftmargin=10,
  innerrightmargin=10,
  outerlinewidth=3pt,
  topline=false,
  rightline=false,
  leftline=false,
  bottomline=false,
  skipabove=\topsep,
  skipbelow=\topsep
}

%\usepackage[compatibility=4.7,language=german]{chemmacros}
\usepackage[backend=biber,sorting=none,autocite =superscript,natbib=true]{biblatex} \addbibresource{books.bib}
\renewcommand{\familydefault}{\sfdefault}
\sisetup{detect-all}
\usepackage{chngcntr}
\counterwithin{table}{section}
\counterwithin{figure}{section}
\usepackage{tabularx}
\usepackage{tikzorbital}
\usepackage{chemfig}
\usepackage{url}

%\usepackage[compatibility=4.7,language=german]{chemmacros}
\renewcommand{\familydefault}{\sfdefault}
\sisetup{detect-all}
\usepackage{chngcntr}
\counterwithin{table}{section}
\counterwithin{figure}{section}

\usepackage{titlesec}
\renewcommand*\printatom[1]{\ensuremath{\mathsf{#1}}}
\titleformat*{\section}{\large\bfseries}
\titleformat*{\subsection}{\normalsize\bfseries}
\usepackage{geometry}
 \geometry{
 a4paper,
 left=20mm,
 top=30mm,
 right=20mm
 }
\usepackage{titlesec}

\titleformat*{\section}{\large\bfseries}
\titleformat*{\subsection}{\normalsize\bfseries}

\usepackage{geometry}
 \geometry{
 a4paper,
 left=20mm,
 top=30mm,
 right=20mm
 }
 \usepackage[justification=justified,singlelinecheck=false]{caption}
% header
\usepackage{fancyhdr} \pagestyle{fancy} \fancyhf{}
 \rhead{23.12.2016}
\lhead{Computerchemie Justin König, Nevroz Arslan} \setlength{\headheight}{15pt}
\cfoot{\thepage }

%Farben
\definecolor{skyblue1}{RGB}{135, 206, 250}
\definecolor{flame}{RGB}{226, 88, 34}
\definecolor{scarletred1}{RGB}{252, 40, 71}
\definecolor{cyanblau}{RGB}{0, 158, 224}
\definecolor{darkblue}{RGB}{0, 0,139}
\definecolor{charcoal}{rgb}{0.21, 0.27, 0.31}
\definecolor{turquoise}{rgb}{0 0.41 0.41}
\definecolor{rouge}{rgb}{0.79 0.0 0.1}
\definecolor{vert}{rgb}{0.15 0.4 0.1}
\definecolor{mauve}{rgb}{0.6 0.4 0.8}
\definecolor{violet}{rgb}{0.58 0. 0.41}
\definecolor{orange}{rgb}{0.8 0.4 0.2}
\definecolor{bleu}{rgb}{0.39, 0.58, 0.93}
\definecolor{azulen}{RGB}{0,218,255}






\begin{document}
\tableofcontents 
\newpage
\begin{onehalfspace}
\raggedright
\section{Berechnung der Hartree-Fock-Energie von Wasser}
\subsection{Berechnung der Gesamtenergie einer Geometrie von Wasser}

Ziel der Aufgabe, ist die Berechnung der Hartree-Fock-Energie von Wasser. Verwendet wird der cc-pVQZ Basissatz.


\textbf{Auswertung 1} Fünf grundlegende Parameter einer quantenmechanischen Rechnung sind die Geometrie, die Ladung, die Multiplizität, die Basisparameter und die QC-Methode. \\
\textbf{Auswertung 2} Für das nicht geometrieoptimierte Wassermolekül wurde eine Gesamtenergie von -76.06099515 \si{\hartree} mit der Punktgruppe $C _s$ berechnet.\\


\subsection{Gesamtenergie in Abhängigkeit von der Geometrie}
Ziel des zweiten Teiles der Aufgabe ist das Aufzeigen der Abhängigkeit der Gesamtenergie des Wassermoleküls von der Geometrie. Hierzu wurde diese optimiert.\\
\textbf{Auswertung 1}:\\
\begin{figure}[!htbp]
\centering
\includegraphics[width=.9\textwidth,height=.9\textheight,keepaspectratio]{data/h2Omolekul_teil1.png}
  \caption{Optimierte Geometrie von Wasser (HF/cc-pVQZ)}
\end{figure}
\noindent
Die Gesamtenergie für die optimierte Geometrie beträgt -76.065518 \si{\hartree}, 
diese ist negativer als die berechnete Energie in Teil 1 der Aufgabe, wodurch eine höhere Stabilität vorher gesagt werden kann. \\
\newpage

\begin{table}[htbp]
\centering
\caption{Die Partialladungen von Sauerstoffatom und Wasserstoffatom im Wassermolekül}
\begin{tabular}{ll}
\toprule
Atom &  Partialladungen in \si{\elementarycharge} ( \si{\elementarycharge}  $= 1.602 \times 10^{-19}$ \si{\coulomb})\\
O & -0.512\\
H &  0.256\\
\midrule
\bottomrule

\end{tabular}
\end{table}
\noindent
Die Summe aller Partialladungen ergibt 0, 
dies entspricht einem neutral geladenem Molekül. 
Das Dipolmoment hat einen Wert von + 1.9648 Debye und ist entlang der \textit{z}-Achse ausgerichtet. 
Der Punktgruppenwechsel beruht auf der Geometrieoptimierung
welche in Teil 2 ausgeführt wurde. 
Gestartet wurde mit einer Geometrie die als Punktgruppe $C_s$ ausgegeben hat, da die einzige Spiegelebene
$\sigma_v$ durch die drei Moleküle (1,2 und 3) lag. 
Durch die Geometrieoptimierung im 2. Teil wurde die Symmetrie verbessert und es ergab sich der Punktgruppe $C _{2v}$, 
da durch die Optimierung zusätzlich eine Spiegelebene $\sigma_v^{\prime}$ und eine Drehachse $C_2$ entlang der \textit{z}-Achse existiert.\\

\noindent
\textbf{Auswertung 2}\\
Die Optimierung der Geometrie sollte nachvollzogen werden.
\begin{figure}[!htp]
\centering
  \includegraphics[width=.9\textwidth,height=.9\textheight,keepaspectratio]{data/water.png}%
  \caption{Auftragung der Gesamtenergie gegen die Optimierungsschritte (RHF/cc-pVTZ)}
\end{figure}
\newpage
Die Geometrieoptimierung lief in 6 Schritten ab.
Im Diagramm ist zu erkennen das die Gesamtenergie vor allem 
in den ersten zwei Optimierungsschritten gesenkt wurde. Hiernach sinkt diese langsamer.
\\

\noindent
\textbf{Auswertung 3}\\
Durch die Änderung der Geometrie sind folgende Terme des Hamilton-Operators betroffen: 
$\hat{V} _{K,K}(\vec{R})$ und $\hat{V} _{K,el}(\vec{r},\vec{R})$.

\begin{equation}
 H\textsubscript{el} = H - T\textsubscript{K} = T\textsubscript{el} + V\textsubscript{k,el} + V\textsubscript{el,el} + V\textsubscript{K,K}   
\end{equation}

Somit würde sich eine Veränderung der Geometrie auf die elektronische Gesamtenergie auswirken, da der elektronische Hamiltonian ebenfalls betroffen wäre.

\subsection{Gesamtenergie in Abhängigkeit des Basissatzes}
In Teil 3 der Aufgabe wird die Energie des optimierten Wassermoleküls,
mit zwei unterschiedlichen Basissätzen berechnet, diese sind STO-3G und cc-pVDZ. 
Die Methoden sollen anschließen mit dem im Teil 2 genutzten Basissatz cc-pVQZ verglichen werden.


\noindent
\textbf{Auswertung 1}\\
\begin{table}[!htpb]
\centering
\caption{Vergleich der Basissätze}
\begin{tabular}{lrrr}
\toprule
 &
STO-3G &
cc-pVDZ & cc-pVQZ \\
\midrule
Punktgruppe & $C _{2v}$ & $C _{2v}$ & $C _{2v}$ \\
Energie \si{\hartree}    & -74.95896952 & -76.02690370 & -76.06551853 \\
$S _O / \textit{e}$ & -0.386 & -0.287  & -0512\\
$S _H / \textit{e}$ & +0.193 & +0.144  & +0.256\\
$\nu / \textit{Debye}$ & 1.7368 & 2.0130  & 1.9648\\
\bottomrule
\end{tabular}
\end{table} 
\textbf{Auswertung 2}\\
Aus den in der Tabelle aufgelisteten Werten lässt sich folgern, das der cc-pVDZ Basissatz
die höchste Qualität der genutzten Basissätze hat, da er die niedrigste und somit negativste Energie aufweist. 
Der STO-3G stellt somit den schlechtesten Basissatz dar. Dies lässt folgern aus 

\begin{equation}
    E\textsubscript{korr} = E\textsubscript{exakt} - E\textsubscript{HF}
\end{equation}

Somit liegt ein möglichst niedriger Wert dem exakten Energiewert näher. Jedoch sollte er auch nicht zu negativ sein da es hierdurch wieder zu Abweichung kommen kann.
 Bei den Partialladungen weisen beide Basissätze große Unterschiede
 nicht nur zueinander, sondern auch zu der optimierten Geometrie aus Teil 2 der ersten Aufgabe auf.
\subsection{Gesamtenergie in Abhängigkeit von der quantenchemischen Methode}
In dieser Aufgabe soll die Abhängigkeit der Gesamtenergie von der quantenmechanischen Methode aufgezeigt werden.
Hierzu wird die Gesamtenergie des optimierten Wassermoleküls mit dem Basissatz cc-pVTZ berechnet, jedoch bei jeder Rechnung eine andere Methode verwendet. Diese sind MP2, CCSD und CCSD(T).\\
\textbf{Auswertung 1}\\

\begin{table}[!htpb]
\centering
\caption{ Vergleich der QM-Methoden}
\begin{tabular}{ll}
\toprule
Methode &   Energie \si{\hartree} \\
\midrule

MP2 & -76.34694522\\
CCSD & -76.35028785 \\
CCSD(T) & -76.35909660\\
\bottomrule
\end{tabular}
\end{table}

Aus den Werten der Tabelle, lässt sich folgern, das die CCSD(T) Methode die beste Qualität besitzt und die HF die geringste.
Somit lässt sich sagen das die CCSD(T) gefolgt von CCSD und MP2 besser sind als HF welche das Schlusslicht bildet.
Es ist zu erkennen, dass die sich für die Gesamtenergie errechneten Werte relativ nah beieinander liegen, jedoch in den Nachkommastellen voneinander abweichen. Die CCSD(T) Methode ist die Komplexeste, welche zugleich die niedrigste der drei Energien aufweist.


\section{Geometrieoptimierung und Frequenzberechnung an Dichlorethen}
In dieser Aufgabe wird die Bedeutung der Geometrieoptimierung erklärt.
Dafür werden die Schwingungsfrequenzen von \textit{cis-} und \textit{trans}-1,2-Dichlorethen auf HF-Niveau und MP2-Niveau berechnet.
Die angewandten quantumchemischen Methoden werden im Zusammenhang von Elektronenkorrelation verglichen.\\
\textbf{Auswertung 1 }
\begin{figure}[!hptb]
    \caption{Die optimierten Geometrien}
    \begin{subfigure}[b]{0.4\textwidth}
       \fbox{  \includegraphics[width=\textwidth]{data/cis_darstellung.png}}
                \subcaption{\textit{cis}-1,2-Dichlorethen }
    \end{subfigure}
    ~ %add desired spacing between images, e. g. ~, \quad, \qquad, \hfill etc.
      %(or a blank line to force the subfigure onto a new line)
    \begin{subfigure}[b]{0.4\textwidth}
       \fbox{  \includegraphics[width=\textwidth]{data/trans_darstellung.png}}
        \subcaption{ \textit{trans}-1,2-Dichlorethen }
    \end{subfigure}
    \label{figure:opt}
\end{figure}


\begin{table}[!htpb]

\caption{Gesamtenergien und Symmetrien von \textit{cis-} und \textit{trans}-1,2-Dichlorethen}
\begin{tabular}{llllc}
\toprule
Molekül & Methode &   Basissatz & Ges. En. \si{\hartree} & Symm. \\
\midrule
cis-1,2-Dichlorethen   & RHF& 6-311G(d,p)& -995.89694703 &$C_ {2v}$  \\
trans-1,2-Dichlorethen & RHF& 6-311G(d,p)& -995.89737964 &$C_ {2h}$ \\
cis-1,2-Dichlorethen   & MP2& 6-311G(d,p)& -996.45258595  &$C_ {2v}$  \\
trans-1,2-Dichlorethen & MP2& 6-311G(d,p)& -996.45188 &$C_ {2h}$    \\
\bottomrule
\label{table:energie}
\end{tabular}
\end{table}
 %\num{4.32e-4}
 %\num{7.05e-4}
% \begin{tabularx}{\textwidth}{lllll}
% &\multicolumn{2}{c}{\textit{cis-}}&\multicolumn{2}{c}{\textit{trans-}}\\
% \toprule
%    &HF/6-311G(d.p) & MP2/6-311G(d.p) & RHF/6-311G(d.p) & MP2/6-311G(d.p)\\
% \midrule
% C-Cl  & 1.723659 &1.715588   & 1.732743 & 1.723728\\
% C-C  & 1.311110 & 1.338136   & 1.308159 & 1.335534\\
% C-H  & 1.071179 & 1.082586   & 1.070440 & 1.082339\\
% \bottomrule
% \end{tabularx}
% \caption{}
% \label{tab:optparameter}
% \end{table}





% \begin{table}[!htpb]

% \caption{theoretische und experimentelle Schwingungsfrequenzen mit irreduziblen Darstel-
% lungen für \textit{cis}-1,2-Dichlorethen}
% \begin{tabularx}{\textwidth}{llll|lll|llll}
% \multicolumn{4}{c}{RHF/6-311G(d,p)}&\multicolumn{3}{c}{MP2/6-311G(d,p)}&\multicolumn{4}{c}{Experimentell} \\
% \midrule
% Irrep &  $\tilde{\nu}$ \si{\per\centi\meter} & \multicolumn{2}{c|}{Int} &
% Irrep &   $\tilde{\nu}$ \si{\per\centi\meter} & Int   &
% Irrep &  $\tilde{\nu}$\si{\per\centi\meter} & \multicolumn{2}{c}{Int}\\
% & & IR & Raman& & & IR && & IR & Raman\\
% \midrule
% $A _1$ & 183  & 0.41 & 2.160  &$A _1$  & 173 & 0.21     & $A_1$&3077 & &\\
% $A _2$ & 458  & 0 & 4.86      &$A _2$  & 411 & 0        & $A_1$&1587 & &\\
% $B _2$ & 614  & 7.93 & 6.74   &$B _2$  & 584 & 3.03     & $A_1$&1179 & & \\
% $A _1$ & 759  & 27.36 & 12.33 &$A _1$  & 714 & 69.15    & $A_1$& 711& &\\
% $B _1$ & 803  & 73.62 & 1.13  &$B _1$  & 747 & 19.58    & $A_1$& 173& &\\
% $B _2$ & 920  & 106.94 & 0.01 &$B _2$  & 888 & 0        & $A_1$& 876& &\\
% $A _2$ & 1051 & 0 & 4.54      &$A _2$  & 897 & 84.06    & $A_2$& 406& &\\
% $A _1$ & 1324 & 0.14 & 20.88  &$A _1$  & 1238 & 0.02    & $A_2$& 3072& &\\
% $B _2$ & 1439 & 36.02 & 1.05  &$B _2$  & 1341 & 22.88   & $B _1$ &1303 & &\\
% $A _1$ & 1818 & 34.83 & 59.81 &$A _1$  & 1648 &31.96    & $B _1$ &857 & &\\
% $B _2$ & 3373 & 15.14 & 53.92 &$B _2$  & 3249 & 14.28   & $B _1$ & 571& &\\
% $A _1$ & 3397 & 2.93 & 141.85 &$A _1$  & 3270 & 2.77    & $B _2$ & 697& &\\
% \bottomrule
% \end{tabularx}
% \label{tab:cisvergleich}

% \end{table}

\begin{table}[!htpb]

  \caption{theoretische und experimentelle Schwingungsfrequenzen mit irreduziblen Darstellungen für \textit{cis}-1,2-Dichlorethen $|^{\colorbox{yellow}{Korr.}}$ }
\begin{tabular}{lllllll}
\multicolumn{2}{c}{RHF/6-311G(d,p)}&\multicolumn{2}{c}{MP2/6-311G(d,p)}&\multicolumn{2}{c}{Experimentell}~\supercite{ciselec} \\
\midrule
Irrep &  $\tilde{\nu}$ \si{\per\centi\meter} & Irrep &   $\tilde{\nu}$ \si{\per\centi\meter} &  &  $\tilde{\nu}$\si{\per\centi\meter} \\
\midrule
$A _1$ & 183  &$A _1$  & 173    & \cellcolor{azulen} $A_1$&  173 \\
$A _2$ & 458  &$A _2$  & 411    & \cellcolor{azulen} $A_1$&  406 \\
$B _2$ & 614  &$B _2$  & 584    & \cellcolor{azulen} $A_1$& 571  \\
$A _1$ & 759  &\cellcolor{azulen} $B _1$  & 714    & \cellcolor{azulen} $A_1$& 697  \\
$B _1$ & 803  &\cellcolor{azulen} $B _1$  & 747    & \cellcolor{azulen} $A_1$& 711  \\
$B _2$ & 920  &\cellcolor{azulen} $B _2$  & 888    & \cellcolor{azulen} $A_2$& 857 \\
$A _2$ & 1051 &\cellcolor{azulen} $A _2$  & 897    & \cellcolor{azulen} $A_2$& 876 \\
$A _1$ & 1324 &$A _1$  & 1238   & \cellcolor{azulen} $B _1$& 1179\\
$B _2$ & 1439 &$B _2$  & 1341   & \cellcolor{azulen} $B _1$ &1303 \\
$A _1$ & 1818 &$A _1$  & 1648   & \cellcolor{azulen} $B _1$ &1587 \\
$B _2$ & 3373 &$B _2$  & 3249   & \cellcolor{azulen} $B _1$ & 3072\\
$A _1$ & 3397 &$A _1$  & 3270   & \cellcolor{azulen} $B _2$ & 3077\\
\bottomrule
\end{tabular}
\label{tab:cisvergleich}

\end{table}





\begin{table}[!htpb]

\caption{Schwingungsintensitäten mit irreduziblen Darstellungen und Schwingungsmodi für
\textit{cis}-1,2-Dichlorethen}
\begin{tabular}{llll}
\midrule
Irrep & IR-Int(RHF)/rel & Raman-Int(RHF)/rel & Schwingungsmodus \supercite{ciselec} \\
\midrule
$A _1$ & 0.41 & 2.160   & CCCl Deformationss.\\
$A _2$ & 0 & 4.86       & Torsion\\
$B _2$ & 7.93 & 6.74    &  CCCl Deformationss.\\
$A _1$ & 27.36 & 12.33  & C-H Deformationss.\\
$B _1$ & 73.62 S & 1.13   & C-Cl Strecks.\\
$B _2$ & 106.94 & 0.01  & C-Cl Strecks.\\
$A _2$ & 0 & 4.54       & C-H Deformationss.\\
$A _1$ & 0.14 & 20.88   & C-H Deformationss.\\
$B _2$ & 36.02 & 1.05   & C-H Deformationss.\\
$A _1$ & 34.83 & 59.81  & C-C Strecks.\\
$B _2$ & 15.14 & 53.92  & C-H Strecks.\\
$A _1$ & 2.93 & 141.85  & C-H Strecks.\\
\bottomrule
\end{tabular}
\label{tab:cisschwings}

\end{table}









% \begin{table}[!htpb]

% \caption{theoretische und experimentelle Schwingungsfrequenzen mit irreduziblen Darstellungen für \textit{trans}-1,2-Dichlorethen}
% \begin{tabular}{llllll}
% \multicolumn{2}{c}{RHF/6-311G(d,p)}&\multicolumn{2}{c}{MP2/6-311G(d,p)}&\multicolumn{2}{c}{Experimentell} \\
% \midrule
% Irrep &  $\tilde{\nu}$ \si{\per\centi\meter} & Irrep &   $\tilde{\nu}$ \si{\per\centi\meter} & Irrep &  $\tilde{\nu}$\si{\per\centi\meter} \\
% \midrule
% $A _u$ & 233  & 0.73   & 0     &   $A _u$ & 215 & 0.15     & & & &\\
% $B _u$ & 257  & 4.58   & 0     &   $B _u$ & 243 & 3.09     & & & &\\
% $A _g$ & 376  & 0.0    & 9.82  &   $A _g$ & 360 & 0        & & & &\\
% $B _u$ & 875  & 156.36 & 0     &   $B _g$ & 753 & 0        & & & &\\
% $B _g$ & 912  & 0      & 6.61  &   $B _u$ & 869 & 115.7    & & & &\\
% $A _g$ & 920  & 0      & 9.56  &   $A _g$ & 892 & 0        & & & &\\
% $A _u$ & 1049 & 82.03  & 0     &   $A _u$ & 939 & 73.37    & & & &\\
% $B _u$ & 1333 & 27.08  & 0     &   $B _u$ & 1258 & 20.34   & & & &\\
% $A _g$ & 1417 & 0      & 24.55 &   $A _g$ & 1325 & 0       & & & &\\
% $A _g$ & 1815 & 0      & 49.17 &   $A _u$ & 1645 & 0       & & & &\\
% $B _u$ & 3393 & 17.72  & 0     &   $B _u$ & 3263 & 15.29   & & & &\\
% $A _g$ & 3400 & 0      & 115.68&   $A _g$ & 3267 & 0       & & & &\\
% \bottomrule
% \end{tabularx}
% \label{tab:transvergleich}

% \end{table}


\begin{table}[!htpb]

\caption{theoretische und experimentelle Schwingungsfrequenzen mit irreduziblen Darstellungen für \textit{trans}-1,2-Dichlorethen}
\begin{tabular}{llllll}
\multicolumn{2}{c}{RHF/6-311G(d,p)}&\multicolumn{2}{c}{MP2/6-311G(d,p)}&\multicolumn{2}{c}{Experimentell \supercite{transelec}} \\
\midrule
Irrep &  $\tilde{\nu}$ \si{\per\centi\meter} & Irrep &   $\tilde{\nu}$ \si{\per\centi\meter} & Irrep &  $\tilde{\nu}$\si{\per\centi\meter} \\
\midrule
$A _u$ & 233  & $A _u$ & 215  & $A_u$ & 227\\
$B _u$ & 257  & $B _u$ & 243  & $B_u$ & 250\\
$A _g$ & 376  & $A _g$ & 360  & $A_g$ & 350\\
$B _u$ & 875  & $B _g$ & 753  & $B_g$ & 763\\
$B _g$ & 912  & $B _u$ & 869  & $B_u$ & 828\\
$A _g$ & 920  & $A _g$ & 892  & $A_g$ & 846\\
$A _u$ & 1049 & $A _u$ & 939  & $A_u$ & 900\\
$B _u$ & 1333 & $B _u$ & 1258 & $B_u$ & 1200\\
$A _g$ & 1417 & $A _g$ & 1325 & $A_g$ & 1274\\
$A _g$ & 1815 & $A _u$ & 1645 & $A_g$ & 1578\\
$B _u$ & 3393 & $B _u$ & 3263 & $A_g$ & 3073\\
$A _g$ & 3400 & $A _g$ & 3267 & $B_u$ & 3090\\
\bottomrule
\end{tabular}
\label{tab:transvergleich}

\end{table}


\begin{table}[!htpb]
\caption{Schwingungsintensitäten mit irreduziblen Darstellungen und Schwingungsmodi für \textit{trans}-1,2-Dichlorethen }
\begin{tabular}{llll}
\midrule
Irrep & IR-Int(RHF)/rel & Raman-Int(RHF)/rel & Schwingungsmodus \supercite{transelec}  \\
\midrule
$A _u$ & 0.73 w& 0 & Torsion\\
$B _u$ & 4.58 w& 0 & CCCl Deformationss.\\
$A _g$ & 0.0 & 9.82 w&CCCl Deformationss.\\
$B _u$ & 156.36 s& 0 & C-H Deformationss.  \\
$B _g$ & 0 & 6.61 w & C-Cl Strecks.\\
$A _g$ & 0 & 9.56 w & C-Cl Strecks.\\
$A _u$  & 82.03 m & 0 & C-H Deformation \\
$B _u$ & 27.08 w& 0 & C-H Deformation\\
$A _g$ & 0 & 24.55 w& C-H Deformation\\
$A _g$ & 0 & 49.17 m& C-C Strecks.\\
$B _u$ & 17.72 w& 0 & C-H Streck. \\
$A _g$ & 0 & 115.68 s& C-H Strecks.\\
\bottomrule
\end{tabular}
\end{table}

\newpage


\textbf{Auswertung 2}

\begin{figure}[!hptb]
    \centering
    \begin{subfigure}[b]{0.4\textwidth}
         \fbox{   \includegraphics[width=\textwidth,height=5cm]{data/cis_ir_spektrum.png}}
        \subcaption{auf RHF/6-311G(d,p) Niveau}
    \end{subfigure}
    ~ %add desired spacing between images, e. g. ~, \quad, \qquad, \hfill etc.
      %(or a blank line to force the subfigure onto a new line)
    \begin{subfigure}[b]{0.4\textwidth}
         \fbox{   \includegraphics[width=\textwidth,height=5cm]{data/cis_ir_mp2.png}}
        \subcaption{auf MP2/6-311G(d,p) Niveau}
    \end{subfigure}
    \caption{IR-Spektren von \textit{cis}-1,2-Dichlorethen}
\end{figure}

\begin{figure}[!hptb]

\begin{tikzpicture}[font=\sffamily]
\begin{axis}[xlabel=Wellenzahl, ylabel=Intensität,  enlargelimits=true,axis x line=middle,
    axis y line=middle,
   y label style={at={(axis description cs:-0.15,.5)},anchor=south,rotate=90},
    x tick label style={rotate=90,anchor=east},
    x label style={at={(axis description cs:0.5,-.3)},anchor=south},
     xmin=0
    ]
\addplot[blue] table[x=frequency ,y=intensity] {data/cisex.txt};
\end{axis}
\end{tikzpicture}
    \caption{IR-Spektrum von \textit{cis}-1,2-Dichlorethen \supercite{cisir}}
\label{figure:vergleichcis}
\end{figure}

\begin{figure}[!hptb]
    \centering
    \begin{subfigure}[b]{0.4\textwidth}
         \fbox{   \includegraphics[width=\textwidth,height=5cm]{data/trans_ir_spektrum.png}}
        \subcaption{auf RHF/6-311G(d,p) Niveau}
    \end{subfigure}
    ~ %add desired spacing between images, e. g. ~, \quad, \qquad, \hfill etc.
      %(or a blank line to force the subfigure onto a new line)
    \begin{subfigure}[b]{0.4\textwidth}
         \fbox{   \includegraphics[width=\textwidth,height=5cm]{data/trans_ir_mp2.png}}
                \subcaption{auf MP2/6-311G(d,p) Niveau}
    \end{subfigure}
        \caption{IR-Spektren von \textit{trans}-1,2-Dichlorethen \supercite{transir}}

\end{figure}

\begin{figure}[!hptb]

\begin{tikzpicture}[font=\sffamily]
\begin{axis}[xlabel=Wellenzahl, ylabel=Intensität,  enlargelimits=true,axis x line=middle,
    axis y line=middle,
   y label style={at={(axis description cs:-0.15,.5)},anchor=south,rotate=90},
    x tick label style={rotate=90,anchor=east},
    x label style={at={(axis description cs:0.5,-.3)},anchor=south},
     xmin=0
    ]
\addplot[blue] table[x=frequency ,y=intensity] {data/trans.txt};
\end{axis}
\end{tikzpicture}
    \caption{IR-Spektrum von \textit{trans}-1,2-Dichlorethen}
\label{figure:vergleichtrans}
\end{figure}


\begin{figure}[!hptb]
    \centering
    \begin{subfigure}[b]{0.4\textwidth}
          \fbox{  \includegraphics[width=\textwidth,height=5cm]{data/cis_raman_spektrum.png}}
        \subcaption{ \textit{cis}-1,2-Dichlorethen }
    \end{subfigure}
    ~ %add desired spacing between images, e. g. ~, \quad, \qquad, \hfill etc.
      %(or a blank line to force the subfigure onto a new line)
    \begin{subfigure}[b]{0.4\textwidth}
         \fbox{   \includegraphics[width=\textwidth,height=5cm]{data/trans_raman_spektrum.png}}
        \subcaption{\textit{trans}-1,2-Dichlorethen}
    \end{subfigure}
            \caption{Raman-Spektren auf RHF/6-311G(d,p) Niveau}
\end{figure}

\newpage 

\textbf{Auswertung 3}\\
Die MP2-Methode ist für die Berechnung von Schwingungsfrequenzen besser geeignet als die RHF-Methode,
da die Spektren der MP2-Methode näher an den experimentellen Spektren sind.
Dies liegt daran, dass die RHF-Methode die Elektronenkorrelation (die Wechselwirkung der Elektronen) vernachlässigt,
die MP2 Methode diese jedoch berücksichtigt.\\
\textbf{Auswertung 4}\\
Mit Elektronenkorrelation ist gemeint, dass die Coulombabstoßung $\dfrac{e^2}{\epsilon r_{12}}$ zwischen den beiden Elektronen einer Bindung dafür sorgt, dass sich
die Elektronen nicht zu nahe kommen. coulsonseite=135 Ist das eine Elektron momentan in der Umgebung des einen Kerns,
so ist die Wahrscheinlichkeit groß, dass sich das andere Elektron in der Umgebung des anderen Kerns aufhält. Die Elektronen bewegen sich derart \glqq
korreliert\grqq, dass sie sich nicht nahe kommen. Die Elektronenkorrelation wirkt sich auf die Geometrie bzw. den
Energiezustand einer Geometrie aus.\\
\textbf{Auswertung 5}\\
Beim \textit{trans}-1,2-Dichlorethen fällt auf, dass es entweder nur ramanaktiv oder infrarotaktiv ist.
Das \textit{cis}-1,2-Dichlorethen hingegen
 ist bei beiden aktiv. Dies liegt daran, dass das \textit{trans}-1,2-Dichlorethen ein
  Inversionszentrum hat und das \textit{cis}-1,2-Dichlorethen nicht.
Das Inversionszentrum verbietet bei einer symmetrischen Schwingung die
Infrarotaktivität und bei einer asymmetrischen Schwingung die Ramanaktivität.\\
\textbf{Auswertung 6}\\

Anhand der Werte für die Hartree-Energien aus ~\ref{table:energie} lässt sich sagen, dass
bei der Verwendung der RHF Methode für das \textit{trans}-1,2-Dichlorethen eine
kleinere Hartree-Energie ermittelt wurde als für das \textit{cis}-1,2-Dichlorethen.
Dementsprechend ist unter Verwendung der RHF Methode das cis-Isomer der
Verbindung die stabilere. Betrachtet man jedoch die Werte, die anhand der MP2
Methode berechnet wurden, so lässt sich feststellen, dass hierbei das \textit{cis}-Isomer
eine kleinere Hartree-Energie aufweist als das \textit{trans}-Isomer. Der Grund dafür ist, die MP2-Methode die destabilisierende Elektronenabstoßung
 zwischen Elektronen der vicinalen Cl-Atomen von \textit{cis}-1,2-Dichlorethen in Betracht zieht. 
\colorbox{yellow}{Korr.}\\
 Da es sich bei den oben genannten Verbindungen um cis-/ trans-Isomere handelt und die Moleküle somit die gleiche Bildungsenergie aufweisen, können ihre
energetischen Stabilitäten miteinander verglichen werden. Um die
Stabilitäten von Molekülen miteinander vergleichen zu können, müssen sie aus
den selben Elementen bestehen, da nur solche Moleküle die gleiche
Bildungsenergie und den gleichen Energienullpunkt besitzen. Sobald Moleküle
aus anderen Atomen zusammengesetzt sind, ist der Energienullpunkt nicht mehr
gleich und die Bindungsenergie, die Auskunft über die Stabilität der verschiedenen Moleküle gibt, 
kann nicht mehr verglichen werden. Daher eine Aussage anhand eines Vergleiches der energetischen 
Stabilität von Monochlorethen und Ethen nicht getroffen werden.



\newpage

\section{Geometrieoptimierung und Berechnung der chemischen Verschiebung von Toluol}
In dieser Aufgabe, soll die \textsuperscript{1}H-NMR-Verschiebung gegen Tetramethylsilan als Standard zu berechnen. \\
\textbf{Auswertung 1:}

\begin{table}[!htpb]


\caption{ Parameter für die optimierte Geometrie von Toluol}
\begin{tabularx}{\textwidth}{llllc}
\toprule
Molekül  & Methode & Basissatz & Energie \si{\hartree} & Punktgruppe \\
\midrule
 Toluol & MP2 & aug-cc-pVDZ & -270.75294 &$C _S$\\
\midrule
 Bindung & Bindungslänge \si{\angstrom} & &  &\\
 C-C (Ring) $C_3 - C_4$ & 1.409800 &&&\\
 C-C (Ring) $C_2 - C_3$ & 1.405891 &&&\\
 C-C (Ring) $C_1 - C_2$ & 1.406102 &&&\\
 C-C (Kette) & 1.511691 &&&\\
 C-H (Orto)  & 1.094643 &&&\\
 C-H (Meta)  & 1.093370 &&&\\
 C-H (Para)  & 1.093007 &&&\\
 C-H (Alkyl) & 1.100292 &&&\\
\bottomrule
\end{tabularx}
\label{tab:toluol}
\end{table}

\textbf{Auswertung 2:}

\begin{figure}[!htpb]
  \includegraphics[width=\textwidth]{data/toluol_bezzifert.png}%
  \caption{Optimierte Geometrie von Toluol auf MP2/aug-cc-pVDZ Niveau für die Bindungslänge siehe ~\ref{tab:toluol}  }
\end{figure}
\begin{figure}[!htbp]
  \includegraphics[width=\textwidth]{data/mp2fullaugccpvddz.png}%
  \caption{Chemische Verschiebung und Intensitäten auf MP2/aug-cc-pVDZ Niveau}
\end{figure}
\pagebreak

\begin{table}[!htpb]
\caption{Chemische Verschiebung der Wasserstoffatome für die optimierten Geometrie von Toluol \colorbox{yellow}{Korr.}}
\begin{tabular}{lcc}
\toprule
H-Atom  & \parbox[t]{4cm}{$\delta$ Toluol/TMS \\ (MP2/aug-cc-pVDZ\\ ppm}  &   Intensitäten\\
\midrule
H-7  & 7.46 &  1 \\
H-8  & 7.51 &  2  \\
H-9  & 7.52 &  2  \\
H-10 & 7.52 &  2 \\
H-11 & 7.51 &  2 \\
H-13 & 2.66 &  1 \\
H-14 & 2.34 &  2 \\
H-15 & 2.34 &  2 \\
\bottomrule
\end{tabular}
\label{table:nmrtoluol}
\end{table}



\textbf{Auswertung 3:}
Die theoretische Berechnung und die reale Messung unterscheiden sich in mehrere Punkten.
 Bei der theoretischen Betrachtung wird außer acht gelassen dass das Molekül in der Realität
 nicht starr ist sondern Bewegungen durchläuft.
 Ebenso spielt die Temperatur der gemessenen Probe eine Rolle,
 sowie natürliche Messfehler und Verunreinigungen der Probe.
 Zudem kann es in der Probe zu Wechselwirkungen kommen die das Ergebnis ebenfalls abweichen lassen. Außerdem kann es auch durch Lösungsmittel Einflüsse zu chemischen Verschiebungen kommen.

Es fällt an der ~\ref{table:nmrtoluol} auf, dass die magnetische Äquivalenz
3 Wasserstoffe der Alkylgruppe nicht berücksichtigt werden. Es liegt daran,
dass die Berechnung der Verschiebungen auf der Geometrie des Moleküls beruht. Obwohl die Wasserstoffatome der Alkylgruppe sich im Magnetfeld gleich verhalten,
sind sie räumlich nicht äquivalent (in Bezug auf $\sigma _h$-Ebene). Dies spiegelt sich in der Berechnung wieder.


\begin{table}[!htpb]
  \caption{Chemische Verschiebung der Wasserstoffatome für die optimierten Geometrie von Toluol und experimentelle Daten\~supercite{hesse} \colorbox{yellow}{Korr.} }

\begin{tabular}{lcc}
\toprule
H-Atom  & \parbox[t]{4cm}{$\delta$ Toluol/TMS \\ (MP2/aug-cc-pVDZ\\ ppm}  &  \parbox[t]{4cm}{$\delta$ Toluol[Experimentelle Daten] \supercite{zeeh} ppm}\\
\midrule
H-7  & 7.46 & 7.17  \\
H-8  & 7.51 & 7.21   \\
H-9  &  7.52 & 7.21   \\
H-10 & 7.52 & 7.17 \\
H-11 & 7.51 & 7.17  \\
H-13 & 2.66 & 2.32  \\
H-14 & 2.34 & 2.32 \\
H-15 & 2.34 & 2.32  \\
\bottomrule
\end{tabular}
\end{table}
\pagebreak
\textbf{Auswertung 4:}Der Vergleich zwischen berechneten und experimentellen Daten zeigt, das  es bis auf 14-H und 15-H der Methylgruppe deutliche Unterschiede gibt. Die größte Abweichung stellt hier das H-13 der Methylgruppe dar, gefolgt von den H-Signalen des Benzolrings. Die Unterschiede der Werte gehen wie in Auswertung 3 aufgezeigt unter anderem auf die Wechselwirkungen in der gemessenen Probe und anderen bereits in Auswertung 3 geschilderten Umständen zurück.



\section{Berechnung der Potentialkurve am Beispiel von \ce{N_2}}
Ziel dieses Aufgabe, ist die Berechnung einer Potentialkurve des Stickstoffmoleküls, so wie die Visualisierung des MO-Orbitals und Betrachtung der Elektronenkonfiguration mit Hilfe des MOs. Daraufhin soll das berechnete Diagramm mit einem MO-Diagramm des N\textsubscript{2} aus Lehrbüchern verglichen werden. 

\subsection{Berechnung der Potenzialkurve auf Hartree-Fock Niveau}
\textbf{Auswertung 1 und 2}
\begin{table}[!htpb]
\small
\begin{tabularx}{\textwidth}{llllcll}
\toprule
Molekül &
Methode &
Basissatz &
Gitter &
Symmetrie &
GGW- &
GGW- \\
&&&\scriptsize{(Min, Max, Incr)}&&Abstand&Energie $E_h$ \\
\midrule
\ce{N _2} & RHF & 6-311G(d,p) & 0.6, 4.6, 0.2 \si{\angstrom}& $D _{\infty h}$ & 1 \si{\angstrom} & -108.95140449 \\
\bottomrule
\end{tabularx}
\end{table}
\begin{figure}[!htpb]
\centering
  \fbox{\includegraphics[width=14cm]{data/morse.png}}
  \caption{Potenzialkurve für Stickstoff auf RHF/6-311G(d,p)-Niveau}
\end{figure}


\pagebreak
\noindent
\textbf{Auswertung 3}\\
Durch die Betrachtung des Inputs „charge: 0, Spin: Singlet“ kann geschlussfolgert werden das es sich bei der Bindungsspaltung um eine heterolytische handelt.

\newpage

\textbf{Auswertung 4 }\\

\begin{table}[!htpb]
\begin{tabular}{c|ccc}
 \large Homolytisch & &\multicolumn{2}{c}{\large Heterolytisch}\\
 & &\\
 \ce{\Lewis{0.2.4:6.,N}} & & \ce{N+} & \ce{N-}\\
  & &\\
\begin{tikzpicture}
%\draw [->,ultra thick] (-1,-2) --(-1,4) node[above] { Energie};
\drawLevel[elec = updown,pos = {(0,0)},    width = 1]{d1};
\drawLevel[elec = updown,pos = {(0,1.3)},  width = 1]{};
\drawLevel[elec = up,pos = {(0,2.6)},  width = 1]{};
\drawLevel[elec = up,pos = {(1.3,2.6)},  width = 1]{};
\drawLevel[elec = up,pos = {(2.6,2.6)},  width = 1]{};
\node[right] at (right d1) { Quartett} ;
\end{tikzpicture}
& &
\begin{tikzpicture}

%\draw [->,ultra thick] (-1,-2) --(-1,4) node[above] { Energie};
\drawLevel[elec = updown,pos = {(0,0)},    width = 1]{d1};
\drawLevel[elec = updown,pos = {(0,1.3)},  width = 1]{};
\drawLevel[elec = up,pos = {(0,2.6)},  width = 1]{};
\drawLevel[elec = up,pos = {(1.3,2.6)},  width = 1]{};
\drawLevel[pos = {(2.6,2.6)},  width = 1]{};
\node[right] at (right d1) { Triplett} ;
\end{tikzpicture}
&
\begin{tikzpicture}
%\draw [->,ultra thick] (-1,-2) --(-1,4) node[above] { Energie};
\drawLevel[elec = updown,pos = {(0,0)},    width = 1]{d1};
\drawLevel[elec = updown,pos = {(0,1.3)},  width = 1]{};
\drawLevel[elec = updown,pos = {(0,2.6)},  width = 1]{};
\drawLevel[elec = up,pos = {(1.3,2.6)},  width = 1]{};
\drawLevel[elec = up,pos = {(2.6,2.6)},  width = 1]{};
\node[right] at (right d1) { Triplett} ;
\end{tikzpicture}\\
&&&\\
&&&\\
\begin{tikzpicture}
%\draw [->,ultra thick] (-1,-2) --(-1,4) node[above] { Energie};
\drawLevel[elec = updown,pos = {(0,0)},    width = 1]{d1};
\drawLevel[elec = updown,pos = {(0,1.3)},  width = 1]{};
\drawLevel[elec = updown,pos = {(0,2.6)},  width = 1]{};
\drawLevel[elec = up,pos = {(1.3,2.6)},  width = 1]{};
\drawLevel[pos = {(2.6,2.6)},  width = 1]{};
\node[right] at (right d1) { Dublett} ;
\end{tikzpicture}
&&
\begin{tikzpicture}
%\draw [->,ultra thick] (-1,-2) --(-1,4) node[above] { Energie};
\drawLevel[elec = updown,pos = {(0,0)},    width = 1]{d1};
\drawLevel[elec = updown,pos = {(0,1.3)},  width = 1]{};
\drawLevel[elec = updown,pos = {(0,2.6)},  width = 1]{};
\drawLevel[pos = {(1.3,2.6)},  width = 1]{};
\drawLevel[pos = {(2.6,2.6)},  width = 1]{};
\node[right] at (right d1) { Singulett} ;
\end{tikzpicture}
&
\begin{tikzpicture}
%\draw [->,ultra thick] (-1,-2) --(-1,4) node[above] { Energie};
\drawLevel[elec = updown,pos = {(0,0)},    width = 1]{d1};
\drawLevel[elec = updown,pos = {(0,1.3)},  width = 1]{};
\drawLevel[elec = updown,pos = {(0,2.6)},  width = 1]{};
\drawLevel[elec = updown,pos = {(1.3,2.6)},  width = 1]{};
\drawLevel[pos = {(2.6,2.6)},  width = 1]{};
\node[right] at (right d1) { Singulett} ;
\end{tikzpicture}\\

\end{tabular}

\caption{Die Elektronenbesetzungschemen der Stickstoff-Fragmenten von möglichen Dissoziationsarten \supercite{wiberg}}
\label{table:besetzung}
\end{table}
\colorbox{yellow}{Korr.}
Die drei möglichen Arten der Bindungsspaltung sind in ~\ref{table:besetzung} dargestellt.
Bei einer heterolytischen Dissoziation erhält man ein
Stickstoffkation und ein Stickstoffanion, die Singulett- und Triplett-Zustand aufweisen.
Möglich ist eine homolytische Bindungsspaltung,
bei der man zwei Stickstoffradikale erhält. Die dabei möglichen Multiplizitäten sind
in der ~\ref{table:besetzung} zu sehen; es ist der Quartett- und der Dublett-Zustand
möglich. \newline 
\textbf{Auswertung 5:}
Nach der Hund'schen Regel erfolgt die Besetzung der entarteten Orbitale so, 
dass die größtmögliche Zahl an ungepaarten Elektronen erreicht wird (max. Spinmultiplizität).
Diese Zustände sind im Gegensatz zu Zuständen bei denen die entarteten Orbitale nicht so besetzt sind, 
das eine maximale Anzahl an ungepaarten Elektronen erreicht wird stabiler.
Somit ergibt sich folgende energetische Reihenfolge der Dissoziationskanäle in aufsteigender Reihenfolge.

\begin{center}
 Quartett < Dublett < Triplett < Singulett.\\
\end{center}
 
Hieraus lässt sich schließen, das eine homolytische Dissoziation energetisch günstiger liegt als eine heterolytische Dissoziation.

\textbf{Auswertung 6}
Unter Berücksichtigung des Morse-Potentials kann aus dem Diagramm die Dissoziationsenergie abgelesen werden, 
diese beträgt -1.1 \si{\hartree} und umgerechnet 2888 \si{\kilo\joule\mol}. 
Dieser Wert ist jedoch im Gegensatz zur Literatur \supercite{riedel} 
viel zu hoch dies kann daran liegen das in den vorhergegangenen Aufgaben von 
einem Singulett Zustand ausgegangen wurde welcher sehr Energiereich ist. 


\subsection{Geometrieoptimierung des Stickstoffmoleküls \colorbox{yellow}{Korr}}
\textbf{Auswertung 1 und 2}
\begin{table}[!htpb]
\centering
\caption{}
\begin{tabular}{lllll}
\toprule
Molekül &
Methode &
Basissatz &
Bindungslänge \si{\angstrom} &
Gesamtenergie \si{\hartree}\\
\midrule
\ce{N _2} & RHF & 6-311G(d,p) & 1.07027 \si{\angstrom} & -108.9514 \\
\bottomrule
\end{tabular}
\end{table}

Im Vergleich zur Gesamtenergie aus dem ersten Aufgabenteil ist die Energie der optimierten Geometrie geringfügig niedriger. Somit lässt sich schließen das die optimierte Geometrie energetisch günstiger und somit stabiler ist als die für Aufgabenteil 1 gewählte.\\

\textbf{Auswertung 3}\\
\begin{table}[!htpb]
\centering
\caption{Der experimentelle Gasphasenabstand im \ce{N_2} Molekül }
\begin{tabular}{ccc}
\toprule
Abstand Teil 1 & Abstand Teil 2 (opt. Geometrie)  & experimenteller Abstand \\
1.0 \si{\angstrom} & 1.07027 \si{\angstrom} & 1.0976 \si{\angstrom} \\
\midrule
\bottomrule
\end{tabular}
\end{table}
\noindent
Bei dem Vergleich der berechneten Ergebnisse mit den experimentellen Werten für Bindungslängen des Stickstoff Moleküls, so ist zu erkennen das die experimentell bestimmte Bindungslänge des N\textsubscript{2}-Moleküls mit 1.0977 \si{\angstrom} ein wenig größer ist als die berechneten Werte für Bindungslängen.

Vergleicht man die von uns berechneten Ergebnisse mit dem experimentell
bestimmten Wert für die Bindungslänge des N2-Moleküls, so stellt man fest, dass
die experimentell bestimmte Bindungslänge mit x = 1,0977 Å etwas größer ist als
die von uns errechneten Werte für die Bindungslänge. 
Durch diesen unterschied lässt sich schließen das durch die gewählte Rechenmethode nicht die tatsächliche Bindungslänge des Moleküls bestimmt werden kann. Möglicherweiße kommt der Fehler daher, das die es der RHF Methode nicht möglich ist die exakte Gesamtenergie und die genauen Parameter des N\textsubscript{2}-Moleküls zu berechnen, da die Hartree-Fock Methode die Elektronenkorrelation nicht berücksichtig. 

\newpage

\textbf{Auswertung 4}\\
\begin{figure}[!htpb]
   \centering
\includegraphics[width=0.5\textwidth,keepaspectratio]{data/mohf.png}
\caption{MO-Diagramm für ein Stickstoff-Molekül auf RHF/6-311G(d,p)}
\end{figure}

\begin{table}[!htpb]
\centering
\caption{Die experimentellen und berechneten Orbital Energien des Stickstoffs}
\begin{tabular}{lll}
\toprule
Orbital  && Orbitalenergien(RHF/6-311G(d,p)) (eV)\\
\midrule
$3\sigma _u$ & &   0.32629\\
$1\pi _g$    & &    -0.09957 \\
$1\pi _g$    & &   -0.09957 \\
$1\pi _u$    &  &  -0.435529 \\
$1\pi _u$    & &  -0.48286 \\
$3\sigma _g$ &  & -0.48286 \\
$2\sigma _u$ && -0.52102 \\
$2\sigma _g$ & & -1.12397 \\
$1\sigma _u$ &  &-14.03231 \\
$1\sigma _g$ & &-14.03375 \\
\bottomrule
\end{tabular}
\end{table}


Beim Vergleich des berechneten MO-Diagrams mit dem MO aus dem Atkins \supercite{atkins}, fällt auf, dass es mehrere Abweichungen gibt. Ein Grund für die Abweichung des berechneten MO's und dem des Atkins könnte sein das dieses mit Hilfe des Koopmanns-Theorem zusammen gestellt wurde. Im Koopmanns-Theorem wird davon ausgegangen, dass die Ioniesierungsenergie des Moleküls der negativen Orbitalenergie entspricht. Da dies jedoch auch nur eine Näherung ist können mit den Berechnungen am N\textsubscript{2}-Moleküls präzisere Ergebnisse hervorgebracht werden.  

\textbf{Auswertung 5 und 6}
\begin{figure}[!hptb]
    \centering
    \begin{subfigure}[b]{0.4\textwidth}
        \fbox{\includegraphics[width=0.4\textwidth]{data/orbitale/1.png}}
      \subcaption*{$1 \sigma _g$  }
    \end{subfigure}
    \begin{subfigure}[b]{0.4\textwidth}
        \fbox{\includegraphics[width=0.4\textwidth]{data/orbitale/2.png}}
         \subcaption*{$1 \sigma _u$ }
    \end{subfigure}

\end{figure}



\begin{figure}[!hptb]
    \centering
    \begin{subfigure}[b]{0.4\textwidth}
        \fbox{\includegraphics[width=0.4\textwidth]{data/orbitale/3.png}}
        \subcaption*{$2 \sigma _g$ }
    \end{subfigure}
    ~ %add desired spacing between images, e. g. ~, \quad, \qquad, \hfill etc.
      %(or a blank line to force the subfigure onto a new line)
    \begin{subfigure}[b]{0.4\textwidth}
       \fbox{ \includegraphics[width=0.4\textwidth]{data/orbitale/4.png}}
     \subcaption*{$2 \sigma _u$}
    \end{subfigure}

\end{figure}

\begin{figure}[!hptb]
    \centering
    \begin{subfigure}[b]{0.4\textwidth}
        \fbox{\includegraphics[width=0.4\textwidth]{data/orbitale/5.png}}
        \subcaption*{$1 \pi _u^x$  }
    \end{subfigure}
    ~ %add desired spacing between images, e. g. ~, \quad, \qquad, \hfill etc.
      %(or a blank line to force the subfigure onto a new line)
    \begin{subfigure}[b]{0.4\textwidth}
        \fbox{\includegraphics[width=0.4\textwidth]{data/orbitale/6.png}}
        \subcaption*{$1 \pi _u^y$}
    \end{subfigure}
\label{figure:orbitalen2}
\end{figure}
\begin{figure}[!hptb]
    \centering
    \begin{subfigure}[b]{0.4\textwidth}
        \fbox{\includegraphics[width=0.4\textwidth]{data/orbitale/7.png}}
                \subcaption*{$3 \sigma _g$}

    \end{subfigure}
    ~ %add desired spacing between images, e. g. ~, \quad, \qquad, \hfill etc.
      %(or a blank line to force the subfigure onto a new line)
    \begin{subfigure}[b]{0.4\textwidth}
        \fbox{\includegraphics[width=0.4\textwidth]{data/orbitale/8.png}}
               \subcaption*{$1 \pi _g$}

    \end{subfigure}
\end{figure}

\begin{figure}[!hptb]
    \centering
    \begin{subfigure}[b]{0.4\textwidth}
        \fbox{\includegraphics[width=0.4\textwidth]{data/orbitale/9.png}}
               \subcaption*{$1 \pi _g$}
    \end{subfigure}
    ~ %add desired spacing between images, e. g. ~, \quad, \qquad, \hfill etc.
      %(or a blank line to force the subfigure onto a new line)
    \begin{subfigure}[b]{0.4\textwidth}
        \fbox{\includegraphics[width=0.4\textwidth]{data/orbitale/10.png}}
          \subcaption*{ $3 \sigma _u$}
    \end{subfigure}

\end{figure}

\newpage

\section{Aufgabe 5 S\textsubscript{N}2-Substitution des Br- durch Cl- an CH\textsubscript{3}Br}

Ziel dieser Aufgabe ist die Bestimmung des Übergangszustandes bei der
Substitution des Bromidions durch ein Chloridion an CH\textsubscript{3}Br,
sowie die Auswahl der Startgeometrie (die Geometrie am höchsten höchsten Punkt
des Energieprofiles, bei der schrittweisen Annäherung des Chloridions an an
das Methylbromid) und die Berechnung der Aktivierungsbarriere und der Ausbeute
dieser Reaktion.

\subsection{Bestimmung des Reaktionspfades und Auswahl einer geeigneten Startgeometrie}



\textbf{Auswertung 1}:

\begin{figure}[!htbp]
\centering
  \includegraphics[width=0.8\textwidth]{data/a5_teil1_scan.png}%
  \caption{Optimierte Geometrie des Übergangszustandes}
\end{figure}
\noindent

 Die Gesamtenergie für die Geometrie beträgt -3071.56972781 \si{\hartree}.
 Der Basissatz ist aug-cc-pVDZ. Die Punktgruppe beträgt C\textsubscript{1}.

\textbf{Auswertung 2 }\\

Nach der Optimierung kommt es zur Änderung der Punktgruppe von
C\textsubscript{1} zu C\textsubscript{3V}. Die Gesamtenergie beträgt nun
-3071.56971543 \si{\hartree}. Die Bildungsenthalpie beträgt -3071.560004
\si{\hartree} Die optimierte Bindungslänge der C-Cl bindung beträgt 2.45
\si{\angstrom}
Der Zustand besitzt die imaginäre Schwingungsfrequenz $i^*$-390.12 \colorbox{yellow}{Korr.}
cm\textsuperscript{-1} Die Bewegung bei dieser Frequenz entspricht der
Walden'sche Umkehr.

\begin{figure}[!htbp]
\centering
  \includegraphics[width=14cm]{data/A5_opt_darstellung.png}%
  \caption{Optimierte Geometrie des Übergangszustandes}
\end{figure}
\noindent

\newpage

\subsection{Optimierung und Frequenzrechnung des Übergangszustandes}

\textbf{Auswertung 1}\\
\begin{table}[!htpb]
\centering
\begin{tabular}{ccc}
\toprule
Molekül & E/\textit{Hartree} & Bildungsenthalpie $\Delta G$  \textit{Hartree}\\
\ce{CH_3Br}  & -2612.00556924 & -2611.99026 \\
\ce{CH_3Cl}  & -499.12227340 & -499.105102\\
\ce{Br^-}  & -2572.46248971 & -2572.478665 \\
\ce{Cl^-}  & -459.56364460 & -459.578667 \\
\midrule
\bottomrule
\end{tabular}
\caption{Molekülbindungen und Energien}
\end{table}
Mit Hilfe der ermittelten Werte kann nun die freie Enthalpie wie folgt berechnet werden:
\begin{equation}
\Delta _R G = \sum\limits_{Produkte} \Delta _B G - \sum\limits_{Edukte} \Delta _B G
\end{equation}
Nach einsetzen der Werte erhält man $ \Delta _R G = -0.014834$ \si{\hartree}. Nach Umrechnung des Ergebnisses in  \si{\kilo\joule\per\mol},erhält man $ \Delta _R G = -38.95$ \si{\kilo\joule\per\mol}.
Die freie Aktivierungsenthalpie wird definiert als Energiedifferenz der Edukte und der reaktiven Zwischenstufe. Und wird wie folgt berechnet:
\begin{equation}
\Delta _R G^{\neq} = \sum\limits_{Edukte} - \Delta _B G _U
\end{equation}
$\Delta _B G _U$ stellt hier die Energie der reaktiven Zwischenstufe dar und  beträgt $\Delta _B G _U = -3071.560004$ \si{\hartree} (Wert aus Aufgabe 5 Teil 1). $\Delta _R G^{\neq}$ steht für die freie Aktivierungsenthalpie. Nach der Berechnung und Umrechnung in SI-Einheiten ergibt sich für die freie Aktivierungsenthalpie: $\Delta _R G^{\neq} = 24.178$ \si{\kilo\joule\per\mol}.

\noindent

\newpage

 \textbf{Auswertung 2:} Bei Betrachtung des Diagrammes ~\ref{figure:pfad} im
Skript auf S.51 mit der Auftragung der Reaktionsenthalpie gegen die Ausbeute
für Reaktionen bei 25 °C bzw. 100 °C, lässt sich für den Berechneten Wert
eine Ausbeute von annähernd 100\% voraus sagen. Aus dem niedrigen Wert für
die freie Aktivierungsenthalpie lässt sich schließen, dass die Reaktion sehr
schnell ablaufen wird, da der Graph der Halbwertszeiten für die Reaktion 2.
Ordnung mit  Aktivierungsenthalpien unter $\Delta _R G^{\neq} = 90$
\si{\kilo\joule\per\mol} stark gegen Null geht. Zur Betrachtung eines Fehlers
von $\pm 10$ \si{\kilo\joule\per\mol} lässt sich sagen das sich dieser nur
geringfügig auf eine Vorhersage auswirkt, da es auch bei einem so hohen
Fehler nur zu unsignifikanten Änderung der Reaktionszeit und der Ausbeute
kommen würde.
\subsection{Bestimmung des Reaktionspfades}
\textbf{Auswertung 1}
 \begin{figure}[!htpb]
 \centering
 \includegraphics[width=\textwidth,height=\textheight,keepaspectratio]{data/potentialkurve.png}%
\captionsetup{justification=raggedright}
 \caption{Reaktionspfad: Auftragung der Energie der optimierten Geometrien gegen den C-Cl-Bindungsabstand}
 \label{figure:pfad}
\end{figure}
\subsection{Bestimmung des Reaktionspfades}


Die Bewegung, welche bei der S\textsubscript{N}2 Reaktion abläuft, ist die
eines Rückseitenangriffs des Chloridions an das Methylbromid. Bei dieser
Reaktion kommt es zunächst zu einem Rückseitenangriff des Chloridions,
daraufhin zur Bildung eines Intermediates und letztlich zum Bindungsbruch zum
Bromidion. Diese Bewegung ist aperiodisch.

Die Vorstellung des \glqq umklappenden Regenschirms \grqq ist unpassend, da
die H-Atome bei der Übergangsbewegung starr auf ihrem Platz bleiben, somit
nicht umklappen, sondern das C-Atom die Fläche der H-Atome durchwandert
(Inversion am C-Zentrum).

\printbibliography
 \end{onehalfspace}
 \end{document}


