\documentclass{article}
\usepackage{amsmath}


%encoding
%--------------------------------------

%----------------------------
\usepackage{microtype}
\usepackage{fourier}
\usepackage[utf8]{inputenc}
\usepackage[T1]{fontenc}
%units
%--------------------------------------
\usepackage{siunitx}
\usepackage{verbatim}
%drawing
%------------------------------------
\usepackage{tikz} % To generate the plot from csv
\usepackage{pgfplots}
\usepackage{pgfplotstable}
\usepackage{caption}
\usepackage{color, colortbl}
\usepackage{tabularx}

\usepackage{chemfig}

\begin{document}
\newcommand\setpolymerdelim[2]{\def\delimleft{#1}\def\delimright{#2}}
\def\makebraces[#1,#2]#3#4#5{%
\edef\delimhalfdim{\the\dimexpr(#1+#2)/2}%
\edef\delimvshift{\the\dimexpr(#1-#2)/2}%
\chemmove{%
\node[at=(#4),yshift=(\delimvshift)]
{$\left\delimleft\vrule height\delimhalfdim depth\delimhalfdim
width0pt\right.$};%
\node[at=(#5),yshift=(\delimvshift)]
{$\left.\vrule height\delimhalfdim depth\delimhalfdim
width0pt\right\delimright_{\rlap{$\scriptstyle#3$}}$};}}
\setpolymerdelim()
\setatomsep{2em}

\renewcommand * \printatom[1]{\ensuremath{\mathsf{#1}}}
\schemestart
\chemfig{ *6(=- * 6(-\chembelow{N}{H}-NH_2)=-=-)}
\+
\chemfig{(=[:-150]O)(-[:-30]R_2)-[2]-[:150]R_1}
\arrow(.mid east--.mid west){->[\chemfig{H^+}]}
\chemfig{[:-30]-[@{left,.75}]N(-[6]H)-[:30](=[2]O)--[:30]--[:30]--[@{right,0.25}:30]}
\makebraces[5pt,25pt]{\!\!\!n}{left}{right}
\chemfig{ * 6(-= * 5(-\chembelow{N}{H}-(-R_2)=(-R_1)-)-=-=)}
\schemestop

\end{document}
