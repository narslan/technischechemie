\section{Erzählung}
\epigraph{
The Black Atlantic developed from my uneven attempts to show these students, that the experience of black people were part of the abstract modernity they found so puzzling and to produce as evidence some of the things that black intellectuals had said – sometimes as defenders of the West, sometimes as its sharpest critics- about their sense of embeddedness in the modern world.
}{Paul Gilroy\footnotemark} \footnotetext{Gilroy,\textit{The Black Atlantic.
Modernity and Double Consciousness},IV.}

Die Unterteilung der Welt in Zentrum und Peripherie wird von postkolonialen
Theoretiker\_innen als eine der maßgebendsten Praktiken der sogenannten
Europäischen Moderne gefasst. Maßgebend in zweierlei Hinsicht. Mit Blick auf
die gewaltvolle Herrschaft der kolonisierenden Länder über die kolonisierten
Länder und mit Blick auf die mit dieser Herrschaftsform einhergehende
dualistische Wissensproduktion, die sich gewiss nicht ohne
Widerstand\footnotemark \footnotetext{Widerstand als allgegenwärtiges Moment
des kolonialen Verhältnisses – bereits widerständige Lesart, da das Phantasma
der Fügsamkeit der Kolonisierten in Ausbeutungsverhältnisse dadurch gebrochen
wird. Vgl.: Troillot, \textit{Silencing The Past. Power and the Production of
History}.} aber
dennoch mit enormer Kraft in das Denken der kolonisierten wie auch
kolonisierenden Gesellschaften und ihren Subjekten einschrieb und
schreibt.\footnotemark \footnotetext{Frantz Fanon, \textit{Schwarze Haut, Weiße
Masken}, Frankfurt: Suhrkamp, 1992 und Smith, \textit{Decolonizing
Methodologies}.} Im letzten Kapitel konnte ich aufzeigen, dass hierbei
insbesondere die Erfindung des 'Menschen' von einer tiefen Spaltung gezeichnet
ist, in der sich vermeintlich Subjekte und Objekte gegenüberstehen. Die Idee
vom Menschen ist damit stets an einen Spiegel gebunden, der einer die eigene
Negation vorhält und erst so die Möglichkeit der Selbstkonstitution eröffnet.
Diese Spaltung, das hat die Auseinandersetzung mit dem Erfahrungsbegriff in der
emanzipatorischen Epistemologie gezeigt, eröffnet den dermaßen verschieden
positionierten Subjekten jedoch auch äußerst verschiedene Spiegelbilder oder
anders, Erfahrungsräume die unterschiedliches Wissen hervorbringen. Subjekte
können sich der Spaltung zwar nicht entziehen, sie können ihre Erfahrungen
jedoch im Verhältnis zu dieser bestimmen, reflektieren und so Wissen erzeugen
das andere Spiegel- und Selbstbilder ermöglicht. Dazu bedarf es nicht nur eines
kritischen Blickes darauf wie sich Dualismen in das eigene Denken einschreiben,
sondern auch, wie jene Dualismen historisch gewachsen sind und so das Denken
mitsamt seinen vermeintlichen Grenzen formen konnten. Dieser Blick in die
Geschichte des Denkens muss, das habe ich im letzten Kapitel aufgezeigt, die
Gewalt der kolonialen Eroberung fokussieren aus der jene Spaltung entspringt.
Das Wissen, das als legitimiertes Wissen aus dieser Gewalt heraus entstanden
ist und die Spaltung des Menschen in Subjekte und Objekte immerwieder von Neuem
vollzieht, schreibt sich dabei auch in die Geschichten ein, die von und über den Menschen erzählt werden.

Im Verlaufe dieses Kapitels möchte ich auf jene Perspektiven Bezug nehmen,
welche diese Praktik der Unterteilung in Zentrum und Peripherie als Erzählung
theoretisieren. Die im vorherigen Kapitel explizierte andro- und eurozentrische
Wissenschaftspraxis, ermöglicht, so die These, eine Erzählung in der sich
Europa nicht nur als Mittelpunkt der Welt, sondern darüber hinaus als
Ursprungsort eines Denkens inszeniert, welches tatsächlich Ausdruck einer
gewaltvollen Begegnungs- und Beziehungsgeschichte war und ist. Meine
Bezugnahme von Repräsentations- und Erzähltheorie zeigt hier auf, dass es sich
bei der Europäischen Moderne nicht um irgendeine Erzählung handelt, sondern in
direkter Abhängigkeit zur hegemonialen Episteme um eine hegemoniale Erzählung.
Dabei wird die Europäische Moderne auf der Ebene ihrer diskursiven Selbst- und
Außenkonstruktion untersucht und der Zusammenhang bzw. die Verschränkung von
Erzählung und Wissensproduktion verdeutlicht. Im Vordergrund steht dabei für
mich nach wie vor die Frage nach der Erfahrung: Welchen Ort erhält Erfahrung?
Wie wird Erfahrung gedacht? Kann Erfahrung für eine dekoloniale Intervention
nutzbar gemacht werden?

Gilroy greift die Bedeutung von Erfahrung auf, wenn er wie im oben stehenden
Zitat zu seinen Schwarzen Student\_innen spricht, und diese auffordert, sich als
Teil der \glqq abstrakten Moderne \grqq \footnotemark \footnotetext{Gilroy,
\textit{The Black Atlantic.}} zu begreifen die ihnen bisweilen \glqq rätselhaft
\grqq \footnotemark \footnotetext{Ebd.}
erscheint. Letzteres verdeutlicht die narrative Macht, die dem kolonialen
Projekt entspringt. Wie sonst kann den Studierenden etwas abstrakt und
rätselhaft erscheinen, das in vielfacher Hinsicht von ihren Erfahrungen und den
Erfahrungen ihrer Vorfahr\_innen konstituiert ist? 

Rätsel haben bei aller Unterschiedlichkeit gemein, dass sie etwas verbergen.
Die Erzählung über die sogenannte Europäische Moderne, das konnte ich in den
letzten Kapiteln herausstellen, verbirgt nicht nur die Gewalt aus der sie
entsprungen ist, sie verbirgt zudem, dass sie nur auf Grund dieser Gewalt
möglich werden konnte. So ist es nicht verwunderlich, dass einer\_m etwas
rätselhaft erscheint, wenn dieses etwas um eine geteilte, gewaltvolle Erfahrung
kreist, die weder beschrieben noch benannt wird. Und es ist noch weniger
verwunderlich, dass diese Ent\_nennung, das Misstrauen jener weckt, die in
dieser Erzählung kategorisch ausgeschlossen werden, wie von Gilroy beschrieben.

Wenn im folgenden die Europäische Moderne, die viel genauer die von Europa
vereinnahmte Moderne genannt werden müsste, als Erzählung theoretisiert wird,
dient dies nicht einer Leugnung der stattgefundenen Gewalt. Eine Erzählung
findet nicht anstatt einer Wirklichkeit statt, sie ist viel eher ein Werkzeug,
das aus der Gewalt entspringt und sie zugleich legitimiert und die darum für
eine Untersuchung interessant ist.\footnotemark \footnotetext{Vgl. Bristol,
\textit{Plantation Pedagogies.}} 

Gilroys Forderung, die tatsächliche Eingebundenheit der Ausgeschlossenen
anzuerkennen, ist Teil eines postkolonialen und feministischen
Theorieprojektes, das es sich zur Aufgabe macht, von und zu dem Außen zu
sprechen, auf das sich Europa in seinem Bestreben, das Zentrum der Welt zu
sein, heimlich bezieht. Dadurch wird die Spaltung, durch die sich die
Europäische Moderne konstituiert, unterwandert und die Europäische Moderne als
Prozess der wechselseitigen Hervorbringung von Zentrum und seinem Außen
theoretisiert. Die Spaltung wurde insbesondere in der feministischen
Epistemologie als Spaltung im Geschlechterverhältnis begriffen.
Dekolonisierung, das Vorhaben die  Europäische Moderne zu Dezentrieren bzw.
Europa zu \glqq provinzialisieren \grqq \footnotemark
\footnotetext{Chakrabarty, \textit{Europa als Provinz.}} problematisiert jedoch nicht nur die Logik der
Zweigeschlechtlichkeit sondern greift darüber hinaus Rassismus als inhärenten
Bestandteil der Europäischen Moderne auf und an. Postkolonialer Feminismus
versucht hier den Dualismus von Körper- Geist, Subjekt-Objekt zu überwinden in
dem es aufzeigt, wie vermeintlich Gegensätzliches zusammen gedacht werden kann
und so Erfahrungen vom Menschsein in das Wissen eingeschrieben werden können,
die im dominierenden Diskurs negiert wurden. Die kontroverse Auseinandersetzung
um Erfahrung in der Wissensproduktion kann hier als ein konkreter Versuch
verstanden werden, emanzipatorische Epistemologien aufzubauen.

Theoretiker\_innen haben in ihrem Versuch, kulturelle Phänomene
nachzuvollziehen, jene Phänomene auf verschiedene Art und Weise
konzeptionalisiert. Was heißt es nun, das soziale Phänomen, der Europäischen
Moderne als Erzählung zu theoretisieren?

In dieser Arbeit, das ist bereits vielfach angeklungen, dominiert eine
diskursive Ebene als theoretischer Zugang. Dies birgt zugleich den Anspruch,
dass das Materielle, Verkörperte etc. auch in dem Diskursiven Ausdruck gewinnen
bzw. durch das Diskursive zugänglich gemacht werden kann. Problematisch ist
hierbei, dass eine Sprache in der Narrativ, Erzählung, Diskurs, Repräsentation
etc. zu den Kerntermini werden es unter Umständen nicht schafft, die Präsenz
der Gewalt, also ihre Gegenwärtigkeit in dem Leben der Menschen, zu
artikulieren.

Die Auseinandersetzung mit Erfahrung kann hier jedoch als Brücke zwischen dem
Materiellen und Diskursiven verstanden werden, da Erfahrungen zum Beispiel im
Anschluss an Duden\footnotemark \footnotetext{Vgl. Duden, \glqq Somatisches
Wissen\grqq. }, als Ausdruck somatischer Erlebnisse verstanden werden
können, die durch ihre Artikulation Einzug in das Diskursive erhalten bzw.
darin intervenieren. Insbesondere Dudens Konzept scheint sich daher für eine
postkoloniale Kritik am Dualismus zu eignen. Dabei muss, wie schon im letzten
Kapitel herausgestellt, stets achtsam damit umgegangen werden, welches Subjekt
imaginiert und damit auch welchen Erfahrungen Raum gegeben wird um die
ausschließende Logik hegemonialen Wissens nicht fortzuschreiben. 

Bevor ich mich, wie eben angekündigt, mit der Bedeutung von Erfahrung für die
Praxis des (Gegen)Erzählens beschäftige, möchte ich mich zunächst noch in
allgemeinerer Form dem Begriff der Erzählung widmen. 

Die Disziplin, die ich hierfür heranziehe ist die Narratologie. Dabei
beschränke ich mich auf Auseinandersetzungen im deutschsprachigen Raum und gehe
nicht auf den, mit dem postmodernen Aufschwung einher gegangenen, und
insbesondere durch Jean-François Lyotard bekannt gewordenen Begriff des Grand
Récit, der Großen Erzählung ein, da es sich hier um ein zwar interessantes,
aber sicherlich den Rahmen dieser Arbeit sprengendes Diskursfeld handelt.

Die Narratologie ist auch ohne die französischen, postmodernen Intellektuellen
ein weites Feld. Deren Aufschwung und Niedergang in Mitten der Disziplinen wird
dabei recht unterschiedlich erzählt. Insbesondere die sogenannten Feinde der
Narratologie könnten gegensätzlicher nicht sein: So wird einmal die Hegemonie
der Naturwissenschaften\footnotemark \footnotetext{Achim Saupe und Felix
Wiedemann, \glqq Narration und Narratologie. Erzähltheorien in der
Geschichtswissenschaft,\grqq Version 1.0, in \textit{Docupedia-Zeitgeschichte}, (2015), 1.} angeführt, ein anderes Mal dem Poststrukturalismus die
Schuld für das sinkende Interesse an der Narratologie in die Schuhe
geschoben\footnotemark \footnotetext{Ansgar Nünning, \glqq Wie Erzählungen Kulturen
erzeugen. Prämissen, Konzepte und Perspektiven für eine kulturwissenschaftliche
Narratologie \grqq in \textit{Kultur-Wissen-Narration. Perspektiven transdisziplinärer
Erzählforschung}. Herausgegeben von Andrea Strohmaier,  (Bielefeld. Transkript,
2013), 16.} Gemeinsam ist jenen Erzählungen jedoch, ihre geteilte Erleichterung
gegenüber dem Come Back der Narratologie in den vielfältigsten Disziplinen. Die
Spanne reicht von literaturwissenschaftlichen Fragestellungen nach der
Erzählstruktur, über psychologische Untersuchungen der narrativen Identität, zu
geschichtswissenschaftlichen Auseinandersetzungen um das Spannungsfeld von
Fiktionalität und Faktizität angesichts der narrativen Aufbereitung von
historischen Ereignissen. Auch wenn es nicht wenig unterhaltsam ist, sich mit
den verschiedenen Niedergangs- und Behauptungsgeschichten der Narratologie zu
befassen, muss an dieser Stelle davon abgesehen werden. Stattdessen möchte ich
mich mit einem Teilbereich der Narratologie befassen, nämlich der
kulturwissenschaftlichen Erzählforschung. Grund für meine Präferenz der
Kulturwissenschaften ist ihre Konzeption der sogenannten
\glqq Wirklichkeitserzählung \grqq \footnotemark \footnotetext{Nünning, 
\glqq Wie Erzählungen Kulturen erzeugen,\grqq  27.}. Anders als andere Zugänge gerät hier das
Wechselspiel, in dem sich das Grand Narrative mit den Partikularen Geschichten
befindet, in den Fokus. Ein zentraler Vertreter der kulturwissenschaftlichen 
Erzählforschung ist Ansgar Nünning. Nünning plädiert für eine Synthese kultur-
und erzählwissenschaftlicher Forschungsansätze.\footnotemark \footnotetext{Ebd. 28.}
Ausgangspunkt dieser Forderung
bilden dabei seine Auffassung, dass sich Kultur u.a. über Praxen des Erzählens
konstituiert bzw. dass Erzählungen immer als kulturelle und damit wandelbare,
an den zeit- und räumlichen Kontext gebundene Phänomene untersucht werden
müssen. Eine kulturwissenschaftliche Narratologie ermöglicht es in diesem Sinne
eine in der Vergangenheit vielfach ahistorisch praktizierte Narratologie für
die Bedeutung des Erzählkontextes sensibel zu machen. Für die
Kulturwissenschaften hingegen ergebe sich durch die Synthese weniger eine
Erweiterung ihrer theoretischen Perspektive, als vielmehr ein Impuls für einen
äußerst relevanten, vielfach jedoch ignorierten Gegenstandsbereich: Erzählungen
werden hier, so Nünnung, nicht auf Prosa reduziert, viel eher interessiert sich
die kulturwissenschaftliche Narratologie für sogenannte
'Wirklichkeitserzählungen' und versteht darunter jene über das literarische
Erzählens hinausgehende Praxen der Bedeutungskonstruktion. Erzählungen werden
hier durch zwei wesentliche Kennzeichen gefasst. Zum einen sind sie stets
Ausdruck \glqq spezifischer Verknüpfungen \grqq \footnotemark
\footnotetext{Saupe und Wiedermann, \glqq Narration und Narratologie,\grqq 2.}, zum anderen werden sie von einer
\glqq temporalen Struktur \grqq \footnotemark \footnotetext{Ebd.}  geformt, sodass sie als \glqq zeitlich strukturierte
Repräsentation von Ereignissequenzen\grqq \footnotemark \footnotetext{Ebd.} begriffen werden können. So könne die
Erzählung auch vom Diskursbegriff abgegrenzt werden, der sich stärker auf die
Ordnungsfunktion von Sprache beziehe und damit synchron angelegt sei, wogegen
sich der Erzählbegriff eher auf linear und temporale Dimensionen
beziehe.\footnotemark \footnotetext{Nünning, \glqq Wie Erzählungen Kulturen
erzeugen,\grqq 28.}
Erzählungen bilden damit eine Form des materialisierten, bzw. performierten
Ausdrucks gesellschaftlicher Ordnungen:
\begin{myenv}
  \textit{
  \glqq Zweifelsohne sind es Erzählungen die kollektiven, nationalen
  Gedächtnissen zugrunde liegen und Politiken der Identität bzw. Differenz
  konstituieren. Kulturen sind immer auch als Erzählgemeinschaften anzusehen,
  die sich gerade im Hinblick auf ihr narratives Reservoir unterscheiden \grqq
  \footnotemark \footnotetext{Wolfang Müller-Funk, \textit{Die Kulturen und ihre
Narrative. Eine Einführung}, (Wien: Springer Verlag, 2008), zitiert  in Nünning, ebd., 29.}
  } \end{myenv}

  Wolfgang Müller-Funk greift hier die politische Dimension kultureller
  Praktiken auf und weist daraufhin, dass jene Ordnungen von
  Auseinandersetzungen um Identität und Differenz geprägt sind, die als Kämpfe
  um Zugehörigkeit zu eben jenen kulturellen Gemeinschaften aufgefasst werden
  können. Die Frage der Macht bahnt sich auf diese Weise ihren Weg in die
  Narratologie und nimmt zugleich Gebrauch von deren formalistischen
  Ausprägungen. Das heißt, dass der Blick auf die Form für die Narratologie
  konstitutiv ist, da er die Möglichkeit bietet, über direkte Erzählbotschaften
  hinaus Aufschlüsse über die ideologische Dimension des Erzählten zu erhalten.
  Frederic Jamesons ideology of the form die davon ausgeht, dass die Art und
  Weise wie erzählt wird mehr über das Motiv der Erzählung aussagt, als eine
  bloße Aufmerksamkeit für den Inhalt es vermag, gilt als einer ihrer
  prominentesten Vertreter. An dieser Stelle wäre eine tiefer gehende
  Auseinandersetzung mit den Methoden der Narratologie bzw. ihrem
  Formverständnis möglich, für meine Arbeit jedoch nicht fruchtbar. Da meine
  Arbeit keine Anwendung dieser Methode beabsichtigt, sondern sich vielmehr der
  Narratologie bedient um den Begriff der Erzählung, wie er im postkolonialen
  Feminismus Verwendung findet, zu schärfen.

  \subsection{Ich-Erzählung als Gegenerzählung}

  Ich knüpfe nun wieder an die feministische Auseinandersetzung um den
  Erfahrungsbegriff in der Erkenntnisproduktion an, konkretisiere die
  Diskussion allerdings auf eine postkoloniale Praxis: die Ich-Erzählung als
  Gegenerzählung. Dazu setze ich mich mit den Möglichkeiten von Repräsentation
  und Dialog auseinander. Kann eine\_r sich selbst erzählen? Was braucht sie\_er
  dafür? Ab wann ist eine Ich-Erzählung eine Gegenerzählung?

  \subsubsection{Repräsentation}

  Eine Auseinandersetzung mit den Möglichkeiten und Grenzen von Repräsentation
  durchzieht diese Arbeit wie ein roter Faden. Bereits im Bildungsbegriff, der
  Wimmer folgend, auf einem Subjektverständnis ruht das niemals in der Lage ist
  sich selbst zu repräsentieren, da es immer auf die Entbehrung eines
  kohärenten Selbst zu Gunsten eines Zwischenraums zwischen Ich und Welt
  zurückgeworfen wird, wird die Problematik der Repräsentation aufgegriffen.
  Diese Problematik wird mit  Mohanty in Anbetracht eines Sprechens über bzw.
  für die 'Dritte Welt' weitergedacht, wenn sie die Herausforderungen
  beschreibt, in ihren Darstellungen nicht zugleich als Vertreterin eben jener
  Dargestellten missverstanden zu werden. Während es also bei Wimmer zunächst
  um die Repräsentation des Selbst geht, beschreibt Mohanty die Schwierigkeit
  andere zu repräsentieren.
  
  Auch die hegemoniale Episteme, die daraufhin von
  mir erörtert wird, beruht letztlich auf einem falschen Versprechen der
  Repräsentation: Sie gibt vor für alle zu sprechen,  in Wirklichkeit
  verkörpert sie jedoch nur die Interessen einer Minderheit.  Die Einordnung
  diesen Wissenssystems als hegemonial, beschreibt darüber hinaus die
  Schwierigkeit sich ihm zu widersetzen und damit aus dem herrschenden
  Repräsentationsregime auszubrechen, da es so stark auf  Zustimmung aufgebaut
  und in der Lage ist auch die Versuche des Widerstands für sich zu
  vereinnahmen.

  Versuche, der hegemonialen Idee des  Menschen eine anderes Fundament
  entgegenzustellen,  kommen hier nicht ohne Widerspruch aus. Emanzipatorische
  Strategien fallen oftmals auf eine von identitären Kategorien durchdrungene
  Sprache zurück – zu Lasten einer Ideen von Erfahrungen als Erfahrung von
  Differenz. Der Repräsentationsanspruch der mit identitären Kategorien, wie
  der Kategorie Frau\_* einhergeht wird hier von jenen Frauen\_* zurückgewiesen,
  die sich in den proklamierten 'weiblichen' Erfahrungen nicht wiedererkennen
  und sich diesem kategorialen Ausschluss widersetzen.

  Die Gegengeschichtsschreibung, oder Gegenerzählung kann nun als Versuch
  beschrieben werden, einen Zwischenraum zu öffnen der dem dissonanten Stimmen
  kein vorgegeben identitären Kategorien überstülpt. Gegenerzählungen  können
  auf diese Weise neue Wege der Repräsentation erkämpfen, die sich dem
  vereinnahmenden Praxen des Verstehens widersetzen und Erfahrung immer als
  Erfahrung von Differenz verstehen. Im Sinne eines strategischen
  Essentialismus, kann sich aus dieser Differenz heraus behauptet werden ohne
  auf die Differenz reduziert zu werden.
 
  Laurette Bristol geht auf dieses emanzipatorische Potential von Erfahrung als
  Wissen in der Praxis des Erzählens  in  \glqq Plantation Pedagogy. A Postcolonial
  and Global Perspective \grqq \footnotemark \footnotetext{Bristol,
  \textit{Plantation Pedagogy.}} ein.

  Geschichten, bzw. das Erzählen von Geschichten, so schreibt Bristol in ihrer
  Monographie können als Praktiken verstanden werden, die die Kraft haben,
  herrschende Repräsentationsregime zu unterwandern. Sie werden damit zu
  Werkzeugen für eine Pädagogik, die es den Erzähler\_innen und Zuhörer\_innen
  ermöglicht, wenn auch nur für einen Moment, aus den “ravages of
  existence”\footnotemark \footnotetext{Ebd., 1.}
  auszubrechen und an an einen anderen Ort zu gelangen. Dieser andere Ort birgt
  zugleich ein Entkommen, eine Distanz zum Selbst  als auch einen Standpunkt,
  von dem aus eine andere Perspektive auf das Selbst in der Welt möglich ist.
  Damit schließt Bristols Konzept der Erzählung an den Bildungsbegriff der im
  ersten Kapitel dargelegt wurde an und beschreibt das emanzipatorische
  Potential, welches im Erzählen von Geschichten liegt:
  \begin{myenv}

\textit{\glqq To tell a story [she claims, is] to seize a political space
  created through talk, to represent self, to claim his/her authority on
  the topic of a discussion and to share his/ her interpretation of social
existence with the community.\grqq \footnotemark \footnotetext{Ebd.}} 
\end{myenv}

Interessant sind hierbei ihre Verwendung der Verben to \textit{seize},
\textit{claim} und \textit{share}.
Das Erzählen von Geschichten beschreibt sie hier als Praxis, in der eine
bestimmte Perspektive zugleich beansprucht, behauptet und geteilt wird. Das
Subjekt, das Autorität über seinen Zugang zur Wirklichkeit beansprucht, wird
dabei nicht als Individuum allein, sondern in der Verbindung zur Gemeinschaft
begriffen mit der es seine Behauptung, die immer als Interpretation verstanden wird, teilt.

Der Versuch, der hegemonialen Erzählung etwas entgegenzusetzen, kann jedoch nur
gelingen, wenn das Subjekt-, Geschichts- und Wissenschaftsverständnis, das sich
in der Europäischen Moderne etablierte, überwunden wird. Denn „even marginal
experiences narratives risk repeating the biases and exclusions of received
narrative logics“\footnotemark \footcitetext[S. 142]{sstone}, wenn sie von einem autonomen Subjekt ausgehen, lineares
Geschichtsverständnis bedienen oder Wissenschaft außerhalb von
gesellschaftlichen Verhältnissen begreifen, so Shari Stone-Mediatore in ihrer
Monographie „Reading across Borders. Storytelling and Knowledges of
Resistance“\footnotemark \footnotetext{Stone-Mediatore, \textit{Reading across
Borders}.}. Um Räume für eine Gegengeschichtsschreibung zu schaffen, ist es
also dringend notwendig, Strategien zu entwickeln mit denen alternative, den
dominanten Vorstellungen widersprechende Subjekt-, Geschichts- und
Wissenschaftsverständnisse ermöglicht werden.

Gloria Anzalduás und Domitila Barrios de Chúngaras Arbeiten liefern für
Stone-Mediatore wichtige Impulse und ich möchte deren Perspektiven an dieser
Stelle als Einstieg nutzen, um konkreter darin zu werden, woran sich Praxen der
Gegenerzählung orientieren können: Anzalduá, so Stone-Mediatore, beschreibt
ihre Arbeit als Versuch, persönlichen Erfahrungen Raum zu geben bei einer
gleichzeitigen Notwendigkeit, neue, eigene Kategorien zu schaffen. Poesie,
Autobiographie und Geschichte würden es Anzalduá ermöglichen sich mit jenen
Erfahrungen auseinanderzusetzen, die die Spannkraft besitzen, herrschende
Bilder von Frauen\_* of Color zu unterwandern.\footnotemark \footnotetext{Anzalduá in Stone-Medatore, ebd. 143.} Auch Barrios de Chúngara sehe in
erfahrungsorientiertem Schreiben die Möglichkeit, zwar keine Revolution, aber
immerhin „Sand ins Getriebe zu streuen“ \footnotemark \footnotetext{Barrios de Chúngara in Stone-Mediatore, ebd.}, da die Schilderung eigener
Erfahrungen eine Möglichkeit für andere böte, sich neu zu orientieren. Stone
Mediatore betont an dieser Stelle, dass sowohl Anzalduá als auch Barrios de
Chúngara insbesondere die Erfahrungen von Widerstand in neue, unorthodoxe
Formen bringen. Ihre Erzählungen seien Ausdruck spezifischer Erfahrungen die
sie an den sozial/geografischen Kontext binden, ohne dabei ein
essentialistisches Verständnis des Kontextes zu bedienen. Damit würden sie sich
gegen eine Aufwertung abgewerteter Subjektpositionen wehren und stattdessen die
Logik angreifen, in welcher der Kontext die Handlungsmöglichkeiten der Subjekte
determiniert.\footnotemark \footnotetext{Stone- Mediatore, 144.} Stone-Mediatore beschreibt dies folgendermaßen:
\begin{myenv}
    \textit{„[...] they each use their writing to explore the forces that
    condition their experience and to renarrate their identities in ways that
  help them to confront those forces more effectively. In the process, they
also recast 'identity' as a historically rooted yet also strategic
category.“\footnotemark \footnotetext{Ebd.}}
\end{myenv}

Es geht also vielmehr darum, die Grenzen der Beschreibbarkeit, mit denen die
eigene Situation sichtbar gemacht werden kann, zu verschieben, also eine
Sprache zu entwickeln die dem Anspruch nach Repräsentation der eignen
Erfahrungen ein wenig mehr gerecht wird, als es herrschende Erzählparadigmen
und ihre Konzepte vermögen.
\subsubsection{Dialog}

Mit der Frage, wie sich unterschiedliche Ausdrucksformen von Wissen miteinander
in Bezug gesetzt werden können, setzt sich Lorraine Code auseinander. Sie
interessiert sich dafür,  wie Wissen über Menschen und die  Situationen in
denen sie sich befinden dialogisch entstehen könnte. Hier richtet sie ihr
Augenmerk auf das Potential, das in sogenannten \textit{First-Person-Accounts} also
Ich-Erzählungen liegt, in denen von eigenen Erfahrungen berichtet wird. Diese
Form der Beschreibung von Wirklichkeit erhält ihrer Ansicht nach in der
\textit{Malestream Epistemology} kaum Bedeutung.\footnotemark \footnotetext{Lorraine
Code, „Experience, Knowledge and Responsibility“ , in A. Garry and M. Pearsall, Hrsg., Women, Knowledge and Reality (Boston: Unwin Hyman, 1989), 168.} Auch Code setzt sich mit der Bedeutung von Erfahrung auseinander. Sie versteht Erfahrung in Abgrenzung zu theoretischen Positionen. 
Die Begegnung mit und Einbeziehung von Erfahrungen, die über Ich-Erzählungen
artikuliert werden kann hierbei, so Code,  der Formalität und Isoliertheit
abstrakter Theoriegebilde eine lebendige, lebensweltorientierte Erkenntnis
gegenüberstellen. Persönliche Erfahrungen sind für Code keineswegs wahrer oder
authentischer als wissenschaftliche Texte. Ihre Affinität für Ich-Erzählungen
liegt hingegen in der enormen Mannigfaltigkeit der Perspektiven die durch die
Praxis des Erzählens zum Ausdruck kommen. Ziel der erkenntnistheoretischen
Nutzbarmachung von Erfahrung sieht sie entsprechend nicht in der Ergänzung
bereits bestehender Theorien durch Ich-Erzählungen. Viel eher soll die
Einbindung unterschiedlicher Wissensformen die Brüche und Leerstellen aufzeigen,
die ihrer Ansicht nach einen unumgehbaren Bestandteil jeder Theorie beinhalten
und zu einem Dialog zwischen theorie- und erfahungsbasiertem Wissen anregen.

Um unterschiedliche Formen des Wissens in einem gleichberechtigten Dialog
miteinander sprechen zu lassen, müsse jedoch die Suche nach dem 'reinen Wissen'
als unerreichbares und ohnehin nicht erstrebenswertes Ziel aufgegeben
werden.\footnotemark \footnotetext{Code, „Experience, Knowledge and Responsibility,“ 157.}

Stattdessen gilt es, so Code, Epistemologie und Ethik als gemeinsame und gleichsam bedeutende Aspekte der Wissensproduktion anzuerkennen. Verantwortungsbewusstes und an Erfahrung orientiertes wissenschaftliches Tun werde dadurch zu einer Grundprämisse.

Code macht hier auch deutlich, dass sie ihrem Plädoyer keine verallgemeinerbare
Erfahrung des 'Frau\_* seins' zu Grunde legt. Eine solche Annahme verdecke die
vielfachen Unterschiede, die zwischen Frauen\_* existierten. Stattdessen
verwendet sie stets den Plural und spricht von Erfahrung\textbf{en} von Frauen\_* um der
Heterogenität von Lebensumständen in denen Frauen\_* entsprechend
unterschiedliche Erfahrungen machen, gerecht zu werden.

Ihr Ansatz, sowohl den Erfahrungen als auch der Verantwortung eine wesentliche
Bedeutung in der Erkenntnisproduktion und ihrer Bewertung zuzuschreiben, kann
dabei weder als komplementär zur traditionellen Erkenntnistheorie, noch als von
ihr vollständig losgelöst verstanden werden. Sie begreift ihre Arbeit viel eher
als eine permanente Auseinandersetzung mit der malestream tradition deren
Lücken, Ausschlüsse und unbefragten Vorannahmen sie durch eine kritische
Bezugnahme weiblicher Erfahrungen zur Diskussion stellen möchte.\footnotemark
\footnotetext{Ebd., 158.}

Code schreibt weiblicher Erfahrung für feministische Erkenntnistheorie zwar
einen zentralen Stellenwert zu, warnt aber davor, essentialisierenden
Universalismen auf der einen Seite, oder subjektivistischen Relativismen auf der
anderen Seite zuzuspielen. Hingegen müsse die klassische Gegenüberstellung und
das damit einhergehende Ausschlussprinzip zwischen Subjektivität und
Objektivität überwunden werden. Denn da Frauen\_* per se ein subjektiver
Standort in Abgrenzung zum objektiven wissenschaftlichen Wissen zugeschrieben
werde, führe dies zu dem fatalen Schluss, dass weibliches Wissen nicht
gleichzeitig wissenschaftliches Wissen sein könne.\footnotemark
\footnotetext{Ernst, \textit{Diskurspiratinnen}, 87.} 

Feministische Standpunktepistemologie, in die ich bereits mit Hartsock eingeführt haben, kehren dieses Prinzip zwar erfolgreich um, indem sie gerade dem weiblichen Wissen einen objektiveren Zugang zu Wirklichkeit zuschreiben, bleiben aber in der binären Logik von Objektivität und Subjektivität verhaftet, die es zu überwinden gelte.

An dieser Stelle führt Code in die Praxis des Dialogs ein und löst damit auch die Dichotomie, nach der zwischen dem Subjekt das Wissen generiert und das Objekt, über das Wissen hergestellt wird, unterschieden werden kann, auf.

Stattdessen werden Subjekt und Objekt gewissermaßen austauschbare Kategorien, da
es fortan stärker um den Dialog zwischen unterschiedlichen
Abstraktionsdimensionen geht, indem verschiedene Formen des Wissens miteinander
in Beziehung gesetzt werden.\footnotemark \footnotetext{Ebd., 88.} 

Ein solcher Dialog setzte jedoch ein gewisses Maß an Bescheidenheit, sich den
Grenzen des eigenen Wissens bzw. der individuellen Perspektive bewusst zu sein,
voraus. Nur ein von wechselseitigem Interesse bestimmter Austausch könne einen
Dialog ermöglichen, der tatsächlich verschiedene Formen des Wissens zueinander
in Beziehung setzt ohne dabei in klassische Hierarchisierungen zu verfallen.
Code stellt an dieser Stelle das Konzept der Freundschaft als interessante
Allegorie zur Diskussion. So sind es ihrer Ansicht nach, und hier nähert sich
ihre Position der von Mohanty bedeutend an, gerade die Getrenntheit, Distanz und
die Anerkennung von Differenz die den Dialog in Freundschaften prägt und die
unabdingbare Voraussetzungen für die Auseinandersetzungen um epistemologische
Fragen darstellt.\footnotemark \footnotetext{Ebd., 89.}

Code unterscheidet in ihrer Argumentation zwischen wissenschaftlich
abgeleiteter, bzw. interpretierter Erfahrung und alltäglicher Erfahrung. Das
Abstraktionsniveau soll jedoch für die Nutzung der Erfahrungen für die
Erkenntnisgenerierung nicht bewertet werden. Vielmehr ist die Einsicht, dass
gerade der Verwobenheit unterschiedlicher Erkenntnisprozesse Raum gegeben werden
muss, zentral für das Zustandekommens eines Dialoges.\footnotemark
\footnotetext{Ebd., 92.}

Erzählungen, so kann Code hier verstanden werden, bedienen unterschiedliche
Abstraktionsniveaus, die insbesondere in ihren Verknüpfungen bzw.
wechselseitigen Bezugnahmen interessant werden.

Ihre Perspekive ist darum interessant, weil sie die poststrukturalistische
Präferenz des abstrakten Denkens und die phänomenologische Affinität für das
unmittelbare als gleichberechtigte Formen des Wissens anerkennt und für ihre
Synthese plädiert. Das sogenannte 'gegen den Strich lesen' von ideologisch
aufgeladenen Erfahrungen, dem vermeintlich eine theoretische Schulung
vorausgehen muss, wird insbesondere auf Grund der Hierarchisierung von Erfahrung
und Theorie, die diesem Ansatz vorausgeht, von ihr in Frage gestellt. Nicht erst
durch die abstrakte Analyse von  Erfahrungen werden Erkenntnisse gewonnen, viel
eher so lässt sich Code resümieren, sind Erkenntnisse das Ergebnis dialogischer
Praxen in denen Wissen mit Wissen in den Austausch geht wenn Erfahrung auf
theoretische Positionen trifft.

\subsection{Transmoderne als Gegenerzählung}

Ein anderer Ansatz des Gegenerzählens stellt das Konzept der Transmoderne von
Enrique Dussel dar. Die Gegenerzählung versucht hier nicht als partikulare
Ich-Erzählung das hegemoniale Narrativ der Europäischen Moderne zu unterwandern,
sondern als eine Metaerzählung, die über die europäische Moderne hinausgeht,
letztere in ihrer Macht einzuschränken. So wird die Europäische Moderne zur
partikularen Erzählung innerhalb einer Transmoderne, in der viele einander
widerstreitende Erzählungen nebeneinander existieren.

Für Dussel stellt der Partikularismus, den ich eben unter den Aspekten
Repräsentation und Dialog vorgestellt habe, keine ausreichende Kritik an der
Europäischen Moderne dar. Seine Position knüpft zwar teilweise an die von Code
und Bristol vorgestellten Strategien an, stellt diese jedoch unter ein anderes
Vorzeichen. Seine Vermutung ist, dass sich Formen des Partikularismus zu schnell
relational zur Europäischen Moderne positionieren und damit leicht von ihr
vereinnahmt werden können. Alcoff beschreibt seine Position hier folgendermaßen:
Die Gefährlichkeit der Europäischen Moderne liege ihm nach nicht in ihrem
Anspruch, global gültige Aussagen zu treffen, sondern in der wissenschaftlichen
Praxis, mit der sie ihre Erkenntnisse gewinnt und legitimiert.\footnotemark
\footnotetext{ Alcoff, „Dussels Transmodernism,“ 62.} Dussel greift somit nicht
nur das gegenwärtige hegemoniale Wissenssystem der Europäischen Moderne, sondern
auch ihr Entstehungsnarrativ und damit einhergehendes Selbstverständnis an.

Er macht darauf aufmerksam, dass die Alternative weder in einem puren
Relativismus, noch in einer radikalen Umkehrung bzw. Neubestimmung des Zentrums
liegen kann. Wir sind viel eher gefordert, die eindimensionale und damit
notwendig gewaltvolle Setzung dessen, von wo aus über wen Wissen produziert
werden kann, durch eine Praxis des Dialoges zu unterwandern, die verschiedene
epistemische Gemeinschaften verbindet und so dem Universalismus der Moderne ein
Pluriversalismus der Transmoderne gegenüberstellt. Das Konzept der Transmoderne
knüpft an die Kritik an der hegemonialen Erzählpraxis der Europäischen Moderne
an, und stellt ihr die Konzepte der Verortung, Exteriorität und Autonomie
entgegen. Die Praxis des Dialoges nimmt hierbei einen zentralen Stellenwert ein
um jenen Konzepten in ihrer wechselseitigen Bedingtheit Wirkungskraft zu
verschaffen. Gegenerzählung verlangt dabei eine situierte, im Dialog mit anderen
Wissensbeständen praktizierte Wissensproduktion, die ein reflexiven Umgang mit
denjenigen Kategorien vornimmt, die sie konstituieren.

Die Dramaturgie Dussels Ausführungen erhält neben einer chronologischen auch
eine analytische Ordnung. Auf beide möchte ich im Folgenden eingehen und mich
auf diese Weise dem Konzept der Transmoderne, wie es von Dussel entwickelt
worden ist, nähern. So ist es nicht überraschend, dass Dussel in seine
Ausführungen auf die Geschichte seiner intellektuellen Auseinandersetzung, als
Schüler\_* und später Student\_* zurückblickt und damit deutlich macht, dass sich
seine Philosophie unmittelbar aus dem eigenen Erleben und seiner Position in
(welt-)gesellschaftlichen Verhältnissen entfaltet:\footnotemark \footnotetext{
    Dussel „Transmodernity and Interculturality. An Interpretation from the
    Philosophy of Liberation,“ \textit{Transmodernity: Journal of Peripheral Cultural
    Production of the Luso-Hispanic World}, Vol. 1 Nr. 3, (2012).}

„Was ist eigentlich Lateinamerika?“ Mit dieser Frage wird Dussel bei seiner
Ankunft in Europa konfrontiert und sie bestimmt seine ersten Texte, die er
während seiner Zeit als Student\_* an europäischen Universitäten publiziert.
Dabei wird ihm die Frage nicht etwa von außen gestellt, viel eher beginnt seine
Auseinandersetzung mit der Geschichte Lateinamerikas auf Grund der
Selbsterkenntnis bzw. Fremdzuschreibung, dass er, entgegen seiner bisherigen
Annahmen, selbst kein Europäer\_* sei:

\begin{myenv}
    \textit{ „With my trip to Europe – in my case, crossing the Atlantic by ship
    in 1957 – we discovered ourselves to be “Latin Americans,” or at least no
    longer “Europeans,” from the moment that we disembarked in Lisbon or
    Barcelona. The differences were obvious and could not be concealed.
    Consequently, the problem of culture—humanistically, philosophically, and
    existencially—was an obsession for me: “Who are we culturally? What is our
    historical identity?” This was not a question of the possibility of
    describing this “identity” objectively; it was something prior. It was the
    existential anguish of knowing oneself.“\footnotemark \footnotetext{Dussel, „Transmodernity and Interculturality,“ 28.}
\end{myenv}

Dussels autobiographischer Einstieg kann an dieser Stelle bereits als Praxis der
Vorortung gelesen werden. Im Mittelpunkt steht dabei eine Reflexion über die
Wissensbestände, die zu Beginn seiner intellektuellen Auseinandersetzung als
legitimes Wissen an ihn herangetragen wurden und entsprechend sein
Selbstverständnis prägten. Dass die Legitimierung von Wissensbeständen immer
auch mit der Delegitimierung anderer Wissensbestände einhergeht, wird deutlich,
wenn er\_* wie eben zitiert, über die Bedeutung indigenen Wissens während seines
Studiums spricht.

Sein Interesse, den lateinamerikanischen Kontinent bzw. dessen Kultur zu
verorten setzte er in den folgenden Jahren mit dem Versuch fort, die
„historische Identität“\footnotemark \footnotetext{Ebd., 29.} Lateinamerikas innerhalb der Weltgeschichte zu
rekonstruieren. Sein zu Beginn statisch und an national-staatlichen Kategorien
geprägtes Verständnis von Kultur wird dabei schon nach kurzer Zeit verworfen und
weicht einem Verständnis von Kultur, das sich durch Hybridität und stetige
Veränderung auszeichnet. Die Schwierigkeit, die lateinamerikanische Kultur zu
beschreiben darf jedoch, so schreibt er nach seiner Rückkehr nach Lateinamerika,
durch ein essentialismus-kritischen Ansatz nicht in der Leugnung ihrer Existenz
münden. Viel eher muss die Existenz gegenüber der Ignoranz eurozentristischer
Philosophien verteidigt werden. Der Austausch zwischen Theoretiker\_innen aus
Asien, Afrika und Lateinamerika, den er in den folgenden Jahren etabliert,
gleicht dementsprechend einer Suche nach einem kulturellen Selbstverständnis
jener Teile der Welt, die in der Sprache des hegemonialen Europas als Peripherie
bezeichnet werden. Statt einer Ergänzung der hegemonialen Geschichtsschreibung
durch eine Rekonstruktion der lateinamerikanischen Geschichte, forderte Dussel
eine Rekonstruktion und damit einhergehende Neuschreibung der \textit{mythical
narratives} der Europäischen Moderne und setzte damit in seiner Analyse bei den
Auslassungen und Verzerrungen jener westeuropäischen Philosophien an, die im
klassischen Verständnis als allumfassend und universell gültig dargestellt wird
und auf die ich in dieser Arbeit bereits ausführlich eingegangen
bin.\footnotemark \footnotetext{Ebd., 30.}

Nun reicht es nicht aus, und dies kann als Dussels zentrale Kritik an vielen
postmodernen Theoretiker\_innen verstanden werden, sich von dieser Meta-Erzählung
abzuwenden um sich voll und ganz dem Partikularen zu widmen. Die Gefahr, die von
der Europäischen Moderne und ihrer Wissensproduktion ausgeht, liegt Dussel nach,
nicht in dem Versuch globale Zusammenhänge in Form einer Metaerzählung
aufzuzeigen und zu theoretisieren. Viel eher muss die Logik von Linearität und
Fortschritt, mit Europa an dessen Spitze aufgebrochen werden um eine radikale
Neuschreibung der Geschichte zu ermöglichen.

Eine solche Neuschreibung der Geschichte muss die Setzung von Zentrum und
Peripherie aufbrechen und dadurch auch die jahrhundertelange Degradierung und
Herabsetzung der\_des Anderen durch Europa ein Ende bereiten. Der Beginn dieser
Geschichte würde nicht mit Europa als Nabel der Welt gemacht werden, sondern auf
die vielfältigen komplexen Beziehungen Bezug nehmen, die die Selbst(er)findung
von Europa ermöglichten:

\begin{myenv}
    \textit{„Transmodernity displaces the linear and geographically enclosed timeline of
    Europe’s myth of autogenesis with a planetary spatialization that includes
    principal players from all parts of the globe.“\footnotemark}
    \footnotetext{Alcoff, „Dussels Transmodernism“, 63.} 
\end{myenv}

Die Idee der Transmoderne kann entsprechend als eine Erzählung verstanden
werden, die die Europäische Moderne nicht als Ausgang oder Maßstab nimmt,
sondern viel eher als Bestandteil eines komplexen Gefüges, partikularer
Geschichten versteht, die in permanenter Aushandlung miteinander entstanden sind
und fortgeschrieben werden. Die Unterteilung in Zentrum und Peripherie wird
damit obsolet, viel eher versteht sich die Transmoderne als ein Raum in dem
multiple Modernen nebeneinander in Solidarität existieren können, ohne dass sie
durch eine einzige Erzählung vereinnahmt, bzw. ihr unter oder übergeordnet
werden.\footnotemark \footnotetext{Ebd.}

Die Transmoderne ist damit nicht Endzustand, sondern viel eher ein Prozess, der
auf einem egalitären, sogenannten \textit{transversal} Dialog aufbaut bzw. diesen
ermöglicht. Denn um an dem Dialog als der multiplen Modernen teilnehmen zu
können, muss sich die Europäische Moderne als partikular begreifen lernen und
damit auch die Geschichte, die sie bis dahin an der Spitze des Fortschritts und
der Rationalität vermutete umschreiben:

\begin{myenv} 
    \textit{„If the modern understands itself, as it so often does, as the
    unique moment of self-conscious reflexivity, with epistemic rigor and a
    capacity to escape conventions of doxa from pre-rational eras, it is not
clear how to achieve a meaningful solidarity.“\footnotemark \footnotetext{Ebd., 64.}} \end{myenv}

Dussel macht hierbei auch darauf aufmerksam, dass es nicht darum gehen sollte
einen neuen Nullpunkt zu finden, von dem aus alles von Neuem beginnen kann und
der einen egalitären Dialog unterschiedlicher Wissensbestände ermöglichen würde.
Das koloniale Verhältnis wirkt fort und macht ein symmetrisches Verhältnis des
Dialoges unmöglich. Für Dussel besteht die Radikalität eher darin, die Sphäre zu
verlassen, in der eine spezifische Form moderner Rationalität die
Legitimitätsgrundlage für Wissen darstellt. Er fordert die Regeln darüber zu
verändern, was als legitimes Wissen gilt bzw. wer darüber entscheiden kann.

Hierzu stellt Dussel drei Kriterien auf: So soll erstens Ort (\textit{Location}) von dem
aus gesprochen wird maßgebend für den Zugang zur Legitimität darstellen. Ideen,
die ungebunden ihres Entstehungskontextes postuliert werden, werden hiermit
disqualifiziert. Zweitens weist Dussel darauf hin, dass kritisches Denken gerade
in den marginalisierten Räumen gesucht werden müsse, und stellt in diesem
Zusammenhang sein Verständnis von Exteriorität vor. Zuletzt, so fordert er, muss
der Anspruch auf Autonomie, und damit die Möglichkeit sich frei für und gegen
Dialoge zu entscheiden, anerkannt werden.

Der Diaolog wird damit zur konstitutiven Praxis einer dezentrierten
Epistemologie, die es den Teilnehmer\_innen ermöglicht durch wechselseitige
Bezüge eigene Perspektiven neu zu denken ohne sich an nur einem, vermeintlich
universellen Maßstab zu messen.

\subsubsection{Verortung}

Die Theoretisierung von Kultur wird vor dem Hintergrund der machtvollen
epistemischen Hierarchien, die mit der Kolonialisierung einhergehen, zu einem
Projekt in dem die ökonomische Dimension mitgedacht und Kulturen immer auch als
Ausdruck eines kolonialen Verhältnisses verstanden werden, das von Dominanz und
Ausbeutung geprägt ist.\footnotemark \footnotetext{ Dussel, „Transmodernity and
Interculturality,“ 32.} Eine Analyse der Europäischen Moderne bzw. ihre
Gegengeschichtsschreibung muss entsprechend immer in ein Dialog eingebettet
sein, der Wissensbestände als spezifische Ausdrücke ihrer
historisch-geografischen Verortung in global-politischen Verhältnissen versteht.

Dussel verdeutlicht an dieser Stelle, dass sich das Denken von Zentrum und
Peripherie mit der gewaltvollen Eroberung durch die Kolonialmächte in das Denken
der Kolonialisierten eingeschrieben hat. Der Dialog, den er zwischen den
Theoretikern aus Asien, Afrika und Lateinamerika praktiziert, ist entsprechend
auch geprägt von neokolonialen Verbindungen. So unterscheidet Dussel zwischen
jenen, die sich mit den Eliten des Westens verbinden, deren Perspektiven
verinnerlichen und somit koloniale Denkmuster fortschreiben und jenen, die auf
ein ursprüngliches, vorkoloniales Außen rekurrieren. Hier wird das
anti-essentialistische Verständnis von Verortung deutlich. Eine Position im
postkolonialen Kontext\footnotemark \footnotetext{Mit postkolonialem Kontext sind an dieser Stelle jene Territorien gemeint, die unter kolonialer Herrschaft waren oder sind. In theoretischen Auseinandersetzungen mit Kolonialismus und Postkolonialismus wird jedoch darauf hingewiesen, dass insbesondere auch die Kolonialmächte und ihre Gesellschaften als postkoloniale Kontexte gedacht werden müssen, da sich das koloniale Projekt hier ebenso in das Selbstverständnis eingeschrieben hat.} ist demnach nicht per se von hegemonialen Diskursen
verschont, sondern reproduziert diese unter Umständen bewusst oder unterbewusst.

An dieser Stelle fragt sich Dussel, wie eine dezentrale, plurale
Erkenntnisproduktion aussehen kann, die sich nicht in einem puren Relativismus
verliert. Denn globale Strukturen der Ausbeutung und des Widerstandes, wie sie
durch den Kolonialismus entstanden sind, verlangen, so Dussel auch eine Analyse
und Kritik auf globalem Level. Wie kann nun vor dem Hintergrund dieser
Überlegungen ein pluraler Dialog entstehen, der nicht von universellen,
normativen Kriterien bestimmt ist, die eben gerade der Idee des Dialoges
entgegenstehen würden?

Hier stellt sich die Frage, wer am Diskurs auf welche Weise teilnehmen kann bzw.
gehört werden kann und wie Kriterien aufgestellt werden können, die einen
offenen und damit für alle zugänglichen Diskurs ermöglichen.

Die Forderung nach Verortung und damit nach kontextgebundener Wissensproduktion
darf hier  jedoch, so Alcoff in Anlehnung an Dussel nicht die Notwendigkeit
ersetzen, globale Zusammenhänge zu benennen und zu analysieren. Der Fetisch des
Lokalen, verkennt, dass das Lokale immer erst durch dominante Logiken hindurch
intelligibel wird:

\begin{myenv}
    \textit{„When we make cultures or knowledges irreducibly local, we truly
    risk ahistorical reifications. We risk losing the sight of how our
    represenstations of the local practices or knowledges my be constituted
    through imperial sign systems, or in other words, mistaking the local as a
    solipsistic spontaneous emergence, rather than implicated – at least in its
    representations and how it is understood- within a larger colonial
    semiosis.\footnotemark \footnotetext{ Alcoff, „Dussels Transmodernism,“ 65.}}
\end{myenv}

Das Lokale ist demnach immer bereits schon ein Konstrukt einer höheren, unter
Umständen hegemonialen Ordnung, in der das Lokale aufgeht. Zugleich verweist
Alcoff auf die Möglichkeit, dass es auch außerhalb der Ebene der Repräsentation
lokales Wissen bzw. lokale Kulturen gibt und verweist damit bereits auf die Idee
der Exteriorität.

Einem Relativismus, in dem das Lokale damit auf das Lokale reduziert wird, kann
nur mit einer 'provisorischen Metaerzählung der globalen Geschichte' begegnet
werden, die die wechselseitigen Abhängigkeiten und Bezugnahmen anerkennt und
damit auch Referenzpunkte und Rahmensetzungen für lokale Narrative bietet, die
ihre Einbettung in größere Zusammenhänge ermöglicht. Relationalität wird damit
als notwendiger Bestandteil von einer Idee der Vernunft konzipiert, und grenzt
sich von einem transzendenten Vernunftbegriff ab. Nur so kann die konstitutive
Bedeutung des Kolonialismus im Zusammenspiel mit und Aufeinanderwirken von
verschiedenen Akteur\_innen berücksichtigt werden ohne dabei Europa und die
Europäische Moderne als zentralen Bezugspunkt für die gesamte Geschichte in den
Vordergrund zu rücken. Alcoff weist an dieser Stelle darauf hin, dass ein
reflexives Bewusstsein einen Prozess beinhaltet, der prinzipiell jedem offen
steht, ganz egal von wo aus er\_sie spricht. Das gleiche gelte jedoch auch für
blinden Dogmatismus und vorsätzliche Ignoranz, von der niemand per se geschützt
sei.\footnotemark \footnotetext{Ebd.}

\subsubsection{Exteriorität}

Der Versuch Lateinamerika in globalen Zusammenhängen zu situieren wurde nach und
nach durch eine Kritik an der „standart vision“\footnotemark \footnotetext{Dussel, „Transmodernity and Interculturality,“ 36.} dieser Globalgeschichte
abgelöst. Nicht nur die Darstellung und damit immer auch Herstellung der
vermeintlichen Peripherie, sondern auch die Selbstverortung- und Beschreibung
Europas wird als verzerrt entlarvt (vgl. Solipsismus).\footnotemark
\footnotetext{Ebd.} Eine postmoderne Analyse
dieser Moderne kann darum, so argumentiert Dussel, nicht aus diesem
vermeintlichen Zentrum geschehen, sondern muss aus der Exteriorität heraus
praktiziert werden. Entgegen der vielfachen Annahmen, dass die durch
industrielle Revolution, koloniale Eroberung und hegemonialen Stellung Europas
im global- politischen Kontext, sich auch dessen Kultur in imperialer Praxis in alle Sphären einschreiben konnte, stellt Dussel die These der Exteriorität:
\begin{myenv} 
    \textit{„These cultures have been partly colonized, but most of the
    structure of their values has been excluded—disdained, negated and ignored—
    rather than annihilated. The economic and political system has been
    dominated in order to exert colonial power and to accumulate massive riches,
    but those cultures were deemed to be unworthy, insignificant, unimportant,
    and useless. This disdain, however, has allowed them to survive in silence,
in the shadows, simultaneously scorned by their own modernized and westernized
elites.“\footnotemark \footnotetext{Ebd., 42.}} \end{myenv}

Für Dussel ist durch das Selbst- und Weltverständnis der Europäischen Moderne,
das alle anderen kulturellen Praxen und Ausdrucksformen als Minderwertig
begreift, eine Situation entstanden in der es manchen 'Kulturen' gelingen konnte
sich subversiv und im Schatten zu behaupten. Dussel sieht in dieser Exteriorität
das Potential, auf Basis einer völlig unvergleichlichen Erfahrung,
Lösungsvorschläge zu entwickeln, die für die Europäische Moderne niemals
erreicht werden können. Eine transmoderne Kultur sieht entsprechend auch vor,
die Weltgeschichte neu zu schreiben und dabei die Perspektiven und kulturellen
Entwicklungen der Exteriorität zu berücksichtigen:

\begin{myenv} \textit{„'Trans-modernity' points towards all of those aspects
    that are situated 'beyond' (and also 'prior to') the structures valorized by
    modern European/North American culture, and which are present in the great
    non-European cultures and have begun to move toward a pluriversal
utopia.“\footnotemark \footnotetext{Ebd., 43.}} \end{myenv}

Statt von einer in sich isolierten Exteriorität auszugehen, plädiert Dussel für
die Zwischenräume, die Grenzsubjekte, die zwischen dem hegemonialen Zentrum und
der Exteriorität übersetzen können und sich jeweils das Beste, was diese zu
bieten haben für sich nutzen lernen. Kritisches Denken entspringt demnach in dem
Vertrauen, sich auf die eigene Kultur berufen zu können und der gleichzeitigen
Offenheit, die intellektuellen Errungenschaften anderer Kulturen, die für das
eigene Vorhaben Nützlichkeit versprechen, sehen und anwenden zu
können.\footnotemark \footnotetext{Ebd., 47.} Kultur verstehe ich hierbei immer
in dem zuvor dargelegten Verständnis von Hybridität, und nicht als monolithisch.
Die Kontingenz die in kulturellen Selbstverständnissen enthalten ist, wird mit
dem Konzept des Dialoges aufgegriffen, da hier eine stetige Veränderung des
Selbst- und Weltverhältisses zur Praxis wird.

Dussel weist an dieser Stelle darauf hin, dass sein Verständnis von Dialog die
Überschreitung der eigenen Perspektive zum Ziel hat, ohne jedoch darin die
Bedeutung der eigenen Wertigkeit zu verkennen. Viel eher gilt letzteres als
Voraussetzung um sich mit dem eigenen und dem anderen kritisch
auseinanderzusetzen, ohne dabei zu schnellen Schlussfolgerungen zu kommen, die
eines von beiden obsolet erscheinen lassen:
\begin{myenv}
    \textit{„It is not a dialogue among those who merely defend their culture
    from its enemies, but rather among those who recreate it, departing from the
    critical assumptions found in their own cultural tradition and in that of
    globalizing Modernity.“\footnotemark \footnotetext{Ebd., 48.}}
\end{myenv}

\subsubsection{Autonomie}

Inwiefern Dussel einen Dialog zwischen hegemonialem Zentrum und der Exteriorität
für möglich hält, bzw. ob er diesen überhaupt befürwortet kann ich seinen
folgenden Ausführungen nicht entnehmen. Zum einen schildert er die
Notwendigkeit, das asymmetrische Verhältnis das zwischen 'dem Westen' und 'dem
Außen' liegt, zu benennen. Eine Asymmetrie, die jedoch aus einer radikalen
eurozentristischen Perspektive nicht erkannt werden kann, weil die Anerkennung
der zu Anderen gemachten, als Subjekte, nicht gegeben ist. In ihrer
eurozentristischen Sprache,  „[...] there can be no cultural dialogue with
China, India, the Islamic world, Mexico, etc., because they are neither
enlightened nor primitive cultures. They are 'no man's land'.”\footnotemark
\footnotetext{Ebd., 40.} Zum anderen argumentiert er, dass selbst wenn die
Asymmetrie anerkannt wird und damit auch einer Anerkennung der Anderen als
Subjekte möglich wird , dies nicht bedeutet, dass die Strukturen überwunden
werden können, die dieser Asymmetrie unterliegen. Solange die Bedingungen des
Dialoges vom hegemonialen Zentrum aus gesetzt werden, wird, so Dussel sein
Ausgang immer auch jenen dienen, die ihre ökonomischen Interessen unter dem
Vorwand kultureller Verständigung durchsetzen wollen.\footnotemark
\footnotetext{Ebd.} Letztlich scheint es, dass Dussel Verständnis von Dialog die
Sprecher\_innen des Westens erst einmal nicht umfasst. Viel eher geht es ihm
darum, zunächst einen Dialog innerhalb der Exteriorität bzw. ihrer Grenzen zu
schaffen, um im Anschluss hieraus einen Dialog mit der Europäischen Moderne
einzugehen:



