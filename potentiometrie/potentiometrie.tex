\documentclass{article}
\usepackage{amsmath}


%----------------------------
\usepackage{microtype}
%\usepackage{fourier}
%\usepackage{mathpazo} % add possibly `sc` and `osf` options

%\usepackage[default,osfigures,scale=0.85]{opensans} %% Alternatively
%% use the option 'defaultsans' instead of 'default' to replace the
%% sans serif font only.

%\usepackage[default]{gillius}
\usepackage{listings}
\usepackage[utf8]{inputenc}
\usepackage{tgheros}
\renewcommand*\familydefault{\sfdefault} %% Only if the base font of the document is to be sans serif

\usepackage[T1]{fontenc}
\usepackage[german]{babel}

\usepackage{mathastext}




%units
%--------------------------------------
\usepackage{siunitx}
%drawing
%------------------------------------
\usepackage{tikz} % To generate the plot from csv
\usepackage{pgfplots}
\usepackage{pgfplotstable}
\usepackage{graphicx}

\usepackage{caption}
\usepackage{color, colortbl}
\usepackage{tabularx}
\usepackage{setspace}
\usepackage{mhchem}
\usepackage{fancyhdr}
\usepackage[landscape]{geometry}

\pagestyle{fancy}

\cfoot{\thepage}

\rhead{\footnotesize Gruppe 24}
\lhead{\footnotesize Potentiometrische Titration - Auswertung}
\setlength{\headheight}{15pt}


\renewcommand{\familydefault}{\sfdefault}
\captionsetup{font=footnotesize}

\definecolor{LightCyan}{rgb}{0.88,1,1}
\usetikzlibrary{datavisualization}
\pgfplotsset{compat=newest} % Allows to place the legend below plot
\usepgfplotslibrary{units} % Allows to enter the units nicely
\usepackage{overcite}
\renewcommand\citeform[1]{[#1]}

\sisetup{
  round-mode          = places,
  round-precision     = 2,
  inter-unit-product =\ensuremath{{}\cdot{}}
}
\makeatletter
\setlength{\@fptop}{0pt}
\makeatother
%\usetikzlibrary{arrows, positioning, calc, datavisualization}

\begin{document}
%\begin{luacode*}
function string:split(sep)
        local sep, fields = sep or "%s", {}
        local pattern = string.format("([^%s]+)", sep)
        self:gsub(pattern, function(c) fields[#fields+1] = c end)
        return fields
end

function printHyperbola()
    local lines={}

    for line in io.lines("dampf.csv") do
            table.insert(lines, line)
    end
    tex.sprint("\\addplot[color=black] coordinates{")

    for i=2,#lines do
        local a=lines[i]:split()
        tex.sprint("("..a[1]..","..a[2]..")")
    end
     tex.sprint("};")
    tex.sprint("\\addplot[color=blue] coordinates{")

    for i=2,#lines do
        local a=lines[i]:split()
        tex.sprint("("..a[1]..","..a[3]..")")
    end
     tex.sprint("};")

end
\end{luacode*}
%\begin{figure}
\centering
     \begin{tikzpicture}
            \begin{axis}[standard,xlabel=Temperatur,ylabel=Druck]
                \directlua{printHyperbola()}
            \end{axis}
        \end{tikzpicture}
     \end{figure}

\noindent
\pagenumbering{gobble}

\begin{figure}[!t]
\caption{ $\Delta E$ vs \ce{V_{AgNO3}} Titrationskurve der Joghurtprobe mit 0.1 M \ce{AgNO3} bei T = 20 \si{\celsius}}

\begin{tikzpicture}[scale=1.5,transform shape]
\begin{axis}[
axis x line=bottom,
axis y line=middle,
legend style={draw=none,legend pos=outer north east},
xlabel= \ce{V_{\ce{AgNO3}}} / \si{\milli\liter},
ylabel= $E_v$ / \si{\volt},
y label style={at={(axis description cs:-0.1,.5)},rotate=90,anchor=south}]
\addplot[black,thick,smooth] file[skip first]{ag.txt};
\addlegendentry{\scriptsize $\Delta E = E_{IE}-E_{RE}$ }
\addplot[blue,thick] file[skip first] {agd.txt};
\addlegendentry{\scriptsize $\frac{(\partial \Delta E)}{\partial V_{\ce{AgNO3}}}$ }
\addplot[restrict y to domain*=-15:15] file[skip first] {agdd.txt };
\addlegendentry{\scriptsize $\frac{(\partial ^2 \Delta E)}{\partial (V_{\ce{AgNO3}})^2}$ }
\node[coordinate,pin=right:{\scriptsize ÄP bei $V_{\ce{AgNO3}}$ = \SI{1.85}{\milli\liter}}]
at (axis cs:1.87,-142) {};
\end{axis}
\end{tikzpicture}

\end{figure}

\tikzset{every mark/.append style={scale=0.5}}


\begin{figure}[!t]

\caption{ pH vs $\ce{V_{NaOH}}$ Titrationskurve einer 0.1 M \ce{H2S04}-Lösung mit 0.1 M \ce{NaOH} in 0.1 \si{\milli\liter} Schritten}
\begin{tikzpicture}[scale=1.5,transform shape]
\begin{axis}[
axis equal=false,
axis x line=bottom,
axis y line=middle,smooth,
legend style={draw=none,legend pos=outer north east},
xlabel= \ce{V_{NaOH}} / \si{\milli\liter},
ylabel= pH,
y label style={at={(axis description cs:-0.1,.5)},rotate=90,anchor=south},
xmin=0.000, xmax=5, ymin=0, ymax=14,
ytick={0,...,14}
]
\addplot[mark=triangle, thin] file[skip first]{naoht.txt};
\addlegendentry{\scriptsize Theoretische Werte}
\addplot[mark=o, thick] file[skip first] {naohp.txt};
\addlegendentry{\scriptsize $\,$Experimentelle Werte}
\addplot[thick,dashed] file[skip first] {naohpd.txt};
\addlegendentry{\scriptsize $\frac{\partial pH}{\partial V}$ }
\addplot[thin,dashed] file[skip first] {naohpdd.txt };
\addlegendentry{\scriptsize $\partial \frac{(\partial pH)}{(\partial V)^2}$}
\node[coordinate,pin=right:{\scriptsize ÄP bei pH =7}] at (axis cs:2.75,7) {};
\end{axis}

\end{tikzpicture}

\end{figure}


\begin{figure}[!t]
\caption{ pH vs $\ce{V_{NaOH}}$ Titrationskurve einer 0.0333 M \ce{H3PO4}-Lösung mit 0.1 M \ce{NaOH} in 0.1 \si{\milli\liter} Schritten}

\begin{tikzpicture}[scale=1.5,transform shape]
\begin{axis}[
axis equal=false,
axis x line=bottom,smooth,
axis y line=middle,
legend style={draw=none,legend pos=outer north east},
xlabel= \ce{V_{\ce{NaOH}}} / \si{\milli\liter},
ylabel= pH,
y label style={at={(axis description cs:-0.1,.5)},rotate=90,anchor=south},
xmin=0.000, xmax=8, ymin=0, ymax=14
]
\addplot file[skip first] {phosphor.txt};
\addlegendentry{\scriptsize Messwerte}
\addplot  file[skip first] {phosphord.txt};
\addlegendentry{\scriptsize $\frac{\partial pH}{\partial V}$ }
\addplot[mark=star]  file[skip first] {phosphordd.txt};
\addlegendentry{\scriptsize $\partial \frac{(\partial pH)}{(\partial V)^2}$}

\node[coordinate,pin=right:{\tiny ÄP bei pH=4.45}]
at (axis cs:1.45, 4.45) {};
\node[coordinate,pin=right:{\tiny ÄP bei pH=9.04}]
at (axis cs:3.85, 9.04) {};
\node (A) at (axis cs:16,0) {};
\node (B) at (axis cs:16,5.70) {};
\node (C) at (axis cs:40,0) {};
\node (D) at (axis cs:40,9.83) {};
\draw[thick] (A) -- (B) node [midway] {};
\draw[thick] (C) -- (D) node [midway] {};

\end{axis}
\end{tikzpicture}
\end{figure}


\end{document}


%\addplot[domain=0:1200]{-1.42e-5*x^(2)+2.836e-2*x+45.47};


