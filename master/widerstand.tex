\subsection{Widerständiges Wissen}
\epigraph{\textit{ 
Jede (Identitäts-) Politik wird von einem Begehren, von einer besetzen
Orientierung, dem Schatten- und Wunschbild eines Anderen, der destillierten
Erfahrung einer aufgeschobenen Erfahrung getragen, die Gegenwarten erträglich
macht und transformative Handlungen weg von der Gegenwart mobilisiert, und
zugleich muss die Politik sich selbst als verhaltenes und gedrosseltes Begehren
artikulieren.}}{Paul Mecheril\footnotemark} \footnotetext{Mecheril, \textit{Politik
der Unreinheit. Ein Essay über die Hybridität} (Wien: Passagen-Verlag, 2003), 14.} 

Die Perspektive, mit der Grosfoguel, Dussel und Mignolo euro- und
androzentrische Wissensproduktionen untersuchen, beschreiben sie selbst als
dekolonial.\\
Ein zentraler Aspekt dekolonialer Intervention ist dabei die
Überlegung \glqq dass auch die Erkenntnis ein Instrument der Kolonialisierung war
und dass die Dekolonisierung daher Wissen und Sein, dh. Subjektivität
impliziert.\grqq\footnotemark\footnotetext{Mignolo, \textit{Epistemischer
Ungehorsam}, 47.}\\
Indem Erkenntnis hier als subjektivierender Prozess verstanden wird, der auch
die kolonisierten Subjekte betrifft, verdeutlicht Mignolo, dass die Erfahrung
der Kolonisierung sich nicht nur in die kolonisierenden, sondern auch in die
kolonisierten Subjekte einschreibt und ihre Subjektverständnisse bestimmt.
Erkenntnis wird damit in der dekolonialen Perspektive auch als Ausdruck einer
Unterwerfung unter bestehenden Wissensregimen darstellt und ergo als Mittel
der Fügsamkeit in kolonialen Verhältnisse entlarvt.\\

\noindent Die eingeführte Kontextualisierung, die Wissensproduktionen aus einer
dekolonialen Perspektive situiert, macht an dieser Stelle deutlich, dass der
auf Emanzipation bzw. Befreiung zielende Bildungs- und Kritikbegriff, der die
Idee von Universität bestimmt, sein Versprechen unter Umständen nur schwer
einlösen kann. Denn für diejenigen, die sich in ihren Erkenntnissen nur auf
ein System berufen dürfen, das sie unterdrückt, bleibt jede Hoffnung auf eine
Befreiung von eben jenem System vergebens. \\

\noindent Das oben stehende Zitat von Mecheril
wirft jedoch noch einen anderen Blick auf Möglichkeiten, sich zu den
Verhältnissen zu verorten. Identitätspolitiken, greifen Sprache als
Wissenssysteme gleichzeitig auf und an. Die Erkenntnis, dass Erfahrungen niemals
authentisch sondern immer schon destilliert, also bereits durch ein System
gewandert sind, das die Erfahrungen spaltet um aus ihr das gewünschte zu
extrahieren, bedeutet nicht, dass sie nicht für Überlebensstrategien genutzt
werden können. Es bedeutet hingegen, dass die Prozesse der Destillation in den Blick
genommem werden müssen um ihre Logik zu hinterfragen.\\

\noindent Postkolonialer Feminismus knüpft hieran an. Der Hinweis, Erkenntnis als
Ergebnis kolonialer Einschreibung in Wissen und Sein ernst zu nehmen, führt
hier nicht dazu, sich von Erkenntnispolitik abzuwenden. Wie bereits mit Castro
Varela und Dhawan erläutert, steckt im Postkolonialer Feminismus trotz dieser
komplexen Ausgangslage der Anspruch nach Wegen zu suchen wie die
Wissensproduktion auf andere, umfassendere und damit weniger auschließende
Subjekt- und Erkenntnistheorien aufgebaut sein kann. \\
Mein Interesse liegt nun
darin, jene Komplexität der erkenntnistheoretischen Auseinandersetzung um die
Bedeutung von Erfahrungen für die Theoriegenerierung Schritt für Schritt zu
entfalten und dabei postkolonial\_feministische Perspektiven und Widerstreits in
den Blick nehmen. Im nächsten Schritt folgende Perspektiven habe ich jedoch
nicht aus dem emanzipatorisch funkelnden Teich des Postkolonialen Feminismus
herausgefischt. Wie so oft verhält es sich mit Etiketten und Dingen nicht ganz
so widerspruchsfrei.\\

\noindent Das Etikett Postkolonialer Feminismus hat in erster Linie \textit{mich} angesprochen, da
es gegenwärtige Verhältnisse postkolonial situiert und theoretische Positionen,
die sich mit feministischer Kritik auseinandersetzen in diesen Verhältnissen
verortet. \\
Die Texte auf die ich mich im folgenden beziehe habe ich jedoch nach
einem anderen Kriterium ausgewählt, da es den glitzernden Teich in der Form
bekanntlich nicht gibt - er wird viel eher durch Arbeiten wie diese immer
wieder neu hergestellt.\\
 Das Kriterium, das für mich in der Auswahl relevant
war, ist die Thematisierung von Erfahrung als Modi und Gegenstand von
feministischer Kritik. Eine postkoloniale Situierung findet dabei nur
sporadisch statt. Die Verzerrungen und Ausblendungen, die dies mit sich bringt
aufzuspüren und zu fragen, inwiefern jene Denkansätze für eine dekoloniale
Strategie fruchtbar sein können, sind darum Fragen die ich im Folgenden stellen
werde.

\subsubsection{Das Epistemische Privileg, Differenz und Erfahrung}

Ausgehend von der marxistischen These, nach der die materiellen Umstände in
denen ein Mensch sich befindet sein Bewusstsein bestimmen, entwickelt Nancy
Hartsock eine feministische Standpunktepistemologie. Während bei Marx, wie eben
beschrieben, das sogenannte Arbeiterbewusstsein durch die Erfahrungen des
weißen Arbeiters im Produktionsprozess entsteht, muss, so Hartsock, für eine
Analyse des Patriarchats der Kontext der Produktion um Reproduktion erweitert
werden. So würden all jene Tätigkeiten miteinbezogen werden, die außerhalb der
Lohnarbeit für letztere die Basis schaffen und in der Regel von Frauen\_*
ausgeübt werden. Die spezifische Position, die Frauen\_* innerhalb der
Reproduktionsarbeit einnehmen, gehe mit einer ebenso spezifischen Erfahrung der
patriarchalen Strukturen einher. Erfahrungen, die Frauen\_* dazu befähigen, eben
jene Strukturen zu erkennen und zu analysieren, welche sie unterdrücken. Erst
die tatsächliche Erfahrung reproduktiver Arbeit und das Bewusstsein, das
Hartsock zufolge hierbei entsteht, ermöglichen demnach eine fundierte,
situierte Erkenntnisproduktion die patriarchale Strukturen erkennen und in
Frage stellen kann.\footnotemark \footnotetext{ Nancy Hartsock, \glqq The
Feminist Standpoint: Developing the Ground for a Specifically Feminist
Historical Materialism \grqq , zitiert in \textit{Waldraut Ernst, Diskurspiratinnen.
Wie feministische Erkenntnisprozesse die Wirklichkeit verändern}. (Wien: Milena
Verlag, 1999), 67.}\\ 

Ähnlich wie Hartsock geht auch Patricia Hill Collings von einem weiblichen
Bewusstsein aus, erweitert diese jedoch noch um das spezifische Bewusstsein
Schwarzer Frauen\_*. Während bei Hartsock die Positionierung von Frauen\_* in
rassistischen Verhältnissen unberücksichtigt bleibt, beansprucht Hill Collins
eine intersektionale Analyse. Dafür befasst sie sich mit der Entstehung und
Bedeutung des Black Feminist Thought, eben auch für epistemologische Fragen und
setzt sie sich explizit mit der Bedeutung der Erfahrung Schwarzer Frauen\_* für
die Wissensproduktion auseinander. Sie entwickelt hierbei das Konzept von
Weisheit das sie unmittelbar an spezifische Erfahrungen von multipler
Unterdrückung knüpft.\footnotemark \footnotetext{Patricia Hill Collins,
  \textit{Black Feminist Thought. Knowledge, Consciousness and the Politics of
Empowerment}, (New York:Routledge, 2000).}\\ 

Für hooks stellte die Anerkennung von Erfahrung, insbesondere in ihren
universitären Anfängen, eine wichtigen Ansatz dar, um die Differenz, die sie
als Schwarze Frau\_* z.B. in weiß dominierten Auseinandersetzungen mit
Feminismus erfuhr, ernst zu nehmen. Gerade weil es zu diesem Zeitpunkt anders
als heute keinen breiten Fundus an wissenschaftlichen Beiträgen über
rassistische Ausschlüsse in der weiß dominierten feministischen Theoriebildung
gab, also weil ihre Erfahrung noch nicht analytisch und wissenschaftlich
Ausdruck fanden, war hooks darauf angewiesen, ihre Erfahrungen von
Nicht-Zugehörigkeit nicht abzutun sondern als Wissensquelle ernst zu
nehmen.\footnotemark \footnotetext{hooks, \textit{Teaching to Transgress}, 90.}
Der Einzug von Critical Race Studies oder Gender Studies in die Universitäten,
hat Räume für diese von hooks beschriebenen Erfahrungen geschaffen und damit
die Frage, wie und von wem Wissen produziert wird, neu zur Disposition
gestellt.\footnotemark \footnotetext{Mohanty, \glqq On Race and Voice,\grqq
185.} \\

Mohantys Forderung nach universitären Räumen, (vgl. Kapitel: Universität)
in denen marginalisierte Erfahrungen artikuliert werden können, liest sich
gewissermaßen als praktische Konsequenz aus oben formulierten Standpunkten.
Diese Räume zeichnen sich durch den Anspruch einer analytischen Offenheit aus,
die die Gleichzeitigkeit verschiedener Machtverhältnisse in ihrer komplexen
Verwobenheit artikulierbar machen. Die Kategorie Frau\_* wird damit zu einem
Ort, der nicht nur Geschlechterverhältnisse, sondern gleichermaßen Class und
Race als Strukturkategorien mit einbezieht. Räume in denen dies möglich wird
sind, so Mohanty, jedoch stets umkämpft, da sie sich permanent gegen eine
Vereinnahmung durch apolitische und ahistorische Diversitätsdiskurse zur Wehr
setzen müssen. Denn Differenz, so führt sie aus, ist nur dann produktiv wenn
die hegemonialen Vorstellungen darüber, wer was sein kann, irritiert werden,
anstatt sich in komplementärer Manier identitären Kategorien
anzuschließen.\footnotemark \footnotetext{Ebd., 184.}\\ 

Anders als Hartsock und Collins geht Mohanty jedoch nicht davon aus, dass
Frauen\_* allein durch ihre Position innerhalb des Geschlechterverhältnisses
bereits eine kritische und entsprechend erkenntnistheoretisch priviligierte
Sichtweise einnehmen. Feminismus ist ihrer Ansicht nach mehr als ein
natürlicher Effekt der mit dem Frau\_* sein einhergeht. Hier beruft sie sich auf
King nach der Lesbe zu sein nicht davor schützt heteronormative Strukturen und
Wissensbestände aufrecht zu erhalten. \\
Eine kritische politische Haltung ergibt
sich ihnen nach nicht auf Grund einer gesellschaftlich marginalisierten
Position oder Praxis allein, sondern nur durch eine bewusste Reflexion
innerhalb derer die Mechanismen, die zur eigenen Marginalisierung führen
erkannt und bewertet werden.\footnotemark \footnotetext{Mohanty, \glqq Feminist
Encounters. Locating the Politics of Experience, \grqq 177.}\\

Auch Bat-Ami Bar On kritisiert die Annahme eines epistemischen Privilegs, das
marginalisierten Gruppen per se zukommt. Dies würde die prekären Verhältnisse,
die an den Rändern der Gesellschaft herrschen, romantisieren und dabei
verkennen, dass der Rand sich nicht außerhalb der dominanten Diskurse befindet.
Unterdrückt zu sein bedeute nicht, von den Denkweisen befreit zu sein, die eben
jene Unterdrückung ermöglichen.\footnotemark \footnotetext{Bat-Ami Bar On,
 \textit{Marginality and Epistemic Privilege}. Zitiert in \textit{Ernst, Waldraut:
Diskurspiratinnen. Wie feministische Erkenntnisprozesse die Wirklichkeit
verändern}. (Wien: Milena Verlag, 1999), 72.}\\

Von der Notwendigkeit einer universellen, weiblichen Unterdrückungserfahrung
für feministische Kritik auszugehen, wird auch von Waltraud Ernst
problematisiert, weil sie diejenige Struktur, die es zu überwinden gilt, als
Bedingung voraussetzt. Das Zweigeschlechtermodell werde reproduziert, da
Frauen\_* nur qua ihrer untergeordneten Rolle den Status der Kritiker\_innen
genießen dürfen. \\
Statt die Erfahrung von Unterdrückung in den Vordergrund zu
stellen, müssten, so Ernst, viel eher Momente des Widerstandes und der
unerwarteten Ausbrüche aus als weiblich erklärten Rollenverständnissen in den
Mittelpunkt der Analyse rücken. Erkenntnis dürfe nicht an die Position gebunden
sein, die das Patriarchat den Frauen\_* zuschreibt, sondern entstehe in den
Momenten der Befreiung.\footnotemark \footnotetext{Ernst,
\textit{Diskurspiratinnen}, 73.}\\

\noindent Dekolonisierung verlangt hier eine
Aufmerksamkeit für die asymmetrischen Machtverhältnissen, in die Sprechpraxen
an der Universität eingebunden bzw. durch sie wirksam werden. Theorien, die auf
einer abstrakteren Ebene funktionieren als es persönliche Erfahrungen tun,
werden dadurch, das betont Mohanty, keineswegs obsolet. Sie werden vielmehr in
einen Dialog mit konkreten Alltagserfahrungen gesetzt und dadurch ergänzt oder
auch in Frage gestellt.\footnotemark \footnotetext{Mohanty, \glqq On Race and
Voice \grqq, 184.} Die Universität muss entsprechend als ein Ort verstanden
werden, an dem einander widerstreitende Positionen und daraus folgende
Interessen aufeinander treffen und Mohanty plädiert dafür, dass diese Kämpfe
gefochten werden und nicht in einem leeren Pluralismus versiegen.\\

\noindent Erfahrung in Wissensproduktion wurde in bisherigen Beiträgen stets als
kollektive Erfahrung konzipiert - ihr wird ein Ort zugewiesen, und zwar in den
meisten Fällen, ein Ort in Geschlechterverhältnissen. Dass diese
Geschlechterverhältnisse auch rassifiziert sind, wird zumeist ignoriert und
damit auch die Sensibilität für die Differenzen zwischen Frauen\_*. Die
Berücksichtigung differenter Erfahrungen wird nur aus Schwarzen und of Color
Perspektiven thematisiert, wie Collins, Mohanty oder Hooks zeigen.\\
Umstritten ist auch welche Erfahrung eigentlich emanzipatorisches Potential
hat. Hierbei spielen Unterdrückungserfahrungen ebenso eine Rolle wie Erfahrung
von Widerstand und Ausbruch aus vorgeschriebenen Wegen. Insgesamt wird jedoch
deutlich, dass Erfahrung einen enorm wichtigen Bestandteil in dem Versuch
bildet, sowohl die Leerstelle, die von hegemonialer Philosophie ausgeht zu
benennen, also auch, um aus dieser Leerstelle heraus eine
erkenntnistheoretische Position zu beziehen und zu legitimieren.\\

\noindent Im letzten Kapitel habe ich mit Dussel und Grosfoguel aufgezeigt, dass die
Erfahrung von Herrschaftsausübung den europäischen weißen männlichen
Philosophen eine Selbstverständnis ermöglichte das ihre Methodologie bestimmt.
Es wurde deutlich dass in der emanzipatorischen Epistemologie, diese Erfahrung
von Ausschluss als Ausgangspunkt gewählt wird, um die eigene Stimme bzw. das
Interesse zu markieren. \\
Der Unterschied ist nun, dass sich in der
postkolonial\_feministischen Epistemologie auf diesen marginalen Standpunkt
explizit bezogen wird und entsprechend sich auch ihre Legitimation hieraus
ableitet. In der hegemonialen Epistemologie hingegen wird der Anspruch zwar
auch aus der Erfahrung heraus entwickelt, wie mit Dussel und Grosfoguel
aufgezeigt, diese Erfahrung wird jedoch nicht reflektiert und entscheidend:
nicht benannt.\\
           
\noindent Bisher habe ich mich mit Perspektiven beschäftigt, in denen Erfahrung als
Standpunkt theoretisiert und problematisiert wird. Eine Beobachtung, die
Canning an anderer Stelle\footnotemark \footnotetext{Kathleen Canning,
 \glqq Problematische Dichotomien. Erfahrung zwischen Narrativität und
 Materialität, \grqq
in Erfahrung: Alles nur Diskurs? Zur Verwendung des Erfahrungsbegriffes in der
Geschlechtergeschichte, herausgegeben von Marguérite Bos, Bettina Vincenz und
Tanja Wirz, (Zürich: Chronos Verlag, 2004), 38.} macht, lässt sich dabei auch
auf die bisher dargestellten Positionen übertragen.\\
 
 Die Thematisierung von
Erfahrung, in Erkenntnisproduktion, so Canning, beschränkt sich oftmals auf
die Erfahrungen des forschenden Subjekts, das auf diese Weise seinen Zugang zum
Forschungsthema transparent macht. Was hier jedoch unberücksichtigt bleibt, und
auch in meiner bisherigen Analyse kaum Ausdruck fand, ist die Erfahrung des
historischen Subjekts. Canning führt damit die Erfahrung der\_des Anderen ein.
Ob er\_sie vergangen oder anwesend ist spielt für mich hierbei eine zweitrangige
Rolle. Für mich ist die Frage interessant, wie sich ein reflexiver Umgang in
dem Wissen gestalten kann, dass die Erkenntnisse der beforschten Subjekte aus
einem spezifischen Standpunkt bzw. einer konkreten Erfahrung heraus entstehen
und nun von einem anderen Standpunkt heraus analysiert bzw. nutzbar gemacht
werden.\\

Joan Scott hat sich sehr explizit mit Erfahrung als Gegenstand von
Wissensgenerierung auseinandergesetzt. Als Historikerin merkt sie an, dass
Sehen, bzw. die Sichtbarkeit von etwas vielfach als einziger oder zumindest
vielversprechendster Zugang zu Erkenntnis gehandelt wird. Im Sehen verberge
sich damit der eigentliche Ursprung des Wissens, wohingegen die Aufgabe des
Schreibens lediglich als Ausführung, oder Reproduktion dieser Erkenntnis
verstanden werde. Diese Hierarchisierung kritisiert Scott in ihrem Essay
\glqq Experience\grqq \footnotemark \footnotetext{Joan Scott: \glqq Experience
 \grqq in
\textit{Feminists Theorize the Political}, herausgegeben von Judith Butler und Joan Scott. ( New York: Routledge, 1992).}, auf das ich im folgenden näher eingehen werde.\\
Die Praxis des Sehens beschreibt Scott als Form der unmittelbaren Erfahrung.
Insbesondere in den Geschichtswissenschaften habe die Sichtbarmachung von
Erfahrungen (anderer) zu einer willkommenen \glqq Pluralisierung von
Geschichten\grqq \footnotemark \footnotetext{Scott, \glqq Experience \grqq,
24.} geführt. Forschungen, die auf Erfahrung (anderer) basieren, würden somit
nicht nur zu einer Diversifizierung dominanter Narrative führen, sondern diese
auch um bis dato unberücksichtigte Subjektpositionen erweitern.\footnotemark \footnotetext{Ebd.} So werde mit
der Dezentrierung des hegemonialen Narratives durch die Inanspruchnahme von
Erfahrungswissen, zugleich der weiße Mann\_* als einzig möglicher Ursprung von
Erkenntnis in Frage gestellt. \\
Nun gilt es aber angesichts des emanzipatorischen Potentials, das
erfahrungsbasierte Forschungen folglich beanspruchen, vorsichtig zu sein. Auf
die Geschichtswissenschaft bezogen merkt Scott an, dass der Hype um Erfahrung
einer notwendigen und zwar fundamentalen Kritik am geschichtswissenschaftlichen
Methodenverständnis im Wege steht: \\
Die Hinwendung zu den Erfahrungen von bis
dato als unwesentlich und unsichtbar gehandelten Subjekten diene der
Geschichtswissenschaft nämlich lediglich als willkommene Ergänzung bereits
existierender Erkenntnisinstrumente. Die Legitimität dieser Erkenntnisquelle
werde dabei über die \glqq Autorität der Erfahrung\grqq \footnotemark
\footnotetext{Ebd.} behauptet. Die Erfahrung als \glqq authentisch, wahrer
Bericht dessen, was ein Mensch erlebt hat, dient entsprechend als Ursprung für
Erklärungen.\grqq \footnotemark \footnotetext{Ebd.}\\

\noindent Scott kritisiert diese Handhabung von Erfahrungswissen auf zwei Ebenen.\\
 Zum
einen hinterfragt sie den erkenntnistheoretischen Wert, der in jenen
Erfahrungsberichten liegt. Zum anderen kritisiert sie die Fokussierung auf die
sogenannte Identität der Betroffenen bzw. Ermächtigten und stellt dem ihr
Interesse an der Herstellung von Identität entgegen.\\
Scotts Problematisierung
von Erfahrungswissen zeigt sich in ihrem Verständnis von Erfahrung als
Narrativ.\\
 Erfahrung, und dies wird in späteren Ausführungen deutlicher, ist für
Scott in erster Linie das, was als Erfahrungswissen Eingang in den Diskurs
erhält, also eine Erzählung. Erfahrungs-Erzählungen dürfen, so argumentiert sie
nun, nicht als Beweis für oder gegen eine theoretische Position gedeutet
werden. Stattdessen müsse die Logik, nach der Erkenntnis erst durch eine vorher
festgelegte Beweisführung legitimiert werden muss, in Frage gestellt werden.
Die komplexe Verwobenheit und Abhängigkeit, die in dem Verhältnis von Narrativ
und Beweis liegt, dürfe nicht umgangen werden sondern müsse eigentlicher
Gegenstand der Kritik werden. An diese grundsätzliche Kritik am
Legitimationszwang von Erkenntnis durch Beweise, schließt sie die die
Problematik der identitären Aufladung von Erfahrung an.\\

\noindent Die Identität jener, die ihre Erfahrungen teilen, werden, so kritisiert Scott,
als gegeben vorausgesetzt und ihre Erzählungen als Intervention in das
dominante Narrativ gewertet. Dies führe zu einer Platzierung der Subjekte
außerhalb von Diskursen und einer damit einhergehenden Individualisierung ihrer
Erfahrung. Scott bezieht sich hier vermutlich auf die Versuche, jenen, die z.B.
von Missbeständen betroffen sind, und entsprechend als Betroffene über jene
Missstände berichten, Gehör zu verschaffen. \\
Auch wenn von einer allgemeinen,
also keiner individuellen Erfahrung ausgegangen wird, und Erfahrung damit als
kollektives identitätsstiftendes Moment verstanden wird, bleibt der Fokus, so
verstehe ich Scott, individualisiert, weil eben jene Prozesse des 'kollektiv
werdens' nicht untersucht werden.\\
Problematisch für Scott ist also nicht, dass
Betroffen aus ihrer Perspektive heraus ihre Erfahrungen teilen, sondern
problematisch ist, dass sie als Betroffene wahrgenommen und ihre Erfahrungen
als Betroffenheitserfahrungen verstanden werden, ohne den diskursiven Rahmen,
der diese Artikulation bestimmt näher zu betrachten und unter Umständen zu
kritisieren. Wenn nun aber davon ausgegangen wird, das Subjekte nicht zufällig
auf eine bestimmte oder vermeintliche Erfahrung reduziert werden, bzw. dass
jene Zuschreibung bereits Teil der Erfahrungen und ihre
Artikulationsmöglichkeit ist, werden andere Fragen interessant: 
\begin{myenv}
 \textit{\glqq Questions about the constructed nature of experience, about how
 subjects are constituted as different in the first place, about how one's
vision is structured, about language (or discourse) and history – [...]\grqq\footnotemark}
\footnotetext{Ebd.,25.}
\end{myenv}
Scott interessiert sich, wie hier deutlich wird also weniger für die Erfahrung
der vermeintlich Anderen, sondern vielmehr dafür wie diese Anderen zu Anderen
gemacht werden und wie dies mit ihrer Sicht auf die Welt verbunden ist.
Erfahrung soll entsprechend nicht als Beweis für einen Unterschied, sondern zum
Medium werden, um die Herstellung diesen Unterschiedes zu untersuchen:
\begin{myenv}
 \textit{\glqq 
  The evidence of experience then becomes evidence for the fact of difference, rather than a way of exploring how difference is established, how it operates, how and in what way it constitutes subjects who see and act in the world.\grqq}\footnotemark \footnotetext{Ebd.}
\end{myenv}
Letztlich geht es Scott also darum, die Herstellung von Differenz zu
untersuchen. Ein Unterfangen, das Differenz nun schon als natürlich gegeben
voraussetzt, verliert das Potential, das Verhältnis von Subjekt und Diskurs in
den Blick zu nehmen um die Momente der wechselseitigen Hervorbringung
untersuchen zu können. \\

\noindent Erfahrung ist für Scott somit nur eine von vielen
Erzählungen, die niemals als Beweis, sondern lediglich als Gegenstand für eine
kritische Hinterfragung der darin zu Grunde liegenden Annahmen etc. genutzt
werden kann. Die Pluralisierung von Erzählungen und damit einhergehende
Sichtbarmachungen marginalisierter Realitäten gibt noch keinen Aufschluss über
ihre Verortung in den kontingenten Strukturen, wie beispielsweise der
Zweigeschlechtlichkeit: \glqq We learn about the existence of difference, but we do
not learn anything about how this difference is being constituted and in
relation to what.\grqq \footnotemark \footnotetext{Ebd.}\\  
Stattdessen fordert Scott dazu auf, einen historisierenden Blick auf jene Diskurse zu werfen, die Subjekte positionieren und somit Erfahrungen erst erzeugen:
\begin{myenv}
 \textit{\glqq 
It is not individuals who have experience, but subjects who are constituted through experience. Experience in this definition then becomes not the origin of our explanation […] but rather that which we seek to explain, that about which knowledge is produced.\grqq}\footnotemark\footnotetext{Ebd.}
\end{myenv}
Diese Vorstellungen, nach der Erfahrung dem Diskurs nicht vorausgeht, sondern
durch ihn ermöglicht wird, stellt Wissenschaftler\_innen damit vor die Aufgabe,
jene Erfahrungen zu historisieren und damit auch die Identitäten, die so
erzeugt werden als geschichtlich gewordene Subjektpositionen nachzuvollziehen.
\begin{myenv}
 \textit{\glqq 
 An jet it is precicely the questions precluded – questions about discourse, difference, and subjectivity, as well as about what counts as experience and who gets to make that determination – that would enable ust o historicize experience, to reflect critically on the history we write about it, rather than premise out history upon it.\grqq}\footnotemark \footnotetext{Ebd., 33.}
\end{myenv}
Scott fordert also dazu auf, einen Schritt zurückzugehen und Erfahrung nicht
als leichte Antwort auf auf schwierige Fragen auszunutzen.\\

\noindent Um Machtverhältnisse
zu untersuchen, müssen, so verstehe ich Scott, jene Verhältnisse in denen
Erfahrungen entstehen und als Erfahrungen markiert und gehört werden, als
machtvolle Verhältnisse untersucht werden. Nicht die Erfahrung an sich, sondern
der Umgang der Wissenschaft mit Erfahrungen und damit, wer wessen Erfahrung
Gültigkeit verleiht, muss damit im Zentrum stehen, um herrschende Verhältnisse
unterwandern zu können.\\


Linda Martin Alcoff schließt sich Scotts Erkenntnisinteresse an, und versteht
wie Scott unter Feminismus im weitesten Sinne eine Ideologiekritik.\\
Ideologien
können im Anschluss an Althusser als Praxen der Legitimierung von Herrschaft
bezeichnet werden, die es vermögen, selbst denjenigen, die unter dieser
Herrschaft leiden, die erlittene Ungerechtigkeit als unveränderlich, u.U. sogar
gerecht zu inszenieren (vgl. Herrschaft, voriges Kapitel). \\
Als Phänomenologin\_* fragt sie
nun in Anbetracht der Tatsache, dass sich phänomenologische Analysen auf
leibliche Erfahrungen stützen, wie die Mechanismen dieser ideologischen Praxis
anhand von Erfahrungen untersucht werden können, die sich folglich nicht
außerhalb sondern innerhalb des ideologischen Apparates befinden und seinen
Praktiken entsprechend nicht nur ausgesetzt sind, sondern in ihnen mitwirken.\\

\noindent Sie greift damit Scotts Bedenken an der Möglichkeit einer auf Erfahrung
basierenden Erkenntnis auf, kommt jedoch zu einem anderen Schluss:\\
Anders als
Scott weist Alcoff die methodologische Bezugnahme von Erfahrung nicht als
unbrauchbar zurück. Sie plädiert stattdessen für einen phänomenologischen
Feminismus, der sich zwar kritisch auf phänomenologische Grundkategorien
bezieht ihren Erfahrungsbegriff jedoch nicht zurückweist sondern für
feministische Theoriebildung fruchtbar macht.\\

\noindent Ihre Argumentation folgt dabei
einer geneologischen Herangehensweise, das heiß, sie leitet die Notwendigkeit,
sich mit dem phänomenologischen Konzept der Erfahrung auseinander zusetzen,
anhand einer historisierenden Perspektive auf die ursprünglichen Anliegen des
Feminismus her. \\
Dazu bezieht sie sich auf das in der europäischen Philosophie
grundlegende Problem, dass Vernunft und Erkenntnis als vergeschlechtlichte
Kategorien konzipiert werden. Dies würde Frauen\_* von den Möglichkeiten an
Wissensproduktion bzw. Erkenntnis teilzuhaben ausschließen: \\
Vernunft, so
schreibt sie, wird durch ihren Gegensatz zur weiblichen Leiblichkeit definiert
und durch die Vorherrschaft über sie manifestiert.\footnotemark\footnotetext{Alcoff, \glqq Phänomenologie, Poststrukturalismus und
 feministische Theorie. Zum Begriff der Erfahrung,\grqq in Phänomenologie und
Geschlechterdifferenz, herausgegeben von Silvia Stoller und Helmuth Vetter,
(Wien: WUV Universitätsverlag, 1997), 229.} \\
Irrationalität, Intuition und
Emotionalität werden hier als weibliche und zugleich der männlichen Vernunft
gegenüber abgewertete Begriffe genutzt, um den wie auch schon von Grosfoguel
beschriebenen Geist- Körper Dualismus aufrecht zu erhalten, bzw. das darin
enthaltenen Machtverhältnis zu legitimieren: Der Geist muss über den Körper
herrschen, \glqq wenn der Mensch Erkenntnis erlangen will.\grqq \footnotemark
\footnotetext{Alcoff, \glqq Zum Begriff der Erfahrung, \grqq Ebd. }\\
Alcoff verdeutlicht die Tragweite dieses Denkens anhand eines Zitates von Rousseau:
\begin{myenv}
 \textit{
 \glqq Der Mann ist nur hie und da Mann, die Frau aber ist immer eine Frau
 […]. Alles erinnert sie an ihr Geschlecht. So empfiehlt er: Ziehe die Meinung
 einer Frau [nur] in Fragen des Körpers und allem, was die Sinne betrifft, zu
 Rate. Bei Fragen der Moral und allem, was den Verstand betrifft, so höre auf
 die Meinung der Männer \grqq \footnotemark \footnotetext{Rousseau in Alcoff,
 ebd., 229.}} und sie resümiert: \textit{ \glqq Die männliche Vernunft wurde
 durch diese wertenden Hierarchie von Körper und Geist paradoxerweise sowohl
 gestützt als auch verdeckt. Indem der empfindende Leib vom erkennenden Geist
 getrennt wurde und nur als roher Empfänger von Wahrnehmungsbilder diente,
 schienen körperliche Unterschiede keine Rolle in der Ausformung von Vernunft
 spielen zu können. Der Körper und die ihm zugeschriebenen Gelüste,
 Leidenschaften etc. mussten alle bezwungen werden, um dem Geiste die
Möglichkeit nach Erkenntnis zu gewähren.\grqq \footnotemark \footnotetext{
Alcoff, Ebd., 230.} } \end{myenv}
Die Vernunft wurde somit zum körperlosen, und damit universell gültigen Maß
erkoren, und all diejenigen, denen diese Vernunft abgesprochen wurde, z.B.
Frauen\_*, wie hier mit Rousseau verdeutlicht wird, vom Wissenschaftsbetrieb
ausgeschlossen. \\
Alcoffs Analyse liest sich somit analog zu Grosfoguels Analyse
des dualistischen Wissenschaftsverständnisses mit dem zentralen Unterschied,
dass sie zumindest hier, nicht auf den Ausschluss rassifizierter Subjekte Bezug
nimmt. Auch das Rousseau mit großer Wahrscheinlichkeit von \textit{w}eißen Männern und
Frauen ausgeht, wird von Alcoff nicht berücksichtigt.\\ 

Auch Louise Lopman kommt in
ihren Auseinandersetzungen mit dem Dualismus in der Soziologie zu einem
ähnlich, weißen Ergebnis: Sie zeigt auf, dass es männliche Perspektiven auf
Frauen\_* sind, die die Theorien von Sokrates bis hin zu Rousseau etc. prägen.
Frauen\_* würden hier stets dem männlichen Blick und entsprechendem Maßstab
untergeordnet und dienten lediglich als Objekte der Analyse. Es fehlen, so
resümiert sie, Selbstdefinitionen von und durch Frauen\_*.\footnotemark
\footnotetext{ Louise Levesque Lopman, \textit{Claiming Reality. Phenomenology
and Women's Experience}. (Totowa: New Jersey, Rowman and Littlefield Publishers, 1988), 1-5.} \\
Dass sowohl die
Männer\_* als auch Frauen\_* in Bezug auf ihre Position in rassistischen
Verhältnissen unmarkiert bleiben, könnte nun, angesichts dessen, dass niemals
alle Differenzverhältnisse mitgedacht werden (können), verharmlost werden. Es
ist immer einfach, anzumerken, woran eine\_r nicht gedacht hat.\\

\noindent Mir geht es hier jedoch nicht nur um das Aufzeigen von Lücken. Das gravierenende an
der Dethematisierung von Rassismus liegt meiner Ansicht nach darin, dass hier
die Entstehung des dualistischen Wissenssystems im kolonialen Verhältnis
unberücksichtigt bleibt. Die Auseinandersetzungen ignorieren damit nicht
irgendein Herrschaftsverhältnis, sondern dasjenige, auf das sich der
Kolonialismus in erste Linie stützte: Rassismus.\\

\noindent Eine feministische Kritik mit
dem Ziel, Frauen\_* eine gleichberechtigte Rolle zu ermöglichen, kann den
Autor\_innen nach jedoch nur gelingen, wenn sie jenen Dualismus aufbricht.
Alcoff beruft sich hierbei auf Genevieve Loyd, die davor warnt, die Lösung
einzig und allein im Anspruch auf Vernunft zu behaupten ohne die gefährliche
Gegenüberstellung von Körper und Geist zu überwinden. Denn dies führe zu nichts
anderem als dem eigenen Ausschluss: \glqq Wenn die Frau es nun wagt, in die Sphäre
der Vernunft einzudringen so löscht sie sich zugleich selbst aus, da die
Vernunft auf dem Ausschluss des Weiblichen basiert\grqq \footnotemark
\footnotetext{Genieve Loyd zitiert in Alcoff,\glqq Zum Begriff der
Erfahrung,\grqq 230.} Für Alcoff besteht also
die Aneignung einer Kategorie, wie der Vernunft, nicht in einer Ermächtigung
über sondern in einer Affirmation eben jenes Dualismus, der über die Bestimmung
dessen, was als Quelle von Wissen Geltung beanspruchen darf, entscheidet. Statt
einer Affirmation steht der Feminismus jedoch vor der Aufgabe, einen
Wahrheitsbegriff zu entwickeln, der nicht vom Körper losgelöst ist, sondern
\glqq
die intensive libidinöse Kraft, auf welche sie sich stützt, respektiert und
deren Spuren trägt\grqq\footnotemark \footnotetext{Ros Braidotti in Alcoff,
Ebd., 231.}\\
So kommt Alcoff zu dem Ergebnis, dass \glqq wenn Frauen\_* eine
epistemische Glaubwürdigkeit und Autorität besitzen sollen, […] wir die Rolle
der leiblichen Erfahrung in der Entwicklung von Wissen neu erwägen\grqq
\footnotemark \footnotetext{Alcoff, Ebd.} müssen.\\
\noindent Doch was hat es mit dem Leib, bzw. der leiblichen Erfahrung eigentlich auf
sich? Bzw. wie kann leibliche Erfahrung zur Entwicklung von Wissen genutzt
werden? Und welche Körper werden hier imaginiert?

\subsubsection{Feminismus/ Körper/ Soma/ Erfahrung}

In ihrem Buch „Claiming Reality. Phenomenology and Women's Experience“,
diskutiert Lopman die Vorzüge der phänomenologischen Soziologie für
feministische Forschung und bezieht sich im Anschluss an Alcoff auf die in
soziologischen Forschungen gängige Ignoranz gegenüber den sogenannten
\glqq Mind/Body Experiences\grqq also Geist/Körpererfahrungen. Insbesondere jene
Frauen\_*spezifischen Erfahrungen, wie \glqq Schwangerschaft, Geburt, Menstruation
und Menopause [...]\grqq \footnotemark \footnotetext{Lopman, \textit{Claiming
Reality}, VIII.} blieben hier, so ihre Beobachtung, unberücksichtigt.\\

Für Lopman ergibt sich hieraus die Notwendigkeit, eine Methode zu entwickeln,
die gegenüber dem \glqq intuitiven und empirischen Wissen\grqq \footnotemark
\footnotetext{Ebd.} der je individuellen
Erfahrungen von Frauen\_* aufmerksam ist.\\
 So hätte die feministische Forschung
nicht allein die Aufgabe, Kritik an z.B. den sexistischen Strukturen
gesellschaftlicher Institutionen zu richten. Sie sei darüber hinaus gefordert,
\glqq den partriarchalen Blick auf unsere Erfahrung der sozialen Welt, auf
fundamentale und tiefgreifende Weise zu transformieren.\grqq Dafür, so fügt sie
hinzu, reicht ein bloßer Standpunkt nicht aus, viel eher braucht es eine
Theorie und eine Methode.\\
 Es folgt eine Übersicht über die unterschiedlichen
Ausgangslagen und Stoßrichungen feministischer Interventionen innerhalb der
klassischen Soziologie in den 70iger und 80iger Jahren. Ein wesentlicher Fokus
der Kritik stellt ihrer Ansicht dabei die geteilte Beobachtung dar, dass
männliche Erfahrungen und daraus entstehende Perspektiven auf Wirklichkeit die
Theorietradition innerhalb der Soziologie bestimmen, sodass Frauen\_* nur durch
den männlichen Blick als Objekte vorkommen. Die feministische Strategie
entgegnet dem, so Lopman, indem sie die weibliche Erfahrung als Ausgangspunkt
von Wissensproduktion stellt. Diese sollte jedoch nicht komplementär zu
bisherigen Erkenntnissen verstanden werden, sondern letztere in ihrem
Erkenntnisanspruch radikal in Frage stellen. Hierbei werden von ihr sowohl
Methodik als auch Gegenstand herkömmlicher soziologischer Theorien kritisiert,
und ein feministischen Projekt entworfen, in dem sowohl der Körper, als auch
ein spezifisch weiblicher subjektiver Zugang zu Wirklichkeit in den Vordergrund
rückt. Damit verschiebt sich die Aufmerksamkeitsrichtung von einem Sprechen und
Forschen über Frauen\_* hin zu einer Forschung aus der Perspektive von
Frauen\_*. \footnotemark \footnotetext{Ebd., vgl. 1-4.}\\

\noindent Mir fällt auf, dass Körper-Geist Dualismus zu durchbrechen, keine leichte
Aufgabe ist. Während dem Poststrukturalismus Affirmation vorgeworfen wird, weil
er wie androzentrische Theorien das körperliche negiert und als Konstrukt
ablehnt, kann der phänomenologischen Position von Lopman, angefangen bei ihren
Beispielen der genuin weiblichen Erfahrung von Schwangerschaft etc. über die
weiblichen Zugang zu Wirklichkeit eine essentialistische, cis-normative und
dualistische Vorstellung von weiblich vs. männlicher Erfahrung unterstellt
werden. \\
Beide Positionen scheitern daran, Geist und Körper zusammen zu denken
und so den Dualismus tatsächlich aufzubrechen. Körper wird hierbei weder von
den poststrukturalistischen noch von den phänomenologischen Theoretiker\_innen
als rassifiziert gedacht. Die Dethematisierung davon, wie sich Rassismus in
Körper einschreibt und dass eine historisierende Perspektive sich mit den
kolonialen Konstruktionen weißer und Schwarzer Körper auseinandersetzen müsste,
zeigt auf, dass der postkoloniale Diskurs von den \textit{w}eiß dominierten
feministischen Auseinandersetzungen um Erfahrung und Körper ignoriert bleibt.\\

Barabara Duden grenzt sich von jenen phänomenologischen Konzeption des Leibes
durch den von ihr genutzten Begriff des Soma ab, knüpft jedoch diskursiv an den
von Alcoff argumentierten Standpunkt an, nach der die Erfahrung des leiblichen,
bzw. des Soma einen notwendige Dimension für die feministische
Erkenntnistheorie beinhaltet. Dabei führt sie auch ein \textit{w}eiß dominiertes Körper
bzw. Soma Verständnis fort. Zugleich kann ihre Theorie des Soma auch für
postkoloniale Perspektiven fruchtbar gemacht werden, da die diskursive
Dimension von Soma zumindest Imganinationsmöglichkeiten für nicht-\textit{w}eiße Körper
eröffnet, selbst wenn das nicht explizit benannt wird.\\

\noindent Die Auseinandersetzung mit Erfahrung erfordert, so argumentiert Duden, das
Verhältnis zwischen dem Leiblichen und dem Diskursiven zu untersuchen auch um
das asymmetrische Verhältnis in dem sowohl auf quantitativer Ebene (\glqq da immer
weniger vom Gewussten […] begriffen oder gar betastet [werden] kann […] und
somit das leibhaft erfahrene Wissen der Fülle an Informationen und Kenntnissen
weichen muss\grqq)\footnotemark \footnotetext{Barbara Duden, \glqq
Somatisches Wissen, Erfahrungswissen und 'diskursive' Gewissheiten.
Überlegungen zum Erfahrungsbegriff aus der Sicht einer Körper-Historikerin.
\grqq in
Erfahrung: Alles nur Diskurs? Zur Verwendung des Erfahrungsbegriffes in der
Geschlechtergeschichte. Herausgegeben Marguérite Bos,et al. (Zürich: Chronos
Verlag, 2004), 27.} als auch auf qualitativer Ebene (medizinisches Wissen vs.
Sinnliche Wahrnehmung des eigenen Körpers) zwischen leiblichem und diskursiven
Wissen gewertet wird, zu berücksichtigen.\\

\noindent Die Gegenüberstellung von leiblichem und diskursivem Wissen, ermöglicht ihr die
Selbsterfahrung des eigenen Körpers als Wissen zu theoretisieren, das außerhalb
von kategorialen Rastern existiert und damit einen eigenständigen
Wissensbestand bildet. \\
Um nicht, in Anschluss an Lopman ein essentialisierenden
Körperbegriff zu reproduzieren grenzt sie sich jedoch von dem
phänomenologischen Begriff des Leibes ab und nutzt stattdessen den Begriff
Soma, der den Körper als \glqq epochenspezifische Erfahrung\grqq \footnotemark
\footnotetext{Vgl. Duden, \glqq Somatisches Wissen \grqq. } konzipiert und somit
etwas Bezeichnet, das niemals immer schon da ist sondern sich im Werden
befindet: „Mit Soma versuche ich, etwas immer Geschichtliches zu bezeichnen und
so den Fehlschluss zu vermeiden, es ginge mir um eine natürlich gegebene oder
gar 'authentische' Befindlichkeit. Interessant ist an dieser Stelle ihr
Geschichtsverständnis, das sie Prodi entlehnt: \glqq [...] Geschichte ist die
Disziplin, die im Licht von heute die Vergangenheit die in uns ist, sucht\grqq
\footnotemark \footnotetext{Paolo Prado in Duden, ebd., 33.}\\

\noindent Mit diesem, verkörperten Geschichtsbegriff macht Duden nun darauf aufmerksam,
dass es \glqq leibliche Erfahrung gab, die nicht das Resultat kategorialer
Konstruktion war, und die doch in ihrer historischen Prägung untersucht werden
kann.\grqq \footnotemark \footnotetext{Duden, Ebd., 25.} Der, u.A. aus poststrukturalistische Perspektive formulierte Zweifel
\glqq an der sinnlichen Erfahrung spiegelt [für Duden] eine epistemische
Bodenlosigkeit sondergleichen [wieder]: Den Verlust der untrüglichen
Gewissheit, leibhaftig die Quelle der eigenen Aussage zu sein.\grqq\footnotemark \footnotetext{Ebd., 26.}\\
Die leibhafte
Erfahrung wird damit für Duden zu unumstößlichem Wissen, ohne dass sie die
historischen Bedingungen, die diese Erfahrung erst ermöglichen, als unwichtig
zurückweist. \\
In gewisser Weise, beschreibt Duden einen Zwischenraum, zwischen
dem diskursiv bestimmten und dem Ursprünglichen, das sich der kategorialen
Macht entzieht. Es ist weder natürlich und immer schon da, noch ist es
ausschließlich durch das Soziale konstruiert und damit abhängig von dem bereits
Gewussten und Artikulierten. \\

\noindent Jener Zwischenraum birgt jedoch stets Gefahr vereinnahmt zu werden, selbst von
jenen die ihn im Sinne der Emanzipation aufspüren und sich gegen die
hegemoniale Ordnung wenden. Cathleen Canning \footnotemark \footnotetext{
Cathleen Canning,\glqq Feminist Discourse after the Linguistic Turn: Historicizing
Discourse and Experience, \grqq in Signs, Vol. 19 Nr. 2, (1994):370.} führt hier an, dass die
Neuschreibung von Subjektivität als Ort von Diskontinuität und Konflikt
zunächst zwar, wie eben beschrieben, einen emanzipatorischen Raum öffnen kann
in dem weibliche Subjektivitäten sich durch die Artikulation von weiblichen
Erfahrungen einen Platz erkämpften und so die Gleichsetzeung von \textit{human} with
\textit{male} korigieren. Postkoloniale Feminist\_innen fragen nun so Canning,
von wessen Befreiung hier die Rede ist. Sie stellen somit nicht nur die
Geschichte des human as male sondern auch die Geschichte der \textit{woman} as
\textit{white} in
Frage:
\begin{myenv}
 \textit{
 \glqq[...]'feminist dream of a common naming of experience' was illusory,
 totalizing, and racist'. As feminists of color rewrote histories of slavery,
 colonialism, and feminism from their oppositionall ocations,they also
 contested their own colonization in the discourses of Western feminist
 humanism.\grqq \footnotemark \footnotetext{ Cathleen Canning, \glqq Historicizing
 Discourse and Experience \grqq, Ebd.} }
\end{myenv}
In diesem Sinne möchte ich mit Mohanty und ihren Reflexionen aus einer
postkolonialen Perspektive das Potential der Selbsterfahrung aus der Sicht von
Frauen of Color und Schwarzer Frauen fortführen.\\

Widerständige, und auf vielfältige Art und Weise artikulierte
Selbstdefinitionen aus Selbsterfahrung ermöglichen, so argumentiert Mohanty,
jene Leerstellen zu füllen, für die es im dominanten Diskurs keinen Ort gibt.
Indem die Selbstbeschreibungen mit den Kategorien brechen die von der
hegemonialen Ordnung für die Marginalisierten vorhergesehen werden,
transformieren sie diese bis zur Unkenntlichkeit und schaffen damit
Bezugspunkte für die Artikulation von Erfahrung die sich in ihrer Differenz
gegenüber der dominanten Erfahrung behaupten.\footnotemark\footnotetext{Mohanty, \glqq On Race and Voice,\grqq ebd. }\\

\noindent Es ist diese Differenz, nicht die Differenz der Diversity Konzepte, sondern die
Differenz die aneckt, die das bestehende irritiert und seiner vermeintlichen
Vollständigkeit beraubt die für Mohanty das Potential hat, einer
Homogenisierung der Marginalisierten entgegenzuwirken. Differenz muss demnach
sowohl als Erfahrung von Differenz innerhalb des vermeintlichen Kollektives als
auch in Abgrenzung zu der Norm gedacht werden. Es ist eine Differenz für die es
keine Sprecher\_in geben kann, sondern die sich über die vielfältigen sich
einander widersprechender Stimmen\footnotemark \footnotetext{Al-Samaray,
\glqq Inspired Topography, \grqq 120.} auszeichnet über die widerspenstige,
uneindeutige Erfahrungen artikuliert werden und die damit jedem Versuch der
Essentialisierung der\_des Anderen entgegensteht. \\
Positionierung wird damit zu
einer Strategie, die sich gegen das Dominante behauptet und sich in der
Behauptung unmittelbar der Einordnung durch den dominanten Diskurs entzieht.
Bündnisse sind darum immer als strategische Zusammenschlüsse zu verstehen, die
die überlebensnotwendig werden, um sich in der hegemonialen Ordnung behaupten
zu können. \glqq You don't go into coalition because you like it. The only reason
you would consider tring to team up with somebody who could possibly kill you,
is because that's the only way you can figure, you can stay alive.\grqq
\footnotemark \footnotetext{Bernice Johnson Reagon zitiert in Mohanty, Ort unbekannt.}\\

\noindent Dies zeigt, dass der weiß dominierte feministische Diskurs durchaus auch für
den Postkolonialismus eine Plattform bildet, solange die Antwort auf die Frage
was mit weibliche Erfahrungen gemeint wird fortlaufend gestellt wird und offen
bleibt für die vielfältigen Positionen aus denen Frauen\_* heraus Erfahrungen
in strukturellen, historisch gewachsenen Gewaltverhältnissen machen.

\subsubsection{Grenzen widerständigen Wissens}

Sowohl in der phänomenologischen Perspektive, in der Erfahrung als Quelle der
Erkenntnis verstanden und ihre subversives Potential auch auf Diskurse
einwirken zu können beansprucht wird\footnotemark \footnotetext{Alcoff, \glqq Zum
Begriff der Erfahrung \grqq.} als auch in poststrukturalistischen
Perspektiven die Erfahrung stärker als Produkt von Diskursen begreift\footnotemark \footnotetext{Scott \glqq Experience, \grqq.} nimmt
die frage nach den Möglichkeiten Standorte zu beschreiben, und damit situiertes
Wissen auch als solches zu markieren eine große Bedeutung ein.
Standortgebundene Erkenntnisproduktion kommt nicht umhin, der Theoretisierung
von Erfahrung im Sinne der Geschichte des Blickwinkels von dem aus Wissen
produziert wird, einen Wert zuzuschreiben.\\
Während die phänomenologische
Perspektive hier in der unterschiedlichen Erfahrung von Wirklichkeit das
Potential sieht, dominante Perspektiven auszuhebeln, und ihre Partikularität
zu entlarven verhält sich der Poststrukturalismus zögerlicher. Die Struktur von
Diskursen kann demnach nicht über die Analyse von Erfahrungen allein aufgedeckt
werden, da letztere erst durch die Diskurse hindurch entstehen, artikulierbar
werden und entsprechend nicht außerhalb machtvoller Normen darüber was wie
sagbar ist, denkbar sind.\\

\noindent Es wird deutlich dass poststrukturalistische und phänomenologische Positionen
bestreben, den Körper-Geist Dualismus zu überwinden und dadurch auch die
jahrhundertelange Tradition von einem Forschenden Subjekt und einem beforschten
Objekt aufzubrechen. \\
Während in der Phänomenologie dazu der Erfahrungsbegriff
produktiv diskutiert wird und wie bei Duden erkennbar, ein
anti-essentialistisches Körperverständnis entwickelt wird, bleibt im
Poststrukturalismus die Vorsicht bestehen, sich auf eine Kategorie zu stürzen
die dermaßen ideologisch aufgeladen und damit nicht fruchtbar für eine Analyse
eben jener Ideologie zu sein scheint. \\
Gerade im Poststrukturalismus wird daran
gezweifelt, ob Objekte von Forschung einfach zu Subjekten transformiert werden
können indem ihre Erfahrungen als Wissen geltend gemacht werden. Erkenntnis ist
hier nicht automatisch an eine soziale Position gebunden, sondern an einen
Prozess der Untersuchung, des Schreiben und Denkens. Also an eine Praktik die
weniger aus einer Erfahrung resultiert als an einem analytischen Blick auf
Verhältnisse und einer Reflexion der eigenen Verwobenheit in ihnen. \\
Erfahrung
wird jedoch auch im Poststrukturalismus nicht vollkommen negiert. Es wird, das
konnte mit Scott gezeigt werden weniger als Quelle von Erkenntnis, sondern als
Medium für Erkenntnis konzipiert.\\

\noindent Für fast alle diskutierten Positionen kann jedoch festgestellt werden, dass
eine Situierung im postkolonialen Verhältnis ausbleibt. Die Kritik am Dualismus
und der Versuch Subjekt-Objekt Verhältnisse zu überwinden wurde weder in den
poststrukturalistischen noch in den phänomenologischen Positionen als Teil
einer Dekolonisierung \glqq of hearts and minds\grqq  begriffen.\\
Dass Körper und Geist
nicht nur vergeschlechtlichte, sondern auch rassifizierte Kategorien sind, die
sich durch das koloniale Verhältnis hindurch formen konnten, wird erst von
Canning in Bezug auf Mohanty und Haraway deutlich. Mohanty beschreibt hier
anschaulich, wie notwendig eine postkoloniale Aneignung des Erfahrungsbegriffs
für eine emanzipatorische Epistemologie ist die hier Erfahrung immer als
Erfahrung von (rassifizierter) Differenz denkt.\\

\noindent Im folgenden möchte ich mit dieser Position weiterdenken und dabei Erfahrung in
Differenzverhältnissen als Ausgangspunkt und Gegenstand von Gegenerzählung diskutieren.
