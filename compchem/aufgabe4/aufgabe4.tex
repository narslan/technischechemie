\documentclass[12pt]{article}
\usepackage{amsmath,mathtools}
\usepackage[usenames,dvipsnames]{xcolor}
\usepackage[german]{babel}
\usepackage[utf8]{inputenc}
\usepackage[T1]{fontenc}
\usepackage{textcomp}
%\usepackage{libertine}
%\usepackage{helvet}

\usepackage{microtype}
% Minion and Myriad fonts
\usepackage[minionint,mathlf]{MinionPro}
\renewcommand{\sfdefault}{Myriad-LF}
\usepackage{siunitx}
\usepackage{fancyhdr}
\usepackage{sectsty}
\usepackage{setspace}
\usepackage{booktabs} % To thicken table lines
\usepackage[version=4]{mhchem}
\usepackage[draft]{graphicx}
\usepackage[labelfont=bf]{caption}
\usepackage{subcaption}
\usepackage{chemstyle}

%\usepackage[compatibility=4.7,language=german]{chemmacros}
\renewcommand{\familydefault}{\sfdefault}
\sisetup{detect-all}
\usepackage{chngcntr}
\counterwithin{table}{section}
\counterwithin{figure}{section}

\usepackage{titlesec}

\titleformat*{\section}{\large\bfseries}
\titleformat*{\subsection}{\normalsize\bfseries}

\usepackage{geometry}
 \geometry{
 a4paper,
 left=20mm,
 top=30mm,
 right=20mm
 }

\begin{document}

\begin{onehalfspace}

\section{Potentialkurve von \ce{N_2}}
Ziel dieser Aufgabe ist es, die Potentialkurve des Stickstoffmoleküls zu berechnen und anschließend die Orbitale visualisiert und ein MO-Diagramm erstellt werden.
\subsection{Berechnung der Potenzialkurve auf Hartree-Fock Niveau}
\textbf{Auswertung 1 und 2 }\\
\begin{figure}[!htpb]
\centering
  \includegraphics[width=0.8\textwidth]{data/potentialkurve.png}%
  \caption{Potenzialkurve für Stickstoff auf RHF/cc-pVTZ-Niveau}
\end{figure}

\begin{table}[!htpb]
\centering
\caption{}
\begin{tabular}{lllclll}
\toprule
Molekül &
Methode &
Basissatz &
\parbox[t]{2cm}{Gitter\\(Min, Max, Incr)} &
Symmetrie &
GGW-Abstand &
GGW-Energie $E _h$ \\
\midrule
\ce{N _2} & RHF & cc-pVTZ &0.6, 4.6, 0.2 \si{\angstrom}& $D _{\infty h}$ & 1 \si{\angstrom} & -108.96801 \\
\bottomrule
\end{tabular}
\end{table}

\noindent
\textbf{Auswertung 3}\\
 Bei der Vorgabe wird der Bindungsabstand in jedem Eingabeschritt 0.2 \si{\angstrom} erhöht, was in 20 Schritten einen Abstand von 4.6 \si{\angstrom} ausmacht. Die Dissoziation erfolgt in etwa 10 Schritten.\\
Bei der Bindungsspaltung handelt es sich um eine homolytische Moleküldissoziation\cite{wiberg71} in der die Bindungselektronen gleichmäßig auf beide Molekülbruchstücke verteilt werden. Es entstehen dabei zwei \ce{N$\cdot$} Radikal-Fragmente\cite{wiberg384}. Der andere Dissoziationweg wäre heterolytisch, in dem
zwei Stickstoff-Fragmente mit entgegengesetzen Ladungen entstehen.\\
 \noindent
\textbf{Auswertung 4 und 5}\\
\begin{figure}[!htpb]
   \centering
\includegraphics[width=0.8\textwidth]{data/hund.png}
\caption{Die Elektronenbesetzungschemen der Stickstoff-Fragmenten von möglichen Dissoziationsarten \cite{wiberg98}}
\end{figure}\\
\noindent
Die Singulett-Konfiguration (\ce{^4_{}S}) der Stickstoff-Fragmente hat die Multiplizität von $2 \cdot \frac{3}{2} + 1 = 4$ und die Dublett-Konfiguration (\ce{^4_{}S}) hat die Multiplizität von $2 \cdot \frac{1}{2} + 1 = 2$. Nach Hunds'chen Regel der Multiplizität, ist der Singulettzustand (\ce{^4_{}S}) energetisch günstiger als der Dublettzustand(\ce{^2_{}D}).\cite{wiberg98}
\subsection{Geometrieoptimierung des Stickstoffmoleküls}
\textbf{Auswertung 1}
Die Gesamtenergie der optimierten Geometrie beträgt -108.98655 \si{\hartree} und der Gleichgewicht-Abstand 1.06711 \si{\angstrom}.\\
\textbf{Auswertung 2}
Im Vergleich zur Gesamtenergie aus dem ersten Aufgabenteil ist die Energie der optimierten Geometrie geringfügig niedriger.\\
\textbf{Auswertung 3}\\
\begin{table}[!htpb]
\centering
\caption{Der experimentelle Gasphasenabstand im \ce{N_2} Molekül \cite{wiberg653}}
\begin{tabular}{ccc}
\toprule
Abstand Teil 1 & Abstand Teil 2 (opt. Geometrie)  & experimenteller Abstand \\
1.0 \si{\angstrom} & 1.07027 \si{\angstrom} & 1.0976 \si{\angstrom} \\
\midrule
\bottomrule
\end{tabular}
\end{table}\\
\noindent
\textbf{Auswertung 4 und Auswertung 5}\\
\begin{figure}[!htpb]
   \centering
\includegraphics[width=0.5\textwidth,height=11cm]{data/mohf.png}
\caption{MO-Diagramm für ein Stickstoff-Molekül (RHF/cc-pVTZ-Niveau)}
\end{figure}

\begin{table}[!htpb]
\centering
\caption{Die experimentellen und berechneten Orbital Energien des Stickstoffs}
\begin{tabular}{lrr}
\toprule
Orbital & Orbitalenergien \cite{miessler} (eV) & Orbitalenergien(RHF/6-311G(d,p)) (eV)\\
\midrule
$3\sigma _u$ & &    0.42837\\
$1\pi _g$    & &    0.17721 \\
$1\pi _g$    & &    0.17721 \\
$1\pi _u$    & - 16.9 &  -0.62551 \\
$1\pi _u$    & - 16.9 & -0.62551 \\
$3\sigma _g$ &  - 15.5 & -0.63513 \\
$2\sigma _u$ & - 18.7 & -0.76739 \\
$2\sigma _g$ & -19.4  & -1.49206 \\
$1\sigma _u$ &  &-15.66636 \\
$1\sigma _g$ & &-15.67063 \\
\bottomrule
\end{tabular}
\end{table}
 Die HF-Energien von p-Orbitalen entsprechen den Photoelektronenspektren Werte \cite{miessler} nicht.
 Der Grund dafür ist, die RHF Methode beschreiben die Orbitale nicht gut, die weit weg vom Kern sind.

\textbf{Auswertung 6}
\begin{figure}[!hptb]
    \centering
    \begin{subfigure}[b]{0.4\textwidth}
        \includegraphics[width=\textwidth]{data/orbitale/1.png}
    \end{subfigure}
    ~ %add desired spacing between images, e. g. ~, \quad, \qquad, \hfill etc.
      %(or a blank line to force the subfigure onto a new line)
    \begin{subfigure}[b]{0.4\textwidth}
        \includegraphics[width=\textwidth]{data/orbitale/2.png}
    \end{subfigure}
    \caption{$1 \sigma _g$  und $1 \sigma _u$ }
\end{figure}

\begin{figure}[!hptb]
    \centering
    \begin{subfigure}[b]{0.4\textwidth}
        \includegraphics[width=\textwidth]{data/orbitale/3.png}
    \end{subfigure}
    ~ %add desired spacing between images, e. g. ~, \quad, \qquad, \hfill etc.
      %(or a blank line to force the subfigure onto a new line)
    \begin{subfigure}[b]{0.4\textwidth}
        \includegraphics[width=\textwidth]{data/orbitale/4.png}
    \end{subfigure}
    \caption{$2 \sigma _g$ und $2 \sigma _u$ }
\end{figure}

\begin{figure}[!hptb]
    \centering
    \begin{subfigure}[b]{0.4\textwidth}
        \includegraphics[width=\textwidth]{data/orbitale/5.png}
    \end{subfigure}
    ~ %add desired spacing between images, e. g. ~, \quad, \qquad, \hfill etc.
      %(or a blank line to force the subfigure onto a new line)
    \begin{subfigure}[b]{0.4\textwidth}
        \includegraphics[width=\textwidth]{data/orbitale/6.png}
    \end{subfigure}
     \caption{$1 \pi _u^x$ und $1 \pi _u^y$ }
\end{figure}
\newpage
\begin{figure}[!hptb]
    \centering
    \begin{subfigure}[b]{0.4\textwidth}
        \includegraphics[width=\textwidth]{data/orbitale/7.png}
    \end{subfigure}
    ~ %add desired spacing between images, e. g. ~, \quad, \qquad, \hfill etc.
      %(or a blank line to force the subfigure onto a new line)
    \begin{subfigure}[b]{0.4\textwidth}
        \includegraphics[width=\textwidth]{data/orbitale/8.png}
    \end{subfigure}
       \caption{$3 \sigma _g$ und $1 \pi _g$}
\end{figure}

\begin{figure}[!hptb]
    \centering
    \begin{subfigure}[b]{0.4\textwidth}
        \includegraphics[width=\textwidth]{data/orbitale/9.png}
    \end{subfigure}
    ~ %add desired spacing between images, e. g. ~, \quad, \qquad, \hfill etc.
      %(or a blank line to force the subfigure onto a new line)
    \begin{subfigure}[b]{0.4\textwidth}
        \includegraphics[width=\textwidth]{data/orbitale/10.png}
    \end{subfigure}
    \caption{$1 \pi _g^x$   und $3 \sigma _u$}
\end{figure}

\end{onehalfspace}
\end{document}