% !TEX encoding   = UTF-8
% !TEX program    = LuaLaTeX
% !TEX spellcheck = de_DE
\documentclass[12pt]{article}
\usepackage{amsmath,mathtools}
\usepackage[usenames,dvipsnames]{xcolor}
%\usepackage[bitstream-charter]{mathdesign}
\usepackage{microtype}
\usepackage{fontspec}

%\usepackage[utf8]{inputenc}
%\usepackage[T1]{fontenc}
%\usepackage{libertine}
\usepackage{graphicx}
\usepackage{siunitx}
\usepackage[german]{babel}
\usepackage{comment}
\usepackage{nicefrac}
\usepackage{booktabs}
\usepackage{float}
\usepackage{tikz}
\usetikzlibrary{arrows,chains,matrix,shapes}
\usepackage{pgfplots}
\pgfplotsset{compat=newest}
\usepackage[version=4]{mhchem}
\usepackage[backend=biber,sorting=none,autocite = superscript,natbib=true]{biblatex} \addbibresource{books.bib}
\usepackage{caption}
\usepackage{subcaption}
\usepackage{luacode}
\floatstyle{plaintop}
\restylefloat{table}
%\usepackage[justification=justified,singlelinecheck=false]{caption}
\usepgfplotslibrary{units}

\usepackage{fancyhdr}
\fancyhf{}
\rhead{13.12.2016}
\lhead{Nevroz Arslan, Justin König Gruppe 6}
\setlength{\headheight}{15pt}
\rfoot{\thepage }
\lfoot{Versuch 1: Fluoreszenzspektroskopie}
\pagestyle{fancy}

\usepackage{url}
\usepackage{csquotes}
\sisetup{
  round-mode          = figures,
  round-precision     = 2,
  inter-unit-product =\cdot,
   group-digits=true,          %% Zifferngruppierung an/aus
        scientific-notation = true,
}
\renewcommand{\arraystretch}{1.5}
\definecolor{skyblue1}{RGB}{135, 206, 250}
\definecolor{flame}{RGB}{226, 88, 34}
\definecolor{scarletred1}{RGB}{252, 40, 71}
\definecolor{cyanblau}{RGB}{0, 158, 224}
\definecolor{charcoal}{rgb}{0.21, 0.27, 0.31}
% header
\defaultfontfeatures{Ligatures=TeX,Numbers=OldStyle,Scale=MatchLowercase}
\setmainfont{Linux Libertine}
\setsansfont{Linux Biolinum}

\begin{luacode*}

fs = dofile("fs.lua")

function drawPU()

 xwerte, ywerte = fs.readcsv("data/ph12anregung420_410.txt",23)
xa, xb= fs.readcsv("data/ph02anregung350_340.txt",23)


    tex.sprint("\\addplot[".."no marks,blue".."] coordinates{")

    for i=1,#xwerte do
      tex.sprint("("..xwerte[i]..","..ywerte[i]..")")
    end
    tex.sprint("};")
     tex.sprint("\\addlegendentry{2-Naphtholat}")

 tex.sprint("\\addplot[".."no marks,red".."] coordinates{")

    for i=1,#xa do
      tex.sprint("("..xa[i]..","..xb[i]..")")
    end
    tex.sprint("};")
     tex.sprint("\\addlegendentry{2-Naphthol}")



end


function drawISO()

xwerte, ywerte = fs.readcsv("data/ph02emmissions_320_330.txt",23)
xa1, xb1= fs.readcsv("data/ph04emmissions_320_330.txt",23)
xa2, xb2= fs.readcsv("data/ph06emmissions_320_330.txt",23)
xa3, xb3= fs.readcsv("data/ph08emmissions_320_330.txt",23)
xa4, xb4= fs.readcsv("data/ph10emmissions_320_330.txt",23)
xa5, xb5= fs.readcsv("data/ph12emmissions_320_330.txt",23)
    tex.sprint("\\addplot[".."no marks,red".."] coordinates{")

    for i=1,#xwerte do
      tex.sprint("("..xwerte[i]..","..ywerte[i]..")")
    end
    tex.sprint("};")
   tex.sprint("\\addlegendentry{ph 2}")

 tex.sprint("\\addplot[".."no marks,blue".."] coordinates{")
    for i=1,#xa1 do
      tex.sprint("("..xa1[i]..","..xb1[i]..")")
    end
    tex.sprint("};")
   tex.sprint("\\addlegendentry{ph 4}")

 tex.sprint("\\addplot[".."no marks,orange".."] coordinates{")
    for i=1,#xa2 do
      tex.sprint("("..xa2[i]..","..xb2[i]..")")
    end
    tex.sprint("};")
   tex.sprint("\\addlegendentry{ph 6}")


     tex.sprint("\\addplot[".."no marks,green".."] coordinates{")
    for i=1,#xa3 do
      tex.sprint("("..xa3[i]..","..xb3[i]..")")
    end
    tex.sprint("};")
   tex.sprint("\\addlegendentry{ph 8}")


     tex.sprint("\\addplot[".."no marks,teal".."] coordinates{")
    for i=1,#xa4 do
      tex.sprint("("..xa4[i]..","..xb4[i]..")")
    end
    tex.sprint("};")
   tex.sprint("\\addlegendentry{ph 10}")

 tex.sprint("\\addplot[".."no marks,black".."] coordinates{")
    for i=1,#xa5 do
      tex.sprint("("..xa5[i]..","..xb5[i]..")")
    end
    tex.sprint("};")
   tex.sprint("\\addlegendentry{ph 12}")

end


\end{luacode*}


 \newcommand\addplotzz{\directlua{drawPU()}}
 \newcommand\addplotiso{\directlua{drawISO()}}



\begin{document}

\section{Ziel des Versuches}
Der Versuch befasst sich mit spektralen Untersuchungen im UV-Bereich. Dafür werden die Fluoreszenzspektren und Anregungsspektren von 2-Naphthol und 2-Naphtolat in Wasser aufgenommen.
\section {Theorie}
\subsection{Absorption und Emission\supercite{og}}
 Eine chemische Verbindung erscheint dann farbig, wenn sie aus dem sichtbaren Teil des Spektrums einen Wellenlängenbereich selektiv absorbiert.
Im Allgemeinen sind gesättigte organische Verbindungen für unser Auge farblos, da sich ihr Absorptionsbereich außerhalb der Sichtbaren im fernen Ultraviolett
befindet. Durch Einführung von $\pi$-Bindungssystemen wie \ce{C=O}, \ce{C=N}, \ce{N=O} u.a., besonders wenn diese miteinander in Konjugation stehen, verschieben sich
die Absorptionsbanden immer mehr zum langwelligeren, sichtbaren Teil des Spektrums.
Für die Farbigkeit ist vor allem die energetisch hochliegenden und damit leichter anregbaren $\pi$ Elektronen verantwortlich.
Je stärker die $\pi$-Elektronen delokasiert sind, umso langwelliger ist das Licht, welches die Verbindung absorbiert.
Man bezeichnet eine Gruppierung in einem Molekül, die Strahlung im UV/Vis Bereich absorbiert, als \textbf{Chromophore} (gr. \textit{phoron} = Träger). Die Anhäufung von Chromophoren,
besonders in konjugierten Systemen, ruft eine Farbvertiefung (\textbf{Bathochromie}, Rotverschiebung) hervor, d.h. eine Verschiebung der Absorptionsmaxima nach längeren Wellen.
Die auxochromen (gr. \textit{auxesis}= Zunahme) Gruppen verschieben die Absorptionsmaxima noch längeren Wellen. Eine typische, auxochrome Gruppe ist die Hydroxgruppe, \ce{-OH}.
Demgegenüber können Substituenten, die über einen \ce{-I}-Effekt verfügen, einen hypsochromen Effekt (Blauverschiebung) (\textbf{Hypsochromie}) auslösen.

\subsection{Physikalische Vorgänge der Fluoreszenz\supercite{harris}}
Wir nehmen an, dass das Molekül M Licht absorbiert und in den angeregten Zustand \ce{M^*}
überführt wird:
\begin{align*}
 \text{Absorption:} &= \ce{M + h \cdot v -> M^*}\\
 \text{Geschwindigkeit:} &= - \frac{d[M^*]}{dt} = k _{Abs.} [M]
\end{align*}
Die Geschwindigkeitskonstante, $k_{Abs.}$ , hängt von der Intensität der Bestrahlung und
dem Extinktionskoeffizienten ($\epsilon$) von $M$ ab. Je intensiver die Einstrahlung und je besser es
absorbiert wird, desto schneller wird $M^*$ gebildet.

Nach der Absorption kann $M^*$ ein Photon emittieren und in den Grundzustand zurückkehren:
\begin{align*}
 \text{Emmision:} &= \ce{M^* + -> M +  h \cdot v}\\
 \text{Geschwindigkeit:} &= - \frac{d[M^*]}{dt} = k _{E} [M^*]
\end{align*}

Alternativ kann das angeregte Molekül die Energie durch Wärme verlieren:
\begin{align*}
 \text{Desaktivierung:} &= \ce{M^* -> M +  W\ddot{a}rme}\\
 \text{Geschwindigkeit:} &= - \frac{d[M^*]}{dt} = k _{D} [M^*]
\end{align*}


Eine weitere Möglichkeit ist die Übertragung der Energie des angeregten Moleküls auf ein
anderes Molekül (Quencher, Q) durch Fluoreszenzlöschung ( Quenching ). Dabei wird
der Quencher angeregt:
\begin{align*}
 \text{Quenching:} &= \ce{M^* + Q -> M +  Q^*}\\
 \text{Geschwindigkeit:} &= - \frac{d[M^*]}{dt} = k _{Q} [M^*] [Q]
\end{align*}

Aus den beschriebenen Einzelprozessen resultiert, dass sich die Geschwindigkeit der Deaktivierung von $M^*$ aus der Summe der Geschwindigkeiten von Emission, Deaktivierung
und Löschung ergibt:
\begin{align*}
 \text{Geschwindigkeit des Verschwindens von } M^* &=  k _{E} [M^*] + k _{D} [M^*] + k _{Q} [M^*] [Q]\\
\end{align*}
Setzt man Bildungs- und Abbaugeschwindigkeit von M* gleich, erhält man:
\begin{align*}
 k_{Abs} [M] &=  k _{E} [M^*] + k _{D} [M^*] + k _{Q} [M^*] [Q]\\
\end{align*}
\subsection{Quantenausbeute\supercite{harris}} % (fold)
\label{ssub:quantenausbeute}

% subsubsection quantenausbeute (end)
\grqq Die Quantenausbeute für einen photophysikalischen Prozess ist der Anteil der absor-
bierten Photonen, der die gewünschten Emissionsprozesse hervorruft. Wenn das für jedes
absorbierte Photon zutrifft, wäre die Quantenausbeute Eins.\grqq
Die Quantenausbeute für die Emission von $M^*$ ist die Geschwindigkeit der Emission
dividiert durch die Geschwindigkeit der Absorption. Bei Abwesenheit des Quenchers :
\begin{align*}
  \Phi _0 = \frac{\text{pro Sekunde Emittierte Photonen}}{\text{pro Sekunde absorbierte Photonen}} = \frac{k_{E} [M^*]}{k_{Abs.} [M]}
\end{align*}
\subsection{Aufbau des Fluoreszenz-Spektrometers\supercite{skript}}
\begin{figure}[h]

\centering
\begin{tikzpicture}[
  triangle/.style={isosceles triangle,draw=,thick,shape border rotate=90, minimum height=15mm,minimum width=15mm,inner sep=0},
  ]{
\node[triangle]{X};
\node at (0,-0.7) {Licht quelle};
\draw[->,ultra thick] (1,0.25)  -- (2,0.25);
 % \draw (0,0) rectangle (1,0.5) node[pos=.5] (a) {Licht quelle};
\draw[->,ultra thick] (1,0.25)  -- (2,0.25);
\draw (2,0) rectangle (5,0.5) node[pos=.5] (b) {Monochromator};
\draw[->,ultra thick] (5,0.25)  -- (6,0.25);
\draw (6,0) rectangle (8,0.5) node[pos=.5] (c) {Probe};
\draw[->,ultra thick] (8,0.25)  -- (9,0.25);
\draw (9,0) rectangle (11,0.5) node[pos=.5] (d) {Detektor};
\draw[->,ultra thick] (11,0.25)  -- (12,0.25);
\draw (12,0) rectangle (14,0.5) node[pos=.5] (e) {Rechner};

}
\end{tikzpicture}

\end{figure}

\section{Auswertung}
\subsection{Messergebnisse}
\begin{table}[!ht]
 \begin{tabular}{cc}
 pH-Wert & Gemessener pH-Wert   \\
\hline
 12 &    12.198    \\
 10 &  9.973 \\
  8 & 8.158  \\
   6 &  5.925 \\
    4 & 3.986  \\
     2 & 2.000  \\
\end{tabular}
  % \renewcommand\thetable{3}
  \caption{ Die Ergebnisse der pH-Messungen}
\end{table}
\subsection{Spektren}
\subsection{Die Fluoreszenzs- und Anregungsspektren von 2-Naphtol (pH 2)}
Die Fluoreszenzs- und Anregungsspektren von 2-Naphtol (pH 2) befinden sich am Anhang 1
\subsection{Die Fluoreszenzs- und Anregungsspektren von 2-Naphtolat (ph 12)}
Die Fluoreszenzs- und Anregungsspektren von 2-Naphtol (pH 2) befinden sich am Anhang 2
\subsection{Der isosbestische Punkt}
Die Präzens eines isobestischen Punkt weis darauf hin, dass eine gute Gleichgewichtseinstellung im
Naphtol-Naphtolat-System erfolgt hat.
\pgfplotsset{
  standard/.style={
    axis x line=bottom,
    axis y line=middle,
     xlabel= Wellenlänge, % Set the labels
    ylabel= Intensität ,
    x unit= \si{\nano\meter},
       legend style={at={(1.2,1)},anchor=north},
    y label style={at={(axis description cs:-.19,.5)},rotate=90,anchor=south},
    x tick label style={rotate=90,anchor=east},
   yticklabel style={/pgf/number format/sci},
scaled y ticks=false
  }
}
\begin{figure}[!hbpt]
\begin{tikzpicture}

\begin{axis}[standard]
    \addplotiso
  \end{axis}
\end{tikzpicture}
\caption{ Der Verlauf des Naphtol-Naphtolat-Gleichgewichts bei $\lambda_{Anr} = 320$}

\end{figure}

\subsection{Bestimmung der Wellenlängemaxima aus den Spektren}
Die Wellenlängemaxima der Anregungsspektren des 2-Naphtolat-2-Naphtol Gleichgewichtes werden bestimmt und tabellerisch dargestellt.
\begin{table}[!htbp]
 \begin{tabular}{ccc}
 pH-Wert & Anregungs-$\lambda$ \si{\nano\meter} & $\lambda_{max}$ \si{\nano\meter}  \\
\hline
2& 420 & 284 ,328\\
2& 350 & 284, 328\\
6& 420 & 284 , 328\\
6& 350 & 284 , 328\\
12& 420 & 340\\
12& 350 & 280, 297, 313\\
\end{tabular}
  % \renewcommand\thetable{3}
 \end{table}

\subsection{Stokes Verschiebung}
\pgfplotsset{
  standard/.style={
    axis x line=bottom,
    axis y line=middle,
     xlabel= Wellenlänge, % Set the labels
    ylabel= Intensität ,
    x unit= \si{\nano\meter},
       legend style={at={(1.2,1)},anchor=north},
    y label style={at={(axis description cs:-.19,.5)},rotate=90,anchor=south},
    x tick label style={rotate=90,anchor=east},
   yticklabel style={/pgf/number format/sci},
scaled y ticks=false
  }
}


\begin{figure}[!h]
\begin{subfigure}{.4\textwidth}
\resizebox{\linewidth}{!}{
  \includegraphics[width=1.5\textwidth]{naphthol.jpg}}
  \caption{zur Verfügung gestellt}
\end{subfigure}%
\begin{subfigure}{.4\textwidth}
\resizebox{\linewidth}{!}{
 \begin{tikzpicture}
\begin{axis}[standard]
    \addplotzz
  \end{axis}
\end{tikzpicture}}
\caption{experimentell}
\end{subfigure}
\caption{Der Vergleich der Absorptions- und Anregungsspektren des Naphtol-Naphtolat-Sytems}

\end{figure}




\subsection{Die Fluoreszenzs- und Anregungsspektren von 2-Naphtol und 2-Naphtolat (ph 6)}
Die Fluoreszenzs- und Anregungsspektren von 2-Naphtol und 2-Naphtolat (ph 6) befinden sich am Anhang 3

\subsection{$pK_s$ Werte}


\begin{table}[!htp]
 \begin{tabular}{lccccc}
 pH-Wert & $\lambda_{An.}$ \si{\nano\meter} & $\lambda_{max}$ \si{\nano\meter} & $I_{max}$ & $I_{350}$ & $I_{420}$ \\
\hline
2.000& 320 & 354 & \num{1.3172177e5}& \num{1.312e5} & \num{1.7145e4} \\
3.986& 320 & 351 & \num{1.12065891e5}& \num{1.11589e5} & \num{4.75454e4}\\
5.925& 320 & 355 & \num{1.2808172e5}& \num{1.12685039e5} & \num{4.9373e4}\\
8.158& 320 & 355 & \num{1.1583568e5}& \num{1.11267211e5}& \num{5.589716e4}\\
9.973& 320 & 415 & \num{1.144213915e5}& \num{2.25556e4} & \num{1.12899375e5}\\
12.198& 320 & 414 & \num{1.61930719e5}& \num{8.50769e2} & \num{1.59646828e5} \\
\end{tabular}
  % \renewcommand\thetable{3}
  \caption{Die Intensitäten aus der Emissionsspektren bei der Anregungswellenlänge $\lambda_{An.}$ = 320 }
\end{table}

\begin{table}[!htp]
 \begin{tabular}{lcccc}
 pH-Wert & $\lambda_{An.}$ \si{\nano\meter} & $\lambda_{max}$ \si{\nano\meter} & $I_{max}$ & $I_{420}$  \\
\hline
2.000& 360 &  411 & \num{2.2639e3}& \num{1.779e2}  \\
3.986& 360 &  411 & \num{2.31473e3}& \num{1.9123e2} \\
5.925& 360 &  411 & \num{2.316608e3}& \num{2.42999e2} \\
8.158& 360 &  411& \num{1.08663e4}& \num{8.87501e3}\\
9.973& 360 &  411 & \num{1.25291e5}& \num{1.2253e5}\\
12.198& 360 & 411 & \num{1.89882531e5}& \num{1.86032e5}  \\
\end{tabular}
  % \renewcommand\thetable{3}
  \caption{Die Intensitäten aus der Emissionsspektren bei der Anregungswellenlänge $\lambda_{An.}$ = 360 }
\end{table}


Aus der Auftragung der Fluoreszenzintensität gegen pH-Wert kann die Gleichgewichtskonstante K und der
pK-Wert ermittelt werden. Am Wendepunkt der Kurve gilt pH = pK.
Experimentell wird ein pKs Wert für 2-Naphthol$^*$ 2.9 festgestellt und ein pKs Wert für 2-Naphthol 9.2.
Der ermittelte pKs Wert 9.2 von 2-Naphthol weicht jedoch von der Literaturangabe
2-Naphthol hat pKs Wert von 9.63 bei T=25 \si{\celsius}~\supercite{crc} ab. Durch Erhöhung der Anzahl der Messungen kann
eine genaure Bestimmung des pKs-Werts gemacht werden.
\begin{figure}[!htbp]
\begin{tikzpicture}
\begin{axis}[
axis x line=bottom,
axis y line=middle,
xlabel= pH,
ylabel= Intensität,
legend style={draw=none,legend pos=outer north east,font=\scriptsize},
y label style={at={(axis description cs:-0.3,.5)},rotate=90,anchor=south},
  legend cell align=left,
 scaled ticks=false,
  y tick label style={/pgf/number format/sci},
 minor x tick num=10
]
\addplot[no marks,smooth,blue] table {
   X Y
2.000 1.77969086e2
3.986  1.91234894e2
 5.925 2.42990112e2
8.158  8.8750166e3
 9.973 1.22534625e5
 12.198 1.86032375e5
  };
  \addlegendentry{$I_f$=(360 nm/420 nm)}
\addplot[no marks,smooth,red] table {
   X Y
2.000 1.31233328e5
3.986  1.11589625e5
 5.925 1.12685039e5
8.158  1.11267211e5
 9.973 2.25562051e4
 12.198 8.50769043e2
  };
    \addlegendentry{$I_f$=(320 nm/350 nm)}

  \addplot[no marks,smooth,green] table {
   X Y
2.000 1.71453184e4
3.986  4.75454219e4
 5.925 4.93732383e4
8.158  5.58971602e4
 9.973 1.12899375e5
 12.198 1.59646828e5
  };
    \addlegendentry{$I_f$=(320 nm/420 nm)}
\node[coordinate,pin=right:{\tiny pH = pKs = 9.2}] at (axis cs:9.12,6.7e4) {};
\node[coordinate,pin=right:{\tiny pH = pKs = 2.9}] at (axis cs:2.9,3.2e4) {};

\end{axis}
\end{tikzpicture}
\caption{Kurvenverlauf der pH-Werte gegen Fluoreszenzintänsitäten und Bestimmung der pKs-Werte}
\end{figure}


\section{Diskussion}

Bei Messung der Spektren des pH Wertes 12 zeigt sich, dass sowohl bei einer Einschusswellenlänge von 320 nm (EM1) und
360 nm (EM2) die Probe eine starke Emission mit Maximum bei 411 nm zeigt. Über die Anregungsspektren
 ergeben sich Anregungsmaxima bei 280nm, 297 nm und 313 nm bei einer Emissionswellenlänge von 420 nm (AN1)
 und bei einer Emissionswellenlänge von 350nm (AN2) ein Anregungsmaxima bei 340 nm.

Es kann somit davon ausgegangen werden, dass an diesem pH Wert, das Naphthol komplett in dissoziierter Form, also dem Naphtholat vorliegt.
Die gemessene emittierte Wellenlänge (Fluoreszenz) mit 420 nm zeigt den in der Theorie bereits angesprochenen „Stokes-Shift“.
 Die Probe wird mit einer Wellenlänge von 360 nm angeregt, während das durch Fluoreszenz wieder abgegebene
 Licht eine Wellenlänge von 420nm aufweist.

% Hier Spektren einblenden, keine Ahnung wie das geht :P

Bei Messungen eines pH-Wertes von 2 ist die bei pH 12 gemessene Fluoreszenz mit einem Emissionsmaximum bei 420 nm verschwunden,
jedoch zeigt sich eine Emission bei 340 nm unter Einstrahlung von 320 nm (EM1), nicht aber bei 360 nm (EM2).
 Des Weiteren zeigen sich zwei Anregungsmaxima bei 275 nm und 330 nm bei einer Emissionswellenlänge von 350 nm (AN2).
Dies legt nahe, dass bei einem pH Wert von 2, das Naphthol in seiner undissoziierten Form vorliegt
und somit kein Naphtholat in der Probe vorhanden ist. Bei der gemessenen Fluoreszenz von 355 nm handelt
es sich um die Emission des Naphthols. Das Licht mit einer Wellenlänge von 360 nm ist nicht dazu in der Lage,
 eine Fluoreszenz des Naphthols anzuregen.

%wieder Spektren einfügen

Die Betrachtung der Spektren bei einem pH Wert von 6 zeigt, dass lediglich eine Emission bei 350 nm
 über eine Anregung mit 320 nm (EM1) gemessen werden kann. Bei 275 nm und 330 nm zeigen sich sowohl Absorptionsmaxima bei 420 nm (AN1),
  als auch bei 350 nm (AN2) Emissionswellenlänge.
Somit ergibt sich aus den Messungen der Spektren bei einem pH Wert von 6,
dass die Naphtholat-Konzentration zu gering ist bzw. zu instabil ist,
als dass eine Emission über 350 nm gemessen werden kann.
Stattdessen kann lediglich die Fluoreszenz des Naphtholats
über Anregung des Naphthols mit einer Wellenlänge von 320 nm gemessen werden.

%Vergleich mit vorgegeben Spektrum

Mittels eines Vergleiches der im Versuch gemessenen Absorptionsspektren mit den Vorgegeben Spektren wird ersichtlich,
dass sich diese auf in mehrfachen Punkten decken. Zum einen kann in beiden die Verschiebung der Absorptionswellenlänge
 vom Naphthol zum Naphtholat hin in den längeren Wellenlängenbereich beobachtet werden. Ausserdem zeigt sich,
  dass die Maxima in beiden Fällen nur minimal voneinander abweiche.
Der Vergleich zeigt, dass die gegebenen Spektren mit den gemessenen im übereinstimmen,
jedoch weisen die gegebenen Spektren eine größere Genauigkeit auf.

\printbibliography

\end{document}
