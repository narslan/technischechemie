% !TEX encoding   = UTF-8
% !TEX program    = LuaLaTeX
% !TEX spellcheck = de_DE
\documentclass{article}
\usepackage[usenames,dvipsnames]{color}
\usepackage{siunitx}

\usepackage{tikz}
\usepackage{pgfplots}
\usepackage{pgfplotstable}
\pgfplotsset{compat=newest}
\usepgfplotslibrary{units}



\usetikzlibrary{arrows,chains,matrix,shapes}
\usepackage{luacode}

\usepackage[bitstream-charter]{mathdesign}
\usepackage{microtype}
\usepackage[T1]{fontenc}


%\usepackage[landscape]{geometry}
%\setmainfont{Ubuntu}
%\defaultfontfeatures{Ligatures=TeX,Numbers=OldStyle,Scale=MatchLowercase}
%\setmainfont{Linux Libertine}
%\setsansfont{Linux Biolinum}
\usepackage{fancyhdr}
\pagestyle{fancy}
\usepackage{subcaption}
\usepackage[landscape]{geometry}
\usepackage{adjustbox}


\rhead{ \Large \textbf{Polyethylenterephthalat}}
\lhead{\large \textbf{ C (Z=6) und O (Z=8)  }}


\fancyheadoffset{-1cm}
\pagenumbering{gobble}
\usetikzlibrary{pgfplots.groupplots}
\usepackage{pgfplotstable}
\begin{luacode*}


  fs = dofile("fs.lua")

  function drawPET()

  px,py = fs.readcsv("data/pet/ubersichtspektrern.txt",0)


  tex.sprint("\\addplot[".."black".."] coordinates{")
    for i=1,#px do
    tex.sprint("("..px[i]..","..py[i]..")")
    end
  tex.sprint("};")

  tex.sprint("\\node[rotate=90] (a) at (290,1.7e5) {\\textbf{C 1s}};")
  tex.sprint("\\node[rotate=90] (a) at (536,1.7e5) {\\textbf{O 1s}};")

  end
  


\end{luacode*}


\newcommand\addplotpr{\directlua{drawPET()}}



\definecolor{skyblue1}{RGB}{135, 206, 250}
\definecolor{flame}{RGB}{226, 88, 34}
\definecolor{scarletred1}{RGB}{252, 40, 71}
\definecolor{cyanblau}{RGB}{0, 158, 224}
\definecolor{darkblue}{RGB}{0, 0,139}
\definecolor{charcoal}{rgb}{0.21, 0.27, 0.31}
\definecolor{turquoise}{rgb}{0 0.41 0.41}
\definecolor{rouge}{rgb}{0.79 0.0 0.1}
\definecolor{vert}{rgb}{0.15 0.4 0.1}
\definecolor{mauve}{rgb}{0.6 0.4 0.8}
\definecolor{violet}{rgb}{0.58 0. 0.41}
\definecolor{orange}{rgb}{0.8 0.4 0.2}
\definecolor{bleu}{rgb}{0.39, 0.58, 0.93}
\definecolor{azulen}{RGB}{0,218,255}



\begin{document}
\begin{figure}[!t]
\begin{tikzpicture}
  \begin{groupplot}[
      group style={
        group name=my plots,
        group size=1 by 1,
        xlabels at=edge bottom,
        xticklabels at=edge bottom,
        vertical sep=20pt
      },
      width=16cm,
      height=12cm,
      xlabel=Bindungsenergie,
      ylabel=Intensität, 
			tick label style={/pgf/number format/assume math mode=true,font=\sffamily},
			x tick label style={rotate=90,anchor=east},
			yticklabel style={/pgf/number format/fixed},
			x unit= \si{\electronvolt},
			tickpos=left,
			ytick align=outside,
			xtick align=outside,
			legend style={draw=none,legend pos=outer north east},
			legend cell align=left,
			x dir=reverse,
      ymin=0
		]
    \nextgroupplot[title=\textbf{Spektrum ID\# VII}]
		\addplotpr
    			\end{groupplot}
		\end{tikzpicture}
    \pgfplotstabletypeset[col sep=&,
   columns/a/.style={string type,column type=l,column name=\textbf{Spektrum ID\#}},
   columns/b/.style={string type,column type=l,column name=\textsc{VII}},
   ]{
   a & b
   \textbf{Materie} & Polyethylenterephthalat 
   \textbf{Technik} & XPS 
   \textbf{Spektral Bereich} & Übersicht
}

\caption*{\textbf{Abbildung 4.} Das Übersichtsspektrum von PET}
	\end{figure}






  \end{document}
