\documentclass[a4paper]{article}
\usepackage{tikz}
\usetikzlibrary{decorations.pathmorphing}
\usepackage{fontspec}
\usepackage{fancyhdr}
\pagestyle{fancy}
\usepackage{fontspec}
\usepackage{chemfig}
\setmainfont{Ubuntu}
\newcommand\setpolymerdelim[2]{\def\delimleft{#1}\def\delimright{#2}}

\def\makebraces(#1,#2)#3#4#5{%
  \edef\delimhalfdim{\the\dimexpr(#1+#2)/2}%
  \edef\delimvshift{\the\dimexpr(#1-#2)/2}%
  \chemmove{
    \node[at=(#4),yshift=(\delimvshift)]
      {$
       \left\delimleft
         \vrule height\delimhalfdim depth\delimhalfdim width0pt
       \right.
      $};
    \node[at=(#5),yshift=(\delimvshift)]
      {$
        \left.
          \vrule height\delimhalfdim depth\delimhalfdim width0pt
        \right\delimright_{\rlap{#3}}
      $};
  }%
}
\begin{document}
\title{Voltametrie}
\author{Nevroz Arslan, Universität Oldenburg}
\lhead{\textbf{HPLC}-Gruppe 18 \footnotesize{Anatoli Schmidt, Nevroz Arslan}}
\subsection*{HPLC}
In der hochauflösenden Flüssigkeitschromotographie \textbf{(HPLC)} wird das
Lösungsmittel bei hohem Druck durch eine kleine Säule gepumpt, die 3-10 $\mu$m 
große Teilchen als stationäre Phase enthält.\footnote{Harris C.D, Lehrbuch der
quantitativen Analyse, vieweg Verlag,1997, s. 843}  Die Säule ist mit nur
wenige Mikrometer großen Partikeln gefüllt, die eine Trennung 
von sehr kleiner Mengen($\mu$g,ng) an Analyt ermöglichen. 
\subsection*{\small Stationäre Phase}
Bei dem Versuch wird es 5$\mu$ Lichrospher 100RP18e Kieselgel verwendet. 
\bigskip
\par
\setpolymerdelim()
\chemname{\chemfig{-[,1,,,decorate,decoration=snake]O-Si(-[:90]CH_3)(-[@{op,.75}]-CH_2-[@{cl,0.25}]-CH_3)(-[:-90]CH_3)}}{polymere Stationäre Phase mit Octadecylschwanz}
\makebraces(10pt,10pt){$\!\!17$}{op}{cl}
\subsection*{\small Mobile Phase}
\begin{itemize}
\item Methanol-Waser Gemisch für Elution
\item Polar-Polar Mischung
\item Verwendung der sauberer Lösüngsmittel erforderlich
\end{itemize}
\subsection*{\small UV-Detektor}

\begin{itemize}
\item Viele Analyte ultraviolettes Licht absorbieren
\item Der UV-Detektor ein Signal zur Konzentration des Analytes
\item Die (HPLC-Anlage A) verwendet eine intensive Emmision bei 280 nm \\
  \textit{Peakwert des Absorptionsspektrums des Koffeins(~275nm).\footnote{http:\/\/webbook.nist.gov\/cgi\/cbook.cgi?ID=C58082\&Mask=400\#UV-Vis-Spec}}
\item  Die HPLC-Anlage B verwendet eine Emmision bei 254 nm\\
 \textit{Peakwert des Absorptionsspektrums der Benzoesäure und der Benzoesäurederivaten(~250nm).\footnote{http:\/\/webbook.nist.gov\/cgi\/cbook.cgi?ID=C65850\&Mask=400\#UV-Vis-Spec}}
\item Die Flieszelle des UV-Detektors besitzt ein Volumen von 20$\mu$l
\end{itemize}

\newpage


\subsection*{\small Stufenhöhe als Säuleneffienz}
\begin{itemize}
\item kleine Stufenhöhe \\
  =>schmale Peaks\\
  =>bessere Trennung
\item Stufenhöhe ist die Proportionalitätskonstante zwischen $\sigma^2$ der
      Bande und der zurückgelegten Strecke (x) \\
      \textbf{\textit{Stufenhöhe}$=\sigma^2/x$}
\item Diffussionskoeffizient(D) charakterisiert die Geschwindigkeit von der
Region hoher Konzentration zur Region niedriger Konzentration\\
       Eine funktion der Molmenge und der Zeit
\item $\sigma^2=(2Dt)=2D(x/u_x)$ , wobei $u_x$ für lineare Fließgeschwindigket
  steht und x für die in der Säule zurückgelegte Strecke steht  
\item Bei dem zweiten Teil des Versuches, wird das Verhältnis des\\
  Kapazitätsfaktors(Retenzionsfaktor) zum Peak untersucht  

\end{itemize}




\end{document}

